\chapter{Referencial Teórico} \label{Referencial Teórico}

Este capítulo reúne informações obtidas sobre usabilidade em ERPs na literatura nacional e internacional. Inicialmente, documenta-se a pesquisa realizada sobre o estudo da usabilidade em sistemas ERP, na qual se optou por usar, como técnica de pesquisa, o mapeamento sistemático (MS), com o objetivo especifico de localizar os principais estudos realizados nos últimos anos. Após o MS abordam-se os conceitos de usabilidade e ERP e conceitua-se a usabilidade com o foco em sistemas ERP. As duas últimas subseções tratam da aplicação de \textit{survey} como metodologia de pesquisa e dos trabalhos relacionados. 


\section{Mapeamento Sistemático} \label{Mapeamento Sistemático}

Para Kitchenham e Charters (2007), o mapeamento sistemático (MS) é projetado para prover uma visão mais ampla de um tópico de pesquisa, de modo a estabelecer se há evidência de pesquisa nesse tópico.\newline
\indent Para Felizardo (2017), as revisões informais de literatura são conduzidas sem um planejamento estabelecido e caracterizam-se por serem pouco abrangentes, não passiveis de repetição, pouco confiáveis e com a qualidade dependente da experiência dos pesquisadores. Já a revisão sistemática (RS) é um método de pesquisa chave para apoiar a pesquisa baseada em evidências, desenvolvida com o objetivo de reunir, avaliar e interpretar as evidências literárias disponíveis, referentes a uma questão de pesquisa, tópico específico ou fenômeno de interesse.\newline
\indent Ainda segundo Kitchenham e Charters (2007), o MS mapeia um domínio em alto nível de granularidade. Permitindo, assim, identificar os agrupamentos e, dessa forma, direciona o foco para futuras revisões sistemáticas e distingue áreas para a condução de novos estudos primários.\newline
\indent Na Figura \ref{fig:figura-revisao-x-mapeamento} é possível verificar as diferenças entre revisão sistemática e mapeamento sistemático.

\begin{figure}[H]
	\centering	
	\caption{Comparativo entre Mapeamento Sistemático e Revisão Sistemática}
	\includegraphics[width=6in, height=2.7in]{frsms}
	\label{fig:figura-revisao-x-mapeamento}
\end{figure}
\vspace{-0.8 cm} \hspace{-0.5 cm} Fonte: Falbo (2017)\newline
\indent Sendo assim, o Mapeamento Sistemático é útil para alunos de mestrado e doutorado que precisam preparar uma visão geral do tópico de pesquisa sobre o qual vão trabalhar (KITCHENHAM; CHARTERS, 2007).\newline
\indent A partir dessa premissa, foi escolhido o Mapeamento Sistemático como ferramenta para a ampliação do domínio de pesquisa, que foi conduzido no mês de abril de 2018, a partir de um total de 140 abordagens retornadas (IEEE=124, IBICT=3, e Literatura Cinzenta=13). Os trabalhos repetidos 66 (sessenta e seis) foram excluídos. Dos trabalhos restantes, 74 (setenta e quatro) foram considerados elegíveis para a segunda fase do mapeamento, que consistiu da leitura e avaliação dos resumos desses artigos, culminando com o resumo elaborado pelo autor dos artigos escolhidos.\newline
\indent Os trabalhos que, pelos seus resumos, não satisfizeram os requisitos para esta pesquisa e os demais critérios definidos para inclusão e exclusão, também foram excluídos. Sobrando 7 pesquisas, que por atenderem apenas os critérios de inclusão, tiveram o seu conteúdo analisado por completo, para que, assim, pudessem compor a síntese da pesquisa.\newline
\indent Nos resumos comparativos dos trabalhos selecionados (seleção final) são apresentadas as principais diferenças e semelhanças identificadas nessas pesquisas.\newline
\indent A leitura das obras possibilitou a identificação de trabalhos relevantes para o objetivo do mapeamento, ou seja, permitiu encontrar as obras que contribuíssem para que as perguntas estabelecidas no foco de pesquisa fossem respondidas.\newline
\indent De forma geral, buscaram-se obras que relacionassem usabilidade com aceitação de sistemas ERP na pós-implantação ou estudos da usabilidade em ERP e sua influência na percepção do usuário. Os detalhes do Mapeamento Sistemático podem ser verificados no Apêndice B.

\section{Contexto}  \label{Contexto}

Para Banhos (2008), o contexto é um dos principais elementos da informação, pois uma informação pode ser importante, ter valor, fazer sentido para uma pessoa, e não causar nenhuma mudança em outra. Conforme nos ensina em uma definição clássica bastante referenciada na literatura Dey (2001), contexto se refere a qualquer informação que possa ser utilizada para caracterizar a situação de uma entidade. Sendo que, entidade pode ser uma pessoa, lugar ou objeto que é considerada relevante para uma interação entre um usuário e uma aplicação, incluindo o próprio usuário e a aplicação. Exemplos típicos de contexto são localização, identidade, uso, usuário.\newline
\indent O procedimento de especificar, controlar e relatar o contexto no qual a medição ocorre é rotineiro há séculos. Esse procedimento assegura que as medições sejam significativas e reprodutíveis. Maguire (2001)  relata que ele e outros descobriram em meados da décáda de 1980 que, embora muitos produtos tenham um bom desempenho em seus experimentos de laboratório, eles não funcionavam quando transferidos para o mundo real.\newline
\indent Bazire e Brézillon (2005), chegaram à conclusão que a definição de contexto depende da área à qual esta relacionada, não existindo assim uma definição absoluta, mas sim, definições que se adaptam às necessidades das áreas às quais são aplicados. Outra conclusão que os autores chegaram foi que o contexto atua como um conjunto de restrições que influenciam o comportamento de um sistema (um usuário ou um computador) envolvido numa data tarefa ou atividade.\newline
\indent O contexto de uso foi reconhecido no campo da usabilidade pela norma ISO 9241 (2010) publicada inicialmente em inglês no ano de 1997, referindo-se ao contexto de uso na definição de usabilidade:\newline

	{\raggedleft
	\hspace*{4.5cm} 
	\begin{minipage}{0.7\textwidth} 
		"Usabilidade é a medida em que um produto pode ser usado por usuários específicos para alcançar objetivos específicos com eficácia, eficiência e satisfação em um contexto de uso especifico."	
		\newline	
	\end{minipage}
	\newline	
	\par}

\indent A norma ISO 9241 (2010) ainda define o contexto de uso como: \newline

{\raggedleft
	\hspace*{4.5cm} 
	\begin{minipage}{0.7\textwidth} 
		"O Contexto de Uso consiste nos usuários, tarefas e equipamentos (hardware, software e materiais) e nos ambientes físico e social nos quais um produto é usado."	
		\newline	
	\end{minipage}
	\newline	
	\par}

\indent Contexto de Uso também é incorporado na norma ISO 13407 (1999) sobre design centrado no ser humano. Isso define o processo de compreender e especificar o Contexto de Uso como um dos principais estágios do processo de design centrado no ser humano.\newline
\indent Em complemento a norma, conforme nos ensina Maguire (2001) geralmente os sistemas são divididos entre aqueles que possuem usabilidade e aqueles que não possuem usabilidade. Na verdade, é incorreto indicar a usabilidade de um sistema, sem que antes seja descrito o seu contexto de uso.  Maguire (2001) ainda nos  indica que quando se esta desenvolvendo um sistema, este será usado em um contexto particular, porém há uma tendência nos desenvolvedores de esquecer o contexto de uso\newline
\indent Segundo Vieira et al. (2009), existem dois conceitos distintos: contexto e elemento contextual. Um elemento contextual é qualquer dado, informação ou conhecimento que permite caracterizar uma entidade em um dado domínio. O contexto é um conjunto de elementos contextuais instanciados que são necessários para apoiar a tarefa atual.\newline
\indent Bazire e Brézillon (2005), após realizarem uma pesquisa com mais de 150 individuos, chegaram a conclusão que contexto é dado por  6 parâmetros: restrição: indica a capacidade de restringir um significado de uma informação; influência: indica a capacidade da informação modificar o ambiente; comportamento: indica a capacidade alterar as ações do usuário diante da informação; natureza: indica a capacidade de influenciar a natureza da informação; estrutura: refere-se a estrutura na qual o contexto se apresenta e; sistema: refere-se a influencia do contexto sobre o sistema.\newline
\indent Schilit et al. (1994) aprofundando-se na teoria sobre o contexto de uso, identificaram 4 sub-divisões: o \textit{\textbf{contexto computacional}} que se refere a rede, conectividade, custo da comunicação, banda passante, recursos tais como impressoras, estações, entre outras; o \textit{\textbf{contexto do usuário}} se refere a atributos como, situação social, estado de espírito, satisfação, percepção, entre outras; o \textit{\textbf{contexto físico}} se refere à luminosidade, nível de ruído, temperatura, umidade e; o \textit{\textbf{contexto de tempo}} se refere à hora do dia, dia/mês/ano, semana ou época do ano.\newline
\indent Dentro do contexto do usuário estudando mais a fundo o atributo percepção, nos ensina Stabile (2001) que a percepção é afetada pelos 3 fatores principais:  Interpretação: capacidade de entendimento da informação que esta sendo exibida; Relevância: refere-se a percepção da importância da informação e; Confiabilidade: percepção da validade da informação conforme a fonte.\newline
\indent Segundo Costa (2010) a percepção de um sistema é formada por um conjunto de fatores que denotam qual o valor que um usuário obtém com o uso do sistema. Esta percepção afeta as decisões dos usuários que, muitas vezes, resistem ao uso do sistema quando não encontram satisfeitas suas necessidades.\newline
\indent Independente do tipo de usuário para Costa (2010), a percepção de usabilidade torna-se distinta caso uma tarefa em uma interface seja realizada esporadicamente ou de forma frequente, ou mesmo executada em computadores rápidos ou lentos.\newline
\indent Ao se explorar outra subdivisão do contexto do usuário a satisfação, verificamos a definição constante na ISO 9126-1 (2003) que o atributo satisfação indica a qualidade software percebido quando ele esta em uso. Verificamos ainda que, como um indicador a satisfação do usuário, é um dos atributos mais inportantes como indicador de sucesso de uma implantação.\newline
\indent Tomando as idéias de Maquire (2001) já citadas e fazendo um paralelo com os sistemas  ERP constata-se o quão dificil é avaliar a usabilidade em sistemas ERP’s, dada a gama de contextos de uso que existem.\newline
\indent Macêdo, Gaete e Joia (2014) ao estudar a resistência a implantação de sistemas ERP sob o contexto do usuário,  indicaram haver inseguranças atribuidas a três indicadores mapeados por eles:
	\begin{flushright}
	\begin{minipage}{.955\textwidth}
		Poder e Politica: este indicador refere-se a insegurança do usuário no que se refere a possivel mudança de poder na empresa e a reorganização estrutural da mesma após a implantação;
	\end{minipage}
	\end{flushright}

	\begin{flushright}
	\begin{minipage}{.955\textwidth}
		Interesse Pessoal: este indicador refere-se a percepção do usuário em relação a implantação do sistema, cruzando sua espectativa inicial versus a realidade nos aspectos da melhora dos processos e procedimentos;
	\end{minipage}
	\end{flushright}
	
	\begin{flushright}
	\begin{minipage}{.955\textwidth}
		Inclinação Pessoal: este indicador se refere a falta de intesesse dos usuários em aprender as novas tecnologias e processos.
	\end{minipage}
	\end{flushright}

Desta forma baseando-se Macêdo, Gaete e Joia (2014) verifica-se que os indicadores acima listados afetam a percepção do usuário e então entende-se, a restrição da pesquisa de Lambeck, et al. (2014b) a pós-implantação, pois decorridos 1 ano, as inseguranças do processo de implantação indicadas por Macêdo, Gaete e Joia (2014) já não mais existem, pois foram mitigadas ou realizadas ao logo do contexto do tempo e não afetam a percepção do usuário.  

\section{Usabilidade} \label{Usabilidade}

O termo usabilidade começou a ser usado na década de 1980, como um substituto da expressão \textit{user-friendly} traduzida para o português como amigável, sobretudo nas áreas de psicologia e ergonomia. O motivo dessa substituição está na constatação de que os usuários não precisavam que as máquinas fossem conceituadas como amigáveis, mas sim, que elas não interferissem nas tarefas que eles desejavam realizar. Mesmo porque, um sistema pode ser considerado amigável para um usuário e não tão amigável para outro, tendo em vista que as necessidades diferem de um usuário para outro. (GONÇALVES, 2009, apud DIAS, 2003)\newline
\indent Com a publicação da norma ISO 9126 (1991) o termo usabilidade passa a não se limitar a ergonomia e psicologia, mas começa a fazer parte de outras áreas de conhecimento.\newline
\indent A ISO 9126 (1991) conceitua o termo usabilidade da seguinte maneira:\newline

	{\raggedleft
	\hspace*{4.5cm} 
	\begin{minipage}{0.7\textwidth} 
		"conjunto de \textbf{atributos} que evidenciam o esforço necessário para se poder utilizar o software, bem como o julgamento individual desse uso, por um conjunto explícito ou implícito de usuários"	
		\newline	
	\end{minipage}
	\newline	
	\par}

\indent Sete anos depois ISO 9126 (1998), evolui a compreenção do conceito de usabilidade:\newline

	{\raggedleft
	\hspace*{4.5cm} 
	\begin{minipage}{0.7\textwidth} 
		"usabilidade é a \textbf{eficiência}, \textbf{eficácia} e \textbf{satisfação} com a qual os públicos do produto alcançam objetivos em um determinado ambiente"	
		\newline	
	\end{minipage}	
	\par}

\indent A ISO 13407 (1999), descreve a prática como o melhor do processo para projetos voltados ao Projeto Centrado no Usuário. Essa norma define quatro atividades que precisam estar presentes logo no início do projeto: compreender e especificar o contexto de uso; especificar os requisitos do usuário e os organizacionais; produzir soluções de projeto e avaliar projeto em relação aos requisitos. Essas atividades são organizadas, linearmente, formando um “ciclo de vida da usabilidade”, que pode ser visto na Figura \ref{fig:iso13407}. 
\begin{figure}[H]
	\centering	
	\caption{Atividades do Ciclo de Vida da Usabilidade}
	\includegraphics[width=6in, height=2.7in]{iso13407}
	\label{fig:iso13407}
\end{figure} %Fonte: Adaptado de ISO 13407 (1999)
\vspace{-0.8 cm} Fonte: Adaptado de ISO 13407 (1999)\newline
\indent A ISO 9241 (2002), descreve o termo usabilidade como:\newline

	{\raggedleft
	\hspace*{4.5cm} 
	\begin{minipage}{0.7\textwidth} 
		"\textbf{medida} na qual um produto pode ser usado por usuários específicos para alcançar objetivos específicos com eficácia, eficiência e satisfação em um contexto específico de uso"	
		\newline	
	\end{minipage}	
	\par}

\indent No ano seguinte a ISO 9126 (2003), define usabilidade como:\newline

	{\raggedleft
	\hspace*{4.5cm} 
	\begin{minipage}{0.7\textwidth} 
		"\textbf{capacidade} do produto de software de ser compreendido, aprendido, operado e atraente ao usuário, quando usado sob condições especificadas"	
		\newline	
	\end{minipage}
	\par}

\indent Como uma evolução dos estudos de usabilidade e com uma perspectiva de focar mais nos usuários do que no sistema ou produto, dez anos depois da sua criação, a  ISO 13407 (1999) foi revisada e renomeada para ISO 9241 (2010), tornando-se a norma padrão internacional de requisitos ergonômicos para o trabalho em terminais de visualização.\newline
\indent A ISO 9241 (2010) foi organizada em um conjunto de 17 partes, sendo que cada parte trata de um aspecto da utilização de um ambiente informatizado, a Parte 11, conceitua o termo usabilidade da seguinte maneira:\newline

{\raggedleft
	\hspace*{4.5cm} 
	\begin{minipage}{0.7\textwidth} 
		"\textbf{extensão} na qual um produto pode ser usado por usuários específicos para alcançar objetivos específicos com efetividade, eficiência e satisfação em um contexto de uso específico" 
		\newline	
	\end{minipage}
	\par}

\indent E ainda como:\newline

	{\raggedleft
	\hspace*{4.5cm} 
	\begin{minipage}{0.7\textwidth} 
		"\textbf{características} que permitem que o usuário alcance seus objetivos e satisfaça suas necessidades dentro de um contexto de utilização determinado" 
		\newline	
	\end{minipage}
	\par}

\indent A Parte 11 da ISO 9241 (2010) sugere as seguintes medidas para a usabilidade: efetividade, que permite que o utilizador atinja os objetivos iniciais de interação; eficiência, que refere-se à quantidade de esforço e recursos necessários para se chegar a um determinado objetivo e; satisfação, que é a mais difícil de medir e quantificar, pois está relacionada aos fatores subjetivos.\newline
\indent Nielsen (1993) e (2012), autor reconhecido pela sua contribuição para o entendimento da usabilidade, definiu a usabilidade como:\newline

	{\raggedleft
	\hspace*{4.5cm} 
	\begin{minipage}{0.7\textwidth} 
		"\textbf{medida} da qualidade da experiência do usuário ao interagir com alguma coisa - seja um site na Internet, um aplicativo de software tradicional ou outro dispositivo que o usuário possa operar de alguma forma" 
		\newline	
	\end{minipage}
	\par}

\indent Para Oliveira (2014), pode-se entender usabilidade como a característica que avalia a qualidade do uso do sistema, englobando itens como: facilidade do usuário em aprender a usá-lo, o reaprendizado após um período de tempo sem utilização, o quão agradável é a experiência com ele, a quantidade e severidade dos erros bem como a velocidade de realização das tarefas.\newline
\indent Já para Sadiq (2017), a usabilidade foi introduzida como um meio para entender a relação entre seres humanos e dispositivos técnicos. No entanto, quando o foco dos estudos humano-computador mudou gradualmente, para os produtos de consumo, o papel da usabilidade também se modificou. Em ambientes industriais, a usabilidade passou a ser considerada um meio para um desempenho preciso e efetivo.\newline
\indent Na área de produtos de consumo, esse tipo de critério tem valor somente se o consumidor o priorizar. O aspecto mais importante, nesse caso, não é a eficiência mensurável, mas a experiência geral de usar o aplicativo. Na comercialização de produtos de consumo, o principal objetivo é fazer com que o consumidor, efetivamente, consuma e não que execute uma determinada tarefa de forma eficaz.\newline
\indent Keinonen (1998) expõe o conceito de usabilidade e o apresenta em três dimensões:
\begin{flushright}
	\begin{minipage}{.96\textwidth}
		a - usabilidade como abordagem de projeto, consistindo em um conjunto de métodos ou abordagens de projeto, aí compreendidos a engenharia de usabilidade e o projeto centrado no usuário;\newline	
		b - usabilidade como atributo do produto sendo que, essa dimensão é levada em conta por meio da listagem de qualidades ou características que, se adicionadas a determinado produto, cooperam para a sua boa usabilidade;\newline
		c - usabilidade como a engenharia de usabilidade, considera o levantamento de medidas quantitativas para avaliar o potencial da usabilidade de um sistema.	
	\end{minipage}	
\end{flushright}

\newline \indent Outros autores definiram a usabilidade de forma diferente:\newline

	{\raggedleft
	\hspace*{4.5cm} 
	\begin{minipage}{0.7\textwidth} 
		“é a \textbf{capacidade} que um sistema interativo oferece a seu usuário, em determinado contexto de operação, para a realização de tarefas de maneira eficaz, eficiente e agradável” (BETIOL, 2007 apud CYBIS et al,  2002) \newline	
	\end{minipage}
	\par}

	{\raggedleft
	\hspace*{4.5cm} 
	\begin{minipage}{0.7\textwidth} 
		“é a \textbf{característica} que determina se o manuseio de um produto é fácil e rapidamente aprendido, com quantidade pequena de erros operacionais e oferecimento de um alto grau de satisfação, atingindo seus objetivos” (FERREIRA; LEITE, 2003) \newline	 	
	\end{minipage}
	\par}

	{\raggedleft
	\hspace*{4.5cm} 
	\begin{minipage}{0.7\textwidth} 
		“consiste em \textbf{propriedades} de interface de um sistema, no que se refere sua adequação ao usuário” (OLIVEIRA, 2008)\newline		 	
	\end{minipage}
	\par}

	{\raggedleft
	\hspace*{4.5cm} 
	\begin{minipage}{0.7\textwidth} 
		“é a \textbf{disciplina} que garante o uso eficiente e confortável dos sistemas computacionais” (BENSON; MULLER-PROVE; MZOUREK, 2004) \newline
	\end{minipage}
	\par}

	{\raggedleft
	\hspace*{4.5cm} 
	\begin{minipage}{0.7\textwidth} 
		“é uma \textbf{qualidade} de uso de um sistema, diretamente associada ao seu contexto operacional e aos diferentes tipos de usuários, tarefas, ambientes físicos e organizacionais. Pode-se dizer, então, que	qualquer alteração em um aspecto relevante do contexto de uso é capaz de alterar a usabilidade de um sistema” (OLIVEIRA, 2008 apud DIAS, 2003)	\newline	 	
	\end{minipage}
	\par}

	{\raggedleft
	\hspace*{4.5cm} 
	\begin{minipage}{0.7\textwidth} 
		“é o \textbf{parâmetro} que define até que ponto um produto de informação, um sistema de informação, um serviço de informação ou uma informação se prestam ao uso” (OLIVEIRA, 2008 apud LE COADIC, 2004) \newline 	
	\end{minipage}
	\par}

	{\raggedleft
	\hspace*{4.5cm} 
	\begin{minipage}{0.7\textwidth} 
		“é definida por 3 \textbf{aspectos}, qualidade do que é usável, característica do que é simples e fácil de usar e capacidade de um objeto, programa de computador, página da internet, etc.” (FERREIRA, 2014)\newline	
	\end{minipage}
	\par}

Nielsen (2012) ainda descreve cinco atributos da usabilidade, que são: facilidade de aprendizado, eficiência, facilidade de memorização, erros e satisfação.\newline
\indent Pelas definições acima e verificando as diversas técnicas que envolvem o estudo da Usabilidade permite-se afirmar que existe uma amplitude de seu escopo.\newline
\indent Conforme informam Yassien (2017 apud Nielsen e Gilutz, 2016), alguns estudos mostram que projetos de software devem gastar pelo menos 10\% de seu orçamento em usabilidade, para aumentar sua eficácia em 100\%. Dessa forma, pode-se dizer que a usabilidade é um item estratégico na construção de um sistema.\newline
\indent Pode-se observar que a usabilidade é um atributo complexo em qualquer produto de software, pois é afetado por algumas características intrínsecas ao próprio produto, bem como por algumas características de seus usuários. É, comumente, reconhecido que a usabilidade de um determinado produto (tanto subjetivamente percebido quanto medido em relação a escalas ou índices objetivos) aumenta para alguns usuários e se degrada para outros, devido às suas características cognitivas e pessoais. (BABAIAN et al, 2014).

\subsection{Engenharia de Usabilidade}  \label{Engenharia Usabilidade}

Mais recentemente, o termo engenharia foi associado à usabilidade. Para Costa e Ramalho (2010, apud Cybis, 2007, p. 17) a Engenharia de Usabilidade emerge como um esforço sistemático das empresas e organizações para desenvolver programas de software interativo com usabilidade. A Engenharia de Usabilidade abrange um conjunto  de três elementos fundamentais: métodos, ferramentas e procedimentos, que possibilita ao gerente o controle do processo de desenvolvimento do software e oferece ao profissional uma base para construção de software de alta qualidade e produtividade.\newline
\indent A Engenharia da Usabilidade compreende práticas, metodologias, modelos, ferramentas e processos dos quais se pode lançar mão para se obter um sistema interativo, centrado no usuário, com melhor usabilidade (MAYHEW, 1999).\newline
\indent O modelo proposto por Mayhew (1999) oferece uma visão holística acerca dessa engenharia e uma descrição detalhada de como podemos realizar os testes de usabilidade.\newline
\indent A primeira etapa do modelo é a análise dos requisitos. Aqui, Mayhew (1999) propõe quatro tipos de atividades de análise de requisitos:  Análise do perfil do usuário, análise do contexto da tarefa, análise das possibilidades e restrições da plataforma e análise dos princípios.\newline
\indent Depois na etapa de projetos, testes e implementação, Mayhew (1999) propôs que os ciclos devem se repetir de forma a tratar três níveis de aspectos de uma interface, sendo eles: O primeiro nível onde a interface é definida conceitualmente, o segundo, onde se fazem as definições em termos de estilo e o terceiro nível, as interações e componentes relacionados com os contextos das tarefas.\newline
\indent A última fase é a de instalação do sistema onde, depois que o usuário já tiver se habituado ao sistema, este poderá fornecer um \textit{feedback} sobre a usabilidade do produto de forma mais fidedigna por já ser um “especialista" da ferramenta.\newline
\indent O ciclo de  vida pode ser observado na Figura \ref{fig:cvum}.

\begin{figure}[H]
	\centering	
	\caption{Modelo de Ciclo de Vida da Usabilidade}
	\includegraphics[scale=1]{cvum}
	\label{fig:cvum}
\end{figure}
Fonte: Rebelo (2018)
\newline
Dentro da nossa proposta estamos localizados na ultima fase do modelo de Mayhew.

\subsection{Critérios de avaliação da usabilidade} \label{Critérios de avaliação da usabilidade}

As técnicas de avaliação nos proporcionam a capacidade de utilizar diferentes processos para verificar a usabilidade em suas diversas dimensões.\newline
\indent Segundo Shackel (1986), são quatro os componentes que devem ser considerados no contexto da avaliação da usabilidade: o usuário, a tarefa, o sistema e o ambiente. Assim, sob o ponto de vista da Usabilidade, são necessários quatro critérios para avaliação da interação do usuário com suas tarefas:\newline
\indent Eficácia, em que se avalia a capacidade do usuário de desempenhar a tarefa em um determinado ambiente. Por exemplo: a aferição da velocidade de execução e do levantamento número de erros. \newline
\indent Aprendizagem, pelo que se avalia o desempenho a partir da instalação do aplicativo, do inicio do seu uso, o tempo gasto em treinamento e o reaprendizado relativo ao uso frequente. \newline
\indent Flexibilidade, onde se avalia a adaptação do usuário às suas tarefas além das previamente especificadas pelo desenvolvedor. \newline
\indent Atitude, pelo que se avaliam o desempenho relacionado ao usuário, seu conforto ou satisfação, tais como as condições aceitáveis de frustração, desconforto, fadiga, esforço e desgaste pessoal.\newline
\indent Para Nielsen (1993) o modo mais recorrente de se avaliar a usabilidade de um software é pela observação da sua capacidade de interação com o usuário, p odendo essa observação ser realizada em um laboratório, com uma quantidade controlada de usuários para o qual o sistema foi desenvolvido, ou mesmo no próprio ambiente de trabalho do usuário onde o sistema está instalado. O que importa no processo de avaliação é que, sempre quando for possível, deve-se utilizar o tipo de usuário adequado para a realização da tarefa, no sentido de se possibilitar a melhor avaliação.\newline
\indent Para Nielsen (1993) a avaliação da usabilidade deve ser realizada sob a ótica de cinco atributos:\newline
\indent − Facilidade de aprender, o sistema deve ser fácil para o usuário aprender, para que possa realizar sua tarefa com agilidade. A interface com o usuário deve ser prática e clara. \newline
\indent − Eficiência de uso, o sistema deve permitir a eficiência na execução da tarefa a ser realizada, para que com isto o usuário amplie seu nível de produtividade. 
− Memorização: As funcionalidades que o sistema possui devem estar de forma que facilite sua memorização pelo usuário, ainda que fique um determinado período sem utilizá-lo, porém sem ocorra a necessidade de mais treinamento. \newline
\indent − Poucos erros, o sistema deve favorecer a redução do número de erros, e caso estes ocorram, o usuário deve poder resolvê-los ou ignorá-los de modo simples e rápido.\newline
\indent - Satisfação, é uma percepção totalmente subjetiva em que o usuário, diante da interface do sistema, transmite a impressão de estar satisfeito e apreciar do seu uso.\newline
\indent Quesenbery (2001) aponta cinco características da usabilidade, também conhecidas como 5E’s (effective, efficient, engaging, error tolerant, easy to learn). Em sua abordagem a usabilidade da interface deve ser analisada pelo confronto dessas características, de modo a permitir a satisfação e o sucesso do usuário. \newline
\indent Eficiência, verifica o tempo gasto total para realização de uma determinada tarefa. \newline
\indent Eficácia, inspeciona se as tarefas foram concluídas conforme seu planejamento, e com qual freqüência produzem erros. \newline
\indent Atração, verifica a satisfação ou conforto do usuário em utilizar o sistema. Focaliza medir sua aceitação ou rejeição em relação ao sistema. \newline
\indent Tolerância a erros, verifica a incidência de erros gerados pelo sistema. Pressupõe-se que tais erros sejam solucionados facilmente e apresentados de forma clara ao usuário.\newline
\indent Facilidade de aprender, verifica a facilidade de uso entre os usuários em diversos níveis de experiência com o sistema. O usuário deve concluir a tarefa com o mínimo de assistência ou ajuda necessária.

\section{Enterprise Resource Planning (ERP)}  \label{Enterprise Resource Planning (ERP)}

Por definição, quando dois ou mais sistemas de gestão se unem, de forma a perder a independência de um deles ou de ambos sem, contudo, deixar suas particularidades de fora, esse movimento consiste no estabelecimento de um Sistema Integrado de Gestão (SIG) ou Sistema de Gestão Integrado (SGI) (KARAPETROVIC; WILLBORN, 1998, apud INÁCIO, 2017, p. 25-26).\newline
\indent A termo \textit{Enterprise Resource Planning} e a sigla ERP, foram cunhados por uma empresa americana de pesquisa a \textit{Gartner Group} e, rapidamente, assimilados no meio comercial, substituindo as siglas SIG/SGI.\newline
\indent A intenção foi definir esses sistemas integrados como uma evolução dos sistemas \textit{Manufacturing Resources Planning} (MRP II). Para Souza (2000) os sistemas ERP surgiram da necessidade de rápido desenvolvimento dos SIG/SGI, as empresas fornecedoras utilizaram-se de modelos de processos obtidos de estudo e comparações em diversas empresas por meio de \textit{benchmarking}, as chamadas melhores práticas. Esse conhecimento foi agregado à empresa no processo de implantação. As melhores práticas, em associação à integração dos departamentos, podem permitir reduções de mão de obra indireta, principalmente, nos setores administrativos da empresa.\newline
\indent Os sistemas ERP podem ser definidos como sistemas de informação integrados, adquiridos na forma de um pacote de software comercial, com a finalidade de dar suporte à maioria das operações de uma empresa. São, geralmente, divididos em módulos que se comunicam e atualizam uma mesma base de dados central, de modo que informações alimentadas em um módulo são, instantaneamente, disponibilizadas para os demais módulos com dependência. (SOUZA, 2000).\newline
\indent Embora os conceitos utilizados em sistemas ERP possam ser usados por empresas que queiram desenvolver, internamente, os seus aplicativos, o termo ERP refere-se, essencialmente, a pacotes comprados. Exemplos de sistemas ERP existentes no mercado são: (SAP S/4HANA, SAP S/4 HANA Cloud e SAP ERP), da alemã SAP, (Protheus, RM, Datasul e Logix), da brasileira Totvs e o Oracle ERP Cloud da americana Oracle Inc.\newline
\indent Conforme Kovalczyk e Kovalczyk (2017) empregar um sistema de informação apropriado, apto a reduzir os contratempos sistemáticos verificados em sistemas arcaicos e que proporcionem a inserção de um novo protótipo de gestão empresarial fundamentado na gestão constituída, é um dos grandes propósitos vigentes nas instituições. O autor ainda diz que o uso de ERP é indispensável na maior parte das organizações. Entretanto, a dimensão do quanto a solução será necessária e a integralidade que a empresa contratada irá proporcionar, são de extrema importância na questão referente ao custo x benefício.

\section{ERP x Usabilidade} \label{ERP x Usabilidade}

As pesquisas do setor reconhecem que os sistemas ERP estão cheios de problemas de usabilidade, mas há uma escassez de pesquisas sobre meios para melhorar a experiência do usuário (BABAIAN et al, 2014), conforme investigação efetuada no Mapeamento Sistemático os aspectos específicos de usabilidade da interface não são, amplamente, discutidos no campo dos ERPs. Dos artigos encontrados, poucos investigaram fatores importantes de interface do usuário (IU), como navegação, orientação do usuário, fatores visuais, carga mínima de memória e capacidade de aprendizado. Essa afirmação é corroborada também por Lamberck et al. (2014a).\newline
\indent Com o incremento de complexidade dos sistemas ERP, os profissionais de desenvolvimento já encaram a usabilidade como um conhecimento indispensável e, de certa forma, indissociável da prática profissional. Entender as dimensões da usabilidade e seu impacto no produto final colabora para o aumento da efetividade e da eficiência.\newline
\indent No dia a dia do suporte às ferramentas de ERP surgem situações em que se percebe que a usabilidade de um sistema tem uma alta variação de situações (erros no sistemas, erros operacionais, dúvidas e etc.), isso se deve a não homogeneidade dos usuários, que vão de funcionários desmotivados ou inexperientes a funcionários altamente exigentes ou experientes.\newline
\indent Percebe-se na Figura \ref{fig:vdpu} uma situação inusitada, relatada por Dums (2015), em que um usuário dialoga com um profissional de TI. Nessa situação o sistema apresenta a funcionalidade de atualização da tela pela tecla F5 e o usuário quer que exista uma duplicidade do mesmo recurso por meio de um botão “Atualizar”. Essa situação inusitada demonstra que a percepção da usabilidade se torna algo, extremamente, pessoal e é afetada por diversos fatores externos.\newline
\indent Segundo Lambeck et al. (2014b), os sistemas ERP sofrem de inúmeros problemas de usabilidade em geral, conforme avaliação histórica. Isso é inevitável, pois esses sistemas se caracterizam por processos complexos que os ERPs precisam implementar e suportar. No entanto, há outros motivos e fatores que, tradicionalmente, afetam a usabilidade de tais sistemas. Para Veneziano (2014), o termo usabilidade não é, frequentemente, associado a sistemas ERP, nem é uma das características principais consideradas, comercialmente. Em vez disso, esses sistemas são, tipicamente, complexos e frustrantes de usar.\newline
\indent Um exemplo dessa percepção pode ser visto na Figura \ref{fig:vdpu}, essa é uma das diversas charges publicadas, semanalmente, que são elaboradas a partir relatos de profissionais do meio de suporte e programação, nelas o personagem ”O Programador” se depara com situações diversas de suporte, desenvolvimento ou a iteração com seus superiores, que demonstram o despreparo tanto do usuário, como do próprio profissional, para percepção da usabilidade do sistema.

%\vspace{0.5 cm}
\begin{figure}[H]
	\centering	
	\caption{Charge Vida de Programador - Botão “Atualizar” x Tecla F5}
	\includegraphics[scale=1]{vdpu}
	\label{fig:vdpu}
\end{figure}
\vspace{-0.8 cm} \hspace{1.85 cm} Fonte: Dums (2015)\newline
%\newline cxx \newline
\indent Para Parks (2012), no que diz respeito à interface do usuário, a menor satisfação do usuário é causada, pelo menos parcialmente, pela implementação de processos de negócios complexos e na concepção de interfaces de usuários gerais que são direcionadas para várias indústrias, simultaneamente.\newline
\indent A importância de interfaces empresariais bem concebidas é destacada por Parks (2012), uma vez que dados, incorretamente, codificados podem diminuir, significativamente, o desempenho da empresa (por exemplo, metas de produção não realizadas ou pedidos incorretos).\newline
\indent Os estudos conduzidos por Babaian et al. (2014); Lambeck et al. (2014b) revelam várias fontes de confusão e frustração dos usuários em geral, sendo que essas fontes são repetidas em trabalhos referenciados pelos autores, o que denota um certo descaso do tema ’usabilidade’ pelas empresas que criam e mantém esses sistemas.\newline
\indent Essa realidade pode ser sentida no convívio cotidiano em grupos de colaboração que se formam, espontaneamente, em plataformas de colaboração como \textit{Skype}, \textit{Whatsapp} e \textit{Telegram}. A aplicação da pesquisa pretendida nesses grupos poderá começar a desvendar o que esses usuários, realmente, querem e qual a sua percepção em relação aos sistemas ERP.\newline
\indent Conforme Babaian et al. (2014) os seguintes princípios de projeto deveriam ser observados nos sistemas ERPs, em geral:

%{\raggedleft
\begin{flushright}
	\begin{minipage}{.96\textwidth}
		1.	a interface do usuário deve fornecer um mecanismo para personalizar o vocabulário dos termos usados pelo sistema em sua comunicação com o usuário, a composição das transações comerciais e o conteúdo das saídas do sistema para corresponder às práticas da organização. Deve haver um mecanismo para incorporar as personalizações de uma versão anterior do sistema a uma versão posterior.
	\end{minipage}
\end{flushright}
%	\newline	
%\par}

%{\raggedleft
\begin{flushright}
	\begin{minipage}{.96\textwidth}
		2.	O sistema deve fornecer orientação de navegação e progresso para um usuário que executa uma transação, indicando o contexto mais amplo de cada interação em termos dos componentes de processos de negócios relacionados e, especificando, as partes concluídas e restantes. Um usuário, suficientemente, competente deve ser capaz de desativar essa orientação se ela se tornar uma distração.		
	\end{minipage}
\end{flushright}
%	\newline	
%\par}


%{\raggedleft
\begin{flushright}
	\begin{minipage}{.96\textwidth}
		3.	Quando o sistema detecta um problema, ele deve identificar as possíveis causas e formas de resolvê-lo. Se a correção for óbvia, o sistema deve informar ao usuário e executar a ação. Se não for, as possíveis causas e cenários de resolução devem ser apresentadas ao usuário possibilitando, prontamente, a correção. Se o sistema não conseguir identificar estratégias de resolução, ele deverá apresentar ao usuário os dados e transações relevantes.		
	\end{minipage}
\end{flushright}
%	\newline	
%\par}


%{\raggedleft
\begin{flushright}
	\begin{minipage}{.96\textwidth}
		4.	Ao apresentar opções de seleção, o sistema deve utilizar o que sabe sobre o usuário, a organização, a tarefa e o contexto e fornecer acesso mais rápido às opções mais prováveis do que as menos prováveis. Onde a escolha de dados ou ação é óbvia, o sistema deve ter a opção de não esperar que o usuário o autorize. O usuário deve ter uma opção para substituir/cancelar a escolha de dados/ação fornecida pelo sistema.		
	\end{minipage}
\end{flushright}
%	\newline	
%\par}

Para Melo (2015 apud Singh, Wesson 2009), por, ainda, não existir uma forma padronizada para determinar a usabilidade em um sistema ERP, os limitados estudos publicados apontam que os problemas de usabilidade encontrados nesses sistemas foram identificados pela comunhão de vários critérios. Entre tais problemas podem ser citados:\newline

%{\raggedleft
\begin{flushright}
	\begin{minipage}{.96\textwidth}
	1.	Complexidade para encontrar funcionalidades;\newline	
	2.	Falta de orientação ao usuário de como concluir precisamente a sua tarefa; 	\newline
	3.	Inexistência de recursos de personalização da interface para apoiar as ações do usuário;\newline 	
	4.	Ineficiência na recuperação de dados e informações acessados frequentemente;\newline 	
	5.	Complexidade na compreensão dos leiautes das telas;	\newline
	6.	Dificuldade na interpretação das saídas do sistema; e	\newline
	7.	Dificuldade em lembrar como operar com as diferentes partes do sistema.
	\end{minipage}
\end{flushright}
%	\newline	
%\par}

Melo (2015 apud Singh, Wesson 2009) usa um conjunto de 5 heurísticas de usabilidade específicas para sistemas ERP:
%{\raggedleft
\begin{flushright}
	\begin{minipage}{.96\textwidth}
	a) navegação: navegação e acesso a informação.\newline
	b) apresentação: apresentação da tela e da saída.\newline
	c) suporte à tarefa: suporte apropriado à tarefa.\newline
	d) aprendizado: grau de facilidade para aprender como usar o sistema efetivamente.\newline
	e) customização: facilidade de customização do sistema para o alinhamento entre o mesmo, o usuário e os processos de negócio.		
	\end{minipage}
\end{flushright}
%	\newline	
%\par}

Segundo Melo (2015 apud Singh, Wesson 2009), devido às 10 heurísticas de Nielsen terem como objetivo a avaliação da usabilidade geral e não serem direcionadas às características de usabilidade que são próprias de sistemas ERP, outras heurísticas são utilizadas no intuito de cobrir aspectos de usabilidade que não são considerados por Nielsen.\newline
\indent Melo (2015) cita os resultados de um estudo de Singh \& Wesson, citado anteriormente, onde usou-se a heurística de Nielsen, juntamente com a heurística dos citados, com a ajuda de três especialistas em usabilidade, que verificaram a capacidade do seu conjunto de heurísticas e das 10 heurísticas de Nielsen em identificar, conjuntamente, os possíveis problemas de usabilidade do sistema ERP.\newline
\indent Os problemas de usabilidade detectados com as 10 heurísticas de Nielsen, revelam a necessidade de ampliar a capacidade de prevenção e recuperação de erros e, com as heurísticas específicas para sistemas ERP, foi constatado como potenciais problemas como a dificuldade de encontrar informações em funcionalidades do sistema e a falta de orientação por parte do sistema para auxiliar os usuários a completarem as suas tarefas.\newline
\indent Nesse sentido, Veneziano (2014), considera a percepção da usabilidade por parte do usuário, como um dos pontos centrais no que tange ao processo de consumo dos ERPs. Considera, ainda, a existência de uma relação quanto à percepção do usuário vinculado ao nível de eficiência para atingir determinado objetivo. Quanto menor essa percepção, menor o índice de satisfação e, consequentemente, de sucesso em suas operações junto ao sistema.\newline
\indent Para o trabalho em questão nos basearemos nas heurísticas de Nielsen (1993) e nas 5 heurísticas de ERP introduzidas por Singh \& Wesson, já citados neste trabalho e utilizados por Lambeck et al (2014a). 

\section{\textit{Survey}} \label{Survey}

\textit{Survey's} são investigações que colhem dados de uma amostra representativa de uma população específica, que são descritos e, analiticamente, explicados. Pretende-se que os resultados sejam generalizáveis ao universo dessa população, evitando-se realizar o censo, ou seja, ouvir todos os indivíduos, o que é, geralmente, impossível, por questão de custo e de tempo (BABBIE, 2005).\newline
\indent Segundo Freitas et al. (2000), a \textit{survey} é apropriada como método de pesquisa quando:

%{\raggedleft
\begin{flushright}
	\begin{minipage}{.96\textwidth}
		a) se deseja responder questões do tipo ”o quê?”, ”por que?”, ”como?” e ”quanto?” Ou seja, quando o foco de interesse é sobre ”o que está acontecendo ”ou” como e por que isso está acontecendo”;\newline		
		b) não se tem interesse ou não é possível controlar as variáveis dependentes e independentes;\newline
		c) o ambiente natural é a melhor situação para estudar o fenômeno de Interesse; \newline
		d) o objeto de interesse ocorre no presente ou no passado recente;
	\end{minipage}
\end{flushright}
%	\newline	
%\par}

\indent Para Freitas et al. (2000 apud Pinsonneault; Kraemer, 1993), um questionário é a forma mais, corriqueiramente, utilizada como instrumento de obtenção de dados em uma pesquisa cujo o método utilizado é o \textit{survey}.

\section{Trabalhos Relacionados} \label{Trabalhos Relacionados}

Os principais trabalhos relacionados a esta pesquisa foram conduzidos em Dresden entre 2013 e 2017, localizou-se também um interessante trabalho nos Estados Unidos de Veneziano et al. (2014), a seguir, são analisadas as características principais.\newline
\indent Šūpulniece et al. (2013), investiga se os problemas de usabilidade em ERPs tradicionais de projeção mundial são válidos para os sistemas locais que dominam as micro, pequenas e médias empresas na Letônia, sendo que em suas conclusões de estudos indicam a necessidade de comparar esses resultados com os demais países da Europa, para que se tenha uma visão mais abrangente.\newline
\indent Já Lambeck et al. (2014a), cita que os estudos nos últimos 20 anos focaram apenas em ERPs individuais em filiais específicas com pequenos grupos de usuários, o estudo em face usa uma ampla amostra de 184 usuários, distribuídos em pequenas e médias empresas.\newline
\indent A pesquisa de Lambeck et al. (2014a) é uma revisitação de uma pesquisa feita em 2005, por esse motivo a pesquisa tem dois objetivos: primeiramente, avaliar se os problemas de usabilidade identificados em 2005 se repetem quase 10 anos depois e ampliar o foco da pesquisa para considerações adicionais, como o papel do tipo de menu, a incerteza no uso do sistema ou o suporte em situações problemáticas.\newline
\indent Lambeck et al. (2014a), avaliam que os problemas encontrados em pesquisas anteriores ainda existem na atualidade., porém, afirmam que esses problemas são menos críticos, assim, concluem que ainda há alguns esforços necessários para alcançar a visão de uma interface ERP fácil de usar.\newline
\indent Veneziano et al. (2014), buscam estabelecer a relação da influência das informações demográficas dos usuários (por exemplo: antecedentes educacionais e experiências de trabalho) sob a avaliação da usabilidade. O estudo concluiu que a existe uma relativa influência do grau de formação educacional na percepção da usabilidade, porém elenca como trabalhos futuros o aprofundamento desse estudo em outras populações.\newline
\indent Lambeck et al. (2014b), baseando-se na pesquisa de Šūpulniece et al. (2013), apresentam uma pesquisa realizada com, aproximadamente, 200 usuários de ERP na Alemanha e na Letônia, sendo que os resultados indicaram que ambos os países têm vários contrastes, mas também pontos comuns diversos na indústria, mercado de ERP e cultura. No entanto, os usuários em ambos os países são muito homogêneos em relação à avaliação de problemas de usabilidade em suas interfaces ERP.\newline
\indent O artigo de Lambeck et al. (2014b) investiga problemas de usabilidade elementares derivados do trabalho relacionado e examina em que medida eles são válidos em ambos os países. A principal hipótese levantada pelo artigo afirma que diversas características nacionais não conduzem, necessariamente, a uma avaliação diferente dos problemas de usabilidade nos sistemas ERP.
