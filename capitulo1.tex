\setcounter{page}{3}
\chapter{Introdução} \label{Introdução}
%\setcounter{chapter}{3}
%"Exemplo de Citação direta" (LEYH; GEBHARDT; BERTON, 2017).
%Para  Leyh, Gebhardt e Berton (2017) este é um exemplo de  citação indireta.
O advento da \textit{internet} alterou todas as relações humanas e profissionais, desde o momento do seu surgimento e caminha ainda no sentido de tornar o mundo mais interativo, rápido e eficaz a cada dia. Porém, junto a esta evolução, com o decorrer do tempo verificou que não somente estas ferramentas precisavam ser mais abrangentes, como também se observou a necessidade de que fossem simples, já que o uso destas não se dava por uma elite mas sim em larga escala mundial, devendo atender a pessoas de todos os níveis, classes, e culturas distintas.\newline
\indent A partir daí, acentuou-se a tendência de simplificarem-se, ainda mais a comunicação e o modo de uso por meio da capacidade dos sistemas em oferecer ações e soluções para problemas e dificuldades em diferentes áreas. Colocar ao alcance de todos ferramentas que podem ser operadas de modo simples e inequívoco, foi uma tendência estudada por Giovanella (2014).\newline
\indent Com a popularização da \textit{internet}, os sistemas de informação ganharam força e se estenderam para além das fronteiras empresariais. Desta forma, retornar ao início da utilização maciça dos sistemas de informação é essencial para compreender a necessidade de otimização.\newline
\indent Se por um lado a evolução na área de sistemas de informação tornou especialmente simples a comunicação e a resolução de questões que passaram a ser abordadas, a partir da união de vários conhecimentos como algo corriqueiro e eficiente, por outro a necessidade de se unir diferentes usuários de diversos modos passou a ser uma condição básica para a evolução e otimização do uso dos sistemas, tais como sistemas conhecidos pelo termo inglês \textit{Enterprise Resource Planning} (ERP). \newline
\indent Compreendendo a importância deste tema, Giovanella (2014) nos ensina que, a troca de informações pela rede mundial de computadores torna a obtenção de informação e principalmente de respostas, uma estratégia rápida e simples, além de estar ao alcance de todos os interessados no tema a qualquer tempo. \newline
\indent Nesse sentido, os novos saberes na área surgiram não somente fornecendo material de pesquisa, mas também métodos próprios, oriundos da necessidade diária como suporte não oficial aos sistema ERP. É o caso, por exemplo, dos grupos de discussão em diversos canais como \textit{Whatsapp}, \textit{Skype}, \textit{Facebook}, \textit{Telegram}, entre outros, chamados neste trabalho de comunidades. \newline
\indent Pode-se dizer que essas comunidades foram criadas com o objetivo inicial de sanar problemas de uma forma mais rápida, por meio da democratização de informações e opiniões. Sendo assim, nessas comunidades todos os que se interessam pelo mesmo tema, podem debater um problema ou dúvida específica com a comunidade. \newline
\indent Uma justificativa para o uso de tais ferramentas, em detrimento a suporte oficial das empresas que provêm este serviço, pode ser verificado pelo trabalho de Giovanella (2014) onde se constata que o aprendizado dos indivíduos que participam de comunidades de prática, que são comunidades onde pessoas com experiência compartilham seu conhecimento prático com usuários iniciantes de maneira \textit{on-line}, é um facilitador do trabalho prático e simplifica o uso de ferramentas e programas. Deste modo, as pessoas com menos conhecimento, podem usufruir da consultoria dos colegas que compartilham experiências vividas e conhecimento, o que consequentemente agrega valor ao conhecimento dos usuários.\newline
\indent Nessas comunidades o conhecimento prático dos usuários e consultores, que passaram por situações semelhantes ou análogas, são compartilhados, possibilitando a identificação de soluções de modo mais rápido e eficaz do que com o auxílio dado pelo suporte das empresas fornecedoras da solução de ERP que, muitas vezes, não apresentam a atualização e conhecimento necessários para fornecer uma resposta rápida e indubitável para a dúvida apresentada.\newline
\indent Atuando desde 1997 com sistemas ERP e utilizando comunidades desde 2003 para troca de informações com usuários, o autor testemunhou uma mudança do sentido no fluxo da resolução de soluções, onde no início se dependia do fornecedor prover a solução, sendo que atualmente as soluções nascem nos grupos, Este comportamento que aqui chamamos de fenômeno autodidata, é um fenômeno que só existe possivelmente graças ao uso das tecnologias de comunicação inseridas em nosso meio nas últimas décadas.\newline
\indent Participando ativamente desse processo, o autor, testemunhou não só o fenômeno autodidata, mas também passou a perceber a importância da usabilidade na aceitação dos sistemas ERP por parte dos usuários. Uma vez que em um ambiente compartilhado e livre a comparação de recursos, as facilidades e possibilidades tornam-se naturais, observa-se que essas comunidades dão origem a novos e diversos questionamentos.\newline
\indent Os aspectos questionados são variados, mas podem ser encaixados em algumas linhas relacionadas à usabilidade de sistemas, tais como: desconhecimento do processo do funcionamento do sistema, telas confusas ou complicadas demais para o uso cotidiano, falta de documentação do sistema, dificuldade em obter suporte, erros graves que inviabilizam a execução dos processos pelos dos usuários e defasagem de informação.\newline
\indent O dia a dia dessas comunidades giram, em certo grau, em torno de perguntas, genericamente, separadas em: “Qual campo devo preencher para ...?”, “Como funciona ...?”, “Vocês já viram esse erro, quando se faz ...?”, “Como faço ....?”, “Alguém tem um Manual de ...?”, “Como tiro esse campo da tela?”, “Para que serve ....?” Esse tipo de questionamento revela aspectos deficitários mais recorrentes de usabilidade desses ERP’s.\newline
\indent Tais problemas de usabilidade em sistemas ERP’s, percebidos nestas comunidades, motivaram a realização de uma pesquisa exploratória mais ampla sobre o tema. Desse modo, após a realização de um levantamento de informações sobre o tema usabilidade em ERP’s, foi localizado um grupo de pesquisadores na Alemanha, cuja a área de concentração são os sistemas caracterizados como ERP’s. Dentre os diversos temas pesquisados destaca-se o estudo da usabilidade. \newline
\indent O trabalho desenvolvido por Lambeck, et al. (2014b) chama a atenção por buscar compreender as influências no uso e aceitação do ERP pelos usuários de diferentes países sendo que, nesta pesquisa são comparados os dados entre dois países Alemanha e Letônia. \newline
\indent Entender as diferenças de percepção entre os  usuários Brasileiros e Alemães, pode contribuir para melhorar o manejo destas ferramentas e compreender a necessidade de torna-las mais simples, para melhorar estes sistemas para as populações alvo deste estudo.

\section{Questões da Pesquisa e Objetivo} \label{Questões da Pesquisa e Objetivo}

Percebe-se no cotidiano das comunidades destinadas à discussão sobre ERP’s, a importância da usabilidade em tópicos que tratam de diversos temas, destacando-se entre eles, as instruções de operacionalização do ambiente, dúvidas na configuração e demora no atendimento por parte dos fornecedores. Também é visível o conflito entre o que os fornecedores querem desenvolver para a evolução do sistema e o que os usuários, realmente, desejam para executar suas tarefas cotidianas.\newline
\indent O mapeamento sistemático efetuado previamente a este trabalho mostrou que existem poucos estudos científicos sobre o tema. Tendo isso em vista e com o objetivo de expandir o conhecimento sobre o assunto, este estudo pretende responder os seguintes questionamentos:

	\begin{flushright}
	\begin{minipage}{.955\textwidth}
		I.	Quais aspectos de usabilidade são importantes para o usuário de sistemas ERP no Brasil? \newline
		Para responder a esta pergunta um questionário foi elaborado e aplicado na população de usuários de ERP no Brasil.
		
	\end{minipage}
	\end{flushright}

	\begin{flushright}
	\begin{minipage}{.955\textwidth}
		II. Qual a importância destes aspectos na percepção da usabilidade da população usuários da Alemanha x Brasil? \newline
		Por meio da comparação entre as respostas das populações de usuários de ERP no Brasil e da Alemanha foi possível responder à questão.
		
	\end{minipage}
	\end{flushright}

	\begin{flushright}
	\begin{minipage}{.955\textwidth}
		III. Existe algum aspecto que se destaca mais que os outros identificados? \newline
		Essa questão foi respondida por meio da análise da discrepância entre as respostas dadas no questionário.
		
	\end{minipage}
	\end{flushright}

As perguntas abaixo são secundárias, pois há um fator condicional para sua análise que é à diversidade das respostas a questão ”Região”:

	\begin{flushright}
	\begin{minipage}{.955\textwidth}
		IV.	Quais são aspectos mais importantes por região no Brasil? \newline
		A análise das respostas a essa pergunta fica condicionada à existência de uma grande diversidade regional.
		
	\end{minipage}
	\end{flushright}

	\begin{flushright}
	\begin{minipage}{.955\textwidth}
		V.	Os aspectos identificados na região sul (colonizada por europeus, principalmente, alemães e Italianos) estão mais próximos do resultado obtido na Alemanha? \newline
		A resposta a essa questão foi obtida a partir da análise das respostas da região sul do país uma vez que aquela região teve um fluxo migratório mais forte de alemães, em comparação com as demais regiões do país.
		
	\end{minipage}
	\end{flushright}


	\begin{flushright}
	\begin{minipage}{.955\textwidth}
		VI.	Tendo em vista que o perfil dos países pesquisados são semelhantes no que se refere as dimensões (Industrialização e PIB) como se dá a percepção da usabilidade? \newline
		Esta pergunta está vinculada a uma das conclusões do estudo comparativo sobre os usuários de ERP da Alemanha e da Letônia, que atribuiu as diferenças encontradas no aspecto da maturidade econômica daquelas nações. Supõe-se então, que ao comparar Alemanha e Brasil se está diminuindo, ou até eliminando, a influência desse aspecto no resultado da comparação.
		
	\end{minipage}
	\end{flushright}

O trabalho de Lambeck et al. (2014b) compara os dados obtidos na Alemanha e na Letônia, ainda que este estudo ressalte uma pequena distância entre os resultados obtidos uma análise mais objetiva mostra usuários de países com diferenças econômicas substanciais, embora tenham características culturais semelhantes: ambos são países com origem na cultura saxã. A Alemanha é o 5º maior PIB do mundo e detentora de uma economia muito industrializada, enquanto a Letônia é apenas o 41º nesse \textit{ranking} e sua economia, ainda, é pouco industrializada.\newline
\indent Ambos países estão, geograficamente, próximos e, historicamente, compartilham uma evolução histórica, relativamente, semelhante. As diferenças apresentadas são muito particulares, porém pequenas e os pesquisadores atribuíram essas diferenças à idade e à maturidade dos pesquisados na Letônia, em relação aos pesquisados na Alemanha.\newline
\indent A partir desse ponto e tomando esses trabalhos como base, o autor propôs uma comparação entre o resultado obtido na pesquisa Alemã, com o que se observa no Brasil. Em linhas gerais, como já foi dito, as diferenças identificadas entre os usuários da Alemanha e da Letônia foram atribuídas à idade e à maturidade dos pesquisados, porém neste trabalho se questionará, também, outros aspectos.\newline
\indent Inicialmente, verifica-se que Alemanha e Brasil não compartilham laços históricos ou proximidade geográfica, como ocorre entre Alemanha e Letônia. Logo, os componentes culturais também se distanciam. Entretanto, a Alemanha com seu substrato cultural saxão e o Brasil com influência latina, se aproximam em outros aspectos de relativa importância, uma vez que a Alemanha é o 5° maior PIB do mundo, enquanto o Brasil ocupa o nono lugar. No que se refere à industrialização a Alemanha é a 4ª nação mais industrializada do mundo, ao passo que o Brasil está, também, na nona posição, segundo a IEDI (2017).\newline
\indent Portanto, as principais diferenças entre o estudo original, que compara Alemanha e Letônia, com o estudo proposto neste trabalho são:\newline
\indent a)	a proximidade do aspecto econômico;\newline
\indent b)	a proximidade do aspecto industrial;\newline
\indent c)	os dois mercados geopoliticamente distintos;\newline
\indent d)	a influência cultural mais distinta e distante; e\newline
\indent e)	pouco vínculo comercial devido à distância.\newline
\indent Portando o objetivo é avaliar a percepção do usuário sobre o uso do ERP e identificar possíveis diferenças na percepção da usabilidade que influenciem na aceitabilidade do sistema por parte dos usuários entre populações heterogêneas sendo estas populações, a população brasileira (conforme a proposição deste trabalho) e a população alemã (pesquisa realizada). Faz-se necessário esclarecer que este estudo, como o trabalho citado, não se concentra sobre a necessidade de identificar os ERP’s utilizados, nem comparar seus recursos para saber qual seria considerado o mais eficaz. 

\section{Contribuição}  \label{Contribuição}

Verificar que a usabilidade em sistemas ERP, para as empresas que os fornecem, tem sido um desafio originado pela incapacidade atual do mercado de entender a grande contribuição que o tema pode trazer, tanto econômica como mercadologicamente. Este trabalho portanto dá continuidade a outros que compreenderam essa necessidade e ao longo dos últimos anos, buscaram suprir essa lacuna do conhecimento. Nesse sentido, destacam-se autores como Šūpulniece et al. (2013) e Lambeck et al. (2014b), que se dedicaram ao estudo do assunto.\newline
\indent A pesquisa conduzida por Šūpulniece et al. (2013) explora se os problemas de usabilidade em ERP’s tradicionais de projeção mundial são válidos para os sistemas locais que dominam as micro, pequenas e médias empresas na Letônia, sendo que em suas conclusões, o estudo indica a necessidade de se comparar os resultados daquele país com os dos demais países da Europa, para que se tivesse uma visão mais abrangente.\newline
\indent A pesquisa conduzida por Lambeck et al. (2014b), que teve como objetivo comparar a percepção dos usuários da Alemanha e da Letônia sobre a usabilidade dos sistemas de ERP, concluiu que existem poucas diferenças entre esses dois países e atribuiu essas diferenças à maturidade da população pesquisada na Alemanha em relação a Letônia.\newline
\indent O presente trabalho, por sua vez, focaliza realidades mercadológica e geograficamente distantes, porém próximas em vários aspectos econômicos e industriais e objetiva dar um outro enfoque aos dados colhidos nos trabalhos citados.\newline
\indent Também será possível analisar melhor quais seriam as soluções regionais para mercados diversos e qual seria a possibilidade de personalização dessas soluções, de acordo com a necessidade individual de cada mercado específico.\newline
\indent Com base nos estudos anteriores, seguem as contribuições pretendidas neste trabalho:
	\begin{flushright}
	\begin{minipage}{.955\textwidth}
		a) compreender quais aspectos influenciam na percepção da usabilidade por parte dos usuários de sistemas ERP’s;
	\end{minipage}
	\end{flushright}

	\begin{flushright}
	\begin{minipage}{.955\textwidth}
		b) colaborar com o aumento da compreensão da percepção da usabilidade sob o ponto de vista do usuário de sistemas ERP’s, possibilitando a ampliação do conhecimento a respeito da usabilidade;
	\end{minipage}
	\end{flushright}

	\begin{flushright}
	\begin{minipage}{.955\textwidth}
		c) ao final do estudo, fornecer insumos para que futuras pesquisas possam ser iniciadas a partir dos dados desta pesquisa, possibilitando a confirmação das evidências ou o confronto;
	\end{minipage}
	\end{flushright}

	\begin{flushright}
	\begin{minipage}{.955\textwidth}
		d) contribuir com o conhecimento sobre as áreas de atuação da usabilidade, além de promover o entendimento do que pode ou não ser caracterizado influência cultural;
	\end{minipage}
	\end{flushright}

	\begin{flushright}
	\begin{minipage}{.955\textwidth}
		e) gerar subsídios que auxiliem no processo de tomada de decisão sobre quais aspectos de usabilidade podem ser considerados e quais não podem, durante o desenvolvimento e aperfeiçoamento das interfaces do ERP e;
	\end{minipage}
	\end{flushright}

	\begin{flushright}
	\begin{minipage}{.955\textwidth}
		f) visto que a compreensão da usabilidade de um nicho tão específico possibilita, a redução de erros e alterações desnecessárias, causando uma relativa economia, além de promover uma melhora contínua do software ao longo do tempo, este trabalho pode propiciar ganhos financeiros e de qualidade às empresas do mercado.		
	\end{minipage}
	\end{flushright}

\section{Método de Trabalho}  \label{Método de Trabalho}

Para este estudo foi proposta uma abordagem de trabalho que tem como base a adaptação da pesquisa realizada por Lambeck et al. (2014b) e a aplicação do método \textit{survey} para a análise de dados. Para Freitas (2000), os métodos de pesquisa podem ser quantitativos (\textit{survey}, experimento, entre outros...) ou qualitativos (estudo de caso, \textit{focus group}, entre outros...), devendo essa escolha estar associada aos objetivos da pesquisa. O método de pesquisa baseado em survey é explicado no Capítulo 2.\newline
\indent A pesquisa será realizada nas comunidades de usuários dos ERP’s mais utilizados por micro, pequenas e médias empresas no Brasil. Segundo Meirelles (2017), os principais fornecedores desse nicho de mercado são as empresas Totvs, SAP e Oracle. Esta pesquisa atuará com usuários que colaboram nessas comunidades por meios modernos como \textit{WhatsApp}, \textit{Skype}, \textit{Facebook} e ou \textit{e-mails}. O trabalho foi desenvolvido seguindo as atividades descritas a seguir:\newline
\indent Mapeamento Sistemático: realização de um estudo bibliográfico sobre a usabilidade e sua aplicação em sistemas ERP’s. Essa atividade teve o objetivo de conceituar a usabilidade e descrever seu uso em ERP’s, ampliando o entendimento sobre o tema.\newline
\indent Pesquisa piloto: A pesquisa piloto foi realizada em um grupo de 10 pessoas, para testar a efetividade da ferramenta escolhida.\newline
\indent Pesquisa: após a avaliação da pesquisa piloto, foi divulgada a pesquisa. Mais detalhes sobre a pesquisa podem ser vistos no Capítulo 3.\newline
\indent Avaliação e comparação dos resultados. Por fim, foram avaliados os resultados e feita a comparação com a pesquisa realizada na Alemanha, com o objetivo de verificar
semelhanças e diferenças.

\section{Organização do Trabalho} \label{Organização do Trabalho}

Na Seção 2 - Referencial teórico, destacam-se os seguintes subtópicos: mapeamento sistemático, que apresenta, inicialmente, os resultados de um mapeamento executado com o objetivo de verificar o que se produziu sobre o tema usabilidade relacionado a sistemas ERP;\newline
\indent Apresentação da conceituação tradicional do termo ‘usabilidade’.\newline
\indent Conceituação ERP, usabilidade versus ERP, em que são relacionados os conceitos específicos de usabilidade em ERP e apresentada a Metodologia;\textit{Survey}.\newline
\indent Trabalhos relacionados: foram avaliados os trabalhos recentes pertinentes ao tema proposto.\newline
\indent A Seção 3 traz a abordagem proposta, descreve as formas de coleta das informações e análise dos dados.\newline
\indent Na Seção 4 encontram-se a análise de resultados, a identificação dos dados resultantes da pesquisa, a discussão dos resultados e as comparações com os resultados obtidos na Alemanha.\newline
\indent A Seção 5 apresenta os resultados atingidos.\newline
\indent Na Seção 6 está a conclusão, onde são apresentadas as considerações finais, as limitações da pesquisa e sugestões para futuros trabalhos.
