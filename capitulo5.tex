\chapter{Resultados} \label{Resultados} 

Na análise dos dados, foi utilizado o programa estatístico SPSS da IBM, versão 23 (Windows).\newline
\indent Foram obtidos 126 participantes durante pesquisa nas seções a seguir os  dados são analisados.

\indent A avaliação da normalidade dos dados foi procedida pelo teste de Shapiro-Wilk e a homogeneidade das variâncias dos grupos tratados foi investigada por meio do teste de Levene. A Comparação entre os dados foi realizada pelo teste F de Snedecor-Fisher para a Análise da Variância (ANOVA), conforme o utilizado por Lambeck et al. (2014a). \newline

\section{Análise Quantitativa}

\subsection{Gênero dos Participantes}
Dos 126 participantes 125 identificaram o seu gênero, destes 12\% são do gênero Feminino e 88\% são do gênero Masculino, conforme pode ser visto na Figura \ref{fig:figura-sexo}.

\begin{figure}[H]
	\centering	
	\caption{Gênero dos participantes}
	\includegraphics[]{sexo}
	\label{fig:figura-sexo}
\end{figure}
\vspace{-0.8 cm} \hspace{3.85 cm} Fonte: Criado pelo Autor\newline

\subsection{Idade dos Participantes}
Todos os 126 participantes responderam a pergunta sobre sua faixa etária que pode ser vista na Figura \ref{fig:figura-fxetaria}.

\begin{figure}[H]
	\centering	
	\caption{Faixa etária dos participantes}
	\includegraphics[]{fxetaria}
	\label{fig:figura-fxetaria}
\end{figure}
\vspace{-0.8 cm} \hspace{3.55 cm} Fonte: Criado pelo Autor\newline

\subsection{Região de Origem}

Foi constatado que há uma predominancia de pesquisados com origem de nascimento nas regiões Sul e Sudeste, estas informações podem ser verificadas na Figura \ref{fig:figura-rgorigem}.

\begin{figure}[H]
	\centering	
	\caption{Região de Origem dos Pesquisados}
	\includegraphics[width=6in, height=2.7in]{regiao_origem}
	\label{fig:figura-rgorigem}
\end{figure}
\vspace{-0.8 cm} \hspace{2.65 cm} Fonte: Criado pelo Autor\newline

\subsection{Tamanho das Empresas}

Referente ao tamanho das empresas pesquisadas, verificou-se um grande número de participantes de grandes empresas (48\%), já as micro, pequenas e médias empresas correspondem a (42\%), estes dados estão representados na Figura \ref{fig:figura-tamanho}.

\begin{figure}[H]
	\centering	
	\caption{Tamanho das Empresas dos Pesquisados}
	\includegraphics[]{tamanho}
	\label{fig:figura-tamanho}
\end{figure}
\vspace{-0.8 cm} \hspace{2.45 cm} Fonte: Criado pelo Autor\newline

\subsection{Nível Hierárquico dos Pesquisados}

Referente ao nível hierárquico dos pesquisados verifica-se que 59\% são funcionários ou terceirizados, 32\% possuem cargo de nível médio de gestão e apenas 9\% são da alta gestão da empresa. Estes resultados podem ser visualizados na Figura \ref{fig:figura-cargos}.

\begin{figure}[H]
	\centering	
	\caption{Nível Hierárquico dos Pesquisados}
	\includegraphics[]{cargos}
	\label{fig:figura-cargos}
\end{figure}
\vspace{-0.8 cm} \hspace{2.85 cm} Fonte: Criado pelo Autor\newline


\subsection{Região de Trabalho}

Não houve uma significativa migração entre a região de origem e a região de trabalho dos pesquisados. Como acontece com a região de origem, foi constatado que há uma predominância de pesquisados que trabalham nas regiões Sul e Sudeste, estas informações podem ser verificadas na Figura \ref{fig:figura-rgtrabalho}.

\begin{figure}[H]
	\centering	
	\caption{Região de Trabalho dos Pesquisados}
	\includegraphics[width=6in, height=2.7in]{regiao_trabalho}
	\label{fig:figura-rgtrabalho}
\end{figure}
\vspace{-0.8 cm} \hspace{2.65 cm} Fonte: Criado pelo Autor\newline

\subsection{Tempo de Experiência na Empresa}

A maioria dos funcionários pesquisados (48\%) trabalham em suas respectivas empresas há menos de um ano, (42\%) dos pesquisados trabalham em suas empresas de 3 á 10 anos, uma minoria dos funcionários (10\%) atua em suas respectivas empresas há mais de 10 anos, estes dados podem ser visualizados na Figura \ref{fig:figura-expempresa}.

\begin{figure}[H]
	\centering	
	\caption{Tempo de Experiência na Empresa}
	\includegraphics[width=6in, height=4.2in]{experiencia_empresa}
	\label{fig:figura-expempresa}
\end{figure}
\vspace{-0.8 cm} \hspace{2.65 cm} Fonte: Criado pelo Autor\newline

\subsection{Experiência dos Participantes com ERP}

A experiência dos funcionários com sistemas ERP é  relativamente grande pois  cerca de (76\%) alegam ter mais de 10 anos de experiência com tais sistemas, os pesquisados que possuem de 4 a 10 anos de experiência com ferramentas ERP totalizam (14\%), enquanto os funcionários com 1 á 3 anos de experiência representam apenas (10\%) do publico pesquisado, os dados podem ser visualizados na Figura \ref{fig:figura-experp}.

\begin{figure}[H]
	\centering	
	\caption{Experiência dos Participantes com ERP}
	\includegraphics[width=6in, height=4.2in]{experiencia_erp}
	\label{fig:figura-experp}
\end{figure}
\vspace{-0.8 cm} \hspace{2.65 cm} Fonte: Criado pelo Autor\newline

\subsection{Método de pesquisa preferido}

Verifica-se pela a analise da porcentagem das respostas referente aos métodos de pesquisa que o publico Brasileiro pesquisado prefere mais [Pesquisa de texto completo] seguido de [Autocompletar], os dados podem ser avaliados na Figura \ref{fig:figura-q22}.

\begin{figure}[H]
	\centering	
	\caption{Método de pesquisa preferido}
	\includegraphics[width=6in, height=1.8in]{q22}
	\label{fig:figura-q22}
\end{figure}
\vspace{-0.8 cm} \hspace{3.15 cm} Fonte: Criado pelo Autor com o Software SPSS\newline

\subsection{Acesso aos dados por departamentos}

Verifica-se pela a analise da porcentagem das respostas que referente ao acesso de dados móveis por departamento, o publico Brasileiro pesquisado acessa mais os dados dos departamentos [Contabilidade - 24\%], [Recursos Humanos - 22\%] e [Produção - 21\%], os dados podem ser avaliados na Figura \ref{fig:figura-q31}.

\begin{figure}[H]
	\centering	
	\caption{Acesso a dados por dispositivos móveis por departamento}
	\includegraphics[width=6in, height=3.5in]{q31}
	\label{fig:figura-q31}
\end{figure}
\vspace{-0.8 cm} \hspace{3.15 cm} Fonte: Criado pelo Autor com o Software SPSS\newline

\subsection{Questão 11 x Questão 31}

A questão 11 indaga ao pesquisado "Em quais departamentos você usa o ERP?", já a questão 31 indaga ao pesquisado "Você usa seu dispositivo móvel para acessar os dados de quais departamentos?", analisamos então a relação entre as duas perguntas.\newline
\indent Dos pesquisados que responderam que usam o [Contabilidade] na questão 11, 21,79\% disseram acessar dados deste departamento por tecnologia móvel.\newline
\indent Dos pesquisados que responderam que usam o [Recursos Humanos] na questão 11, 44,68\% disseram acessar dados deste departamento por tecnologia móvel.\newline
\indent Dos pesquisados que responderam que usam o [Produção] na questão 11, 36,07\% disseram acessar dados deste departamento por tecnologia móvel.\newline
\indent Dos pesquisados que responderam que usam o [Compras / Gerenciamento de Cadeia de Suprimentos] na questão 11, 63,89\% disseram acessar dados deste departamento por tecnologia móvel.\newline
\indent Dos pesquisados que responderam que usam o [Gerenciamento de Projetos] na questão 11, 36,36\% disseram acessar dados deste departamento por tecnologia móvel.

\subsection{Uso de dispositivos móveis na empresa}

Verifica-se pela a analise da porcentagem das respostas que referente ao uso de dispositivos móveis, o publico Brasileiro pesquisado usa mais os dispositivos [\textit{Laptop} - 46\%], [\textit{Nenhum} - 30\%] e [\textit{Smartphone} - 12\%], os dados podem ser avaliados na Figura \ref{fig:figura-q30}.

\begin{figure}[H]
	\centering	
	\caption{Uso de dispositivos móveis na empresa}
	\includegraphics[width=4.7in, height=3.5in]{q30}
	\label{fig:figura-q30}
\end{figure}
\vspace{-0.8 cm} \hspace{3.15 cm} Fonte: Criado pelo Autor com o Software SPSS\newline

\subsection{Uso de dispositivos móveis tempo livre}

Verifica-se pela a analise da porcentagem das respostas que referente ao uso de dispositivos móveis, o publico Brasileiro pesquisado usa mais os dispositivos [\textit{Laptop} - 35\%] e [\textit{Smartphone} - 49\%], os dados podem ser avaliados na Figura \ref{fig:figura-q37}.

\begin{figure}[H]
	\centering	
	\caption{Uso de dispositivos móveis tempo livre}
	\includegraphics[width=4.7in, height=3.5in]{q37}
	\label{fig:figura-q37}
\end{figure}
\vspace{-0.8 cm} \hspace{3.15 cm} Fonte: Criado pelo Autor com o Software SPSS\newline

\section{Método Qualitativo}

Numa pesquisa qualitativa as respostas não são objetivas, e o propósito não é contabilizar quantidades como resultado, mas sim conseguir compreender o comportamento de determinado grupo. Os resultados estatísticos serão publicados no github acessível no endereço:https://github.com/cicerocruz/Mestrado\_IPT\_2019.

\subsection{Utilizando o teste-t no SPSS}

O método utilizado para fazer esta análise foi o Teste de Levene, para verificar se  a variança é homogênea ou não, e, logo após isso avaliar o teste-t para investigar a influência da escolha de uma opção, sobre a outra.\newline
\indent Abaixo na figura \ref{fig:figura-q23_q16_001}, podemos verificar que o software SPSS nos dá duas vias de calculo, sendo que a via  que  tomaremos é  determinada pela variável "Sig." do teste de levene, se o "Sig." for menor que 0,05 isto indica que a variança é homogênea consideraremos os dados da primeira linha, caso seja maior que 0,05 isto indica que a variança é não homogênea e devemos usar a segunda linha.\newline

\begin{figure}[H]
	\centering	
	\caption{teste-t Questão 23 opção 6 x Questão 16 opção 1}
	\includegraphics[width=6in, height=1.8in]{q23-q16-001}
	\label{fig:figura-q23_q16_001}
\end{figure}
\vspace{-0.8 cm} \hspace{1.45 cm} Fonte: Criado pelo Autor via software SPSS\newline

No exemplo da figura \ref{fig:figura-q23_q16_001} acima, verificamos a análise da questão “Questão 23. Como você avalia as seguintes estratégias para lidar com problemas no uso do sistema ERP?”, opção [Tipos de Menu e Estruturas Aprimorados] que tem sua relação com a “Questão 16. Quais tipos de menu são oferecidos pelo seu sistema?”, opção menu [1], podemos aqui avaliar que não existe uma correlação significativa isso pode ser avaliado usando a variável "Sig. bilateral", a fórmula desta relação é dada por (t(113.708) = 1009; p>0.05). Isto indica portanto que não há uma influencia da escolha de uma alternativa sobre outra.

Em outro exemplo que pode ser visto na figura \ref{fig:figura-q23_q16_002} abaixo, avaliamos a primeira linha devido a variável "Sig." do teste de levene. Na primeira linha observamos a variável "Sig. bilateral", sendo assim, a variável nos indica que existe uma correlação significativa de polaridade positiva, dada por (t(123)= 2019; p<0.05). Isto indica então que há uma influencia da escolha de uma alternativa sobre outra.

\begin{figure}[H]
	\centering	
	\caption{teste-t Questão 23 opção 6 x Questão 16 opção 2}
	\includegraphics[width=6in, height=1.8in]{q23-q16-002}
	\label{fig:figura-q23_q16_002}
\end{figure}
\vspace{-0.8 cm} \hspace{1.45 cm} Fonte: Criado pelo Autor via software SPSS\newline

\subsection{Utilizando a correlação de pearson no SPSS}

A correlação é um teste utilizado para avaliar a relação entre duas  variáveis continuas, ao se analisar a correlação de pearson via SPSS deve-se avaliar o seu o valor absoluto, sendo que a correlação é mais  forte quando se aproxima de um e mais fraca quando se aproxima de 0, referente ao sinal da correlação, quando a polaridade for negativa isto nos indica que correlação é inversa, ou seja, quanto maior for uma variável estudada, menor será a outra variável, o critério que utilizamos para  verificar a força da  correlação é o seu índice, sendo que se este for menor que 0,4 a correlação é considerada fraca, caso esteja entre 0,4 e 0,7 pode ser considerada moderada e acima de 0,7 é uma correlação forte. Referente ao valor de Sig.(bilateral) ele nos indicará a força da correlação caso seu valor seja menor que 0,05. Podemos ver uma analise de correlação na figura \ref{fig:figura-q23_q22_exp}.

\begin{figure}[H]
	\centering	
	\caption{Correlação de pearson exemplo}
	\includegraphics[width=6in, height=1.8in]{q23_q22_exp}
	\label{fig:figura-q23_q22_exp}
\end{figure}
\vspace{-0.8 cm} \hspace{2.85 cm} Fonte: Criado pelo Autor via software SPSS\newline

\subsection{Utilizando análise de efeitos com ANOVA no SPSS}

Inicialmente verificamos que existe esfericidade pelo teste de Mauchly. Abaixo vemos um exemplo onde não foi efetuado o calculo de  esfericidade pois só há duas alternativas,  conforme destacado na Figura \ref{fig:figura-com_esfericidade}, neste caso consideramos diretamente que há esfericidade, em outros casos verificariamos se há esfericidade avaliando a variável "W de Mauchly" ou "Mauchly's W", se ela é igual ou próxima a 1, no outro exemplo da Figura \ref{fig:figura-sem_esfericidade}, verificamos que a variável "W de Mauchly" ou "Mauchly's W" é menor que 0,5 então, vamos avaliar o Sig que é menor que 0,001 isso nos indica que não há esfericidade.

\begin{figure}[H]
	\centering	
	\caption{Teste de esfericidade de Mauchly - com Esfericidade}
	\includegraphics[width=6in, height=1.8in]{com_esfericidade}
	\label{fig:figura-com_esfericidade}
\end{figure}
\vspace{-0.8 cm} \hspace{2.85 cm} Fonte: Criado pelo Autor via software SPSS\newline

\begin{figure}[H]
	\centering	
	\caption{Teste de esfericidade de Mauchly - sem Esfericidade}
	\includegraphics[width=6in, height=1.8in]{sem_esfericidade}
	\label{fig:figura-sem_esfericidade}
\end{figure}
\vspace{-0.8 cm} \hspace{2.85 cm} Fonte: Criado pelo Autor via software SPSS\newline

Após determinar a esfericidade dos dados analisados verificamos qual método utilizar, o SPSS já faz o calculo considerando a existência ou não de esfericidade e nos apresenta um quadro com o calculo usando a esfericidade e apresentando o calculo para não esfericidade por 3 métodos isso pode ser visto na Figura \ref{fig:figura-metodo_teste_dentre_sujeitos} - Teste dentre sujeitos, caso haja esfericidade usaremos os dados da primeira linha do erro e da variável analisada, caso não seja utilizaremos um dos métodos de correção indicados, neste trabalho usaremos para dados sem esfericidade o método Greenhouse-Geisser.

\begin{figure}[H]
	\centering	
	\caption{Teste dentre sujeitos}
	\includegraphics[width=6in, height=2.8in]{teste_dentre_sujeitos}
	\label{fig:figura-metodo_teste_dentre_sujeitos}
\end{figure}
\vspace{-0.8 cm} \hspace{2.85 cm} Fonte: Criado pelo Autor via software SPSS\newline

Após a avaliação de qual método utilizar, temos que compreender o efeito de uma variável sobre a outra, para isso avaliamos o post-hoc, o SPSS usa o método de pairwise com ajuste de Bonferroni que é considerado mais conservador, que é exibido na Figura \ref{fig:figura-post-hoc}, sendo que neste quadro analisamos a coluna "Sig." esta coluna nos indicará se existe diferença entre uma variável e outra se ela for menor ou igual que 0,05. Caso exista diferença será necessário descreve-la através da analise do post-hoc.

\begin{figure}[H]
	\centering	
	\caption{Post-hoc comparações pelo método pairwise }
	\includegraphics[width=6in, height=1.8in]{post-hoc}
	\label{fig:figura-post-hoc}
\end{figure}
\vspace{-0.8 cm} \hspace{2.85 cm} Fonte: Criado pelo Autor via software SPSS\newline

\subsection{Questão 23 x Questão 16}

A questão 23 do questionário aplicado indagou aos participantes "Como você avalia as seguintes estratégias para lidar com problemas no uso do sistema ERP?", já a questão 16 indagou aos participantes "Quais tipos de menu são oferecidos pelo seu sistema?". A análise buscou verificar qual a influencia da escolha das opções 6 e 8 da questão 23 sobre os menus indicados na questão 16. Desta forma é possível verificar se as opções visuais facilitam as estratégias para resolução de problemas.\newline
\indent Após avaliar a relação entre a  respostas dadas na  questão 16 com as opções  escolhidas na questão 23, utilizando-se a  analise de médias e um Teste T, cuja técnica foi descrita acima, usando o software SPSS, verificou-se que houveram significativas divergências entre estes públicos.\newline
\indent No publico pesquisado no Brasil a analise indicou que a escolha dos tipos de menu [2] BreadCrumb e [4] Tree, são influenciados e influenciam a escolha da opção [Tipos de Menu e Estruturas Aprimorados] e que o tipo de menu [4] Tree, influência a escolha também da opção [Funcionalidade de Pesquisa Avançada], o que indica que estes dois tipos de menu satisfazem o publico Brasileiro e  são os tipos de menu mais utilizados como estratégias para lidar com problemas no uso do sistema ERP. Em contrapartida no publico Alemão não foi encontrada uma associação entre estas duas opções.

\subsection{Questão 23 x Questão 22}

A questão 23 do questionário aplicado indagou aos participantes "Como você avalia as seguintes estratégias para lidar com problemas no uso do sistema ERP?", já a questão 22 indagou aos participantes "Qual é o seu método preferido para pesquisar informações?".\newline
\indent A análise da correlação de Pearson entre a questão 23 com a questão 22, indica que existe uma correlação negativa, fraca e pouco significativa, sendo que esta correlação é  dada por (r=-0,30, p>0,05).\newline
\indent Os dados da  análise da correlação podem ser vistos na Figura \ref{fig:figura-q23_q22}  abaixo:

\begin{figure}[H]
	\centering	
	\caption{Correlação entre Questão 23 opção 8 x Questão 22}
	\includegraphics[width=6in, height=1.8in]{q23_q22}
	\label{fig:figura-q23_q22}
\end{figure}
\vspace{-0.8 cm} \hspace{2.15 cm} Fonte: Criado pelo Autor via software SPSS\newline

\subsection{Questão 23 x Questão 18}

Aqui analisamos a relação entre os participantes que escolheram a opção "Orientação e Suporte ao Usuário" na questão 23, em relação as opções da questão 18 solicitou aos participantes uma avaliação de seus ERP's "Por favor, avalie seu ERP de acordo com a escala na tabela a seguir:".\newline
\indent Referente a opção [O meu sistema ERP oferece uma ampla gama de funcionalidades de suporte para lidar com problemas] da questão 18, nossa analise indica que há uma correlação moderada, significativa e positiva, com a Questão 23 opção [Orientação e Suporte ao Usuário] dada por (r= 0,431,p<0,001), isto indica que o usuário que prefere a opção [Orientação e Suporte ao Usuário] tende a achar que seu ERP tem uma quantidade de recursos ideal, esta correlação pode ser verificada na Figura \ref{fig:figura-q235_q181}.\newline

\begin{figure}[H]
	\centering	
	\caption{Correlação entre Questão 23 opção 5 x Questão 18 opção 1}
	\includegraphics[width=6in, height=3.9in]{q23-5_q18-1}
	\label{fig:figura-q235_q181}
\end{figure}
\vspace{-0.8 cm} \hspace{1.45 cm} Fonte: Criado pelo Autor via software SPSS\newline

\indent Referente a opção [O meu sistema ERP é muito complexo, o que muitas vezes me faz sentir perdido] da questão 18, nossa analise indica que há uma correlação moderada, significativa e negativa, com a Questão 23 opção [Orientação e Suporte ao Usuário] dada por (r= -0,253,p<0,01), isto indica que o usuário que prefere a opção [Orientação e Suporte ao Usuário] não se sente perdido, esta correlação pode ser verificada na Figura \ref{fig:figura-q235_q182}.\newline

\begin{figure}[H]
	\centering	
	\caption{Correlação entre Questão 23 opção 5 x Questão 18 opção 2}
	\includegraphics[width=6in, height=3.9in]{q23-5_q18-2}
	\label{fig:figura-q235_q182}
\end{figure}
\vspace{-0.8 cm} \hspace{1.45 cm} Fonte: Criado pelo Autor via software SPSS\newline

\indent Referente a opção [O meu sistema ERP oferece inúmeras e úteis visualizações, as quais eu posso escolher] da questão 18, nossa analise indica que há uma correlação moderada, significativa e positiva, com a Questão 23 opção [Orientação e Suporte ao Usuário] dada por (r= 0,538,p<0,01), isto indica que o usuário que prefere a opção [Orientação e Suporte ao Usuário] percebe que seu sistema oferece recursos úteis, esta correlação pode ser verificada na Figura \ref{fig:figura-q235_q183}.\newline

\begin{figure}[H]
	\centering	
	\caption{Correlação entre Questão 23 opção 5 x Questão 18 opção 4}
	\includegraphics[width=6in, height=3.9in]{q23-5_q18-3}
	\label{fig:figura-q235_q183}
\end{figure}
\vspace{-0.8 cm} \hspace{1.45 cm} Fonte: Criado pelo Autor via software SPSS\newline

\subsection{Correlação entre respostas Questão 18}

Referente as opções [O meu sistema ERP oferece uma ampla gama de funcionalidades de suporte para lidar com problemas] e [O meu sistema ERP oferece inúmeras e úteis visualizações, as quais eu posso escolher] da questão 18, nossa analise indica que há uma correlação moderada, significativa e positiva, dada por (r= 0,514,p<0,01), isto indica que os usuários que escolheram uma opção tendem a escolher a outra, esta relação pode ser verificada na Figura  \ref{fig:figura-q181_q184}.\newline

\begin{figure}[H]
	\centering	
	\caption{Correlação entre Questão 18 opção 1 x Questão 18 opção 4}
	\includegraphics[width=6in, height=3.9in]{q18-1_q18-4}
	\label{fig:figura-q181_q184}
\end{figure}
\vspace{-0.8 cm} \hspace{0.45 cm} Fonte: Criado pelo Autor via software SPSS\newline

Referente as opções [O meu sistema ERP é muito complexo, o que muitas vezes me faz sentir perdido] e [O meu sistema ERP oferece inúmeras e úteis visualizações, as quais eu posso escolher] da questão 18, nossa analise indica que há uma correlação moderada, significativa e negativa, dada por (r= -0,583,p<0,01), isto indica que os usuários que escolheram uma opção tendem a não escolher a outra, esta relação pode ser verificada na Figura \ref{fig:figura-q182_q184}.\newline

\begin{figure}[H]
	\centering	
	\caption{Correlação entre Questão 18 opção 2 x Questão 18 opção 4}
	\includegraphics[width=6in, height=3.9in]{q18-2_q18-4}
	\label{fig:figura-q182_q184}
\end{figure}
\vspace{-0.8 cm} \hspace{0.45 cm} Fonte: Criado pelo Autor via software SPSS\newline

\subsection{Questão 17, Questão 18 e Questão 22}

Investigamos a correlação entre a questão 17 "Você conhece plenamente todas as etapas necessárias do processo para realizar suas tarefas", opção [O meu sistema ERP oferece uma ampla gama de funcionalidades de suporte para lidar com problemas] da questão 18 e a questão 21 "Você está sempre ciente das consequências de suas ações", nossa analise indica que há uma correlação fraca, significativa e positiva entre a questão 18 e a questão 21, dada por (r= 0,372,p<0,01), isto indica que quando se escolhe a opção analisada na questão 18 existe uma pequena possibilidade de se escolher "Sim Sempre" na questão 21, esta relação pode ser verificada na Figura \ref{fig:figura-q17_q18-1_q22}.\newline

\begin{figure}[H]
	\centering	
	\caption{Correlação entre Questão 17, Questão 18 opção 1 x Questão 22}
	\includegraphics[width=6in, height=3.9in]{q17_q18-1_q22}
	\label{fig:figura-q17_q18-1_q22}
\end{figure}
\vspace{-0.8 cm} \hspace{0.45 cm} Fonte: Criado pelo Autor via software SPSS\newline

Investigamos também a correlação entre a questão 17 "Você conhece plenamente todas as etapas necessárias do processo para realizar suas tarefas", opção [O meu sistema ERP é muito complexo, o que muitas vezes me faz sentir perdido] da questão 18 e a questão 21 "Você está sempre ciente das consequências de suas ações", nossa analise indica que há uma correlação entre moderada e forte, significativa e negativa entre a questão 18 e a questão 17, dada por (r= -0,687, p<0,01), ainda foi identificada uma correlação fraca e positiva, entre a questão 21 e a questão 17, dada por (r= 0,187, p<0,05), também foi identificada uma correlação fraca e negativa, entre a questão 21 e a questão 18, dada por (r= -0,194, p<0,05), isto indica que quando se escolhe a opção analisada na questão 18 existe a possibilidade de se não escolher "Sim Sempre" na questão 21, que quando se escolhe "Sim Sempre" na questão 21 existe uma pequena possibilidade de se escolher "Sim Sempre" na questão 17, que quando se escolhe "Sim Sempre" na questão 21 existe uma pequena possibilidade de se não escolher a opção analisada da questão 18, esta relação pode ser verificada na Figura  \ref{fig:figura-q17_q18-2_q22}.\newline

\begin{figure}[H]
	\centering	
	\caption{Correlação entre Questão 17, Questão 18 opção 2 x Questão 22}
	\includegraphics[width=6in, height=3.9in]{q17_q18-2_q22}
	\label{fig:figura-q17_q18-2_q22}
\end{figure}
\vspace{-0.8 cm} \hspace{0.45 cm} Fonte: Criado pelo Autor via software SPSS\newline

Por fim, investigamos a correlação entre a questão 17 "Você conhece plenamente todas as etapas necessárias do processo para realizar suas tarefas", opção [O meu sistema ERP oferece inúmeras e úteis visualizações, as quais eu posso escolher] da questão 18 e a questão 21 "Você está sempre ciente das consequências de suas ações", nossa analise indica que há uma correlação moderada, significativa e positiva entre a questão 18 e a questão 17, dada por (r= 0,511,p<0,01), também verificamos a indicação de uma correlação fraca, significativa e positiva entre a questão 18 e a questão 21, dada por (r= 0,213,p<0,05), por fim, ainda verificamos a indicação de uma correlação fraca, significativa e positiva entre a questão 17 e a questão 21, dada por (r= 0,187,p<0,05), isto indica que quando se escolhe a opção analisada na questão 18 existe uma possibilidade de se escolher "Sim Sempre" na questão 17 e uma possibilidade pequena de se escolher "Sim Sempre" na questão 21, esta relação pode ser verificada na Figura \ref{fig:figura-q17_q18-4_q22}.\newline

\begin{figure}[H]
	\centering	
	\caption{Correlação entre Questão 17, Questão 18 opção 4 x Questão 22}
	\includegraphics[width=6in, height=3.9in]{q17_q18-4_q22}
	\label{fig:figura-q17_q18-4_q22}
\end{figure}
\vspace{-0.8 cm} \hspace{0.45 cm} Fonte: Criado pelo Autor via software SPSS\newline

\subsection{Questão 17, Questão 18, Questão 22, Questão 34, Questão 35, Questão 36}

Investigamos a correlação entre a questão 17 "Você conhece plenamente todas as etapas necessárias do processo para realizar suas tarefas", opção [O meu sistema ERP oferece uma ampla gama de funcionalidades de suporte para lidar com problemas] da questão 18, questão 21 "Você está sempre ciente das consequências de suas ações", questão 34 "Há quantos anos você trabalha na empresa?", questão 35 "Há quanto tempo você usa Sistemas ERP no Geral?", e questão 36 "Como você auto avalia sua experiência com sistemas ERP?", a correlação entre as questões 17, 18 e 22 já  foram estudadas anteriormente, analisaremos aqui a correlação entre estas questões e as demais, nossa analise indica que há uma correlação moderada, significativa e de polaridade negativa entre a questão 17 e a questão 35, dada por (r= -0,583,p<0,01), isto indica que os  usuários com mais experiência em sistemas ERP tendem a não escolher a opção "Sim Sempre" na questão 17.\newline
\indent Ainda encontramos correlações entre a questão 17 e a questão 36 dada por (r= 0,254,p<0,01), entre a questão 18 e a questão 34 dada por (r= -0,385,p<0,01), entre a questão 18 e a questão 35 dada por (r= 0,365,p<0,01), entre a questão 35 e a questão 36 dada por (r= -0,346,p<0,01), sendo todas estas correlações fracas e significativas sendo que sua polaridade indicada na variável r.\newline
\indent Analisando estas  correlações  fracas verificamos que indivíduos que conhecem plenamente as etapas necessárias do processo para realizar suas tarefas,  tendem  a avaliar melhor seu ERP, já entre indivíduos que usam a pouco tempo o ERP tem uma pequena possibilidade de a achar que o sistema oferece mais opções do que  necessitam,  também esta analise de correlação nos indicou que indivíduos que usam o ERP a mais tempo tem uma pequena possibilidade de não estar satisfeitos com sua experiencia com estes sistemas, estes dados podem ser consultados na Figura  \ref{fig:figura-q17_q18-1_q22_q34_q35_q36}.

\begin{figure}[H]
	\centering	
	\caption{Correlação entre Questões 17, 18 opção 1, 22, 34, 35, 36}
	\includegraphics[width=6in, height=3.9in]{q17_q18-1_q22_q34_q35_q36}
	\label{fig:figura-q17_q18-1_q22_q34_q35_q36}
\end{figure}
\vspace{-0.8 cm} \hspace{0.85 cm} Fonte: Criado pelo Autor via software SPSS\newline

\subsection{Questão 2 x Questão 23 opções 4 e 7}

Inicialmente verificamos a correlação entre duas opções da questão 23, entre os participantes que escolheram a opção [Feedback: Visual, Tátil ou Auditivo] na questão 23, em relação a opção [Suporte a Dispositivos Sensíveis ao Toque], nossa analise indica que há uma correlação fraca, significativa e positiva, dada por (r= 0,338, p<0,001), isto indica que o usuário que seleciona a opção [Feedback: Visual, Tátil ou Auditivo] tende a ter uma pequena possibilidade de escolher também [Suporte a Dispositivos Sensíveis ao Toque], esta correlação pode ser verificada na Figura \ref{fig:figura-q23-4_q23-7}.

\begin{figure}[H]
	\centering	
	\caption{Correlação entre as opções 4 e 7 da Questão 24}
	\includegraphics[width=6in, height=3.1in]{q23-4_q23-7}
	\label{fig:figura-q23-4_q23-7}
\end{figure}
\vspace{-0.8 cm} \hspace{1.35 cm} Fonte: Criado pelo Autor via software SPSS\newline

Após usar uma ANOVA de duas vias, obtivemos os dados das Figuras \ref{fig:figura-q23-4-7_v2_esfericidade}, \ref{fig:figura-q23-4-7_v2_teste_dentre_sujeitos} e \ref{fig:figura-q23-4-7_v2_post-hoc}, verificamos então  não há efeito de uma opção sobre a outra na questão 23, F[(1 107) = 16,874; p < 0,001]. Analisando o teste post-hoc usando o ajuste de Bonferroni mostrou que  as duas opções analisadas diferem entre si.\newline
\indent Já referente a iteração entre a questão “2.Qual o setor em que sua  empresa atua?” e as opções 4,7 da questão 23, a ANOVA de duas vias mostrou que há efeito da questão 2 sobre estas opções da questão 23, F[(6 107) = 0,958; p < 0,001], analisando o teste de post-hoc verificamos que que quem escolhe a  opção 4 da questão 23 tende não escolher ser do setor 3 ser do setor 7, já quem escolhe a  opção 7 da questão 23, tende não ser do setor 2 e sim do setor 4.

\begin{figure}[H]
	\centering	
	\caption{Teste de esfericidade de Mauchly}
	\includegraphics[width=6in, height=2.7in]{q23-4-7_v2_esfericidade}
	\label{fig:figura-q23-4-7_v2_esfericidade}
\end{figure}
\vspace{-0.8 cm} \hspace{1.55 cm} Fonte: Criado pelo Autor via software SPSS

\begin{figure}[H]
	\centering	
	\caption{Teste dentre os sujeitos}
	\includegraphics[width=6in, height=2.7in]{q23-4-7_v2_teste_dentre_sujeitos}
	\label{fig:figura-q23-4-7_v2_teste_dentre_sujeitos}
\end{figure}
\vspace{-0.8 cm} \hspace{1.55 cm} Fonte: Criado pelo Autor via software SPSS

\begin{figure}[H]
	\centering	
	\caption{Post-hoc por pairwise com ajuste de Bonferroni}
	\includegraphics[width=6in, height=2.7in]{q23-4-7_v2_post-hoc}
	\label{fig:figura-q23-4-7_v2_post-hoc}
\end{figure}
\vspace{-0.8 cm} \hspace{1.55 cm} Fonte: Criado pelo Autor via software SPSS\newline

\subsection{Questão 25 x Questão 26}

Em média, o número de participantes que escolheu a opção inovação na questão 25 (M=2,04; EP=0,65) foi menor do que os que escolherem inovação na questão 26 (M=1,98; EP=0,89), t(121)= 0,688; p>0,05, sendo que o intervalo de confiança passa pelo 0, a analise da correlação indica que há uma correlação fraca, significativa e  de polaridade positiva entre as opções. Com base nestes dados isso nos indica que não houve diferenças significativas.

Esses dados podem ser visualizados na Figura \ref{fig:figura-q25_q26}
\begin{figure}[H]
	\centering	
	\caption{Teste-t Questão 25 e 26 opção inovação}
	\includegraphics[width=6in, height=2.7in]{q25_q26}
	\label{fig:figura-q25_q26}
\end{figure}
\vspace{-0.8 cm} \hspace{1.55 cm} Fonte: Criado pelo Autor via software SPSS\newline

Em média, o número de participantes que escolheu a opção utilidade na questão 25 (M=2,53; EP=0,685) foi menor do que os que escolherem utilidade na questão 26 (M=2,18; EP=0,904), t(119)= 4925; p<0,05, a analise da correlação indica que há uma correlação moderada, significativa e de polaridade positiva entre as opções. Com base nestes dados isso nos indica que houve diferenças na escolha das opções pelos pesquisados.

Esses dados podem ser visualizados na Figura \ref{fig:figura-q25-2_q26-2}
\begin{figure}[H]
	\centering	
	\caption{Teste-t Questão 25 e 26 opção utilidade}
	\includegraphics[width=6in, height=2.7in]{q25-2_q26-2}
	\label{fig:figura-q25-2_q26-2}
\end{figure}
\vspace{-0.8 cm} \hspace{1.55 cm} Fonte: Criado pelo Autor via software SPSS\newline

\subsection{Questão 18 x Questão 19}

A a análise de variância (ANOVA) mostrou que há um efeito da interação entre as questões 18 e 19 [F(3,328) = 18190; p<0,001]. O post-hoc de Bonferroni mostrou que quando se escolhe a opção 2, 3, 4 e 5 na questão 18, influência a escolha das mesmas opções na questão 19.

Esses dados podem ser visualizados nas Figuras \ref{fig:figura-q19_q20-esfericidade}, \ref{fig:figura-q19_q20_tst_sujeitos} e \ref{fig:figura-q19_q20_post-hoc}.
\begin{figure}[H]
	\centering	
	\caption{Teste de esfericidade de Mauchly}
	\includegraphics[width=6in, height=2.7in]{q19_q20-esfericidade}
	\label{fig:figura-q19_q20-esfericidade}
\end{figure}
\vspace{-0.8 cm} \hspace{1.55 cm} Fonte: Criado pelo Autor via software SPSS

\begin{figure}[H]
	\centering	
	\caption{Teste dentre os sujeitos}
	\includegraphics[width=6in, height=5.7in]{q19_q20_tst_sujeitos}
	\label{fig:figura-q19_q20_tst_sujeitos}
\end{figure}
\vspace{-0.8 cm} \hspace{1.55 cm} Fonte: Criado pelo Autor via software SPSS

\begin{figure}[H]
	\centering	
	\caption{Post-hoc por pairwise com ajuste de Bonferroni}
	\includegraphics[width=6in, height=3.7in]{q19_q20_post-hoc}
	\label{fig:figura-q19_q20_post-hoc}
\end{figure}
\vspace{-0.8 cm} \hspace{1.55 cm} Fonte: Criado pelo Autor via software SPSS

\subsection{Dispositivo móvel uso pessoal x uso trabalho}

Em média, o número de participantes escolheu mais o uso pessoal na  relação entre o uso pessoal (M=1,81; EP=0,059) em relação ao uso privado (M=0,85; EP=0,073), t(124)= -10,699; p<0,05. Com base nestes dados isso nos indica que houve diferenças na relação entre uso pessoal e no trabalho.

Esses dados podem ser visualizados na Figura \ref{fig:figura-privado_trabalho}
\begin{figure}[H]
	\centering	
	\caption{Teste-t uso de dispositivo móvel pessoal x trabalho}
	\includegraphics[width=6in, height=2.7in]{privado_trabalho}
	\label{fig:figura-privado_trabalho}
\end{figure}
\vspace{-0.8 cm} \hspace{1.55 cm} Fonte: Criado pelo Autor via software SPSS\newline

\subsection{Questão 18 x Questão 37}

A ANOVA de duas vias mostrou que há um efeito da interação entre as questões 18 e 37 dada por [F(1,965, 241,682) = 6,001; p<0,05].
O post-hoc de Bonferroni mostrou que quando se usa dispositivos móveis na questão 37, influência a escolha das mesmas opções na questão 18 sendo que há maior possibilidade de escolha da opção 5 na questão 18.

Esses dados podem ser visualizados nas Figuras \ref{fig:figura-q19_hb_post-hoc}, \ref{fig:figura-q19_hb_tst_sujeitos} e \ref{fig:figura-q19_q20_post-hoc}.
\begin{figure}[H]
	\centering	
	\caption{Teste de esfericidade de Mauchly}
	\includegraphics[width=6in, height=2.7in]{q19_hb_esfericidade}
	\label{fig:figura-q19_hb_esfericidade}
\end{figure}
\vspace{-0.8 cm} \hspace{1.55 cm} Fonte: Criado pelo Autor via software SPSS

\begin{figure}[H]
	\centering	
	\caption{Teste dentre os sujeitos}
	\includegraphics[width=6in, height=3.7in]{q19_hb_tst_sujeitos}
	\label{fig:figura-q19_hb_tst_sujeitos}
\end{figure}
\vspace{-0.8 cm} \hspace{1.55 cm} Fonte: Criado pelo Autor via software SPSS

\begin{figure}[H]
	\centering	
	\caption{Post-hoc por pairwise com ajuste de Bonferroni}
	\includegraphics[width=6in, height=3.7in]{q19_hb_post-hoc}
	\label{fig:figura-q19_hb_post-hoc}
\end{figure}
\vspace{-0.8 cm} \hspace{1.55 cm} Fonte: Criado pelo Autor via software SPSS

\subsection{Questão 23 x Questão 37}

A ANOVA de duas vias mostrou que não há interação entre as questões 23 e 37.

Esses dados podem ser visualizados nas Figuras \ref{fig:figura-q23_hb_post-hoc}, \ref{fig:figura-q23_hb_tst_sujeitos} e \ref{fig:figura-q23_q20_post-hoc}.
\begin{figure}[H]
	\centering	
	\caption{Teste de esfericidade de Mauchly}
	\includegraphics[width=6in, height=2.3in]{q23_hb_esfericidade}
	\label{fig:figura-q23_hb_esfericidade}
\end{figure}
\vspace{-0.8 cm} \hspace{1.55 cm} Fonte: Criado pelo Autor via software SPSS

\begin{figure}[H]
	\centering	
	\caption{Teste dentre os sujeitos}
	\includegraphics[width=6in, height=3.7in]{q23_hb_tst_sujeitos}
	\label{fig:figura-q23_hb_tst_sujeitos}
\end{figure}
\vspace{-0.8 cm} \hspace{1.55 cm} Fonte: Criado pelo Autor via software SPSS

\begin{figure}[H]
	\centering	
	\caption{Post-hoc por pairwise com ajuste de Bonferroni}
	\includegraphics[width=6in, height=3.7in]{q23_hb_post-hoc}
	\label{fig:figura-q23_hb_post-hoc}
\end{figure}
\vspace{-0.8 cm} \hspace{1.55 cm} Fonte: Criado pelo Autor via software SPSS

\subsection{Questão 17 x Questão 16 opção 4}

Em média, o número de participantes que escolheu de  "Sim Sempre" ou "Sim, na maioria das vezes" para a questão 17 "Você conhece plenamente todas as etapas necessárias do processo para realizar suas tarefas?" (M=1,43; EP=0,075), foi maior do que os não escolheram (M=2,16; EP=0,125), t(82,054)= 5,044; p<0,001. Esses dados podem ser visualizados na Figura \ref{fig:figura-v18_q16_opt4}.
\begin{figure}[H]
	\centering	
	\caption{Teste-t Questão 17 x Questão 16 opção 4}
	\includegraphics[width=6in, height=1.8in]{v18_q16_opt4}
	\label{fig:figura-v18_q16_opt4}
\end{figure}
\vspace{-0.8 cm} \hspace{1.55 cm} Fonte: Criado pelo Autor via software SPSS

\subsection{Questão 18 opção 2 x Questão 16 opção 4}

Em média, o número de participantes que escolheram o tipo de menu [4] na questão 16 que, escolheram a opção [O meu sistema ERP é muito complexo, o que muitas vezes me faz sentir perdido] na questão 18 (M=4,13; EP=0,193), foi maior do que os não escolheram (M=2,88; EP=0,193),  t(93,320)= -6,307; p<0,001. Esses dados podem ser visualizados na Figura \ref{fig:figura-v19_u2_q16_opt4}.
\begin{figure}[H]
	\centering	
	\caption{Teste-t Questão 18 opção 2 x Questão 16 opção 4}
	\includegraphics[width=6in, height=1.8in]{v19_u2_q16_opt4}
	\label{fig:figura-v19_u2_q16_opt4}
\end{figure}
\vspace{-0.8 cm} \hspace{1.55 cm} Fonte: Criado pelo Autor via software SPSS

\subsection{Questão 18 opção 3 x Questão 16 opção 4}

Em média, o número de participantes que escolheram o tipo de menu [4] na questão 16 que, escolheram a opção [A quantidade de informações e detalhes fornecidos é alta para as minhas necessidades]  na questão 18 (M=4,44; EP=0,133), foi maior do que os não escolheram (M=2,61; EP=0,142), t(93,320)= -6,307; p<0,001. Esses dados podem ser visualizados na Figura \ref{fig:figura-v19_u3_q16_opt4}.
\begin{figure}[H]
	\centering	
	\caption{Teste-t Questão 18 opção 3 x Questão 16 opção 4}
	\includegraphics[width=6in, height=1.8in]{v19_u3_q16_opt4}
	\label{fig:figura-v19_u3_q16_opt4}
\end{figure}
\vspace{-0.8 cm} \hspace{1.55 cm} Fonte: Criado pelo Autor via software SPSS

\subsection{Questão 22 x Grupo de Questões}

Criamos uma variável chamada search\_group nela aplicamos uma função de agrupamento, separando as opções em três grupos, o grupo 1 selecionou somente quem escolheu a opção [1] da questão  "Qual é o seu método preferido para pesquisar informações?", o grupo 2 selecionou somente os que escolheram a primeira e a segunda opção, juntamente com os que escolheram a primeira a segunda e a terceira opção e por  fim os que escolheram a primeira e a terceira opção por ultimo o grupo 3 os que escolheram a quarta opção.\newline
\indent Verificamos então que a ANCOVA revelou que há um efeito do grupo de perguntas sobre a opção [Sinônimos e correção automática] dada por [F(3,122) = 6752; p<0,05]. O post-hoc de Bonferroni mostrou que a opção 3 da questão 22 em relação a variável search\_group difere das opções 2 e 3.\newline
\indent Os dados podem ser consultados nas Figuras \ref{fig:figura-search_group_v19_u3} e \ref{fig:figura-search_group_v19_u3-post-hoc}.

\begin{figure}[H]
	\centering	
	\caption{Teste dentre sujeitos}
	\includegraphics[width=6in, height=2.8in]{search_group_v19_u3}
	\label{fig:figura-search_group_v19_u3}
\end{figure}
\vspace{-0.8 cm} \hspace{1.55 cm} Fonte: Criado pelo Autor via software SPSS

\begin{figure}[H]
	\centering	
	\caption{Post-hoc método de pairwise com correção de Bonferroni}
	\includegraphics[width=6in, height=3.8in]{search_group_v19_u3-post-hoc}
	\label{fig:figura-search_group_v19_u3-post-hoc}
\end{figure}
\vspace{-0.8 cm} \hspace{1.55 cm} Fonte: Criado pelo Autor via software SPSS

Verificamos também que a ANCOVA revelou que não há um efeito do grupo de perguntas sobre a opção [Auto-completar].\newline
\indent Os dados podem ser consultados nas Figuras \ref{fig:figura-search_group_v19_u2}.

\begin{figure}[H]
	\centering	
	\caption{Teste dentre sujeitos}
	\includegraphics[width=6in, height=2.8in]{search_group_v19_u2}
	\label{fig:figura-search_group_v19_u2}
\end{figure}
\vspace{-0.8 cm} \hspace{1.55 cm} Fonte: Criado pelo Autor via software SPSS

\chapter{Conclusões} \label{Conclusões} 

\section{Empresa, sistema ERP e Usuários}

Em comparação com o publico alemão (184 pesquisados) o publico brasileiro ( 126 pesquisados ), verificou-se que há uma diferença no perfil das empresas representadas pelos pesquisados. Se na  alemanha a maior parte (70,86\%) são de empresas de médio porte ( 50 á 250 empregados ) no brasil este grupo representa somente (23\%), na Alemanha as pequenas empresas ( 10 a 49 empregados ) representam (24,57\%) enquanto no brasil esse grupo representa (11,9\%) outro fato interessante é que no brasil (48\%) dos  pesquisados informam trabalhar em empresas grandes (mais de 250 empregados). Visto que o investimento em sistemas ERP demanda um esforço financeiro por parte da empresa é compreensível que no Brasil o maior numero de usuários esteja em  empresas de grande porte.\newline
\indent As áreas mais indicadas na Alemanha são produção (52,30\%), comércio atacadista e varejista (16,67\%) e serviços de informação e comunicação (10,92\%). No brasil foram houve um equilíbrio entre os  setores sendo os mais escolhidos compras (18\%), Contabilidade (17\%), RH(16\%) e Armazenamento/Inventário (16\%). O tempo médio de uso dos sistemas ERP  na Alemanha é de 8,6 anos e varia entre um e 23 anos. Já no brasil o tempo médio 4 anos e varia de um a 29 anos. Uma ampla gama de fornecedores de ERP pode ser encontrada na Alemanha, enquanto o SAP é o sistema mais prevalente com (28,26\%). Já no Brasil predominou o uso do Protheus fornecido pela Totvs com (61,9\%) enquanto o SAP responde por (18,3\%).\newline
\indent Os sistemas ERP são usados por participantes com diferentes cargos em suas empresas, na Alemanha empregados (38,51\%) enquanto no brasil esse grupo representa (50,70\%), na Alemanha gerentes de departamento representam (42,53\%) já no brasil esse grupo representa (32,50\%) e por fim na Alemanha CEOs ou CIOs representam (18,97\%) enquanto esse grupo no brasil representa (8,8\%). Verificamos então que no Brasil o grupo empregados é mais representativo.

\section{Resultados da avaliação do sistema ERP}

Uma primeira ênfase da nossa pesquisa é derivada dos problemas de usabilidade identificados na literatura recente Lambeck et al. (2014a), dedica-se à avaliação do usuário do sistema ERP de acordo com as afirmações apresentadas na Figura \ref{fig:figura-erp_x_sa}. Esses itens abrangem o suporte em situações de erro (1), a complexidade geral do sistema (2), a quantidade e o nível de detalhamento das informações (3 ), a disponibilidade de visualizações (4) e a confusão causada por janelas abertas simultaneamente (5). Os participantes foram convidados a avaliar as cinco afirmações mostradas na Figura \ref{fig:figura-erp_x_sa}, em uma escala Likert descentralizada de seis pontos, variando de “1 - concordo totalmente” a “6 - discordo totalmente”.\newline
\indent Os resultados indicam que no brasil Figura \ref{fig:figura-erp_x_sa} os  usuários avaliam estes problemas na médios e menores, como na alemanha. Sendo que a complexidade do sistema foi avaliada em um indice menor no brasil (3,64) do que os da alemanha (4,01).  No publico brasileiro, a complexidade do sistema está relacionada a utilidade das informações exibidas (Figura \ref{fig:figura-erp_x_sa}, no. 4) (r=-0,583, p<0,01). Também no publico alemão, a disponibilidade de numerosas e úteis visualizações (Figura \ref{fig:figura-erp_x_sa}, no. 4) melhora a classificação de usuário da complexidade do sistema (r = -.312, p <.001). Já no publico alemão, a complexidade do sistema está significativamente relacionada à abundância percebida de informações e seu nível de detalhamento (Figura \ref{fig:figura-erp_x_sa}, no. 3) (r = 0,632, p <0,001). Isso não ocorre no publico Brasileiro.\newline
\indent No publico brasileiro, a gama de funcionalidades oferecidas pelo sistema (Figura \ref{fig:figura-erp_x_sa}, no. 2) melhora a classificação da percepção das inúmeras e úteis visualizações das opções do sistema (Figura \ref{fig:figura-erp_x_sa}, no. 4). O software suplementar nos dois países, é empregado como recurso para aumentar ou estender a flexibilidade do sistema ERP. \newline
\indent Os participantes da pesquisa realizada na Alemanha relataram usar duas aplicações adicionais. Software de planilha (por exemplo, o Microsoft Excel) (N = 174, 95,83\%), seguido por aplicações desenvolvidas internamente (N = 174, 55,95\%). Já os participantes no brasil preferem software de planilha (N = 125, 94\%), porém como segundo recurso no brasil se usa softwares web/app's  (N = 125, 66\%).\newline
\indent Na pesquisa alemã, a análise de variância (ANOVA) expôs diferenças na classificação de ERP e aplicação adicional nos itens 2, 3 e 4 [F (4,448) = 21,53, p <0,001)].  Já na pesquisa brasileira, a análise de variância (ANOVA) mostrou diferenças entre a classificação do ERP e aplicação adicional, através do post-hoc de Bonferroni nos itens 2, 3, 4 e 5  [F(3,328) = 18,19; p<0,001].\newline
\indent Os resultados indicam que sistemas ERP ainda apresentam deficiências, que estas, apesar de serem diferentes entre os públicos ressaltam as facilidades dos softwares adicionais, que devem ser abordadas em trabalhos futuros para acompanhar o uso do software adicional avaliando os diferenciais de sua usabilidade.\newline

\begin{figure}[H]
	\centering	
	\caption{Comparativo entre ERP e Software Alternativo}
	\includegraphics[width=6in, height=1.8in]{erp_x_sa}
	\label{fig:figura-erp_x_sa}
\end{figure}
\vspace{-0.8 cm} \hspace{1.55 cm} Fonte: Criado pelo Autor via software SPSS

\section{Incerteza no uso do sistema}

A identificação da funcionalidade ERP necessária e o acesso subsequente a ela é uma tarefa essencial no uso do sistema ERP, mas continua sendo um grande desafio. [5], [6] Para poder executar uma transação apropriadamente, os usuários precisam possuir o conhecimento sobre o próprio processo de negócios, precisam identificar e acessar a funcionalidade necessária na interface do usuário e também devem estar cientes das consequências ao cometer uma transação. Esses três aspectos formam nossa definição de certeza no uso do sistema. Os participantes foram convidados a avaliar com que frequência eles tiveram dificuldades nesses três aspectos em uma escala Likert de cinco pontos, de “1 - nunca” a “5 - sempre”.\newline
\indent Os resultados no Brasil, revelaram que os usuários sofrem principalmente com a falta de conhecimento do processo (N = 122, M = 4,33, SD = 0,743) e não tem consciência suficiente das consequências de sua ação (N = 122, M = 4,19, SD = 0,708 ). Já em relação a Alemanha, os dados revelaram que os usuários não sofrem com a falta de conhecimento do processo (N = 150, M = 1,93, SD = 0,95) ou com a consciência insuficiente das consequências de sua ação (N = 147, M = 1,99, SD = 0,91 ). Esses dados nos levam a concluir que  basicamente os  usuários brasileiros alegam não conhecer seu processo e não possuir consciência das consequências de suas ações.\newline
\indent Portanto, a capacidade de localizar a funcionalidade corporativa desejada, continua sendo um problema geral de usabilidade em diferentes níveis de experiência, sendo que isto se agrava no Brasil, pelo fato de haver indícios de não haver consciência de suas ações e também falta de conhecimento do processo.\newline
\indent No Brasil verificamos que o tipo de menu melhor classificado foi o menu de contexto, opção [5] com 1,80 em média (N = 10, DP = 0,632), seguido pelo menu barra tarefas, opção [6] com 1,78 em média (N = 18, DP = 0,647). Já na Alemanha foi verificado que o menu melhor classificado foi o menu de árvore, opção [4] com 2,66 em média (N = 107, DP = 0,82), seguido pelo menu de contexto, opção [5] com 2,33 em média (N = 51, SD = 0,84). É importante ressaltar que na Alemanha houve uma considerável parcela de pessoas que não utilizavam ERP (93) enquanto os que utilizam ERP totalizam (184), já no Brasil a totalidade dos pesquisados (126) utilizam ERP.\newline
\indent Juntamente com o desempenho dos pesquisados alemães nas questões 17 e 18, isso levou a conclusão pelos Alemães de que o melhor tipo de menu para localizar funções requeridas é o de contexto.  No Brasil não é possível chegar a esta conclusão pois os pesquisados alegam desconhecer os processos e as consequências de suas ações, isso também pode explicar o fraco desempenho da média dos melhores menus no Brasil.\newline
\indent Os resultados no Brasil, indicam uma razão hipotética que o treinamento nos sistemas ERP são deficitários, pois não transmitem ao pesquisado seus processos e as consequências de suas ações no sistema.  

\section{Avaliação de abordagem}

Esta seção é dedicada à avaliação de possíveis abordagens, que podem ser apropriadas para solucionar algumas das deficiências em sistemas ERP.\newline
\indent As interfaces atuais de ERP ainda estão lidando com conceitos de interface do usuário, que foram introduzidos na década de 1990. Para atenuar pelo menos algumas das deficiências examinadas, sugere-se a aplicação de conceitos de interface inovadores sempre que o cenário permitir. Primeiro, essas abordagens devem contar com interfaces visualmente ricas. Em segundo lugar, eles devem empregar conceitos de interação direta-manipulativa, que provaram seus benefícios em vários cenários no domínio empresarial, considera-se que ao oferecer um uso intuitivo do sistema, as barreiras que atualmente dificultam a interação entre usuário e sistema podem ser reduzidas.\newline
\indent Os autores da pesquisa Alemã levantam a hipótese de que conceitos como dispositivos sensíveis ao toque não são a primeira escolha de um usuário de ERP, ao serem solicitados a identificar conceitos potenciais para melhorar a usabilidade do ERP. O estudo Alemão ofereceu ao usuário oito abordagens gerais e bastante abstratas para avaliação dos pesquisados que também continham a opção “dispositivos sensíveis ao toque (por exemplo, \textit{multi-touch}, sistema de mesa)”. O resultado alemão mostrou que esta opção foi a pior avaliada de todas as abordagens apresentadas (N = 109, M = 3,55, DP = 1,33). Na pesquisa  brasileira a opção “dispositivos sensíveis ao toque (por exemplo, \textit{multi-touch}, sistema de mesa)” também foi a pior avaliada  tanto no sentido de inovação (N = 124, M=1,98, DP = 0,888), como no sentido de utilidade (N = 122, M=2,16, DP = 0,909), porém no Brasil a omissão foi bem baixa  o que indica que o publico Brasileiro tem mais contato com este tipo de tecnologia ao contrario do publico Alemão pesquisado a época, o tempo decorrido entre uma pesquisa e outra pode explicar esta divergência entre os públicos.\newline
\indent Na Alemanha a Questão 25 recebeu classificações bastante favoráveis tanto para inovação (N = 58, M = 1,91, SD = 0,93) como para utilidade (N = 58, M = 1,97, SD = 1,02) em uma escala ordinal de seis pontos variando de “1 - muito bom ”Para“ 6 - muito ruim ”. Porém também houve uma grande abstenção nesta questão como na questão 26. No Brasil também a Questão 25 recebeu classificações favoráveis em inovação (N = 122, M = 2,04, DP = 0,648) como em utilidade (N = 121, M = 2,53, DP = 0,684), porém a abstenção nesta questão foi bem reduzida.

\section{Trabalhos Futuros}

A pesquisa brasileira mostrou que existem diferenças significativas entre o publico Alemão e o público Brasileiro, sendo que o perfil dos usuários são bem semelhantes em indicadores como experiência em sistemas ERP e faixa etária, outros indicadores como tamanho das empresas e cargo diferem.\newline
\indent Propomos como trabalho futuro efetuar uma comparação entre as regiões do Brasil pois por se tratar de um país com proporções continentais e uma diversidade cultural grande, o que não foi possível fazer neste trabalho devido a baixa resposta das regiões Norte, Nordeste, Centro-Oeste do país.\newline
\indent Uma segunda proposta seria fazer uma comparação entre os países das Américas do Sul, central e do Norte.\newline
