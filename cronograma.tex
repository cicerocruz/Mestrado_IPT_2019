%\begin{enumerate}
%	\item \label{ela-pro} levantamento do \textit{design} da pesquisa:
%	nessa tarefa o pesquisador cria o \textit{draft} da pesquisa e faz testes com a ferramenta;
%	\item \label{anI} aprovação questionário pelo orientador: 
%	o pesquisador submete a pesquisa e o texto da qualificação ao orientador para obter sua aprovação;
%	\item \label{anII} aprovação qualificação pela banca:
%	a banca analisa o trabalho de qualificação;
%	\item \label{anIII} execução da pesquisa:  
%	com a aprovação da banca o pesquisador executa a pesquisa;
%	\item \label{dI} analise de resultados:
%	após fechar a pesquisa são analisados os resultados dos dados obtidos;
%	\item \label{dII} apresentação banca:
%	apresentação da banca final.
%	%\item \label{dIII} Implementação do módulo de caputra de imagem. 
%	%\item \label{esc-tcI}  Escrita do TC I.
%	%\item \label{imI} Implementação da máquina de estados para controle da análise de imagem.
%	%\item \label{imII} Desenvolvimento da camada de integração.
%	%\item \label{imIII} Integração dos módulos que compõe o sistema.
%	%\item \label{tec} Teste e correções.
%	%	\subitem Comparar desempenho entre hardware e software.
%	%\item \label{esc-tcII} Escrita do TC II.
%\end{enumerate}
%{\raggedleft
\begin{flushright}
	\begin{minipage}{.96\textwidth}
		\label{ela-pro} 1.  levantamento do \textit{design} da pesquisa: nessa tarefa o pesquisador cria o \textit{draft} da pesquisa e faz testes com a ferramenta; \newline
		\label{anI} 2.  aprovação questionário pelo orientador:	o pesquisador submete a pesquisa e o texto da qualificação ao orientador para obter sua aprovação; \newline
		\label{anII} 3.  aprovação qualificação pela banca:	a banca analisa o trabalho de qualificação;
		\label{anIII} 4.  execução da pesquisa:  com a aprovação da banca o pesquisador executa a pesquisa; \newline
		\label{dI} 5.  analise de resultados: após fechar a pesquisa são analisados os resultados dos dados obtidos; \newline
		\label{dII} 6.  apresentação banca: apresentação da banca final.
	\end{minipage}
\end{flushright}
%	\newline	
%\par}


Observe-se o cronograma a seguir:

\definecolor{midgray}{gray}{.5}
\begin{table}[!htbp]
	\centering
		\begin{tabular}{|c|c|c|c|c|c|c|c|c|c|c|}
		\hline
		&\multicolumn{2}{c|}{2018}&\multicolumn{8}{c|}{2019}\\
		\hline
		&NOV&DEZ&JAN&FEV&MAR&ABR&MAI&JUN&JUL&AGO\\
		\hline
		\ref{ela-pro}&\cellcolor{midgray}&&&&&&&&&\\
		\hline
		\ref{anI}&&\cellcolor{midgray}&\cellcolor{midgray}&\cellcolor{blue}&&&&&&\\
		\hline	
		\ref{anII}&&&&&\cellcolor{red}&&&&&\\
		\hline			
		\ref{anIII}&&&&&\cellcolor{red}&\cellcolor{red}&\cellcolor{red}&&&\\
		\hline	
		\ref{dI}&&&&&&&&\cellcolor{red}&&\\
		\hline
		\ref{dII}&&&&&&&&&\cellcolor{red}&\\
		\hline	
		%\ref{dIII}&&&&\cellcolor{midgray}&\cellcolor{midgray}&&&&&\\
		%\hline	
		%\ref{esc-tcI}&&&\cellcolor{midgray}&\cellcolor{midgray}&\cellcolor{midgray}&&&&&\\
		%\hline	
		%\ref{imI}&&&&&\cellcolor{midgray}&&&&&\\
		%\hline	
		%\ref{imII}&&&&&&\cellcolor{midgray}&&&&\\
		%\hline	
		%\ref{imIII}&&&&&&\cellcolor{midgray}&\cellcolor{midgray}&\cellcolor{midgray}&&\\
		%\hline	
		%\ref{tec}&&&&&&&&\cellcolor{midgray}&\cellcolor{midgray}&\\
		%\hline	
		%\ref{esc-tcII}&&&&&&&&\cellcolor{midgray}&\cellcolor{midgray}&\cellcolor{midgray}\\
		%\hline	
		\end{tabular}
\end{table}

Legenda: \newline
\indent Vermelho - Indica Tarefa não Executada;\newline
\indent Azul - Indica Tarefa em Execução; \newline
\indent Cinza - Indica Tarefa Executada;\newline