\documentclass[
	% -- opções da classe memoir --
	12pt,				% tamanho da fonte
    openany,            % capítulos começam em qualquer página
%	openright,			% capítulos começam em pág ímpar (insere página vazia caso preciso) ou openany
	oneside,			% ou twoside
%	twoside,			% ou oneside
	a4paper,			% tamanho do papel.
	% -- opções da classe abntex2 --
	chapter=TITLE,		% títulos de capítulos convertidos em letras maiúsculas
	%section=TITLE,		% títulos de seções convertidos em letras maiúsculas
	%subsection=TITLE,	% títulos de subseções convertidos em letras maiúsculas
	%subsubsection=TITLE,% títulos de subsubseções convertidos em letras maiúsculas
	%Sumario
	sumario=abnt-6027-2012,
	% -- opções do pacote babel --
	english,			% idioma adicional para hifenização
	brazil,				% o último idioma é o principal do documento
	]{abntex2}

% ---
% Pacotes fundamentais
% ---
\usepackage[T1]{fontenc}		% Selecao de codigos de fonte.
\usepackage{lastpage}			% Usado pela Ficha catalográfica
%\usepackage{pifont}
\usepackage{indentfirst}		% Indenta o primeiro parágrafo de cada seção.
\usepackage{color}				% Controle das cores
\usepackage{graphicx}			% Inclusão de gráficos
\usepackage[brazil]{babel}		% Para definir a língua - abntex2.cls já carrega
% Fonte Arial >
\usepackage{fontspec}
\usepackage{xltxtra}
\setromanfont[Mapping=tex-text]{Arial}
\setsansfont[Mapping=tex-text]{Arial}
\usepackage{leading} 
% Fonte Arial <
\usepackage{setspace}			% Permite Configuracao de Espaçamento entre Linhas
\usepackage{microtype}			% Ajuda a ajustar hifenização
%\hyphenpenalty = 10000			% Remove a Hifenização
%\hyphenpenalty = 1000			% Reduz a Hifenização
\usepackage{type1cm}			% Fontes Escaláveis
\usepackage{float}				% Permite Posicionar as Figuras de Forma Correta
\usepackage{csquotes}			% Configuração adicional para aspas
\usepackage{lscape}				% Possibilita a criação de paginas em formato paisagem
\usepackage{enumerate}          % permite listas numeradas
%\usepackage{cite}
\usepackage{iptabntex2}		% Pacote com Customizações do abntex2 + IPT para esta dissertaçao
%\usepackage[alf]{abntex2cite}
%\usepackage[alf,abnt-emphasize=bf,abnt-etal-list=2,abnt-etal-cite=2,abnt-etal-text=2,abnt-repeated-title-omit=yes]{abntex2cite}	% Citações padrão ABNT conforme orientação do IPT
%\usepackage{apacite}
\usepackage{changepage}
\usepackage{sourcecodepro}

% para criar tabela que pode ser quebrado em várias páginas
\usepackage{longtable} 
\usepackage{caption}

\usepackage[table]{xcolor}
%\definecolor{midgray}{gray}{0.6}

% Legenda da Figura
%\def\figcaption{%
%	\refstepcounter{figure}%
%	\@dblarg{ \@caption{ figure } } }
%\newcommand{\figCaption}[2]{\caption[#1]{\textbf{#1}. #2}}

%\newcommand{\figcaption}[1]{\refstepcounter{figure} #1 }

% Define o caminho das figuras
\graphicspath{{imagens/}}
%Define o espaçamento entre o titulo da figura e a figura
\usepackage[skip=0pt]{caption}

% ---
% Informações de dados para CAPA e FOLHA DE ROSTO
% ---
\titulo{Percepção de Usabilidade em Sistemas ERPs baseado no Contexto dos Usuários}
\autor{Cícero Odilio Cruz}
\data{2019}
\mes{07}
\local{São Paulo}
\instituicao{\mbox{Instituto} de Pesquisas Tecnológicas do \mbox{Estado} de São Paulo}
\tipotrabalho{Dissertação de Mestrado}
\curso{Engenharia da Computação: Engenharia de Software}
\concentracao{Área de Concentração: Engenharia de Software}
\orientador{Prof. Dr. Plinio Thomaz Aquino Junior}
\orientadorafiliacao{Mestrado em Engenharia de Computação}
\membroum{Prof. Dr. Flavio Tonidandel}
\membroumAfiliacao{Centro Universitário FEI}
\membrodois{Prof. Dr. Leandro Alves da Silva}
\membrodoisAfiliacao{Instituto Superior de Administração e Economia}
\palavraschave{Mestrado, IPT, LATEX, ABNTEX2.}
\keyword{Master, IPT, LATEX, ABNTEX2.}
%% Preambulo para Qualificacao
\preambulo{Dissertação apresentada ao \imprimirinstituicao~- IPT, como parte dos requisitos para a obtenção do título de Mestre em \imprimircurso}
% Preambulo para Defesa
%\preambulo{Dissertação de Mestrado apresentada ao \imprimirinstituicao~- IPT, como parte dos requisitos para a obtenção do título de Mestre em  \imprimircurso}

% informações do PDF
\makeatletter
\hypersetup{
plainpages=false,
colorlinks=true,
citecolor=black,
linkcolor=black,
urlcolor=black,
filecolor=black,
bookmarksopen=true,
pdftitle={\imprimirtitulo},
pdfauthor={\imprimirautor},
pdfsubject={\imprimirpreambulo}, 
pdfcreator={XeLaTeX com abnTeX2},
pdfkeywords={\imprimirpalavraschave},
bookmarksdepth=4
}
\makeatother
% ---

% ---
% compila o indice
% ---
%\makeindex
% ---


\usepackage{lipsum}
% Configurar Margens do Documento
% Esquerda e Direita
%\setlrmarginsandblock{3cm}{3.85cm}{*}
% Cabeçalho e Rodapé
%\setulmarginsandblock{2.21cm}{3.01cm}{*}
\checkandfixthelayout

%Nivel do Sumário
\settocdepth{section}
\setsecnumdepth{section}
% Início do documento
% ----
\begin{document}
\chapterstyle{iptabntex2}			% Estilo de capitulo adotado pelo IPT
%\sloppy							% Ajusta linhas sem a hifenização
\frenchspacing					% Retira espaço extra, obsoleto, entre as frases.

% ----------------------------------------------------------
% ELEMENTOS PRÉ-TEXTUAIS
% ----------------------------------------------------------
\pretextual
\imprimircapa

\imprimirfolhadeaprovacao

\imprimirfolhaderosto

% ---
% Inserir a ficha bibliografica
% ---

% Isto é um exemplo de Ficha Catalográfica, ou ``Dados internacionais de
% catalogação-na-publicação''. Você pode utilizar este modelo como referência.
% Porém, provavelmente a biblioteca da sua universidade lhe fornecerá um PDF
% com a ficha catalográfica definitiva após a defesa do trabalho. Quando estiver
% com o documento, salve-o como PDF no diretório do seu projeto e substitua todo
% o conteúdo de implementação deste arquivo pelo comando abaixo:
%
% \begin{fichacatalografica}
%     \includepdf{fig_ficha_catalografica.pdf}
% \end{fichacatalografica}
%\begin{fichacatalografica}
%	\vspace*{\fill}					% Posição vertical
%	\hrule							% Linha horizontal
%	\begin{center}					% Minipage Centralizado
%	\begin{minipage}[c]{12.5cm}		% Largura
%
%	\imprimirautor
%
%	\hspace{0.5cm} \imprimirtitulo  / \imprimirautor. --
%	\imprimirlocal, \imprimirdata-
%
%	\hspace{0.5cm} \pageref{LastPage} p. : il. (algumas color.) ; 30 cm.\\
%
%	\hspace{0.5cm} \imprimirorientadorRotulo~\imprimirorientador\\
%
%	\hspace{0.5cm}
%	\parbox[t]{\textwidth}{\imprimirtipotrabalho~--~\imprimirinstituicao,
%	\imprimirdata.}\\
%
%	\hspace{0.5cm}
%		1. Palavra-chave1.
%		2. Palavra-chave2.
%		I. Orientador.
%		II. Universidade xxx.
%		III. Faculdade de xxx.
%		IV. Título\\
%
%	\hspace{8.75cm} CDU 02:141:005.7\\
%
%	\end{minipage}
%	\end{center}
%	\hrule
%\end{fichacatalografica}
% ---



% ---
% Dedicatória
% ---
% Cícero-INICIO
%\begin{dedicatoria}
%\vspace*{\fill}
%\OnehalfSpacing
%Dedico este trabalho...

%\lipsum[1]

%\vspace*{\fill}
%\end{dedicatoria}
% ---
% Cícero-fim
% ---
% Agradecimentos
% ---
% Cícero-INICIO
%\begin{agradecimentos}
%\vspace*{\fill}
%\OnehalfSpacing
%Gostaria de agradecer...

%\lipsum[2]

%\vspace*{\fill}
%\end{agradecimentos}
% Cícero-FIM
% ---

% ---
% Epígrafe
% ---
%\begin{epigrafe}
%    \vspace*{\fill}
%	\begin{flushright}
%		\textit{``Não vos amoldeis às estruturas deste mundo, \\
%		mas transformai-vos pela renovação da mente, \\
%		a fim de distinguir qual é a vontade de Deus: \\
%		o que é bom, o que Lhe é agradável, o que é perfeito.\\
%		(Bíblia Sagrada, Romanos 12, 2)}
%	\end{flushright}
%\end{epigrafe}
% ---


\begin{resumo}
\normalsize

A realidade atual do mercado de sistemas integrados de gestão que popularmente são conhecidos como \textit{Enterprise Resource Planning} ou pela sigla (ERP), demandam soluções dinâmicas, fáceis e eficazes. Ocorre que os fornecedores de tais sistemas, não conseguem atender as expectativas cada vez mais prementes de celeridade e interação exigidas pelo mercado. Apesar da demanda crescente por sistemas simples e ambientes intuitivos, atendendo assim as expectativas cotidianas cada vez mais simples dos usuários, há uma barreira na usabilidade destes sistemas, que ao longo dos anos se faz cada vez mais insuperável. Existe, portanto, uma demanda por sistemas ERP fáceis de usar com o melhor resultado possível, sendo que a característica da usabilidade ganha importância, especialmente, quando se trata da seleção de ERP, o que torna o tema da usabilidade dos sistemas ERP de grande relevância com importância indiscutível nos campos da pesquisa acadêmica e na prática dos contextos da seleção, implantação e utilização de tais sistemas. Essa tendência mundial é antiga com referências que datam de 2005, podendo, ainda, ser verificada nos dias atuais em vasto material sobre o tema, sendo que nos trabalhos atuais pode-se citar as contribuições de Parks (2012), Veneziano , et al (2014), Lambeck, et al (2014b), Corinna (2014), Babaian, Xu e Lucas (2014), Sadiq, Pirhonen (2017). Em 2014 um estudo amplo realizado com 208 usuários em pequenas e  médias empresas em dois países (Alemanha e Letônia), expôs o tema sobre a percepção da Usabilidade em populações diferentes. A hipótese principal é que diversas características regionais levam a uma avaliação diferente dos problemas de usabilidade em sistemas ERP, deste modo, comparamos a pesquisa efetuada na Alemanha com a pesquisa efetuada no Brasil, utilizando-se a metodologia survey como na pesquisa alemã. Verificamos muitos indicadores semelhantes e outros que divergiram entre a população dos dois países.


\vspace{\onelineskip}

\noindent
\textbf{Palavras-chave:} Usabilidade, \textit{Enterprise Resource Planning}, Sistema Integrado de Gestão, Interface do Usuário, Iteração Humano Computador
 %\imprimirpalavraschave
\end{resumo}

% resumo em inglês
% Cícero-INICIO
%\begin{resumo}[Abstract]
%Resumo da dissertação em inglês.

%\lipsum[4]

%\vspace{\onelineskip}

%\noindent
%\textbf{Keywords:} \imprimirkeyword   
%\end{resumo}
% Cícero-FIM

% ---
% inserir lista de ilustrações
% ---
% Cícero-INICIO
{\SingleSpacing
\pdfbookmark[0]{\listfigurename}{lof}
\listoffigures*
\cleardoublepage
}
% Cícero-FIM
% ---

% ---
% inserir lista de quadros
% ---
%{\SingleSpacing
%\pdfbookmark[0]{\listofquadrosname}{loq}
%\listofquadros*
%\cleardoublepage
%}
% ---

% ---
% inserir lista de tabelas
% ---
% Cícero-INICIO
{\SingleSpacing
\pdfbookmark[0]{\listtablename}{lot}
\listoftables*
\cleardoublepage
}
% ---
% Cícero-fim

% ---
% inserir lista de abreviaturas e siglas
% ---

\begin{siglas}
  \item[ERP] Enterprise Resource Planning
  \item[IPT] Instituto de Pesquisas Tecnológicas do Estado de São Paulo
  \item[MRP] Material Requirement Planning
  \item[MRP II] Manufacturing Resources Planning
  \item[SGI] Sistema de Gestão Integrado
  \item[SIG] Sistema Integrado de Gestão
\end{siglas}

% ---

% ---
% inserir lista de símbolos
% ---
%\begin{simbolos}
%  \item[$ \Gamma $] Letra grega Gama
%  \item[$ \Lambda $] Lambda
%  \item[$ \zeta $] Letra grega minúscula zeta
%  \item[$ \in $] Pertence
%\end{simbolos}
% ---

% ---
% inserir o sumario
% ---
% Cícero-INICIO
{\SingleSpacing
\pdfbookmark[0]{\contentsname}{toc}
\tableofcontents*
\cleardoublepage
}
% ---

% ----------------------------------------------------------
% ELEMENTOS TEXTUAIS
% ----------------------------------------------------------
\textual
\OnehalfSpacing

\setcounter{page}{3}
\chapter{Introdução} \label{Introdução}
%\setcounter{chapter}{3}
%"Exemplo de Citação direta" (LEYH; GEBHARDT; BERTON, 2017).
%Para  Leyh, Gebhardt e Berton (2017) este é um exemplo de  citação indireta.
O advento da \textit{internet} alterou todas as relações humanas e profissionais, desde o momento do seu surgimento e caminha ainda no sentido de tornar o mundo mais interativo, rápido e eficaz a cada dia. Porém, junto a esta evolução, com o decorrer do tempo verificou que não somente estas ferramentas precisavam ser mais abrangentes, como também se observou a necessidade de que fossem simples, já que o uso destas não se dava por uma elite mas sim em larga escala mundial, devendo atender a pessoas de todos os níveis, classes, e culturas distintas.\newline
\indent A partir daí, acentuou-se a tendência de simplificarem-se, ainda mais a comunicação e o modo de uso por meio da capacidade dos sistemas em oferecer ações e soluções para problemas e dificuldades em diferentes áreas. Colocar ao alcance de todos ferramentas que podem ser operadas de modo simples e inequívoco, foi uma tendência estudada por Giovanella (2014).\newline
\indent Com a popularização da \textit{internet}, os sistemas de informação ganharam força e se estenderam para além das fronteiras empresariais. Desta forma, retornar ao início da utilização maciça dos sistemas de informação é essencial para compreender a necessidade de otimização.\newline
\indent Se por um lado a evolução na área de sistemas de informação tornou especialmente simples a comunicação e a resolução de questões que passaram a ser abordadas, a partir da união de vários conhecimentos como algo corriqueiro e eficiente, por outro a necessidade de se unir diferentes usuários de diversos modos passou a ser uma condição básica para a evolução e otimização do uso dos sistemas, tais como sistemas conhecidos pelo termo inglês \textit{Enterprise Resource Planning} (ERP). \newline
\indent Compreendendo a importância deste tema, Giovanella (2014) nos ensina que, a troca de informações pela rede mundial de computadores torna a obtenção de informação e principalmente de respostas, uma estratégia rápida e simples, além de estar ao alcance de todos os interessados no tema a qualquer tempo. \newline
\indent Nesse sentido, os novos saberes na área surgiram não somente fornecendo material de pesquisa, mas também métodos próprios, oriundos da necessidade diária como suporte não oficial aos sistema ERP. É o caso, por exemplo, dos grupos de discussão em diversos canais como \textit{Whatsapp}, \textit{Skype}, \textit{Facebook}, \textit{Telegram}, entre outros, chamados neste trabalho de comunidades. \newline
\indent Pode-se dizer que essas comunidades foram criadas com o objetivo inicial de sanar problemas de uma forma mais rápida, por meio da democratização de informações e opiniões. Sendo assim, nessas comunidades todos os que se interessam pelo mesmo tema, podem debater um problema ou dúvida específica com a comunidade. \newline
\indent Uma justificativa para o uso de tais ferramentas, em detrimento a suporte oficial das empresas que provêm este serviço, pode ser verificado pelo trabalho de Giovanella (2014) onde se constata que o aprendizado dos indivíduos que participam de comunidades de prática, que são comunidades onde pessoas com experiência compartilham seu conhecimento prático com usuários iniciantes de maneira \textit{on-line}, é um facilitador do trabalho prático e simplifica o uso de ferramentas e programas. Deste modo, as pessoas com menos conhecimento, podem usufruir da consultoria dos colegas que compartilham experiências vividas e conhecimento, o que consequentemente agrega valor ao conhecimento dos usuários.\newline
\indent Nessas comunidades o conhecimento prático dos usuários e consultores, que passaram por situações semelhantes ou análogas, são compartilhados, possibilitando a identificação de soluções de modo mais rápido e eficaz do que com o auxílio dado pelo suporte das empresas fornecedoras da solução de ERP que, muitas vezes, não apresentam a atualização e conhecimento necessários para fornecer uma resposta rápida e indubitável para a dúvida apresentada.\newline
\indent Atuando desde 1997 com sistemas ERP e utilizando comunidades desde 2003 para troca de informações com usuários, o autor testemunhou uma mudança do sentido no fluxo da resolução de soluções, onde no início se dependia do fornecedor prover a solução, sendo que atualmente as soluções nascem nos grupos, Este comportamento que aqui chamamos de fenômeno autodidata, é um fenômeno que só existe possivelmente graças ao uso das tecnologias de comunicação inseridas em nosso meio nas últimas décadas.\newline
\indent Participando ativamente desse processo, o autor, testemunhou não só o fenômeno autodidata, mas também passou a perceber a importância da usabilidade na aceitação dos sistemas ERP por parte dos usuários. Uma vez que em um ambiente compartilhado e livre a comparação de recursos, as facilidades e possibilidades tornam-se naturais, observa-se que essas comunidades dão origem a novos e diversos questionamentos.\newline
\indent Os aspectos questionados são variados, mas podem ser encaixados em algumas linhas relacionadas à usabilidade de sistemas, tais como: desconhecimento do processo do funcionamento do sistema, telas confusas ou complicadas demais para o uso cotidiano, falta de documentação do sistema, dificuldade em obter suporte, erros graves que inviabilizam a execução dos processos pelos dos usuários e defasagem de informação.\newline
\indent O dia a dia dessas comunidades giram, em certo grau, em torno de perguntas, genericamente, separadas em: “Qual campo devo preencher para ...?”, “Como funciona ...?”, “Vocês já viram esse erro, quando se faz ...?”, “Como faço ....?”, “Alguém tem um Manual de ...?”, “Como tiro esse campo da tela?”, “Para que serve ....?” Esse tipo de questionamento revela aspectos deficitários mais recorrentes de usabilidade desses ERP’s.\newline
\indent Tais problemas de usabilidade em sistemas ERP’s, percebidos nestas comunidades, motivaram a realização de uma pesquisa exploratória mais ampla sobre o tema. Desse modo, após a realização de um levantamento de informações sobre o tema usabilidade em ERP’s, foi localizado um grupo de pesquisadores na Alemanha, cuja a área de concentração são os sistemas caracterizados como ERP’s. Dentre os diversos temas pesquisados destaca-se o estudo da usabilidade. \newline
\indent O trabalho desenvolvido por Lambeck, et al. (2014b) chama a atenção por buscar compreender as influências no uso e aceitação do ERP pelos usuários de diferentes países sendo que, nesta pesquisa são comparados os dados entre dois países Alemanha e Letônia. \newline
\indent Entender as diferenças de percepção entre os  usuários Brasileiros e Alemães, pode contribuir para melhorar o manejo destas ferramentas e compreender a necessidade de torna-las mais simples, para melhorar estes sistemas para as populações alvo deste estudo.

\section{Questões da Pesquisa e Objetivo} \label{Questões da Pesquisa e Objetivo}

Percebe-se no cotidiano das comunidades destinadas à discussão sobre ERP’s, a importância da usabilidade em tópicos que tratam de diversos temas, destacando-se entre eles, as instruções de operacionalização do ambiente, dúvidas na configuração e demora no atendimento por parte dos fornecedores. Também é visível o conflito entre o que os fornecedores querem desenvolver para a evolução do sistema e o que os usuários, realmente, desejam para executar suas tarefas cotidianas.\newline
\indent O mapeamento sistemático efetuado previamente a este trabalho mostrou que existem poucos estudos científicos sobre o tema. Tendo isso em vista e com o objetivo de expandir o conhecimento sobre o assunto, este estudo pretende responder os seguintes questionamentos:

	\begin{flushright}
	\begin{minipage}{.955\textwidth}
		I.	Quais aspectos de usabilidade são importantes para o usuário de sistemas ERP no Brasil? \newline
		Para responder a esta pergunta um questionário foi elaborado e aplicado na população de usuários de ERP no Brasil.
		
	\end{minipage}
	\end{flushright}

	\begin{flushright}
	\begin{minipage}{.955\textwidth}
		II. Qual a importância destes aspectos na percepção da usabilidade da população usuários da Alemanha x Brasil? \newline
		Por meio da comparação entre as respostas das populações de usuários de ERP no Brasil e da Alemanha foi possível responder à questão.
		
	\end{minipage}
	\end{flushright}

	\begin{flushright}
	\begin{minipage}{.955\textwidth}
		III. Existe algum aspecto que se destaca mais que os outros identificados? \newline
		Essa questão foi respondida por meio da análise da discrepância entre as respostas dadas no questionário.
		
	\end{minipage}
	\end{flushright}

As perguntas abaixo são secundárias, pois há um fator condicional para sua análise que é à diversidade das respostas a questão ”Região”:

	\begin{flushright}
	\begin{minipage}{.955\textwidth}
		IV.	Quais são aspectos mais importantes por região no Brasil? \newline
		A análise das respostas a essa pergunta fica condicionada à existência de uma grande diversidade regional.
		
	\end{minipage}
	\end{flushright}

	\begin{flushright}
	\begin{minipage}{.955\textwidth}
		V.	Os aspectos identificados na região sul (colonizada por europeus, principalmente, alemães e Italianos) estão mais próximos do resultado obtido na Alemanha? \newline
		A resposta a essa questão foi obtida a partir da análise das respostas da região sul do país uma vez que aquela região teve um fluxo migratório mais forte de alemães, em comparação com as demais regiões do país.
		
	\end{minipage}
	\end{flushright}


	\begin{flushright}
	\begin{minipage}{.955\textwidth}
		VI.	Tendo em vista que o perfil dos países pesquisados são semelhantes no que se refere as dimensões (Industrialização e PIB) como se dá a percepção da usabilidade? \newline
		Esta pergunta está vinculada a uma das conclusões do estudo comparativo sobre os usuários de ERP da Alemanha e da Letônia, que atribuiu as diferenças encontradas no aspecto da maturidade econômica daquelas nações. Supõe-se então, que ao comparar Alemanha e Brasil se está diminuindo, ou até eliminando, a influência desse aspecto no resultado da comparação.
		
	\end{minipage}
	\end{flushright}

O trabalho de Lambeck et al. (2014b) compara os dados obtidos na Alemanha e na Letônia, ainda que este estudo ressalte uma pequena distância entre os resultados obtidos uma análise mais objetiva mostra usuários de países com diferenças econômicas substanciais, embora tenham características culturais semelhantes: ambos são países com origem na cultura saxã. A Alemanha é o 5º maior PIB do mundo e detentora de uma economia muito industrializada, enquanto a Letônia é apenas o 41º nesse \textit{ranking} e sua economia, ainda, é pouco industrializada.\newline
\indent Ambos países estão, geograficamente, próximos e, historicamente, compartilham uma evolução histórica, relativamente, semelhante. As diferenças apresentadas são muito particulares, porém pequenas e os pesquisadores atribuíram essas diferenças à idade e à maturidade dos pesquisados na Letônia, em relação aos pesquisados na Alemanha.\newline
\indent A partir desse ponto e tomando esses trabalhos como base, o autor propôs uma comparação entre o resultado obtido na pesquisa Alemã, com o que se observa no Brasil. Em linhas gerais, como já foi dito, as diferenças identificadas entre os usuários da Alemanha e da Letônia foram atribuídas à idade e à maturidade dos pesquisados, porém neste trabalho se questionará, também, outros aspectos.\newline
\indent Inicialmente, verifica-se que Alemanha e Brasil não compartilham laços históricos ou proximidade geográfica, como ocorre entre Alemanha e Letônia. Logo, os componentes culturais também se distanciam. Entretanto, a Alemanha com seu substrato cultural saxão e o Brasil com influência latina, se aproximam em outros aspectos de relativa importância, uma vez que a Alemanha é o 5° maior PIB do mundo, enquanto o Brasil ocupa o nono lugar. No que se refere à industrialização a Alemanha é a 4ª nação mais industrializada do mundo, ao passo que o Brasil está, também, na nona posição, segundo a IEDI (2017).\newline
\indent Portanto, as principais diferenças entre o estudo original, que compara Alemanha e Letônia, com o estudo proposto neste trabalho são:\newline
\indent a)	a proximidade do aspecto econômico;\newline
\indent b)	a proximidade do aspecto industrial;\newline
\indent c)	os dois mercados geopoliticamente distintos;\newline
\indent d)	a influência cultural mais distinta e distante; e\newline
\indent e)	pouco vínculo comercial devido à distância.\newline
\indent Portando o objetivo é avaliar a percepção do usuário sobre o uso do ERP e identificar possíveis diferenças na percepção da usabilidade que influenciem na aceitabilidade do sistema por parte dos usuários entre populações heterogêneas sendo estas populações, a população brasileira (conforme a proposição deste trabalho) e a população alemã (pesquisa realizada). Faz-se necessário esclarecer que este estudo, como o trabalho citado, não se concentra sobre a necessidade de identificar os ERP’s utilizados, nem comparar seus recursos para saber qual seria considerado o mais eficaz. 

\section{Contribuição}  \label{Contribuição}

Verificar que a usabilidade em sistemas ERP, para as empresas que os fornecem, tem sido um desafio originado pela incapacidade atual do mercado de entender a grande contribuição que o tema pode trazer, tanto econômica como mercadologicamente. Este trabalho portanto dá continuidade a outros que compreenderam essa necessidade e ao longo dos últimos anos, buscaram suprir essa lacuna do conhecimento. Nesse sentido, destacam-se autores como Šūpulniece et al. (2013) e Lambeck et al. (2014b), que se dedicaram ao estudo do assunto.\newline
\indent A pesquisa conduzida por Šūpulniece et al. (2013) explora se os problemas de usabilidade em ERP’s tradicionais de projeção mundial são válidos para os sistemas locais que dominam as micro, pequenas e médias empresas na Letônia, sendo que em suas conclusões, o estudo indica a necessidade de se comparar os resultados daquele país com os dos demais países da Europa, para que se tivesse uma visão mais abrangente.\newline
\indent A pesquisa conduzida por Lambeck et al. (2014b), que teve como objetivo comparar a percepção dos usuários da Alemanha e da Letônia sobre a usabilidade dos sistemas de ERP, concluiu que existem poucas diferenças entre esses dois países e atribuiu essas diferenças à maturidade da população pesquisada na Alemanha em relação a Letônia.\newline
\indent O presente trabalho, por sua vez, focaliza realidades mercadológica e geograficamente distantes, porém próximas em vários aspectos econômicos e industriais e objetiva dar um outro enfoque aos dados colhidos nos trabalhos citados.\newline
\indent Também será possível analisar melhor quais seriam as soluções regionais para mercados diversos e qual seria a possibilidade de personalização dessas soluções, de acordo com a necessidade individual de cada mercado específico.\newline
\indent Com base nos estudos anteriores, seguem as contribuições pretendidas neste trabalho:
	\begin{flushright}
	\begin{minipage}{.955\textwidth}
		a) compreender quais aspectos influenciam na percepção da usabilidade por parte dos usuários de sistemas ERP’s;
	\end{minipage}
	\end{flushright}

	\begin{flushright}
	\begin{minipage}{.955\textwidth}
		b) colaborar com o aumento da compreensão da percepção da usabilidade sob o ponto de vista do usuário de sistemas ERP’s, possibilitando a ampliação do conhecimento a respeito da usabilidade;
	\end{minipage}
	\end{flushright}

	\begin{flushright}
	\begin{minipage}{.955\textwidth}
		c) ao final do estudo, fornecer insumos para que futuras pesquisas possam ser iniciadas a partir dos dados desta pesquisa, possibilitando a confirmação das evidências ou o confronto;
	\end{minipage}
	\end{flushright}

	\begin{flushright}
	\begin{minipage}{.955\textwidth}
		d) contribuir com o conhecimento sobre as áreas de atuação da usabilidade, além de promover o entendimento do que pode ou não ser caracterizado influência cultural;
	\end{minipage}
	\end{flushright}

	\begin{flushright}
	\begin{minipage}{.955\textwidth}
		e) gerar subsídios que auxiliem no processo de tomada de decisão sobre quais aspectos de usabilidade podem ser considerados e quais não podem, durante o desenvolvimento e aperfeiçoamento das interfaces do ERP e;
	\end{minipage}
	\end{flushright}

	\begin{flushright}
	\begin{minipage}{.955\textwidth}
		f) visto que a compreensão da usabilidade de um nicho tão específico possibilita, a redução de erros e alterações desnecessárias, causando uma relativa economia, além de promover uma melhora contínua do software ao longo do tempo, este trabalho pode propiciar ganhos financeiros e de qualidade às empresas do mercado.		
	\end{minipage}
	\end{flushright}

\section{Método de Trabalho}  \label{Método de Trabalho}

Para este estudo foi proposta uma abordagem de trabalho que tem como base a adaptação da pesquisa realizada por Lambeck et al. (2014b) e a aplicação do método \textit{survey} para a análise de dados. Para Freitas (2000), os métodos de pesquisa podem ser quantitativos (\textit{survey}, experimento, entre outros...) ou qualitativos (estudo de caso, \textit{focus group}, entre outros...), devendo essa escolha estar associada aos objetivos da pesquisa. O método de pesquisa baseado em survey é explicado no Capítulo 2.\newline
\indent A pesquisa será realizada nas comunidades de usuários dos ERP’s mais utilizados por micro, pequenas e médias empresas no Brasil. Segundo Meirelles (2017), os principais fornecedores desse nicho de mercado são as empresas Totvs, SAP e Oracle. Esta pesquisa atuará com usuários que colaboram nessas comunidades por meios modernos como \textit{WhatsApp}, \textit{Skype}, \textit{Facebook} e ou \textit{e-mails}. O trabalho foi desenvolvido seguindo as atividades descritas a seguir:\newline
\indent Mapeamento Sistemático: realização de um estudo bibliográfico sobre a usabilidade e sua aplicação em sistemas ERP’s. Essa atividade teve o objetivo de conceituar a usabilidade e descrever seu uso em ERP’s, ampliando o entendimento sobre o tema.\newline
\indent Pesquisa piloto: A pesquisa piloto foi realizada em um grupo de 10 pessoas, para testar a efetividade da ferramenta escolhida.\newline
\indent Pesquisa: após a avaliação da pesquisa piloto, foi divulgada a pesquisa. Mais detalhes sobre a pesquisa podem ser vistos no Capítulo 3.\newline
\indent Avaliação e comparação dos resultados. Por fim, foram avaliados os resultados e feita a comparação com a pesquisa realizada na Alemanha, com o objetivo de verificar
semelhanças e diferenças.

\section{Organização do Trabalho} \label{Organização do Trabalho}

Na Seção 2 - Referencial teórico, destacam-se os seguintes subtópicos: mapeamento sistemático, que apresenta, inicialmente, os resultados de um mapeamento executado com o objetivo de verificar o que se produziu sobre o tema usabilidade relacionado a sistemas ERP;\newline
\indent Apresentação da conceituação tradicional do termo ‘usabilidade’.\newline
\indent Conceituação ERP, usabilidade versus ERP, em que são relacionados os conceitos específicos de usabilidade em ERP e apresentada a Metodologia;\textit{Survey}.\newline
\indent Trabalhos relacionados: foram avaliados os trabalhos recentes pertinentes ao tema proposto.\newline
\indent A Seção 3 traz a abordagem proposta, descreve as formas de coleta das informações e análise dos dados.\newline
\indent Na Seção 4 encontram-se a análise de resultados, a identificação dos dados resultantes da pesquisa, a discussão dos resultados e as comparações com os resultados obtidos na Alemanha.\newline
\indent A Seção 5 apresenta os resultados atingidos.\newline
\indent Na Seção 6 está a conclusão, onde são apresentadas as considerações finais, as limitações da pesquisa e sugestões para futuros trabalhos.

\chapter{Referencial Teórico} \label{Referencial Teórico}

Este capítulo reúne informações obtidas sobre usabilidade em ERPs na literatura nacional e internacional. Inicialmente, documenta-se a pesquisa realizada sobre o estudo da usabilidade em sistemas ERP, na qual se optou por usar, como técnica de pesquisa, o mapeamento sistemático (MS), com o objetivo especifico de localizar os principais estudos realizados nos últimos anos. Após o MS abordam-se os conceitos de usabilidade e ERP e conceitua-se a usabilidade com o foco em sistemas ERP. As duas últimas subseções tratam da aplicação de \textit{survey} como metodologia de pesquisa e dos trabalhos relacionados. 


\section{Mapeamento Sistemático} \label{Mapeamento Sistemático}

Para Kitchenham e Charters (2007), o mapeamento sistemático (MS) é projetado para prover uma visão mais ampla de um tópico de pesquisa, de modo a estabelecer se há evidência de pesquisa nesse tópico.\newline
\indent Para Felizardo (2017), as revisões informais de literatura são conduzidas sem um planejamento estabelecido e caracterizam-se por serem pouco abrangentes, não passiveis de repetição, pouco confiáveis e com a qualidade dependente da experiência dos pesquisadores. Já a revisão sistemática (RS) é um método de pesquisa chave para apoiar a pesquisa baseada em evidências, desenvolvida com o objetivo de reunir, avaliar e interpretar as evidências literárias disponíveis, referentes a uma questão de pesquisa, tópico específico ou fenômeno de interesse.\newline
\indent Ainda segundo Kitchenham e Charters (2007), o MS mapeia um domínio em alto nível de granularidade. Permitindo, assim, identificar os agrupamentos e, dessa forma, direciona o foco para futuras revisões sistemáticas e distingue áreas para a condução de novos estudos primários.\newline
\indent Na Figura \ref{fig:figura-revisao-x-mapeamento} é possível verificar as diferenças entre revisão sistemática e mapeamento sistemático.

\begin{figure}[H]
	\centering	
	\caption{Comparativo entre Mapeamento Sistemático e Revisão Sistemática}
	\includegraphics[width=6in, height=2.7in]{frsms}
	\label{fig:figura-revisao-x-mapeamento}
\end{figure}
\vspace{-0.8 cm} \hspace{-0.5 cm} Fonte: Falbo (2017)\newline
\indent Sendo assim, o Mapeamento Sistemático é útil para alunos de mestrado e doutorado que precisam preparar uma visão geral do tópico de pesquisa sobre o qual vão trabalhar (KITCHENHAM; CHARTERS, 2007).\newline
\indent A partir dessa premissa, foi escolhido o Mapeamento Sistemático como ferramenta para a ampliação do domínio de pesquisa, que foi conduzido no mês de abril de 2018, a partir de um total de 140 abordagens retornadas (IEEE=124, IBICT=3, e Literatura Cinzenta=13). Os trabalhos repetidos 66 (sessenta e seis) foram excluídos. Dos trabalhos restantes, 74 (setenta e quatro) foram considerados elegíveis para a segunda fase do mapeamento, que consistiu da leitura e avaliação dos resumos desses artigos, culminando com o resumo elaborado pelo autor dos artigos escolhidos.\newline
\indent Os trabalhos que, pelos seus resumos, não satisfizeram os requisitos para esta pesquisa e os demais critérios definidos para inclusão e exclusão, também foram excluídos. Sobrando 7 pesquisas, que por atenderem apenas os critérios de inclusão, tiveram o seu conteúdo analisado por completo, para que, assim, pudessem compor a síntese da pesquisa.\newline
\indent Nos resumos comparativos dos trabalhos selecionados (seleção final) são apresentadas as principais diferenças e semelhanças identificadas nessas pesquisas.\newline
\indent A leitura das obras possibilitou a identificação de trabalhos relevantes para o objetivo do mapeamento, ou seja, permitiu encontrar as obras que contribuíssem para que as perguntas estabelecidas no foco de pesquisa fossem respondidas.\newline
\indent De forma geral, buscaram-se obras que relacionassem usabilidade com aceitação de sistemas ERP na pós-implantação ou estudos da usabilidade em ERP e sua influência na percepção do usuário. Os detalhes do Mapeamento Sistemático podem ser verificados no Apêndice B.

\section{Contexto}  \label{Contexto}

Para Banhos (2008), o contexto é um dos principais elementos da informação, pois uma informação pode ser importante, ter valor, fazer sentido para uma pessoa, e não causar nenhuma mudança em outra. Conforme nos ensina em uma definição clássica bastante referenciada na literatura Dey (2001), contexto se refere a qualquer informação que possa ser utilizada para caracterizar a situação de uma entidade. Sendo que, entidade pode ser uma pessoa, lugar ou objeto que é considerada relevante para uma interação entre um usuário e uma aplicação, incluindo o próprio usuário e a aplicação. Exemplos típicos de contexto são localização, identidade, uso, usuário.\newline
\indent O procedimento de especificar, controlar e relatar o contexto no qual a medição ocorre é rotineiro há séculos. Esse procedimento assegura que as medições sejam significativas e reprodutíveis. Maguire (2001)  relata que ele e outros descobriram em meados da décáda de 1980 que, embora muitos produtos tenham um bom desempenho em seus experimentos de laboratório, eles não funcionavam quando transferidos para o mundo real.\newline
\indent Bazire e Brézillon (2005), chegaram à conclusão que a definição de contexto depende da área à qual esta relacionada, não existindo assim uma definição absoluta, mas sim, definições que se adaptam às necessidades das áreas às quais são aplicados. Outra conclusão que os autores chegaram foi que o contexto atua como um conjunto de restrições que influenciam o comportamento de um sistema (um usuário ou um computador) envolvido numa data tarefa ou atividade.\newline
\indent O contexto de uso foi reconhecido no campo da usabilidade pela norma ISO 9241 (2010) publicada inicialmente em inglês no ano de 1997, referindo-se ao contexto de uso na definição de usabilidade:\newline

	{\raggedleft
	\hspace*{4.5cm} 
	\begin{minipage}{0.7\textwidth} 
		"Usabilidade é a medida em que um produto pode ser usado por usuários específicos para alcançar objetivos específicos com eficácia, eficiência e satisfação em um contexto de uso especifico."	
		\newline	
	\end{minipage}
	\newline	
	\par}

\indent A norma ISO 9241 (2010) ainda define o contexto de uso como: \newline

{\raggedleft
	\hspace*{4.5cm} 
	\begin{minipage}{0.7\textwidth} 
		"O Contexto de Uso consiste nos usuários, tarefas e equipamentos (hardware, software e materiais) e nos ambientes físico e social nos quais um produto é usado."	
		\newline	
	\end{minipage}
	\newline	
	\par}

\indent Contexto de Uso também é incorporado na norma ISO 13407 (1999) sobre design centrado no ser humano. Isso define o processo de compreender e especificar o Contexto de Uso como um dos principais estágios do processo de design centrado no ser humano.\newline
\indent Em complemento a norma, conforme nos ensina Maguire (2001) geralmente os sistemas são divididos entre aqueles que possuem usabilidade e aqueles que não possuem usabilidade. Na verdade, é incorreto indicar a usabilidade de um sistema, sem que antes seja descrito o seu contexto de uso.  Maguire (2001) ainda nos  indica que quando se esta desenvolvendo um sistema, este será usado em um contexto particular, porém há uma tendência nos desenvolvedores de esquecer o contexto de uso\newline
\indent Segundo Vieira et al. (2009), existem dois conceitos distintos: contexto e elemento contextual. Um elemento contextual é qualquer dado, informação ou conhecimento que permite caracterizar uma entidade em um dado domínio. O contexto é um conjunto de elementos contextuais instanciados que são necessários para apoiar a tarefa atual.\newline
\indent Bazire e Brézillon (2005), após realizarem uma pesquisa com mais de 150 individuos, chegaram a conclusão que contexto é dado por  6 parâmetros: restrição: indica a capacidade de restringir um significado de uma informação; influência: indica a capacidade da informação modificar o ambiente; comportamento: indica a capacidade alterar as ações do usuário diante da informação; natureza: indica a capacidade de influenciar a natureza da informação; estrutura: refere-se a estrutura na qual o contexto se apresenta e; sistema: refere-se a influencia do contexto sobre o sistema.\newline
\indent Schilit et al. (1994) aprofundando-se na teoria sobre o contexto de uso, identificaram 4 sub-divisões: o \textit{\textbf{contexto computacional}} que se refere a rede, conectividade, custo da comunicação, banda passante, recursos tais como impressoras, estações, entre outras; o \textit{\textbf{contexto do usuário}} se refere a atributos como, situação social, estado de espírito, satisfação, percepção, entre outras; o \textit{\textbf{contexto físico}} se refere à luminosidade, nível de ruído, temperatura, umidade e; o \textit{\textbf{contexto de tempo}} se refere à hora do dia, dia/mês/ano, semana ou época do ano.\newline
\indent Dentro do contexto do usuário estudando mais a fundo o atributo percepção, nos ensina Stabile (2001) que a percepção é afetada pelos 3 fatores principais:  Interpretação: capacidade de entendimento da informação que esta sendo exibida; Relevância: refere-se a percepção da importância da informação e; Confiabilidade: percepção da validade da informação conforme a fonte.\newline
\indent Segundo Costa (2010) a percepção de um sistema é formada por um conjunto de fatores que denotam qual o valor que um usuário obtém com o uso do sistema. Esta percepção afeta as decisões dos usuários que, muitas vezes, resistem ao uso do sistema quando não encontram satisfeitas suas necessidades.\newline
\indent Independente do tipo de usuário para Costa (2010), a percepção de usabilidade torna-se distinta caso uma tarefa em uma interface seja realizada esporadicamente ou de forma frequente, ou mesmo executada em computadores rápidos ou lentos.\newline
\indent Ao se explorar outra subdivisão do contexto do usuário a satisfação, verificamos a definição constante na ISO 9126-1 (2003) que o atributo satisfação indica a qualidade software percebido quando ele esta em uso. Verificamos ainda que, como um indicador a satisfação do usuário, é um dos atributos mais inportantes como indicador de sucesso de uma implantação.\newline
\indent Tomando as idéias de Maquire (2001) já citadas e fazendo um paralelo com os sistemas  ERP constata-se o quão dificil é avaliar a usabilidade em sistemas ERP’s, dada a gama de contextos de uso que existem.\newline
\indent Macêdo, Gaete e Joia (2014) ao estudar a resistência a implantação de sistemas ERP sob o contexto do usuário,  indicaram haver inseguranças atribuidas a três indicadores mapeados por eles:
	\begin{flushright}
	\begin{minipage}{.955\textwidth}
		Poder e Politica: este indicador refere-se a insegurança do usuário no que se refere a possivel mudança de poder na empresa e a reorganização estrutural da mesma após a implantação;
	\end{minipage}
	\end{flushright}

	\begin{flushright}
	\begin{minipage}{.955\textwidth}
		Interesse Pessoal: este indicador refere-se a percepção do usuário em relação a implantação do sistema, cruzando sua espectativa inicial versus a realidade nos aspectos da melhora dos processos e procedimentos;
	\end{minipage}
	\end{flushright}
	
	\begin{flushright}
	\begin{minipage}{.955\textwidth}
		Inclinação Pessoal: este indicador se refere a falta de intesesse dos usuários em aprender as novas tecnologias e processos.
	\end{minipage}
	\end{flushright}

Desta forma baseando-se Macêdo, Gaete e Joia (2014) verifica-se que os indicadores acima listados afetam a percepção do usuário e então entende-se, a restrição da pesquisa de Lambeck, et al. (2014b) a pós-implantação, pois decorridos 1 ano, as inseguranças do processo de implantação indicadas por Macêdo, Gaete e Joia (2014) já não mais existem, pois foram mitigadas ou realizadas ao logo do contexto do tempo e não afetam a percepção do usuário.  

\section{Usabilidade} \label{Usabilidade}

O termo usabilidade começou a ser usado na década de 1980, como um substituto da expressão \textit{user-friendly} traduzida para o português como amigável, sobretudo nas áreas de psicologia e ergonomia. O motivo dessa substituição está na constatação de que os usuários não precisavam que as máquinas fossem conceituadas como amigáveis, mas sim, que elas não interferissem nas tarefas que eles desejavam realizar. Mesmo porque, um sistema pode ser considerado amigável para um usuário e não tão amigável para outro, tendo em vista que as necessidades diferem de um usuário para outro. (GONÇALVES, 2009, apud DIAS, 2003)\newline
\indent Com a publicação da norma ISO 9126 (1991) o termo usabilidade passa a não se limitar a ergonomia e psicologia, mas começa a fazer parte de outras áreas de conhecimento.\newline
\indent A ISO 9126 (1991) conceitua o termo usabilidade da seguinte maneira:\newline

	{\raggedleft
	\hspace*{4.5cm} 
	\begin{minipage}{0.7\textwidth} 
		"conjunto de \textbf{atributos} que evidenciam o esforço necessário para se poder utilizar o software, bem como o julgamento individual desse uso, por um conjunto explícito ou implícito de usuários"	
		\newline	
	\end{minipage}
	\newline	
	\par}

\indent Sete anos depois ISO 9126 (1998), evolui a compreenção do conceito de usabilidade:\newline

	{\raggedleft
	\hspace*{4.5cm} 
	\begin{minipage}{0.7\textwidth} 
		"usabilidade é a \textbf{eficiência}, \textbf{eficácia} e \textbf{satisfação} com a qual os públicos do produto alcançam objetivos em um determinado ambiente"	
		\newline	
	\end{minipage}	
	\par}

\indent A ISO 13407 (1999), descreve a prática como o melhor do processo para projetos voltados ao Projeto Centrado no Usuário. Essa norma define quatro atividades que precisam estar presentes logo no início do projeto: compreender e especificar o contexto de uso; especificar os requisitos do usuário e os organizacionais; produzir soluções de projeto e avaliar projeto em relação aos requisitos. Essas atividades são organizadas, linearmente, formando um “ciclo de vida da usabilidade”, que pode ser visto na Figura \ref{fig:iso13407}. 
\begin{figure}[H]
	\centering	
	\caption{Atividades do Ciclo de Vida da Usabilidade}
	\includegraphics[width=6in, height=2.7in]{iso13407}
	\label{fig:iso13407}
\end{figure} %Fonte: Adaptado de ISO 13407 (1999)
\vspace{-0.8 cm} Fonte: Adaptado de ISO 13407 (1999)\newline
\indent A ISO 9241 (2002), descreve o termo usabilidade como:\newline

	{\raggedleft
	\hspace*{4.5cm} 
	\begin{minipage}{0.7\textwidth} 
		"\textbf{medida} na qual um produto pode ser usado por usuários específicos para alcançar objetivos específicos com eficácia, eficiência e satisfação em um contexto específico de uso"	
		\newline	
	\end{minipage}	
	\par}

\indent No ano seguinte a ISO 9126 (2003), define usabilidade como:\newline

	{\raggedleft
	\hspace*{4.5cm} 
	\begin{minipage}{0.7\textwidth} 
		"\textbf{capacidade} do produto de software de ser compreendido, aprendido, operado e atraente ao usuário, quando usado sob condições especificadas"	
		\newline	
	\end{minipage}
	\par}

\indent Como uma evolução dos estudos de usabilidade e com uma perspectiva de focar mais nos usuários do que no sistema ou produto, dez anos depois da sua criação, a  ISO 13407 (1999) foi revisada e renomeada para ISO 9241 (2010), tornando-se a norma padrão internacional de requisitos ergonômicos para o trabalho em terminais de visualização.\newline
\indent A ISO 9241 (2010) foi organizada em um conjunto de 17 partes, sendo que cada parte trata de um aspecto da utilização de um ambiente informatizado, a Parte 11, conceitua o termo usabilidade da seguinte maneira:\newline

{\raggedleft
	\hspace*{4.5cm} 
	\begin{minipage}{0.7\textwidth} 
		"\textbf{extensão} na qual um produto pode ser usado por usuários específicos para alcançar objetivos específicos com efetividade, eficiência e satisfação em um contexto de uso específico" 
		\newline	
	\end{minipage}
	\par}

\indent E ainda como:\newline

	{\raggedleft
	\hspace*{4.5cm} 
	\begin{minipage}{0.7\textwidth} 
		"\textbf{características} que permitem que o usuário alcance seus objetivos e satisfaça suas necessidades dentro de um contexto de utilização determinado" 
		\newline	
	\end{minipage}
	\par}

\indent A Parte 11 da ISO 9241 (2010) sugere as seguintes medidas para a usabilidade: efetividade, que permite que o utilizador atinja os objetivos iniciais de interação; eficiência, que refere-se à quantidade de esforço e recursos necessários para se chegar a um determinado objetivo e; satisfação, que é a mais difícil de medir e quantificar, pois está relacionada aos fatores subjetivos.\newline
\indent Nielsen (1993) e (2012), autor reconhecido pela sua contribuição para o entendimento da usabilidade, definiu a usabilidade como:\newline

	{\raggedleft
	\hspace*{4.5cm} 
	\begin{minipage}{0.7\textwidth} 
		"\textbf{medida} da qualidade da experiência do usuário ao interagir com alguma coisa - seja um site na Internet, um aplicativo de software tradicional ou outro dispositivo que o usuário possa operar de alguma forma" 
		\newline	
	\end{minipage}
	\par}

\indent Para Oliveira (2014), pode-se entender usabilidade como a característica que avalia a qualidade do uso do sistema, englobando itens como: facilidade do usuário em aprender a usá-lo, o reaprendizado após um período de tempo sem utilização, o quão agradável é a experiência com ele, a quantidade e severidade dos erros bem como a velocidade de realização das tarefas.\newline
\indent Já para Sadiq (2017), a usabilidade foi introduzida como um meio para entender a relação entre seres humanos e dispositivos técnicos. No entanto, quando o foco dos estudos humano-computador mudou gradualmente, para os produtos de consumo, o papel da usabilidade também se modificou. Em ambientes industriais, a usabilidade passou a ser considerada um meio para um desempenho preciso e efetivo.\newline
\indent Na área de produtos de consumo, esse tipo de critério tem valor somente se o consumidor o priorizar. O aspecto mais importante, nesse caso, não é a eficiência mensurável, mas a experiência geral de usar o aplicativo. Na comercialização de produtos de consumo, o principal objetivo é fazer com que o consumidor, efetivamente, consuma e não que execute uma determinada tarefa de forma eficaz.\newline
\indent Keinonen (1998) expõe o conceito de usabilidade e o apresenta em três dimensões:
\begin{flushright}
	\begin{minipage}{.96\textwidth}
		a - usabilidade como abordagem de projeto, consistindo em um conjunto de métodos ou abordagens de projeto, aí compreendidos a engenharia de usabilidade e o projeto centrado no usuário;\newline	
		b - usabilidade como atributo do produto sendo que, essa dimensão é levada em conta por meio da listagem de qualidades ou características que, se adicionadas a determinado produto, cooperam para a sua boa usabilidade;\newline
		c - usabilidade como a engenharia de usabilidade, considera o levantamento de medidas quantitativas para avaliar o potencial da usabilidade de um sistema.	
	\end{minipage}	
\end{flushright}

\newline \indent Outros autores definiram a usabilidade de forma diferente:\newline

	{\raggedleft
	\hspace*{4.5cm} 
	\begin{minipage}{0.7\textwidth} 
		“é a \textbf{capacidade} que um sistema interativo oferece a seu usuário, em determinado contexto de operação, para a realização de tarefas de maneira eficaz, eficiente e agradável” (BETIOL, 2007 apud CYBIS et al,  2002) \newline	
	\end{minipage}
	\par}

	{\raggedleft
	\hspace*{4.5cm} 
	\begin{minipage}{0.7\textwidth} 
		“é a \textbf{característica} que determina se o manuseio de um produto é fácil e rapidamente aprendido, com quantidade pequena de erros operacionais e oferecimento de um alto grau de satisfação, atingindo seus objetivos” (FERREIRA; LEITE, 2003) \newline	 	
	\end{minipage}
	\par}

	{\raggedleft
	\hspace*{4.5cm} 
	\begin{minipage}{0.7\textwidth} 
		“consiste em \textbf{propriedades} de interface de um sistema, no que se refere sua adequação ao usuário” (OLIVEIRA, 2008)\newline		 	
	\end{minipage}
	\par}

	{\raggedleft
	\hspace*{4.5cm} 
	\begin{minipage}{0.7\textwidth} 
		“é a \textbf{disciplina} que garante o uso eficiente e confortável dos sistemas computacionais” (BENSON; MULLER-PROVE; MZOUREK, 2004) \newline
	\end{minipage}
	\par}

	{\raggedleft
	\hspace*{4.5cm} 
	\begin{minipage}{0.7\textwidth} 
		“é uma \textbf{qualidade} de uso de um sistema, diretamente associada ao seu contexto operacional e aos diferentes tipos de usuários, tarefas, ambientes físicos e organizacionais. Pode-se dizer, então, que	qualquer alteração em um aspecto relevante do contexto de uso é capaz de alterar a usabilidade de um sistema” (OLIVEIRA, 2008 apud DIAS, 2003)	\newline	 	
	\end{minipage}
	\par}

	{\raggedleft
	\hspace*{4.5cm} 
	\begin{minipage}{0.7\textwidth} 
		“é o \textbf{parâmetro} que define até que ponto um produto de informação, um sistema de informação, um serviço de informação ou uma informação se prestam ao uso” (OLIVEIRA, 2008 apud LE COADIC, 2004) \newline 	
	\end{minipage}
	\par}

	{\raggedleft
	\hspace*{4.5cm} 
	\begin{minipage}{0.7\textwidth} 
		“é definida por 3 \textbf{aspectos}, qualidade do que é usável, característica do que é simples e fácil de usar e capacidade de um objeto, programa de computador, página da internet, etc.” (FERREIRA, 2014)\newline	
	\end{minipage}
	\par}

Nielsen (2012) ainda descreve cinco atributos da usabilidade, que são: facilidade de aprendizado, eficiência, facilidade de memorização, erros e satisfação.\newline
\indent Pelas definições acima e verificando as diversas técnicas que envolvem o estudo da Usabilidade permite-se afirmar que existe uma amplitude de seu escopo.\newline
\indent Conforme informam Yassien (2017 apud Nielsen e Gilutz, 2016), alguns estudos mostram que projetos de software devem gastar pelo menos 10\% de seu orçamento em usabilidade, para aumentar sua eficácia em 100\%. Dessa forma, pode-se dizer que a usabilidade é um item estratégico na construção de um sistema.\newline
\indent Pode-se observar que a usabilidade é um atributo complexo em qualquer produto de software, pois é afetado por algumas características intrínsecas ao próprio produto, bem como por algumas características de seus usuários. É, comumente, reconhecido que a usabilidade de um determinado produto (tanto subjetivamente percebido quanto medido em relação a escalas ou índices objetivos) aumenta para alguns usuários e se degrada para outros, devido às suas características cognitivas e pessoais. (BABAIAN et al, 2014).

\subsection{Engenharia de Usabilidade}  \label{Engenharia Usabilidade}

Mais recentemente, o termo engenharia foi associado à usabilidade. Para Costa e Ramalho (2010, apud Cybis, 2007, p. 17) a Engenharia de Usabilidade emerge como um esforço sistemático das empresas e organizações para desenvolver programas de software interativo com usabilidade. A Engenharia de Usabilidade abrange um conjunto  de três elementos fundamentais: métodos, ferramentas e procedimentos, que possibilita ao gerente o controle do processo de desenvolvimento do software e oferece ao profissional uma base para construção de software de alta qualidade e produtividade.\newline
\indent A Engenharia da Usabilidade compreende práticas, metodologias, modelos, ferramentas e processos dos quais se pode lançar mão para se obter um sistema interativo, centrado no usuário, com melhor usabilidade (MAYHEW, 1999).\newline
\indent O modelo proposto por Mayhew (1999) oferece uma visão holística acerca dessa engenharia e uma descrição detalhada de como podemos realizar os testes de usabilidade.\newline
\indent A primeira etapa do modelo é a análise dos requisitos. Aqui, Mayhew (1999) propõe quatro tipos de atividades de análise de requisitos:  Análise do perfil do usuário, análise do contexto da tarefa, análise das possibilidades e restrições da plataforma e análise dos princípios.\newline
\indent Depois na etapa de projetos, testes e implementação, Mayhew (1999) propôs que os ciclos devem se repetir de forma a tratar três níveis de aspectos de uma interface, sendo eles: O primeiro nível onde a interface é definida conceitualmente, o segundo, onde se fazem as definições em termos de estilo e o terceiro nível, as interações e componentes relacionados com os contextos das tarefas.\newline
\indent A última fase é a de instalação do sistema onde, depois que o usuário já tiver se habituado ao sistema, este poderá fornecer um \textit{feedback} sobre a usabilidade do produto de forma mais fidedigna por já ser um “especialista" da ferramenta.\newline
\indent O ciclo de  vida pode ser observado na Figura \ref{fig:cvum}.

\begin{figure}[H]
	\centering	
	\caption{Modelo de Ciclo de Vida da Usabilidade}
	\includegraphics[scale=1]{cvum}
	\label{fig:cvum}
\end{figure}
Fonte: Rebelo (2018)
\newline
Dentro da nossa proposta estamos localizados na ultima fase do modelo de Mayhew.

\subsection{Critérios de avaliação da usabilidade} \label{Critérios de avaliação da usabilidade}

As técnicas de avaliação nos proporcionam a capacidade de utilizar diferentes processos para verificar a usabilidade em suas diversas dimensões.\newline
\indent Segundo Shackel (1986), são quatro os componentes que devem ser considerados no contexto da avaliação da usabilidade: o usuário, a tarefa, o sistema e o ambiente. Assim, sob o ponto de vista da Usabilidade, são necessários quatro critérios para avaliação da interação do usuário com suas tarefas:\newline
\indent Eficácia, em que se avalia a capacidade do usuário de desempenhar a tarefa em um determinado ambiente. Por exemplo: a aferição da velocidade de execução e do levantamento número de erros. \newline
\indent Aprendizagem, pelo que se avalia o desempenho a partir da instalação do aplicativo, do inicio do seu uso, o tempo gasto em treinamento e o reaprendizado relativo ao uso frequente. \newline
\indent Flexibilidade, onde se avalia a adaptação do usuário às suas tarefas além das previamente especificadas pelo desenvolvedor. \newline
\indent Atitude, pelo que se avaliam o desempenho relacionado ao usuário, seu conforto ou satisfação, tais como as condições aceitáveis de frustração, desconforto, fadiga, esforço e desgaste pessoal.\newline
\indent Para Nielsen (1993) o modo mais recorrente de se avaliar a usabilidade de um software é pela observação da sua capacidade de interação com o usuário, p odendo essa observação ser realizada em um laboratório, com uma quantidade controlada de usuários para o qual o sistema foi desenvolvido, ou mesmo no próprio ambiente de trabalho do usuário onde o sistema está instalado. O que importa no processo de avaliação é que, sempre quando for possível, deve-se utilizar o tipo de usuário adequado para a realização da tarefa, no sentido de se possibilitar a melhor avaliação.\newline
\indent Para Nielsen (1993) a avaliação da usabilidade deve ser realizada sob a ótica de cinco atributos:\newline
\indent − Facilidade de aprender, o sistema deve ser fácil para o usuário aprender, para que possa realizar sua tarefa com agilidade. A interface com o usuário deve ser prática e clara. \newline
\indent − Eficiência de uso, o sistema deve permitir a eficiência na execução da tarefa a ser realizada, para que com isto o usuário amplie seu nível de produtividade. 
− Memorização: As funcionalidades que o sistema possui devem estar de forma que facilite sua memorização pelo usuário, ainda que fique um determinado período sem utilizá-lo, porém sem ocorra a necessidade de mais treinamento. \newline
\indent − Poucos erros, o sistema deve favorecer a redução do número de erros, e caso estes ocorram, o usuário deve poder resolvê-los ou ignorá-los de modo simples e rápido.\newline
\indent - Satisfação, é uma percepção totalmente subjetiva em que o usuário, diante da interface do sistema, transmite a impressão de estar satisfeito e apreciar do seu uso.\newline
\indent Quesenbery (2001) aponta cinco características da usabilidade, também conhecidas como 5E’s (effective, efficient, engaging, error tolerant, easy to learn). Em sua abordagem a usabilidade da interface deve ser analisada pelo confronto dessas características, de modo a permitir a satisfação e o sucesso do usuário. \newline
\indent Eficiência, verifica o tempo gasto total para realização de uma determinada tarefa. \newline
\indent Eficácia, inspeciona se as tarefas foram concluídas conforme seu planejamento, e com qual freqüência produzem erros. \newline
\indent Atração, verifica a satisfação ou conforto do usuário em utilizar o sistema. Focaliza medir sua aceitação ou rejeição em relação ao sistema. \newline
\indent Tolerância a erros, verifica a incidência de erros gerados pelo sistema. Pressupõe-se que tais erros sejam solucionados facilmente e apresentados de forma clara ao usuário.\newline
\indent Facilidade de aprender, verifica a facilidade de uso entre os usuários em diversos níveis de experiência com o sistema. O usuário deve concluir a tarefa com o mínimo de assistência ou ajuda necessária.

\section{Enterprise Resource Planning (ERP)}  \label{Enterprise Resource Planning (ERP)}

Por definição, quando dois ou mais sistemas de gestão se unem, de forma a perder a independência de um deles ou de ambos sem, contudo, deixar suas particularidades de fora, esse movimento consiste no estabelecimento de um Sistema Integrado de Gestão (SIG) ou Sistema de Gestão Integrado (SGI) (KARAPETROVIC; WILLBORN, 1998, apud INÁCIO, 2017, p. 25-26).\newline
\indent A termo \textit{Enterprise Resource Planning} e a sigla ERP, foram cunhados por uma empresa americana de pesquisa a \textit{Gartner Group} e, rapidamente, assimilados no meio comercial, substituindo as siglas SIG/SGI.\newline
\indent A intenção foi definir esses sistemas integrados como uma evolução dos sistemas \textit{Manufacturing Resources Planning} (MRP II). Para Souza (2000) os sistemas ERP surgiram da necessidade de rápido desenvolvimento dos SIG/SGI, as empresas fornecedoras utilizaram-se de modelos de processos obtidos de estudo e comparações em diversas empresas por meio de \textit{benchmarking}, as chamadas melhores práticas. Esse conhecimento foi agregado à empresa no processo de implantação. As melhores práticas, em associação à integração dos departamentos, podem permitir reduções de mão de obra indireta, principalmente, nos setores administrativos da empresa.\newline
\indent Os sistemas ERP podem ser definidos como sistemas de informação integrados, adquiridos na forma de um pacote de software comercial, com a finalidade de dar suporte à maioria das operações de uma empresa. São, geralmente, divididos em módulos que se comunicam e atualizam uma mesma base de dados central, de modo que informações alimentadas em um módulo são, instantaneamente, disponibilizadas para os demais módulos com dependência. (SOUZA, 2000).\newline
\indent Embora os conceitos utilizados em sistemas ERP possam ser usados por empresas que queiram desenvolver, internamente, os seus aplicativos, o termo ERP refere-se, essencialmente, a pacotes comprados. Exemplos de sistemas ERP existentes no mercado são: (SAP S/4HANA, SAP S/4 HANA Cloud e SAP ERP), da alemã SAP, (Protheus, RM, Datasul e Logix), da brasileira Totvs e o Oracle ERP Cloud da americana Oracle Inc.\newline
\indent Conforme Kovalczyk e Kovalczyk (2017) empregar um sistema de informação apropriado, apto a reduzir os contratempos sistemáticos verificados em sistemas arcaicos e que proporcionem a inserção de um novo protótipo de gestão empresarial fundamentado na gestão constituída, é um dos grandes propósitos vigentes nas instituições. O autor ainda diz que o uso de ERP é indispensável na maior parte das organizações. Entretanto, a dimensão do quanto a solução será necessária e a integralidade que a empresa contratada irá proporcionar, são de extrema importância na questão referente ao custo x benefício.

\section{ERP x Usabilidade} \label{ERP x Usabilidade}

As pesquisas do setor reconhecem que os sistemas ERP estão cheios de problemas de usabilidade, mas há uma escassez de pesquisas sobre meios para melhorar a experiência do usuário (BABAIAN et al, 2014), conforme investigação efetuada no Mapeamento Sistemático os aspectos específicos de usabilidade da interface não são, amplamente, discutidos no campo dos ERPs. Dos artigos encontrados, poucos investigaram fatores importantes de interface do usuário (IU), como navegação, orientação do usuário, fatores visuais, carga mínima de memória e capacidade de aprendizado. Essa afirmação é corroborada também por Lamberck et al. (2014a).\newline
\indent Com o incremento de complexidade dos sistemas ERP, os profissionais de desenvolvimento já encaram a usabilidade como um conhecimento indispensável e, de certa forma, indissociável da prática profissional. Entender as dimensões da usabilidade e seu impacto no produto final colabora para o aumento da efetividade e da eficiência.\newline
\indent No dia a dia do suporte às ferramentas de ERP surgem situações em que se percebe que a usabilidade de um sistema tem uma alta variação de situações (erros no sistemas, erros operacionais, dúvidas e etc.), isso se deve a não homogeneidade dos usuários, que vão de funcionários desmotivados ou inexperientes a funcionários altamente exigentes ou experientes.\newline
\indent Percebe-se na Figura \ref{fig:vdpu} uma situação inusitada, relatada por Dums (2015), em que um usuário dialoga com um profissional de TI. Nessa situação o sistema apresenta a funcionalidade de atualização da tela pela tecla F5 e o usuário quer que exista uma duplicidade do mesmo recurso por meio de um botão “Atualizar”. Essa situação inusitada demonstra que a percepção da usabilidade se torna algo, extremamente, pessoal e é afetada por diversos fatores externos.\newline
\indent Segundo Lambeck et al. (2014b), os sistemas ERP sofrem de inúmeros problemas de usabilidade em geral, conforme avaliação histórica. Isso é inevitável, pois esses sistemas se caracterizam por processos complexos que os ERPs precisam implementar e suportar. No entanto, há outros motivos e fatores que, tradicionalmente, afetam a usabilidade de tais sistemas. Para Veneziano (2014), o termo usabilidade não é, frequentemente, associado a sistemas ERP, nem é uma das características principais consideradas, comercialmente. Em vez disso, esses sistemas são, tipicamente, complexos e frustrantes de usar.\newline
\indent Um exemplo dessa percepção pode ser visto na Figura \ref{fig:vdpu}, essa é uma das diversas charges publicadas, semanalmente, que são elaboradas a partir relatos de profissionais do meio de suporte e programação, nelas o personagem ”O Programador” se depara com situações diversas de suporte, desenvolvimento ou a iteração com seus superiores, que demonstram o despreparo tanto do usuário, como do próprio profissional, para percepção da usabilidade do sistema.

%\vspace{0.5 cm}
\begin{figure}[H]
	\centering	
	\caption{Charge Vida de Programador - Botão “Atualizar” x Tecla F5}
	\includegraphics[scale=1]{vdpu}
	\label{fig:vdpu}
\end{figure}
\vspace{-0.8 cm} \hspace{1.85 cm} Fonte: Dums (2015)\newline
%\newline cxx \newline
\indent Para Parks (2012), no que diz respeito à interface do usuário, a menor satisfação do usuário é causada, pelo menos parcialmente, pela implementação de processos de negócios complexos e na concepção de interfaces de usuários gerais que são direcionadas para várias indústrias, simultaneamente.\newline
\indent A importância de interfaces empresariais bem concebidas é destacada por Parks (2012), uma vez que dados, incorretamente, codificados podem diminuir, significativamente, o desempenho da empresa (por exemplo, metas de produção não realizadas ou pedidos incorretos).\newline
\indent Os estudos conduzidos por Babaian et al. (2014); Lambeck et al. (2014b) revelam várias fontes de confusão e frustração dos usuários em geral, sendo que essas fontes são repetidas em trabalhos referenciados pelos autores, o que denota um certo descaso do tema ’usabilidade’ pelas empresas que criam e mantém esses sistemas.\newline
\indent Essa realidade pode ser sentida no convívio cotidiano em grupos de colaboração que se formam, espontaneamente, em plataformas de colaboração como \textit{Skype}, \textit{Whatsapp} e \textit{Telegram}. A aplicação da pesquisa pretendida nesses grupos poderá começar a desvendar o que esses usuários, realmente, querem e qual a sua percepção em relação aos sistemas ERP.\newline
\indent Conforme Babaian et al. (2014) os seguintes princípios de projeto deveriam ser observados nos sistemas ERPs, em geral:

%{\raggedleft
\begin{flushright}
	\begin{minipage}{.96\textwidth}
		1.	a interface do usuário deve fornecer um mecanismo para personalizar o vocabulário dos termos usados pelo sistema em sua comunicação com o usuário, a composição das transações comerciais e o conteúdo das saídas do sistema para corresponder às práticas da organização. Deve haver um mecanismo para incorporar as personalizações de uma versão anterior do sistema a uma versão posterior.
	\end{minipage}
\end{flushright}
%	\newline	
%\par}

%{\raggedleft
\begin{flushright}
	\begin{minipage}{.96\textwidth}
		2.	O sistema deve fornecer orientação de navegação e progresso para um usuário que executa uma transação, indicando o contexto mais amplo de cada interação em termos dos componentes de processos de negócios relacionados e, especificando, as partes concluídas e restantes. Um usuário, suficientemente, competente deve ser capaz de desativar essa orientação se ela se tornar uma distração.		
	\end{minipage}
\end{flushright}
%	\newline	
%\par}


%{\raggedleft
\begin{flushright}
	\begin{minipage}{.96\textwidth}
		3.	Quando o sistema detecta um problema, ele deve identificar as possíveis causas e formas de resolvê-lo. Se a correção for óbvia, o sistema deve informar ao usuário e executar a ação. Se não for, as possíveis causas e cenários de resolução devem ser apresentadas ao usuário possibilitando, prontamente, a correção. Se o sistema não conseguir identificar estratégias de resolução, ele deverá apresentar ao usuário os dados e transações relevantes.		
	\end{minipage}
\end{flushright}
%	\newline	
%\par}


%{\raggedleft
\begin{flushright}
	\begin{minipage}{.96\textwidth}
		4.	Ao apresentar opções de seleção, o sistema deve utilizar o que sabe sobre o usuário, a organização, a tarefa e o contexto e fornecer acesso mais rápido às opções mais prováveis do que as menos prováveis. Onde a escolha de dados ou ação é óbvia, o sistema deve ter a opção de não esperar que o usuário o autorize. O usuário deve ter uma opção para substituir/cancelar a escolha de dados/ação fornecida pelo sistema.		
	\end{minipage}
\end{flushright}
%	\newline	
%\par}

Para Melo (2015 apud Singh, Wesson 2009), por, ainda, não existir uma forma padronizada para determinar a usabilidade em um sistema ERP, os limitados estudos publicados apontam que os problemas de usabilidade encontrados nesses sistemas foram identificados pela comunhão de vários critérios. Entre tais problemas podem ser citados:\newline

%{\raggedleft
\begin{flushright}
	\begin{minipage}{.96\textwidth}
	1.	Complexidade para encontrar funcionalidades;\newline	
	2.	Falta de orientação ao usuário de como concluir precisamente a sua tarefa; 	\newline
	3.	Inexistência de recursos de personalização da interface para apoiar as ações do usuário;\newline 	
	4.	Ineficiência na recuperação de dados e informações acessados frequentemente;\newline 	
	5.	Complexidade na compreensão dos leiautes das telas;	\newline
	6.	Dificuldade na interpretação das saídas do sistema; e	\newline
	7.	Dificuldade em lembrar como operar com as diferentes partes do sistema.
	\end{minipage}
\end{flushright}
%	\newline	
%\par}

Melo (2015 apud Singh, Wesson 2009) usa um conjunto de 5 heurísticas de usabilidade específicas para sistemas ERP:
%{\raggedleft
\begin{flushright}
	\begin{minipage}{.96\textwidth}
	a) navegação: navegação e acesso a informação.\newline
	b) apresentação: apresentação da tela e da saída.\newline
	c) suporte à tarefa: suporte apropriado à tarefa.\newline
	d) aprendizado: grau de facilidade para aprender como usar o sistema efetivamente.\newline
	e) customização: facilidade de customização do sistema para o alinhamento entre o mesmo, o usuário e os processos de negócio.		
	\end{minipage}
\end{flushright}
%	\newline	
%\par}

Segundo Melo (2015 apud Singh, Wesson 2009), devido às 10 heurísticas de Nielsen terem como objetivo a avaliação da usabilidade geral e não serem direcionadas às características de usabilidade que são próprias de sistemas ERP, outras heurísticas são utilizadas no intuito de cobrir aspectos de usabilidade que não são considerados por Nielsen.\newline
\indent Melo (2015) cita os resultados de um estudo de Singh \& Wesson, citado anteriormente, onde usou-se a heurística de Nielsen, juntamente com a heurística dos citados, com a ajuda de três especialistas em usabilidade, que verificaram a capacidade do seu conjunto de heurísticas e das 10 heurísticas de Nielsen em identificar, conjuntamente, os possíveis problemas de usabilidade do sistema ERP.\newline
\indent Os problemas de usabilidade detectados com as 10 heurísticas de Nielsen, revelam a necessidade de ampliar a capacidade de prevenção e recuperação de erros e, com as heurísticas específicas para sistemas ERP, foi constatado como potenciais problemas como a dificuldade de encontrar informações em funcionalidades do sistema e a falta de orientação por parte do sistema para auxiliar os usuários a completarem as suas tarefas.\newline
\indent Nesse sentido, Veneziano (2014), considera a percepção da usabilidade por parte do usuário, como um dos pontos centrais no que tange ao processo de consumo dos ERPs. Considera, ainda, a existência de uma relação quanto à percepção do usuário vinculado ao nível de eficiência para atingir determinado objetivo. Quanto menor essa percepção, menor o índice de satisfação e, consequentemente, de sucesso em suas operações junto ao sistema.\newline
\indent Para o trabalho em questão nos basearemos nas heurísticas de Nielsen (1993) e nas 5 heurísticas de ERP introduzidas por Singh \& Wesson, já citados neste trabalho e utilizados por Lambeck et al (2014a). 

\section{\textit{Survey}} \label{Survey}

\textit{Survey's} são investigações que colhem dados de uma amostra representativa de uma população específica, que são descritos e, analiticamente, explicados. Pretende-se que os resultados sejam generalizáveis ao universo dessa população, evitando-se realizar o censo, ou seja, ouvir todos os indivíduos, o que é, geralmente, impossível, por questão de custo e de tempo (BABBIE, 2005).\newline
\indent Segundo Freitas et al. (2000), a \textit{survey} é apropriada como método de pesquisa quando:

%{\raggedleft
\begin{flushright}
	\begin{minipage}{.96\textwidth}
		a) se deseja responder questões do tipo ”o quê?”, ”por que?”, ”como?” e ”quanto?” Ou seja, quando o foco de interesse é sobre ”o que está acontecendo ”ou” como e por que isso está acontecendo”;\newline		
		b) não se tem interesse ou não é possível controlar as variáveis dependentes e independentes;\newline
		c) o ambiente natural é a melhor situação para estudar o fenômeno de Interesse; \newline
		d) o objeto de interesse ocorre no presente ou no passado recente;
	\end{minipage}
\end{flushright}
%	\newline	
%\par}

\indent Para Freitas et al. (2000 apud Pinsonneault; Kraemer, 1993), um questionário é a forma mais, corriqueiramente, utilizada como instrumento de obtenção de dados em uma pesquisa cujo o método utilizado é o \textit{survey}.

\section{Trabalhos Relacionados} \label{Trabalhos Relacionados}

Os principais trabalhos relacionados a esta pesquisa foram conduzidos em Dresden entre 2013 e 2017, localizou-se também um interessante trabalho nos Estados Unidos de Veneziano et al. (2014), a seguir, são analisadas as características principais.\newline
\indent Šūpulniece et al. (2013), investiga se os problemas de usabilidade em ERPs tradicionais de projeção mundial são válidos para os sistemas locais que dominam as micro, pequenas e médias empresas na Letônia, sendo que em suas conclusões de estudos indicam a necessidade de comparar esses resultados com os demais países da Europa, para que se tenha uma visão mais abrangente.\newline
\indent Já Lambeck et al. (2014a), cita que os estudos nos últimos 20 anos focaram apenas em ERPs individuais em filiais específicas com pequenos grupos de usuários, o estudo em face usa uma ampla amostra de 184 usuários, distribuídos em pequenas e médias empresas.\newline
\indent A pesquisa de Lambeck et al. (2014a) é uma revisitação de uma pesquisa feita em 2005, por esse motivo a pesquisa tem dois objetivos: primeiramente, avaliar se os problemas de usabilidade identificados em 2005 se repetem quase 10 anos depois e ampliar o foco da pesquisa para considerações adicionais, como o papel do tipo de menu, a incerteza no uso do sistema ou o suporte em situações problemáticas.\newline
\indent Lambeck et al. (2014a), avaliam que os problemas encontrados em pesquisas anteriores ainda existem na atualidade., porém, afirmam que esses problemas são menos críticos, assim, concluem que ainda há alguns esforços necessários para alcançar a visão de uma interface ERP fácil de usar.\newline
\indent Veneziano et al. (2014), buscam estabelecer a relação da influência das informações demográficas dos usuários (por exemplo: antecedentes educacionais e experiências de trabalho) sob a avaliação da usabilidade. O estudo concluiu que a existe uma relativa influência do grau de formação educacional na percepção da usabilidade, porém elenca como trabalhos futuros o aprofundamento desse estudo em outras populações.\newline
\indent Lambeck et al. (2014b), baseando-se na pesquisa de Šūpulniece et al. (2013), apresentam uma pesquisa realizada com, aproximadamente, 200 usuários de ERP na Alemanha e na Letônia, sendo que os resultados indicaram que ambos os países têm vários contrastes, mas também pontos comuns diversos na indústria, mercado de ERP e cultura. No entanto, os usuários em ambos os países são muito homogêneos em relação à avaliação de problemas de usabilidade em suas interfaces ERP.\newline
\indent O artigo de Lambeck et al. (2014b) investiga problemas de usabilidade elementares derivados do trabalho relacionado e examina em que medida eles são válidos em ambos os países. A principal hipótese levantada pelo artigo afirma que diversas características nacionais não conduzem, necessariamente, a uma avaliação diferente dos problemas de usabilidade nos sistemas ERP.

\chapter{Execução da Pesquisa} \label{Execução da Pesquisa}

Este capítulo é dividido em elaboração e estruturação do questionário, que trata das características do questionário que será aplicado, meio de distribuição da pesquisas e amostra e as características das amostras de dados que, se espera conseguir com a aplicação da pesquisa.\newline 
\indent A pesquisa que terá o prazo de 8 semanas, serão distribuídos os links para 2500 pessoas e espera-se um retorno de 10 \%, ou seja, 250 questionários respondidos por completo. O perfil dos participantes se resume a duas variáveis: eles são usuários de sistemas ERP que usam o ERP há no mínimo um ano.\newline 
\indent Usando a Sinhorini (2017), como referência estima-se a seguinte meta de respostas por região: sudeste (60\%), sul (18\%), centro-oeste e nordeste (20\%), norte(2\%).

\section{Elaboração e Estruturação do Questionário} \label{Elaboração e Estruturação do Questionário}

O questionário da \textit{survey} desta pesquisa foi utilizado em pesquisa que deu origem aos artigos Lambeck et al. (2014a) e Lambeck et al. (2014b), cedidos pelos autores, inclusive com os dados que foram colhidos, o que possibilitará a comparação dos resultados. Conforme informaram os autores da pesquisa original, a preocupação durante a criação do questionário foi minimizar o número de perguntas e estruturá-las de forma a poupar esforço e tempo dos respondentes durante a coleta de dados, sendo que o tempo de resposta máximo estipulado foi de 15 minutos.\newline 
\indent A escala definida para o registro de algumas variáveis foi: “Concordo Completamente”, “Concordo Parcialmente”, “Não Concordo, Nem Discordo”, “Discordo Parcialmente”, ”Discordo Completamente” e “Eu Não Sei”. A decisão de formatar a pergunta dessa forma deveu-se ao fato de se poder inferir mais que uma questão em uma única pergunta. O questionário foi dividido em seis partes e suas seções são descritas abaixo, o questionário completo, pode ser consultado no Apêndice A deste documento.\newline 
\newline
\noindent \textbf{Parte 1: Ambiente Empresarial} \newline

\indent Esta Seção tem o objetivo de posicionar a empresa no mercado em que atua, sendo solicitadas informações a respeito do número de funcionários, setor de atuação, área de influência e qual a posição hierárquica do pesquisado dentro da empresa.\newline 
\newline
\noindent \textbf{Parte 2: Informações sobre o sistema ERP} \newline

\indent O objetivo desta Seção é saber se o pesquisado usa o ERP, plenamente, sem o auxílio de ferramentas ou sistemas auxiliares, se usa o ERP em conjunto de ferramentas ou sistemas auxiliares ou se não usa o ERP e faz uso de ferramentas ou outros sistemas para executar seu trabalho. Esta seção possibilitará avaliar quais ERPs são utilizados pelo público pesquisado.\newline 
\newline
\noindent \textbf{Parte 3: Usabilidade} \newline

\indent Nesta Seção, são feitas questões sobre a usabilidade do sistema ERP, com o objetivo de avaliar a capacidade crítica do usuário em relação ao seu sistema ERP e das ferramentas ou softwares adicionais, sendo avaliadas a percepção do usuário acerca dos recursos disponibilizados nas ferramentas, formato do menu, acesso ao conhecimento dos processos e operações na ferramenta, acesso a funcionalidades e facilitadores. Esta seção contribuirá com a avaliação da percepção do usuário relativa à usabilidade do sistema ERP.\newline 
\newline
\noindent \textbf{Parte 4: Acesso ao Sistema} \newline

\indent Nesta Seção são avaliadas as formas de acesso ao ERP e se o ERP é compatível para utilização em dispositivos móveis. A utilidade dessas informações será avaliar a percepção do usuário relativa à forma de acesso ao sistema ERP.\newline 
\newline
\noindent \textbf{Parte 5: Dados Funcionais} \newline

\indent Composta de oito perguntas que ajudam a caracterizar o perfil do respondente e garantir que sua participação na pesquisa agregue valor aos resultados coletados. A anonimidade dos dados foi garantida na apresentação da pesquisa. Foram definidas questões para identificar o nível de experiência do respondente em relação ao uso de sistemas ERP. \newline 
\indent Essas informações são úteis na fase de análise para identificar correlações entre determinados grupos e suas posturas em relação à percepção do uso dos sistemas ERP. Em relação à pesquisa original foi incluída uma questão sobre a região do país onde o pesquisado trabalha. Essa pergunta adicionada teve o objetivo de possibilitar e considerar a diferenciação da influência cultural por região em nosso país.\newline 
\newline
\noindent \textbf{Parte 6: Dados de Contato} \newline

\indent Para motivar os participantes foi oferecido o envio dos dados resumidos da pesquisa, por e-mail ou cópia impressa, da dissertação para aqueles que fornecerem seus dados de contato.

\section{Meio de Distribuição} \label{Meio de Distribuição}

A construção do questionário é feita no \textit{Lime Survey}, hospedado em um \textit{site}, incluindo uma apresentação inicial, contendo o objetivo da pesquisa, o foco do trabalho, a garantia de anonimato dos dados coletados e informações para contato em caso de necessidade de maiores esclarecimentos. As perguntas foram configuradas para que não fosse possível o envio de respostas em branco, a não ser em campos adicionais de texto livre, disponíveis para os respondentes que sintam a necessidade de conceder maiores informações sobre suas respostas.\newline
\indent Também foram incluídos campos de nome e \textit{e-mail}, em uma tentativa de validar as respostas dos pesquisados, garantindo uma resposta por participante. O questionário proposto encontra-se no apêndice A da pesquisa.\newline

\section{Amostra} \label{Amostra}

Para Freitas et al. (2000), nenhuma amostra é perfeita, podendo haver variação no erro ou no viés. Para mitigar o risco em uma amostra alguns aspectos devem ser, fortemente, considerados como ter, claramente, definido o objetivo que se tem com a realização da pesquisa, o que dará melhores condições de assegurar que a amostra será adequada ou não; definir, objetivamente, os critérios de elegibilidade dos respondentes, ou seja, quais as condições definem se uma pessoa pode ou não participar da pesquisa.\newline
\indent Uma amostra pode ser caracterizada como probabilística e não probabilística. A principal característica de uma amostra probabilística é o fato de todos os elementos da população terem a mesma chance de serem escolhidos, resultando em uma amostra representativa da população. Uma amostra probabilística pode ser classificada em estratificada ou não estratificada. A amostra probabilística estratificada assegura que todos os tipos de intervenientes estejam presentes; cada subgrupo da população considerada dará origem a uma amostra, segundo o fator discriminante para a segmentação da população (FREITAS et al., 2000).\newline
\indent A amostra não probabilística é obtida a partir de algum tipo de critério e nem todos os elementos da população tem a mesma chance de ser selecionados, o que torna os resultados não generalizáveis. Guardando suas limitações esse tipo de amostra pode ser conveniente quando os responsáveis são difíceis de se identificar ou pertencem a grupos específicos ou, ainda, quando existe restrição no orçamento da pesquisa (FREITAS et al., 2000).\newline
\indent Conforme Cooper e Schindler (2011) há dois tipos principais de amostragem não probabilística, por julgamento e amostragem por quota. A amostragem por julgamento é uma forma de amostragem em que os elementos da população são selecionados, deliberadamente, com base no julgamento do pesquisador. Quando usada nos estágios iniciais de um estudo exploratório, uma amostra por julgamento é apropriada.\newline
\indent Quando desejamos selecionar um grupo diferenciado para fins de filtragem, esse método de amostragem também é uma boa escolha. A amostragem por quota é uma forma em que os participantes são escolhidos, proporcionalmente, atendendo a determinados critérios e a amostra é composta por subgrupos.\newline
\indent A decisão de adotar uma amostra não probabilística é influenciada por: tempo reduzido disponível para a distribuição, execução e análise da pesquisa; ausência de recursos financeiros e materiais; necessidade de um processo de amostragem acessível e descomplicado. Neste trabalho será usada uma amostragem não probabilística por julgamento.\newline
\indent Nesse formato são escolhidos membros da população, que são boas fontes de informação sobre o assunto pesquisado e que possam representar, da melhor maneira possível, dentro do contexto de uma amostra não probabilística, a população estudada.\newline
\indent Como uma maneira de garantir a amostragem por julgamento, algumas perguntas abordam o tempo de experiência do pesquisado com sistemas ERP e se é usuário do sistema ou um técnico. Essas informações servirão como uma ferramenta de filtragem na seleção dos elementos que serão incluídos na amostra.\newline
\indent O \textit{link} do questionário será distribuído, principalmente, por \textit{Whatsapp}, mas também será enviado por \textit{e-mail} e compartilhamento na rede social \textit{Facebook}.\newline
\indent Após o término da pesquisa, será efetuada uma filtragem para assegurar o critério usado na amostragem por julgamento, quando serão excluídos os registros que não se enquadravam no perfil da população pesquisada.

\section{Planejamento} \label{Planejamento}

Segundo Babbie (2005), uma \textit{survey} se divide em sete etapas: (1) identificação da questão de pesquisa; (2) elaboração do questionário; (3) pré-teste do questionário; (4) aplicação; (5) coleta dos dados; (6) tabulação dos dados e (7) análise dos dados.\newline
\indent O primeiro passo para realizar um \textit{survey} foi identificar, claramente, uma questão de pesquisa e objetivos a serem atingidos, o que foi feito na introdução deste trabalho. Depois disso, o próximo passo foi traduzir e adaptar o questionário, subsequentemente, foram testadas as funcionalidades da ferramenta de pesquisa e submetido o questionário para a avaliação da banca de qualificação. Após os ajustes solicitados pela banca, o questionário será divulgado e aplicado.\newline
\indent Após a finalização da pesquisa \textit{on-line} serão coletados os dados e realizada a tabulação das informações. Por fim serão analisados os dados do Brasil comparando-os com os da Alemanha.

\chapter{Análise de Resultados} \label{Análise de Resultados}

A base para análise dos resultados foi obtida por meio de uma \textit{survey} aplicada com participantes dos grupos de discussão sobre os sistemas ERPs.\newline
\indent O questionário desta \textit{survey} foi adaptado para a execução da pesquisa foi criado a partir da tradução do questionário aplicado na Alemanha, tendo em vista a pluralidade cultural das regiões brasileiras, foi adicionada a questão referente a região onde o pesquisado reside e trabalha.\newline
\indent Outro fator importante é o \textit{turnover} do colaborador brasileiro que é o maior do mundo, segundo Bispo (2013), esse fato fez com que o fator de exclusão que é o tempo de experiência do colaborador na empresa, fosse reduzido de 2 anos para um ano.\newline
\indent No apêndice A, é possível verificar a cópia do questionário que será divulgado, esse questionário se divide em 6 partes que foram descritas no capítulo execução da pesquisa, subtópico elaboração e estruturação do questionário.\newline
\indent Os resultados compreenderam a avaliação dos sistemas ERP usados, a incerteza dos usuários no uso do sistema, o suporte em situações problemáticas e, finalmente, a avaliação de possíveis soluções para superar as deficiências.\newline
\indent Os resultados incluíram todos os participantes, que tiveram a pergunta específica em seu questionário e não pularam, nem responderam “Não sei”, isso porque a pesquisa original foi realizada dessa forma, logo para que se evite um viés serão utilizados os mesmos parâmetros da pesquisa original. 

\section{Métodos de Análise Utilizados} \label{Métodos de Análise Utilizados}

Conforme Ghosh (1999), para se analisar uma variável de resposta contínua deve-se usar a análise multivariada, que permite estudar e evidenciar as ligações, as semelhanças e diferenças existentes entre todas as variáveis envolvidas no processo.  A necessidade de entender a relação entre diversas variáveis aleatórias faz da análise multivariada uma metodologia com grande potencial de uso.\newline
\indent Desta forma, Ghosh (1999), ainda reitera que o modelo ANOVA tem como objetivo principal verificar se existe uma diferença significativa entre as médias e se os fatores analisados exercem influência na variável de interesse (variável resposta).\newline
\indent Se por um lado, as técnicas estatísticas multivariadas são mais complexas do que aquelas da estatística univariada, por outro lado, apesar de uma razoável complexidade teórica fundamentada na matemática, as técnicas multivariadas, permitem o tratamento de diversas variáveis ao mesmo tempo, podem oferecer ao pesquisador um material bastante robusto para a análise dos dados da pesquisa.

\subsection{Análise Discriminante} \label{Analise Discriminante}

A Análise Discriminante, também denominada Análise do Fator Discriminante ou Análise Discriminante Canônica, foi originalmente desenvolvida na Botânica, e sua aplicação teve como objetivo fazer a distinção de grupos de plantas com base no tamanho e no tipo de folhas, para que, posteriormente, fosse possível classificar as novas espécies encontradas (PREARO et al., 2010).\newline
\indent Entretanto, a aplicação da Análise Discriminante logo se generalizou a outras áreas do conhecimento, inclusive a área de Marketing, sempre em situações em que é possível encontrar grupos de indivíduos e conhecer quais as características que os distinguem uns dos outros.\newline
\indent A seguir, Prearo (2010) indica as premissas presentes na Análise Discriminante:\newline
\indent - Tamanho da amostra (número de casos, indivíduos, observações, entrevistas), deve ser
adequado para permitir a generalização dos resultados, os quais podem ser verificados quanto
à significância estatística dos testes.\newline
\indent - Homoscedasticidade, ocorre quando a variância dos termos de erro parece constante ao longo do domínio da variável preditora, caso os erros não sejam aleatórios há heteroscedasticidade.\newline
\indent - Normalidade multivariada, tem a forma de sinos tridimensionais simétricos quando o eixo de x apresentar os valores de uma determinada variável; o eixo y apresentar a contagem para cada valor da variável de x; e o eixo de z apresentar os valores de qualquer outra variável em consideração.\newline
\indent - Multicolinearidade: refere-se à existência de mais de uma relação linear exata, ao passo que o termo colinearidade refere-se à existência de uma única relação linear.

\subsection{Teste F}

Os testes-F recebem seu nome da sua estatística de teste, F, que recebeu seu nome em homenagem a Sir Ronald Fisher.\newline
\indent O teste F é obtido através de uma razão entre duas variâncias e é  utilizado como um fator para comparar médias de populações normais independentes, sendo que, apresenta desvios no que tange ao tamanho do teste, quando os grupos possuem variâncias populacionais diferentes.\newline
\indent A variância por sua vez é o quadrado do desvio padrão. As variâncias são uma medida de dispersão, sendo que os valores maiores representam uma maior dispersão.\newline
\indent Este teste é mais utilizado quando se comparam modelos estatísticos que foram ajustados a um conjunto de dados, a fim de identificar o modelo que melhor se ajusta à população da qual os dados foram amostrados.

\subsection{Teste t$-$student}

A estatística t foi introduzida em 1908 por William Sealy Gosset, químico da cervejaria Guinness em Dublin, Irlanda (\textit{student} era seu pseudônimo). Gosset havia sido contratado devido à política inovadora de Claude Guinness de recrutar os melhores graduados de Oxford e Cambridge para os cargos de bioquímico e estatístico da indústria Guinness. Gosset desenvolveu o Teste t como um modo barato de monitorar a qualidade da cerveja tipo \textit{stout}. Ele publicou o Teste t na revista acadêmica \textit{Biometrika} em 1908, mas foi forçado a usar seu pseudônimo pelo seu empregador, que acreditava que o fato de usar estatística era um segredo industrial. De fato, a identidade de Gosset não foi reconhecido por seus colegas estatísticos.\newline
\indent O teste t$-$Student ou somente teste t é um teste de hipótese que usa conceitos estatísticos para rejeitar ou não uma hipótese nula quando a estatística de teste (t) segue uma distribuição t$-$student.\newline
\indent Essa premissa é normalmente usada quando a estatística de teste, na verdade, segue uma distribuição normal, mas a variância da população $\sigma$\textsuperscript{2} é desconhecida. Nesse caso, é usada a variância amostral S\textsuperscript{2} e, com esse ajuste, a estatística de teste passa a seguir uma
distribuição t$-$student.\newline
\indent O teste t$-$Student, ou simplesmente teste t é o método mais utilizado para se avaliar as diferenças entre as médias, entre dois grupos. 

\subsection{Teste de Levene}

Levene propôs uma estatística para testar igualdade de variâncias para estudos balanceados, posteriormente foi generalizada para estudos desbalanceados. A estatística é obtida a partir de uma análise de variância com um único fator, já que os níveis são as populações, cada observação i substituída pelo desvio absoluto da variável em relação à média do seu respectivo grupo.\newline
\indent Trata-se de um teste insensível a desvios da normalidade, é um teste robusto, já que, na ausência de normalidade, seu tamanho real é próximo do nível de significância fixado para uma grande variedade de distribuições de probabilidade.

\subsection{Teste de Mann-Whitney}

Os testes não paramétricos são baseados nas posições das observações e não em suas grandezas numéricas. Por isso que se diz que o teste Mann-Whitney compara a mediana ao invés da média, o teste de Mann-Whitney não faz nenhuma suposição quanto a distribuição, populacional.\newline
\indent Testar a mediana ao invés da média pode ser muito vantajoso, pois, a mediana é uma medida de informação mais eficiente que a média, uma vez que não é sensível a valores extremos.

\subsection{Correlação}

Um coeficiente de correlação mede o grau pelo qual duas variáveis tendem a mudar juntas. O coeficiente descreve a força e a direção da relação. \newline
\indent A correlação de pearson avalia a relação linear entre duas variáveis contínuas. Uma relação é linear quando a mudança em uma variável é associada a uma mudança proporcional na outra variável.\newline
\indent O coeficiente de correlação de Pearson pode variar em valor de -1 a +1. Para o coeficiente de correlação de Pearson ser +1, quando uma variável aumenta, as outras variáveis aumentam por uma quantidade consistente.\newline
\indent


\chapter{Resultados} \label{Resultados} 

Na análise dos dados, foi utilizado o programa estatístico SPSS da IBM, versão 23 (Windows).\newline
\indent Foram obtidos 126 participantes durante pesquisa nas seções a seguir os  dados são analisados.

\indent A avaliação da normalidade dos dados foi procedida pelo teste de Shapiro-Wilk e a homogeneidade das variâncias dos grupos tratados foi investigada por meio do teste de Levene. A Comparação entre os dados foi realizada pelo teste F de Snedecor-Fisher para a Análise da Variância (ANOVA), conforme o utilizado por Lambeck et al. (2014a). \newline

\section{Análise Quantitativa}

\subsection{Gênero dos Participantes}
Dos 126 participantes 125 identificaram o seu gênero, destes 12\% são do gênero Feminino e 88\% são do gênero Masculino, conforme pode ser visto na Figura \ref{fig:figura-sexo}.

\begin{figure}[H]
	\centering	
	\caption{Gênero dos participantes}
	\includegraphics[]{sexo}
	\label{fig:figura-sexo}
\end{figure}
\vspace{-0.8 cm} \hspace{3.85 cm} Fonte: Criado pelo Autor\newline

\subsection{Idade dos Participantes}
Todos os 126 participantes responderam a pergunta sobre sua faixa etária que pode ser vista na Figura \ref{fig:figura-fxetaria}.

\begin{figure}[H]
	\centering	
	\caption{Faixa etária dos participantes}
	\includegraphics[]{fxetaria}
	\label{fig:figura-fxetaria}
\end{figure}
\vspace{-0.8 cm} \hspace{3.55 cm} Fonte: Criado pelo Autor\newline

\subsection{Região de Origem}

Foi constatado que há uma predominancia de pesquisados com origem de nascimento nas regiões Sul e Sudeste, estas informações podem ser verificadas na Figura \ref{fig:figura-rgorigem}.

\begin{figure}[H]
	\centering	
	\caption{Região de Origem dos Pesquisados}
	\includegraphics[width=6in, height=2.7in]{regiao_origem}
	\label{fig:figura-rgorigem}
\end{figure}
\vspace{-0.8 cm} \hspace{2.65 cm} Fonte: Criado pelo Autor\newline

\subsection{Tamanho das Empresas}

Referente ao tamanho das empresas pesquisadas, verificou-se um grande número de participantes de grandes empresas (48\%), já as micro, pequenas e médias empresas correspondem a (42\%), estes dados estão representados na Figura \ref{fig:figura-tamanho}.

\begin{figure}[H]
	\centering	
	\caption{Tamanho das Empresas dos Pesquisados}
	\includegraphics[]{tamanho}
	\label{fig:figura-tamanho}
\end{figure}
\vspace{-0.8 cm} \hspace{2.45 cm} Fonte: Criado pelo Autor\newline

\subsection{Nível Hierárquico dos Pesquisados}

Referente ao nível hierárquico dos pesquisados verifica-se que 59\% são funcionários ou terceirizados, 32\% possuem cargo de nível médio de gestão e apenas 9\% são da alta gestão da empresa. Estes resultados podem ser visualizados na Figura \ref{fig:figura-cargos}.

\begin{figure}[H]
	\centering	
	\caption{Nível Hierárquico dos Pesquisados}
	\includegraphics[]{cargos}
	\label{fig:figura-cargos}
\end{figure}
\vspace{-0.8 cm} \hspace{2.85 cm} Fonte: Criado pelo Autor\newline


\subsection{Região de Trabalho}

Não houve uma significativa migração entre a região de origem e a região de trabalho dos pesquisados. Como acontece com a região de origem, foi constatado que há uma predominância de pesquisados que trabalham nas regiões Sul e Sudeste, estas informações podem ser verificadas na Figura \ref{fig:figura-rgtrabalho}.

\begin{figure}[H]
	\centering	
	\caption{Região de Trabalho dos Pesquisados}
	\includegraphics[width=6in, height=2.7in]{regiao_trabalho}
	\label{fig:figura-rgtrabalho}
\end{figure}
\vspace{-0.8 cm} \hspace{2.65 cm} Fonte: Criado pelo Autor\newline

\subsection{Tempo de Experiência na Empresa}

A maioria dos funcionários pesquisados (48\%) trabalham em suas respectivas empresas há menos de um ano, (42\%) dos pesquisados trabalham em suas empresas de 3 á 10 anos, uma minoria dos funcionários (10\%) atua em suas respectivas empresas há mais de 10 anos, estes dados podem ser visualizados na Figura \ref{fig:figura-expempresa}.

\begin{figure}[H]
	\centering	
	\caption{Tempo de Experiência na Empresa}
	\includegraphics[width=6in, height=4.2in]{experiencia_empresa}
	\label{fig:figura-expempresa}
\end{figure}
\vspace{-0.8 cm} \hspace{2.65 cm} Fonte: Criado pelo Autor\newline

\subsection{Experiência dos Participantes com ERP}

A experiência dos funcionários com sistemas ERP é  relativamente grande pois  cerca de (76\%) alegam ter mais de 10 anos de experiência com tais sistemas, os pesquisados que possuem de 4 a 10 anos de experiência com ferramentas ERP totalizam (14\%), enquanto os funcionários com 1 á 3 anos de experiência representam apenas (10\%) do publico pesquisado, os dados podem ser visualizados na Figura \ref{fig:figura-experp}.

\begin{figure}[H]
	\centering	
	\caption{Experiência dos Participantes com ERP}
	\includegraphics[width=6in, height=4.2in]{experiencia_erp}
	\label{fig:figura-experp}
\end{figure}
\vspace{-0.8 cm} \hspace{2.65 cm} Fonte: Criado pelo Autor\newline

\subsection{Método de pesquisa preferido}

Verifica-se pela a analise da porcentagem das respostas referente aos métodos de pesquisa que o publico Brasileiro pesquisado prefere mais [Pesquisa de texto completo] seguido de [Autocompletar], os dados podem ser avaliados na Figura \ref{fig:figura-q22}.

\begin{figure}[H]
	\centering	
	\caption{Método de pesquisa preferido}
	\includegraphics[width=6in, height=1.8in]{q22}
	\label{fig:figura-q22}
\end{figure}
\vspace{-0.8 cm} \hspace{3.15 cm} Fonte: Criado pelo Autor com o Software SPSS\newline

\subsection{Acesso aos dados por departamentos}

Verifica-se pela a analise da porcentagem das respostas que referente ao acesso de dados móveis por departamento, o publico Brasileiro pesquisado acessa mais os dados dos departamentos [Contabilidade - 24\%], [Recursos Humanos - 22\%] e [Produção - 21\%], os dados podem ser avaliados na Figura \ref{fig:figura-q31}.

\begin{figure}[H]
	\centering	
	\caption{Acesso a dados por dispositivos móveis por departamento}
	\includegraphics[width=6in, height=3.5in]{q31}
	\label{fig:figura-q31}
\end{figure}
\vspace{-0.8 cm} \hspace{3.15 cm} Fonte: Criado pelo Autor com o Software SPSS\newline

\subsection{Questão 11 x Questão 31}

A questão 11 indaga ao pesquisado "Em quais departamentos você usa o ERP?", já a questão 31 indaga ao pesquisado "Você usa seu dispositivo móvel para acessar os dados de quais departamentos?", analisamos então a relação entre as duas perguntas.\newline
\indent Dos pesquisados que responderam que usam o [Contabilidade] na questão 11, 21,79\% disseram acessar dados deste departamento por tecnologia móvel.\newline
\indent Dos pesquisados que responderam que usam o [Recursos Humanos] na questão 11, 44,68\% disseram acessar dados deste departamento por tecnologia móvel.\newline
\indent Dos pesquisados que responderam que usam o [Produção] na questão 11, 36,07\% disseram acessar dados deste departamento por tecnologia móvel.\newline
\indent Dos pesquisados que responderam que usam o [Compras / Gerenciamento de Cadeia de Suprimentos] na questão 11, 63,89\% disseram acessar dados deste departamento por tecnologia móvel.\newline
\indent Dos pesquisados que responderam que usam o [Gerenciamento de Projetos] na questão 11, 36,36\% disseram acessar dados deste departamento por tecnologia móvel.

\subsection{Uso de dispositivos móveis na empresa}

Verifica-se pela a analise da porcentagem das respostas que referente ao uso de dispositivos móveis, o publico Brasileiro pesquisado usa mais os dispositivos [\textit{Laptop} - 46\%], [\textit{Nenhum} - 30\%] e [\textit{Smartphone} - 12\%], os dados podem ser avaliados na Figura \ref{fig:figura-q30}.

\begin{figure}[H]
	\centering	
	\caption{Uso de dispositivos móveis na empresa}
	\includegraphics[width=4.7in, height=3.5in]{q30}
	\label{fig:figura-q30}
\end{figure}
\vspace{-0.8 cm} \hspace{3.15 cm} Fonte: Criado pelo Autor com o Software SPSS\newline

\subsection{Uso de dispositivos móveis tempo livre}

Verifica-se pela a analise da porcentagem das respostas que referente ao uso de dispositivos móveis, o publico Brasileiro pesquisado usa mais os dispositivos [\textit{Laptop} - 35\%] e [\textit{Smartphone} - 49\%], os dados podem ser avaliados na Figura \ref{fig:figura-q37}.

\begin{figure}[H]
	\centering	
	\caption{Uso de dispositivos móveis tempo livre}
	\includegraphics[width=4.7in, height=3.5in]{q37}
	\label{fig:figura-q37}
\end{figure}
\vspace{-0.8 cm} \hspace{3.15 cm} Fonte: Criado pelo Autor com o Software SPSS\newline

\section{Método Qualitativo}

Numa pesquisa qualitativa as respostas não são objetivas, e o propósito não é contabilizar quantidades como resultado, mas sim conseguir compreender o comportamento de determinado grupo. Os resultados estatísticos serão publicados no github acessível no endereço:https://github.com/cicerocruz/Mestrado\_IPT\_2019.

\subsection{Utilizando o teste-t no SPSS}

O método utilizado para fazer esta análise foi o Teste de Levene, para verificar se  a variança é homogênea ou não, e, logo após isso avaliar o teste-t para investigar a influência da escolha de uma opção, sobre a outra.\newline
\indent Abaixo na figura \ref{fig:figura-q23_q16_001}, podemos verificar que o software SPSS nos dá duas vias de calculo, sendo que a via  que  tomaremos é  determinada pela variável "Sig." do teste de levene, se o "Sig." for menor que 0,05 isto indica que a variança é homogênea consideraremos os dados da primeira linha, caso seja maior que 0,05 isto indica que a variança é não homogênea e devemos usar a segunda linha.\newline

\begin{figure}[H]
	\centering	
	\caption{teste-t Questão 23 opção 6 x Questão 16 opção 1}
	\includegraphics[width=6in, height=1.8in]{q23-q16-001}
	\label{fig:figura-q23_q16_001}
\end{figure}
\vspace{-0.8 cm} \hspace{1.45 cm} Fonte: Criado pelo Autor via software SPSS\newline

No exemplo da figura \ref{fig:figura-q23_q16_001} acima, verificamos a análise da questão “Questão 23. Como você avalia as seguintes estratégias para lidar com problemas no uso do sistema ERP?”, opção [Tipos de Menu e Estruturas Aprimorados] que tem sua relação com a “Questão 16. Quais tipos de menu são oferecidos pelo seu sistema?”, opção menu [1], podemos aqui avaliar que não existe uma correlação significativa isso pode ser avaliado usando a variável "Sig. bilateral", a fórmula desta relação é dada por (t(113.708) = 1009; p>0.05). Isto indica portanto que não há uma influencia da escolha de uma alternativa sobre outra.

Em outro exemplo que pode ser visto na figura \ref{fig:figura-q23_q16_002} abaixo, avaliamos a primeira linha devido a variável "Sig." do teste de levene. Na primeira linha observamos a variável "Sig. bilateral", sendo assim, a variável nos indica que existe uma correlação significativa de polaridade positiva, dada por (t(123)= 2019; p<0.05). Isto indica então que há uma influencia da escolha de uma alternativa sobre outra.

\begin{figure}[H]
	\centering	
	\caption{teste-t Questão 23 opção 6 x Questão 16 opção 2}
	\includegraphics[width=6in, height=1.8in]{q23-q16-002}
	\label{fig:figura-q23_q16_002}
\end{figure}
\vspace{-0.8 cm} \hspace{1.45 cm} Fonte: Criado pelo Autor via software SPSS\newline

\subsection{Utilizando a correlação de pearson no SPSS}

A correlação é um teste utilizado para avaliar a relação entre duas  variáveis continuas, ao se analisar a correlação de pearson via SPSS deve-se avaliar o seu o valor absoluto, sendo que a correlação é mais  forte quando se aproxima de um e mais fraca quando se aproxima de 0, referente ao sinal da correlação, quando a polaridade for negativa isto nos indica que correlação é inversa, ou seja, quanto maior for uma variável estudada, menor será a outra variável, o critério que utilizamos para  verificar a força da  correlação é o seu índice, sendo que se este for menor que 0,4 a correlação é considerada fraca, caso esteja entre 0,4 e 0,7 pode ser considerada moderada e acima de 0,7 é uma correlação forte. Referente ao valor de Sig.(bilateral) ele nos indicará a força da correlação caso seu valor seja menor que 0,05. Podemos ver uma analise de correlação na figura \ref{fig:figura-q23_q22_exp}.

\begin{figure}[H]
	\centering	
	\caption{Correlação de pearson exemplo}
	\includegraphics[width=6in, height=1.8in]{q23_q22_exp}
	\label{fig:figura-q23_q22_exp}
\end{figure}
\vspace{-0.8 cm} \hspace{2.85 cm} Fonte: Criado pelo Autor via software SPSS\newline

\subsection{Utilizando análise de efeitos com ANOVA no SPSS}

Inicialmente verificamos que existe esfericidade pelo teste de Mauchly. Abaixo vemos um exemplo onde não foi efetuado o calculo de  esfericidade pois só há duas alternativas,  conforme destacado na Figura \ref{fig:figura-com_esfericidade}, neste caso consideramos diretamente que há esfericidade, em outros casos verificariamos se há esfericidade avaliando a variável "W de Mauchly" ou "Mauchly's W", se ela é igual ou próxima a 1, no outro exemplo da Figura \ref{fig:figura-sem_esfericidade}, verificamos que a variável "W de Mauchly" ou "Mauchly's W" é menor que 0,5 então, vamos avaliar o Sig que é menor que 0,001 isso nos indica que não há esfericidade.

\begin{figure}[H]
	\centering	
	\caption{Teste de esfericidade de Mauchly - com Esfericidade}
	\includegraphics[width=6in, height=1.8in]{com_esfericidade}
	\label{fig:figura-com_esfericidade}
\end{figure}
\vspace{-0.8 cm} \hspace{2.85 cm} Fonte: Criado pelo Autor via software SPSS\newline

\begin{figure}[H]
	\centering	
	\caption{Teste de esfericidade de Mauchly - sem Esfericidade}
	\includegraphics[width=6in, height=1.8in]{sem_esfericidade}
	\label{fig:figura-sem_esfericidade}
\end{figure}
\vspace{-0.8 cm} \hspace{2.85 cm} Fonte: Criado pelo Autor via software SPSS\newline

Após determinar a esfericidade dos dados analisados verificamos qual método utilizar, o SPSS já faz o calculo considerando a existência ou não de esfericidade e nos apresenta um quadro com o calculo usando a esfericidade e apresentando o calculo para não esfericidade por 3 métodos isso pode ser visto na Figura \ref{fig:figura-metodo_teste_dentre_sujeitos} - Teste dentre sujeitos, caso haja esfericidade usaremos os dados da primeira linha do erro e da variável analisada, caso não seja utilizaremos um dos métodos de correção indicados, neste trabalho usaremos para dados sem esfericidade o método Greenhouse-Geisser.

\begin{figure}[H]
	\centering	
	\caption{Teste dentre sujeitos}
	\includegraphics[width=6in, height=2.8in]{teste_dentre_sujeitos}
	\label{fig:figura-metodo_teste_dentre_sujeitos}
\end{figure}
\vspace{-0.8 cm} \hspace{2.85 cm} Fonte: Criado pelo Autor via software SPSS\newline

Após a avaliação de qual método utilizar, temos que compreender o efeito de uma variável sobre a outra, para isso avaliamos o post-hoc, o SPSS usa o método de pairwise com ajuste de Bonferroni que é considerado mais conservador, que é exibido na Figura \ref{fig:figura-post-hoc}, sendo que neste quadro analisamos a coluna "Sig." esta coluna nos indicará se existe diferença entre uma variável e outra se ela for menor ou igual que 0,05. Caso exista diferença será necessário descreve-la através da analise do post-hoc.

\begin{figure}[H]
	\centering	
	\caption{Post-hoc comparações pelo método pairwise }
	\includegraphics[width=6in, height=1.8in]{post-hoc}
	\label{fig:figura-post-hoc}
\end{figure}
\vspace{-0.8 cm} \hspace{2.85 cm} Fonte: Criado pelo Autor via software SPSS\newline

\subsection{Questão 23 x Questão 16}

A questão 23 do questionário aplicado indagou aos participantes "Como você avalia as seguintes estratégias para lidar com problemas no uso do sistema ERP?", já a questão 16 indagou aos participantes "Quais tipos de menu são oferecidos pelo seu sistema?". A análise buscou verificar qual a influencia da escolha das opções 6 e 8 da questão 23 sobre os menus indicados na questão 16. Desta forma é possível verificar se as opções visuais facilitam as estratégias para resolução de problemas.\newline
\indent Após avaliar a relação entre a  respostas dadas na  questão 16 com as opções  escolhidas na questão 23, utilizando-se a  analise de médias e um Teste T, cuja técnica foi descrita acima, usando o software SPSS, verificou-se que houveram significativas divergências entre estes públicos.\newline
\indent No publico pesquisado no Brasil a analise indicou que a escolha dos tipos de menu [2] BreadCrumb e [4] Tree, são influenciados e influenciam a escolha da opção [Tipos de Menu e Estruturas Aprimorados] e que o tipo de menu [4] Tree, influência a escolha também da opção [Funcionalidade de Pesquisa Avançada], o que indica que estes dois tipos de menu satisfazem o publico Brasileiro e  são os tipos de menu mais utilizados como estratégias para lidar com problemas no uso do sistema ERP. Em contrapartida no publico Alemão não foi encontrada uma associação entre estas duas opções.

\subsection{Questão 23 x Questão 22}

A questão 23 do questionário aplicado indagou aos participantes "Como você avalia as seguintes estratégias para lidar com problemas no uso do sistema ERP?", já a questão 22 indagou aos participantes "Qual é o seu método preferido para pesquisar informações?".\newline
\indent A análise da correlação de Pearson entre a questão 23 com a questão 22, indica que existe uma correlação negativa, fraca e pouco significativa, sendo que esta correlação é  dada por (r=-0,30, p>0,05).\newline
\indent Os dados da  análise da correlação podem ser vistos na Figura \ref{fig:figura-q23_q22}  abaixo:

\begin{figure}[H]
	\centering	
	\caption{Correlação entre Questão 23 opção 8 x Questão 22}
	\includegraphics[width=6in, height=1.8in]{q23_q22}
	\label{fig:figura-q23_q22}
\end{figure}
\vspace{-0.8 cm} \hspace{2.15 cm} Fonte: Criado pelo Autor via software SPSS\newline

\subsection{Questão 23 x Questão 18}

Aqui analisamos a relação entre os participantes que escolheram a opção "Orientação e Suporte ao Usuário" na questão 23, em relação as opções da questão 18 solicitou aos participantes uma avaliação de seus ERP's "Por favor, avalie seu ERP de acordo com a escala na tabela a seguir:".\newline
\indent Referente a opção [O meu sistema ERP oferece uma ampla gama de funcionalidades de suporte para lidar com problemas] da questão 18, nossa analise indica que há uma correlação moderada, significativa e positiva, com a Questão 23 opção [Orientação e Suporte ao Usuário] dada por (r= 0,431,p<0,001), isto indica que o usuário que prefere a opção [Orientação e Suporte ao Usuário] tende a achar que seu ERP tem uma quantidade de recursos ideal, esta correlação pode ser verificada na Figura \ref{fig:figura-q235_q181}.\newline

\begin{figure}[H]
	\centering	
	\caption{Correlação entre Questão 23 opção 5 x Questão 18 opção 1}
	\includegraphics[width=6in, height=3.9in]{q23-5_q18-1}
	\label{fig:figura-q235_q181}
\end{figure}
\vspace{-0.8 cm} \hspace{1.45 cm} Fonte: Criado pelo Autor via software SPSS\newline

\indent Referente a opção [O meu sistema ERP é muito complexo, o que muitas vezes me faz sentir perdido] da questão 18, nossa analise indica que há uma correlação moderada, significativa e negativa, com a Questão 23 opção [Orientação e Suporte ao Usuário] dada por (r= -0,253,p<0,01), isto indica que o usuário que prefere a opção [Orientação e Suporte ao Usuário] não se sente perdido, esta correlação pode ser verificada na Figura \ref{fig:figura-q235_q182}.\newline

\begin{figure}[H]
	\centering	
	\caption{Correlação entre Questão 23 opção 5 x Questão 18 opção 2}
	\includegraphics[width=6in, height=3.9in]{q23-5_q18-2}
	\label{fig:figura-q235_q182}
\end{figure}
\vspace{-0.8 cm} \hspace{1.45 cm} Fonte: Criado pelo Autor via software SPSS\newline

\indent Referente a opção [O meu sistema ERP oferece inúmeras e úteis visualizações, as quais eu posso escolher] da questão 18, nossa analise indica que há uma correlação moderada, significativa e positiva, com a Questão 23 opção [Orientação e Suporte ao Usuário] dada por (r= 0,538,p<0,01), isto indica que o usuário que prefere a opção [Orientação e Suporte ao Usuário] percebe que seu sistema oferece recursos úteis, esta correlação pode ser verificada na Figura \ref{fig:figura-q235_q183}.\newline

\begin{figure}[H]
	\centering	
	\caption{Correlação entre Questão 23 opção 5 x Questão 18 opção 4}
	\includegraphics[width=6in, height=3.9in]{q23-5_q18-3}
	\label{fig:figura-q235_q183}
\end{figure}
\vspace{-0.8 cm} \hspace{1.45 cm} Fonte: Criado pelo Autor via software SPSS\newline

\subsection{Correlação entre respostas Questão 18}

Referente as opções [O meu sistema ERP oferece uma ampla gama de funcionalidades de suporte para lidar com problemas] e [O meu sistema ERP oferece inúmeras e úteis visualizações, as quais eu posso escolher] da questão 18, nossa analise indica que há uma correlação moderada, significativa e positiva, dada por (r= 0,514,p<0,01), isto indica que os usuários que escolheram uma opção tendem a escolher a outra, esta relação pode ser verificada na Figura  \ref{fig:figura-q181_q184}.\newline

\begin{figure}[H]
	\centering	
	\caption{Correlação entre Questão 18 opção 1 x Questão 18 opção 4}
	\includegraphics[width=6in, height=3.9in]{q18-1_q18-4}
	\label{fig:figura-q181_q184}
\end{figure}
\vspace{-0.8 cm} \hspace{0.45 cm} Fonte: Criado pelo Autor via software SPSS\newline

Referente as opções [O meu sistema ERP é muito complexo, o que muitas vezes me faz sentir perdido] e [O meu sistema ERP oferece inúmeras e úteis visualizações, as quais eu posso escolher] da questão 18, nossa analise indica que há uma correlação moderada, significativa e negativa, dada por (r= -0,583,p<0,01), isto indica que os usuários que escolheram uma opção tendem a não escolher a outra, esta relação pode ser verificada na Figura \ref{fig:figura-q182_q184}.\newline

\begin{figure}[H]
	\centering	
	\caption{Correlação entre Questão 18 opção 2 x Questão 18 opção 4}
	\includegraphics[width=6in, height=3.9in]{q18-2_q18-4}
	\label{fig:figura-q182_q184}
\end{figure}
\vspace{-0.8 cm} \hspace{0.45 cm} Fonte: Criado pelo Autor via software SPSS\newline

\subsection{Questão 17, Questão 18 e Questão 22}

Investigamos a correlação entre a questão 17 "Você conhece plenamente todas as etapas necessárias do processo para realizar suas tarefas", opção [O meu sistema ERP oferece uma ampla gama de funcionalidades de suporte para lidar com problemas] da questão 18 e a questão 21 "Você está sempre ciente das consequências de suas ações", nossa analise indica que há uma correlação fraca, significativa e positiva entre a questão 18 e a questão 21, dada por (r= 0,372,p<0,01), isto indica que quando se escolhe a opção analisada na questão 18 existe uma pequena possibilidade de se escolher "Sim Sempre" na questão 21, esta relação pode ser verificada na Figura \ref{fig:figura-q17_q18-1_q22}.\newline

\begin{figure}[H]
	\centering	
	\caption{Correlação entre Questão 17, Questão 18 opção 1 x Questão 22}
	\includegraphics[width=6in, height=3.9in]{q17_q18-1_q22}
	\label{fig:figura-q17_q18-1_q22}
\end{figure}
\vspace{-0.8 cm} \hspace{0.45 cm} Fonte: Criado pelo Autor via software SPSS\newline

Investigamos também a correlação entre a questão 17 "Você conhece plenamente todas as etapas necessárias do processo para realizar suas tarefas", opção [O meu sistema ERP é muito complexo, o que muitas vezes me faz sentir perdido] da questão 18 e a questão 21 "Você está sempre ciente das consequências de suas ações", nossa analise indica que há uma correlação entre moderada e forte, significativa e negativa entre a questão 18 e a questão 17, dada por (r= -0,687, p<0,01), ainda foi identificada uma correlação fraca e positiva, entre a questão 21 e a questão 17, dada por (r= 0,187, p<0,05), também foi identificada uma correlação fraca e negativa, entre a questão 21 e a questão 18, dada por (r= -0,194, p<0,05), isto indica que quando se escolhe a opção analisada na questão 18 existe a possibilidade de se não escolher "Sim Sempre" na questão 21, que quando se escolhe "Sim Sempre" na questão 21 existe uma pequena possibilidade de se escolher "Sim Sempre" na questão 17, que quando se escolhe "Sim Sempre" na questão 21 existe uma pequena possibilidade de se não escolher a opção analisada da questão 18, esta relação pode ser verificada na Figura  \ref{fig:figura-q17_q18-2_q22}.\newline

\begin{figure}[H]
	\centering	
	\caption{Correlação entre Questão 17, Questão 18 opção 2 x Questão 22}
	\includegraphics[width=6in, height=3.9in]{q17_q18-2_q22}
	\label{fig:figura-q17_q18-2_q22}
\end{figure}
\vspace{-0.8 cm} \hspace{0.45 cm} Fonte: Criado pelo Autor via software SPSS\newline

Por fim, investigamos a correlação entre a questão 17 "Você conhece plenamente todas as etapas necessárias do processo para realizar suas tarefas", opção [O meu sistema ERP oferece inúmeras e úteis visualizações, as quais eu posso escolher] da questão 18 e a questão 21 "Você está sempre ciente das consequências de suas ações", nossa analise indica que há uma correlação moderada, significativa e positiva entre a questão 18 e a questão 17, dada por (r= 0,511,p<0,01), também verificamos a indicação de uma correlação fraca, significativa e positiva entre a questão 18 e a questão 21, dada por (r= 0,213,p<0,05), por fim, ainda verificamos a indicação de uma correlação fraca, significativa e positiva entre a questão 17 e a questão 21, dada por (r= 0,187,p<0,05), isto indica que quando se escolhe a opção analisada na questão 18 existe uma possibilidade de se escolher "Sim Sempre" na questão 17 e uma possibilidade pequena de se escolher "Sim Sempre" na questão 21, esta relação pode ser verificada na Figura \ref{fig:figura-q17_q18-4_q22}.\newline

\begin{figure}[H]
	\centering	
	\caption{Correlação entre Questão 17, Questão 18 opção 4 x Questão 22}
	\includegraphics[width=6in, height=3.9in]{q17_q18-4_q22}
	\label{fig:figura-q17_q18-4_q22}
\end{figure}
\vspace{-0.8 cm} \hspace{0.45 cm} Fonte: Criado pelo Autor via software SPSS\newline

\subsection{Questão 17, Questão 18, Questão 22, Questão 34, Questão 35, Questão 36}

Investigamos a correlação entre a questão 17 "Você conhece plenamente todas as etapas necessárias do processo para realizar suas tarefas", opção [O meu sistema ERP oferece uma ampla gama de funcionalidades de suporte para lidar com problemas] da questão 18, questão 21 "Você está sempre ciente das consequências de suas ações", questão 34 "Há quantos anos você trabalha na empresa?", questão 35 "Há quanto tempo você usa Sistemas ERP no Geral?", e questão 36 "Como você auto avalia sua experiência com sistemas ERP?", a correlação entre as questões 17, 18 e 22 já  foram estudadas anteriormente, analisaremos aqui a correlação entre estas questões e as demais, nossa analise indica que há uma correlação moderada, significativa e de polaridade negativa entre a questão 17 e a questão 35, dada por (r= -0,583,p<0,01), isto indica que os  usuários com mais experiência em sistemas ERP tendem a não escolher a opção "Sim Sempre" na questão 17.\newline
\indent Ainda encontramos correlações entre a questão 17 e a questão 36 dada por (r= 0,254,p<0,01), entre a questão 18 e a questão 34 dada por (r= -0,385,p<0,01), entre a questão 18 e a questão 35 dada por (r= 0,365,p<0,01), entre a questão 35 e a questão 36 dada por (r= -0,346,p<0,01), sendo todas estas correlações fracas e significativas sendo que sua polaridade indicada na variável r.\newline
\indent Analisando estas  correlações  fracas verificamos que indivíduos que conhecem plenamente as etapas necessárias do processo para realizar suas tarefas,  tendem  a avaliar melhor seu ERP, já entre indivíduos que usam a pouco tempo o ERP tem uma pequena possibilidade de a achar que o sistema oferece mais opções do que  necessitam,  também esta analise de correlação nos indicou que indivíduos que usam o ERP a mais tempo tem uma pequena possibilidade de não estar satisfeitos com sua experiencia com estes sistemas, estes dados podem ser consultados na Figura  \ref{fig:figura-q17_q18-1_q22_q34_q35_q36}.

\begin{figure}[H]
	\centering	
	\caption{Correlação entre Questões 17, 18 opção 1, 22, 34, 35, 36}
	\includegraphics[width=6in, height=3.9in]{q17_q18-1_q22_q34_q35_q36}
	\label{fig:figura-q17_q18-1_q22_q34_q35_q36}
\end{figure}
\vspace{-0.8 cm} \hspace{0.85 cm} Fonte: Criado pelo Autor via software SPSS\newline

\subsection{Questão 2 x Questão 23 opções 4 e 7}

Inicialmente verificamos a correlação entre duas opções da questão 23, entre os participantes que escolheram a opção [Feedback: Visual, Tátil ou Auditivo] na questão 23, em relação a opção [Suporte a Dispositivos Sensíveis ao Toque], nossa analise indica que há uma correlação fraca, significativa e positiva, dada por (r= 0,338, p<0,001), isto indica que o usuário que seleciona a opção [Feedback: Visual, Tátil ou Auditivo] tende a ter uma pequena possibilidade de escolher também [Suporte a Dispositivos Sensíveis ao Toque], esta correlação pode ser verificada na Figura \ref{fig:figura-q23-4_q23-7}.

\begin{figure}[H]
	\centering	
	\caption{Correlação entre as opções 4 e 7 da Questão 24}
	\includegraphics[width=6in, height=3.1in]{q23-4_q23-7}
	\label{fig:figura-q23-4_q23-7}
\end{figure}
\vspace{-0.8 cm} \hspace{1.35 cm} Fonte: Criado pelo Autor via software SPSS\newline

Após usar uma ANOVA de duas vias, obtivemos os dados das Figuras \ref{fig:figura-q23-4-7_v2_esfericidade}, \ref{fig:figura-q23-4-7_v2_teste_dentre_sujeitos} e \ref{fig:figura-q23-4-7_v2_post-hoc}, verificamos então  não há efeito de uma opção sobre a outra na questão 23, F[(1 107) = 16,874; p < 0,001]. Analisando o teste post-hoc usando o ajuste de Bonferroni mostrou que  as duas opções analisadas diferem entre si.\newline
\indent Já referente a iteração entre a questão “2.Qual o setor em que sua  empresa atua?” e as opções 4,7 da questão 23, a ANOVA de duas vias mostrou que há efeito da questão 2 sobre estas opções da questão 23, F[(6 107) = 0,958; p < 0,001], analisando o teste de post-hoc verificamos que que quem escolhe a  opção 4 da questão 23 tende não escolher ser do setor 3 ser do setor 7, já quem escolhe a  opção 7 da questão 23, tende não ser do setor 2 e sim do setor 4.

\begin{figure}[H]
	\centering	
	\caption{Teste de esfericidade de Mauchly}
	\includegraphics[width=6in, height=2.7in]{q23-4-7_v2_esfericidade}
	\label{fig:figura-q23-4-7_v2_esfericidade}
\end{figure}
\vspace{-0.8 cm} \hspace{1.55 cm} Fonte: Criado pelo Autor via software SPSS

\begin{figure}[H]
	\centering	
	\caption{Teste dentre os sujeitos}
	\includegraphics[width=6in, height=2.7in]{q23-4-7_v2_teste_dentre_sujeitos}
	\label{fig:figura-q23-4-7_v2_teste_dentre_sujeitos}
\end{figure}
\vspace{-0.8 cm} \hspace{1.55 cm} Fonte: Criado pelo Autor via software SPSS

\begin{figure}[H]
	\centering	
	\caption{Post-hoc por pairwise com ajuste de Bonferroni}
	\includegraphics[width=6in, height=2.7in]{q23-4-7_v2_post-hoc}
	\label{fig:figura-q23-4-7_v2_post-hoc}
\end{figure}
\vspace{-0.8 cm} \hspace{1.55 cm} Fonte: Criado pelo Autor via software SPSS\newline

\subsection{Questão 25 x Questão 26}

Em média, o número de participantes que escolheu a opção inovação na questão 25 (M=2,04; EP=0,65) foi menor do que os que escolherem inovação na questão 26 (M=1,98; EP=0,89), t(121)= 0,688; p>0,05, sendo que o intervalo de confiança passa pelo 0, a analise da correlação indica que há uma correlação fraca, significativa e  de polaridade positiva entre as opções. Com base nestes dados isso nos indica que não houve diferenças significativas.

Esses dados podem ser visualizados na Figura \ref{fig:figura-q25_q26}
\begin{figure}[H]
	\centering	
	\caption{Teste-t Questão 25 e 26 opção inovação}
	\includegraphics[width=6in, height=2.7in]{q25_q26}
	\label{fig:figura-q25_q26}
\end{figure}
\vspace{-0.8 cm} \hspace{1.55 cm} Fonte: Criado pelo Autor via software SPSS\newline

Em média, o número de participantes que escolheu a opção utilidade na questão 25 (M=2,53; EP=0,685) foi menor do que os que escolherem utilidade na questão 26 (M=2,18; EP=0,904), t(119)= 4925; p<0,05, a analise da correlação indica que há uma correlação moderada, significativa e de polaridade positiva entre as opções. Com base nestes dados isso nos indica que houve diferenças na escolha das opções pelos pesquisados.

Esses dados podem ser visualizados na Figura \ref{fig:figura-q25-2_q26-2}
\begin{figure}[H]
	\centering	
	\caption{Teste-t Questão 25 e 26 opção utilidade}
	\includegraphics[width=6in, height=2.7in]{q25-2_q26-2}
	\label{fig:figura-q25-2_q26-2}
\end{figure}
\vspace{-0.8 cm} \hspace{1.55 cm} Fonte: Criado pelo Autor via software SPSS\newline

\subsection{Questão 18 x Questão 19}

A a análise de variância (ANOVA) mostrou que há um efeito da interação entre as questões 18 e 19 [F(3,328) = 18190; p<0,001]. O post-hoc de Bonferroni mostrou que quando se escolhe a opção 2, 3, 4 e 5 na questão 18, influência a escolha das mesmas opções na questão 19.

Esses dados podem ser visualizados nas Figuras \ref{fig:figura-q19_q20-esfericidade}, \ref{fig:figura-q19_q20_tst_sujeitos} e \ref{fig:figura-q19_q20_post-hoc}.
\begin{figure}[H]
	\centering	
	\caption{Teste de esfericidade de Mauchly}
	\includegraphics[width=6in, height=2.7in]{q19_q20-esfericidade}
	\label{fig:figura-q19_q20-esfericidade}
\end{figure}
\vspace{-0.8 cm} \hspace{1.55 cm} Fonte: Criado pelo Autor via software SPSS

\begin{figure}[H]
	\centering	
	\caption{Teste dentre os sujeitos}
	\includegraphics[width=6in, height=5.7in]{q19_q20_tst_sujeitos}
	\label{fig:figura-q19_q20_tst_sujeitos}
\end{figure}
\vspace{-0.8 cm} \hspace{1.55 cm} Fonte: Criado pelo Autor via software SPSS

\begin{figure}[H]
	\centering	
	\caption{Post-hoc por pairwise com ajuste de Bonferroni}
	\includegraphics[width=6in, height=3.7in]{q19_q20_post-hoc}
	\label{fig:figura-q19_q20_post-hoc}
\end{figure}
\vspace{-0.8 cm} \hspace{1.55 cm} Fonte: Criado pelo Autor via software SPSS

\subsection{Dispositivo móvel uso pessoal x uso trabalho}

Em média, o número de participantes escolheu mais o uso pessoal na  relação entre o uso pessoal (M=1,81; EP=0,059) em relação ao uso privado (M=0,85; EP=0,073), t(124)= -10,699; p<0,05. Com base nestes dados isso nos indica que houve diferenças na relação entre uso pessoal e no trabalho.

Esses dados podem ser visualizados na Figura \ref{fig:figura-privado_trabalho}
\begin{figure}[H]
	\centering	
	\caption{Teste-t uso de dispositivo móvel pessoal x trabalho}
	\includegraphics[width=6in, height=2.7in]{privado_trabalho}
	\label{fig:figura-privado_trabalho}
\end{figure}
\vspace{-0.8 cm} \hspace{1.55 cm} Fonte: Criado pelo Autor via software SPSS\newline

\subsection{Questão 18 x Questão 37}

A ANOVA de duas vias mostrou que há um efeito da interação entre as questões 18 e 37 dada por [F(1,965, 241,682) = 6,001; p<0,05].
O post-hoc de Bonferroni mostrou que quando se usa dispositivos móveis na questão 37, influência a escolha das mesmas opções na questão 18 sendo que há maior possibilidade de escolha da opção 5 na questão 18.

Esses dados podem ser visualizados nas Figuras \ref{fig:figura-q19_hb_post-hoc}, \ref{fig:figura-q19_hb_tst_sujeitos} e \ref{fig:figura-q19_q20_post-hoc}.
\begin{figure}[H]
	\centering	
	\caption{Teste de esfericidade de Mauchly}
	\includegraphics[width=6in, height=2.7in]{q19_hb_esfericidade}
	\label{fig:figura-q19_hb_esfericidade}
\end{figure}
\vspace{-0.8 cm} \hspace{1.55 cm} Fonte: Criado pelo Autor via software SPSS

\begin{figure}[H]
	\centering	
	\caption{Teste dentre os sujeitos}
	\includegraphics[width=6in, height=3.7in]{q19_hb_tst_sujeitos}
	\label{fig:figura-q19_hb_tst_sujeitos}
\end{figure}
\vspace{-0.8 cm} \hspace{1.55 cm} Fonte: Criado pelo Autor via software SPSS

\begin{figure}[H]
	\centering	
	\caption{Post-hoc por pairwise com ajuste de Bonferroni}
	\includegraphics[width=6in, height=3.7in]{q19_hb_post-hoc}
	\label{fig:figura-q19_hb_post-hoc}
\end{figure}
\vspace{-0.8 cm} \hspace{1.55 cm} Fonte: Criado pelo Autor via software SPSS

\subsection{Questão 23 x Questão 37}

A ANOVA de duas vias mostrou que não há interação entre as questões 23 e 37.

Esses dados podem ser visualizados nas Figuras \ref{fig:figura-q23_hb_post-hoc}, \ref{fig:figura-q23_hb_tst_sujeitos} e \ref{fig:figura-q23_q20_post-hoc}.
\begin{figure}[H]
	\centering	
	\caption{Teste de esfericidade de Mauchly}
	\includegraphics[width=6in, height=2.3in]{q23_hb_esfericidade}
	\label{fig:figura-q23_hb_esfericidade}
\end{figure}
\vspace{-0.8 cm} \hspace{1.55 cm} Fonte: Criado pelo Autor via software SPSS

\begin{figure}[H]
	\centering	
	\caption{Teste dentre os sujeitos}
	\includegraphics[width=6in, height=3.7in]{q23_hb_tst_sujeitos}
	\label{fig:figura-q23_hb_tst_sujeitos}
\end{figure}
\vspace{-0.8 cm} \hspace{1.55 cm} Fonte: Criado pelo Autor via software SPSS

\begin{figure}[H]
	\centering	
	\caption{Post-hoc por pairwise com ajuste de Bonferroni}
	\includegraphics[width=6in, height=3.7in]{q23_hb_post-hoc}
	\label{fig:figura-q23_hb_post-hoc}
\end{figure}
\vspace{-0.8 cm} \hspace{1.55 cm} Fonte: Criado pelo Autor via software SPSS

\subsection{Questão 17 x Questão 16 opção 4}

Em média, o número de participantes que escolheu de  "Sim Sempre" ou "Sim, na maioria das vezes" para a questão 17 "Você conhece plenamente todas as etapas necessárias do processo para realizar suas tarefas?" (M=1,43; EP=0,075), foi maior do que os não escolheram (M=2,16; EP=0,125), t(82,054)= 5,044; p<0,001. Esses dados podem ser visualizados na Figura \ref{fig:figura-v18_q16_opt4}.
\begin{figure}[H]
	\centering	
	\caption{Teste-t Questão 17 x Questão 16 opção 4}
	\includegraphics[width=6in, height=1.8in]{v18_q16_opt4}
	\label{fig:figura-v18_q16_opt4}
\end{figure}
\vspace{-0.8 cm} \hspace{1.55 cm} Fonte: Criado pelo Autor via software SPSS

\subsection{Questão 18 opção 2 x Questão 16 opção 4}

Em média, o número de participantes que escolheram o tipo de menu [4] na questão 16 que, escolheram a opção [O meu sistema ERP é muito complexo, o que muitas vezes me faz sentir perdido] na questão 18 (M=4,13; EP=0,193), foi maior do que os não escolheram (M=2,88; EP=0,193),  t(93,320)= -6,307; p<0,001. Esses dados podem ser visualizados na Figura \ref{fig:figura-v19_u2_q16_opt4}.
\begin{figure}[H]
	\centering	
	\caption{Teste-t Questão 18 opção 2 x Questão 16 opção 4}
	\includegraphics[width=6in, height=1.8in]{v19_u2_q16_opt4}
	\label{fig:figura-v19_u2_q16_opt4}
\end{figure}
\vspace{-0.8 cm} \hspace{1.55 cm} Fonte: Criado pelo Autor via software SPSS

\subsection{Questão 18 opção 3 x Questão 16 opção 4}

Em média, o número de participantes que escolheram o tipo de menu [4] na questão 16 que, escolheram a opção [A quantidade de informações e detalhes fornecidos é alta para as minhas necessidades]  na questão 18 (M=4,44; EP=0,133), foi maior do que os não escolheram (M=2,61; EP=0,142), t(93,320)= -6,307; p<0,001. Esses dados podem ser visualizados na Figura \ref{fig:figura-v19_u3_q16_opt4}.
\begin{figure}[H]
	\centering	
	\caption{Teste-t Questão 18 opção 3 x Questão 16 opção 4}
	\includegraphics[width=6in, height=1.8in]{v19_u3_q16_opt4}
	\label{fig:figura-v19_u3_q16_opt4}
\end{figure}
\vspace{-0.8 cm} \hspace{1.55 cm} Fonte: Criado pelo Autor via software SPSS

\subsection{Questão 22 x Grupo de Questões}

Criamos uma variável chamada search\_group nela aplicamos uma função de agrupamento, separando as opções em três grupos, o grupo 1 selecionou somente quem escolheu a opção [1] da questão  "Qual é o seu método preferido para pesquisar informações?", o grupo 2 selecionou somente os que escolheram a primeira e a segunda opção, juntamente com os que escolheram a primeira a segunda e a terceira opção e por  fim os que escolheram a primeira e a terceira opção por ultimo o grupo 3 os que escolheram a quarta opção.\newline
\indent Verificamos então que a ANCOVA revelou que há um efeito do grupo de perguntas sobre a opção [Sinônimos e correção automática] dada por [F(3,122) = 6752; p<0,05]. O post-hoc de Bonferroni mostrou que a opção 3 da questão 22 em relação a variável search\_group difere das opções 2 e 3.\newline
\indent Os dados podem ser consultados nas Figuras \ref{fig:figura-search_group_v19_u3} e \ref{fig:figura-search_group_v19_u3-post-hoc}.

\begin{figure}[H]
	\centering	
	\caption{Teste dentre sujeitos}
	\includegraphics[width=6in, height=2.8in]{search_group_v19_u3}
	\label{fig:figura-search_group_v19_u3}
\end{figure}
\vspace{-0.8 cm} \hspace{1.55 cm} Fonte: Criado pelo Autor via software SPSS

\begin{figure}[H]
	\centering	
	\caption{Post-hoc método de pairwise com correção de Bonferroni}
	\includegraphics[width=6in, height=3.8in]{search_group_v19_u3-post-hoc}
	\label{fig:figura-search_group_v19_u3-post-hoc}
\end{figure}
\vspace{-0.8 cm} \hspace{1.55 cm} Fonte: Criado pelo Autor via software SPSS

Verificamos também que a ANCOVA revelou que não há um efeito do grupo de perguntas sobre a opção [Auto-completar].\newline
\indent Os dados podem ser consultados nas Figuras \ref{fig:figura-search_group_v19_u2}.

\begin{figure}[H]
	\centering	
	\caption{Teste dentre sujeitos}
	\includegraphics[width=6in, height=2.8in]{search_group_v19_u2}
	\label{fig:figura-search_group_v19_u2}
\end{figure}
\vspace{-0.8 cm} \hspace{1.55 cm} Fonte: Criado pelo Autor via software SPSS

\chapter{Conclusões} \label{Conclusões} 

\section{Empresa, sistema ERP e Usuários}

Em comparação com o publico alemão (184 pesquisados) o publico brasileiro ( 126 pesquisados ), verificou-se que há uma diferença no perfil das empresas representadas pelos pesquisados. Se na  alemanha a maior parte (70,86\%) são de empresas de médio porte ( 50 á 250 empregados ) no brasil este grupo representa somente (23\%), na Alemanha as pequenas empresas ( 10 a 49 empregados ) representam (24,57\%) enquanto no brasil esse grupo representa (11,9\%) outro fato interessante é que no brasil (48\%) dos  pesquisados informam trabalhar em empresas grandes (mais de 250 empregados). Visto que o investimento em sistemas ERP demanda um esforço financeiro por parte da empresa é compreensível que no Brasil o maior numero de usuários esteja em  empresas de grande porte.\newline
\indent As áreas mais indicadas na Alemanha são produção (52,30\%), comércio atacadista e varejista (16,67\%) e serviços de informação e comunicação (10,92\%). No brasil foram houve um equilíbrio entre os  setores sendo os mais escolhidos compras (18\%), Contabilidade (17\%), RH(16\%) e Armazenamento/Inventário (16\%). O tempo médio de uso dos sistemas ERP  na Alemanha é de 8,6 anos e varia entre um e 23 anos. Já no brasil o tempo médio 4 anos e varia de um a 29 anos. Uma ampla gama de fornecedores de ERP pode ser encontrada na Alemanha, enquanto o SAP é o sistema mais prevalente com (28,26\%). Já no Brasil predominou o uso do Protheus fornecido pela Totvs com (61,9\%) enquanto o SAP responde por (18,3\%).\newline
\indent Os sistemas ERP são usados por participantes com diferentes cargos em suas empresas, na Alemanha empregados (38,51\%) enquanto no brasil esse grupo representa (50,70\%), na Alemanha gerentes de departamento representam (42,53\%) já no brasil esse grupo representa (32,50\%) e por fim na Alemanha CEOs ou CIOs representam (18,97\%) enquanto esse grupo no brasil representa (8,8\%). Verificamos então que no Brasil o grupo empregados é mais representativo.

\section{Resultados da avaliação do sistema ERP}

Uma primeira ênfase da nossa pesquisa é derivada dos problemas de usabilidade identificados na literatura recente Lambeck et al. (2014a), dedica-se à avaliação do usuário do sistema ERP de acordo com as afirmações apresentadas na Figura \ref{fig:figura-erp_x_sa}. Esses itens abrangem o suporte em situações de erro (1), a complexidade geral do sistema (2), a quantidade e o nível de detalhamento das informações (3 ), a disponibilidade de visualizações (4) e a confusão causada por janelas abertas simultaneamente (5). Os participantes foram convidados a avaliar as cinco afirmações mostradas na Figura \ref{fig:figura-erp_x_sa}, em uma escala Likert descentralizada de seis pontos, variando de “1 - concordo totalmente” a “6 - discordo totalmente”.\newline
\indent Os resultados indicam que no brasil Figura \ref{fig:figura-erp_x_sa} os  usuários avaliam estes problemas na médios e menores, como na alemanha. Sendo que a complexidade do sistema foi avaliada em um indice menor no brasil (3,64) do que os da alemanha (4,01).  No publico brasileiro, a complexidade do sistema está relacionada a utilidade das informações exibidas (Figura \ref{fig:figura-erp_x_sa}, no. 4) (r=-0,583, p<0,01). Também no publico alemão, a disponibilidade de numerosas e úteis visualizações (Figura \ref{fig:figura-erp_x_sa}, no. 4) melhora a classificação de usuário da complexidade do sistema (r = -.312, p <.001). Já no publico alemão, a complexidade do sistema está significativamente relacionada à abundância percebida de informações e seu nível de detalhamento (Figura \ref{fig:figura-erp_x_sa}, no. 3) (r = 0,632, p <0,001). Isso não ocorre no publico Brasileiro.\newline
\indent No publico brasileiro, a gama de funcionalidades oferecidas pelo sistema (Figura \ref{fig:figura-erp_x_sa}, no. 2) melhora a classificação da percepção das inúmeras e úteis visualizações das opções do sistema (Figura \ref{fig:figura-erp_x_sa}, no. 4). O software suplementar nos dois países, é empregado como recurso para aumentar ou estender a flexibilidade do sistema ERP. \newline
\indent Os participantes da pesquisa realizada na Alemanha relataram usar duas aplicações adicionais. Software de planilha (por exemplo, o Microsoft Excel) (N = 174, 95,83\%), seguido por aplicações desenvolvidas internamente (N = 174, 55,95\%). Já os participantes no brasil preferem software de planilha (N = 125, 94\%), porém como segundo recurso no brasil se usa softwares web/app's  (N = 125, 66\%).\newline
\indent Na pesquisa alemã, a análise de variância (ANOVA) expôs diferenças na classificação de ERP e aplicação adicional nos itens 2, 3 e 4 [F (4,448) = 21,53, p <0,001)].  Já na pesquisa brasileira, a análise de variância (ANOVA) mostrou diferenças entre a classificação do ERP e aplicação adicional, através do post-hoc de Bonferroni nos itens 2, 3, 4 e 5  [F(3,328) = 18,19; p<0,001].\newline
\indent Os resultados indicam que sistemas ERP ainda apresentam deficiências, que estas, apesar de serem diferentes entre os públicos ressaltam as facilidades dos softwares adicionais, que devem ser abordadas em trabalhos futuros para acompanhar o uso do software adicional avaliando os diferenciais de sua usabilidade.\newline

\begin{figure}[H]
	\centering	
	\caption{Comparativo entre ERP e Software Alternativo}
	\includegraphics[width=6in, height=1.8in]{erp_x_sa}
	\label{fig:figura-erp_x_sa}
\end{figure}
\vspace{-0.8 cm} \hspace{1.55 cm} Fonte: Criado pelo Autor via software SPSS

\section{Incerteza no uso do sistema}

A identificação da funcionalidade ERP necessária e o acesso subsequente a ela é uma tarefa essencial no uso do sistema ERP, mas continua sendo um grande desafio. [5], [6] Para poder executar uma transação apropriadamente, os usuários precisam possuir o conhecimento sobre o próprio processo de negócios, precisam identificar e acessar a funcionalidade necessária na interface do usuário e também devem estar cientes das consequências ao cometer uma transação. Esses três aspectos formam nossa definição de certeza no uso do sistema. Os participantes foram convidados a avaliar com que frequência eles tiveram dificuldades nesses três aspectos em uma escala Likert de cinco pontos, de “1 - nunca” a “5 - sempre”.\newline
\indent Os resultados no Brasil, revelaram que os usuários sofrem principalmente com a falta de conhecimento do processo (N = 122, M = 4,33, SD = 0,743) e não tem consciência suficiente das consequências de sua ação (N = 122, M = 4,19, SD = 0,708 ). Já em relação a Alemanha, os dados revelaram que os usuários não sofrem com a falta de conhecimento do processo (N = 150, M = 1,93, SD = 0,95) ou com a consciência insuficiente das consequências de sua ação (N = 147, M = 1,99, SD = 0,91 ). Esses dados nos levam a concluir que  basicamente os  usuários brasileiros alegam não conhecer seu processo e não possuir consciência das consequências de suas ações.\newline
\indent Portanto, a capacidade de localizar a funcionalidade corporativa desejada, continua sendo um problema geral de usabilidade em diferentes níveis de experiência, sendo que isto se agrava no Brasil, pelo fato de haver indícios de não haver consciência de suas ações e também falta de conhecimento do processo.\newline
\indent No Brasil verificamos que o tipo de menu melhor classificado foi o menu de contexto, opção [5] com 1,80 em média (N = 10, DP = 0,632), seguido pelo menu barra tarefas, opção [6] com 1,78 em média (N = 18, DP = 0,647). Já na Alemanha foi verificado que o menu melhor classificado foi o menu de árvore, opção [4] com 2,66 em média (N = 107, DP = 0,82), seguido pelo menu de contexto, opção [5] com 2,33 em média (N = 51, SD = 0,84). É importante ressaltar que na Alemanha houve uma considerável parcela de pessoas que não utilizavam ERP (93) enquanto os que utilizam ERP totalizam (184), já no Brasil a totalidade dos pesquisados (126) utilizam ERP.\newline
\indent Juntamente com o desempenho dos pesquisados alemães nas questões 17 e 18, isso levou a conclusão pelos Alemães de que o melhor tipo de menu para localizar funções requeridas é o de contexto.  No Brasil não é possível chegar a esta conclusão pois os pesquisados alegam desconhecer os processos e as consequências de suas ações, isso também pode explicar o fraco desempenho da média dos melhores menus no Brasil.\newline
\indent Os resultados no Brasil, indicam uma razão hipotética que o treinamento nos sistemas ERP são deficitários, pois não transmitem ao pesquisado seus processos e as consequências de suas ações no sistema.  

\section{Avaliação de abordagem}

Esta seção é dedicada à avaliação de possíveis abordagens, que podem ser apropriadas para solucionar algumas das deficiências em sistemas ERP.\newline
\indent As interfaces atuais de ERP ainda estão lidando com conceitos de interface do usuário, que foram introduzidos na década de 1990. Para atenuar pelo menos algumas das deficiências examinadas, sugere-se a aplicação de conceitos de interface inovadores sempre que o cenário permitir. Primeiro, essas abordagens devem contar com interfaces visualmente ricas. Em segundo lugar, eles devem empregar conceitos de interação direta-manipulativa, que provaram seus benefícios em vários cenários no domínio empresarial, considera-se que ao oferecer um uso intuitivo do sistema, as barreiras que atualmente dificultam a interação entre usuário e sistema podem ser reduzidas.\newline
\indent Os autores da pesquisa Alemã levantam a hipótese de que conceitos como dispositivos sensíveis ao toque não são a primeira escolha de um usuário de ERP, ao serem solicitados a identificar conceitos potenciais para melhorar a usabilidade do ERP. O estudo Alemão ofereceu ao usuário oito abordagens gerais e bastante abstratas para avaliação dos pesquisados que também continham a opção “dispositivos sensíveis ao toque (por exemplo, \textit{multi-touch}, sistema de mesa)”. O resultado alemão mostrou que esta opção foi a pior avaliada de todas as abordagens apresentadas (N = 109, M = 3,55, DP = 1,33). Na pesquisa  brasileira a opção “dispositivos sensíveis ao toque (por exemplo, \textit{multi-touch}, sistema de mesa)” também foi a pior avaliada  tanto no sentido de inovação (N = 124, M=1,98, DP = 0,888), como no sentido de utilidade (N = 122, M=2,16, DP = 0,909), porém no Brasil a omissão foi bem baixa  o que indica que o publico Brasileiro tem mais contato com este tipo de tecnologia ao contrario do publico Alemão pesquisado a época, o tempo decorrido entre uma pesquisa e outra pode explicar esta divergência entre os públicos.\newline
\indent Na Alemanha a Questão 25 recebeu classificações bastante favoráveis tanto para inovação (N = 58, M = 1,91, SD = 0,93) como para utilidade (N = 58, M = 1,97, SD = 1,02) em uma escala ordinal de seis pontos variando de “1 - muito bom ”Para“ 6 - muito ruim ”. Porém também houve uma grande abstenção nesta questão como na questão 26. No Brasil também a Questão 25 recebeu classificações favoráveis em inovação (N = 122, M = 2,04, DP = 0,648) como em utilidade (N = 121, M = 2,53, DP = 0,684), porém a abstenção nesta questão foi bem reduzida.

\section{Trabalhos Futuros}

A pesquisa brasileira mostrou que existem diferenças significativas entre o publico Alemão e o público Brasileiro, sendo que o perfil dos usuários são bem semelhantes em indicadores como experiência em sistemas ERP e faixa etária, outros indicadores como tamanho das empresas e cargo diferem.\newline
\indent Propomos como trabalho futuro efetuar uma comparação entre as regiões do Brasil pois por se tratar de um país com proporções continentais e uma diversidade cultural grande, o que não foi possível fazer neste trabalho devido a baixa resposta das regiões Norte, Nordeste, Centro-Oeste do país.\newline
\indent Uma segunda proposta seria fazer uma comparação entre os países das Américas do Sul, central e do Norte.\newline


% ----------------------------------------------------------
% ELEMENTOS PÓS-TEXTUAIS
% ----------------------------------------------------------
\postextual
% ----------------------------------------------------------
% Referências bibliográficas
% ----------------------------------------------------------
%\setboolean{isBib}{true} % Obriga o titulo das referências à esquerda

%\setlength\bibitemsep{12pt} % Espaço de uma linha entre as referencias

%\bibliography{qualificacao}
%\bibliographystyle{qualificacao}
\newpage
\setboolean{isBib}{true} % Obriga o titulo das referências à esquerda
\bibsection
\noindent
ASSOCIAÇÃO BRASILEIRA DE NORMAS TÉCNICAS. \textbf{NBR ISO/IEC 9126}: Engenharia de Software - Qualidade de Produto - Parte1: Modelo de Qualidade. 11 ed. Rio de Janeiro: Abnt, 2003. 21 p.
\newline
\newline
\noindent
ASSOCIAÇÃO BRASILEIRA DE NORMAS TÉCNICAS. \textbf{NBR 9241}: Requisitos Ergonômicos para Trabalho de Escritórios com Computadores. 11 ed. Rio de Janeiro: Abnt, 2002. 21 p.
\newline
\newline
\noindent
ASSIS, Francisco Carlos de. Indústria brasileira cai no ranking global: Estudo mostra que País estava na 7ª posição em 2005, caiu para 9º em 2016 e pode deixar a lista dos 10 maiores este ano por causa da baixa produtividade. \textbf{O Estado de São Paulo}. São Paulo, p. 20-22. 16 out. 2017.
\newline
\newline
\noindent 
ALMEIDA, Antônia and ELIAN, Silvia and NOBRE, Juvêncio, Modificações e alternativas aos testes de Levene e de Brown e Forsythe para igualdade de variâncias e médias, \textbf{Revista Colombiana de Estadística}, Dez, 2008, Vol.31 p.241 - 260.
\newline
\newline
\noindent 
BABAIAN, Tamara et al. Applying design principles for enhancing enterprise system usability. In: INTERNATIONAL CONFERENCE ON SOFTWARE ENGINEERING AND APPLICATIONS, 9., 2014, Vienna, Austria. \textbf{Proceedings...}. Washington, D.c., Eua: Ieee Computer Society, 2014. p. 162 - 169.
\newline
\newline
\noindent 
BABBIE, Earl. \textbf{Métodos de pesquisa de survey}. Belo Horizonte: Ufmg, 2005. 519p.
\newline
\newline
\noindent 
BANHOS, Vangela Tatiana. \textbf{Usabilidade na Recuperação de Informação}: um enfoque no Catálogo Athena . 2008. 120f. Dissertação (Mestrado) - Faculdade de Filosofia e Ciências, Universidade Estadual Paulista, Marília, 2008.
\newline
\newline
\noindent
BAZIRE, Mary; BRÉZILLON, Patrick. Understanding context before using it. In: \textbf{5th International and Interdisciplinary Conference}, CONTEXT-05, v. LNAI 3554, p. 29-40, Springer Verlag, Paris, France, 2005.
\newline
\newline
\noindent 
BISHU, Ram R.; KLEINER, Brian M; DRURY, Colin G. Ergonomic Concerns in Enterprise Resource Planning (ERP) Systems and Its Implementations. In: MANUFACTURING AND ENTERPRISE NETWORKS (DIISM), 4., 2001, Melbourne, Victoria, Australia. \textbf{Proceedings of the Fourth International Conference on the Design of Information Infrastructure Systems for Manufacturing: Global Engineering}. Deventer, The Netherlands: Kluwer, 2001. p. 146 - 155.
\newline
\newline
\noindent
BISPO, Nathaly, Brasil tem o maior índice de rotatividade, \textbf{Catho}. São Paulo, https://goo. gl/sDRPb7, 13 dez. 2013
\newline
\newline
\noindent
COOPER, C. R.; SCHINDLER, P. S. \textbf{Métodos de Pesquisa em Administração}. Tradução de:Iuri Durquia Abreu. Porto Alegre: Bookman, 2011.
\newline
\newline
\noindent
COSTA, Eduardo Marques. \textbf{Avaliação de usabilidade}: um estudo da percepção de qualidade no comércio eletrônico. 2010. 203 f. Dissertação (Mestrado) - Curso de Administração e Controladoria, Universidade Federal do Ceará, Fortaleza, 2010.
\newline
\newline
\noindent
COSTA, Luciana Ferreira da; RAMALHO, Francisca Arruda. \textbf{A usabilidade nos estudos de uso da informação:} em cena usuários e sistemas interativos de informação. Perspectivas em Ciência da Informação, Belo Horizonte, v. 15, n. 1, p.92-117, abr. 2010.
\newline
\newline
\noindent
DEY, Anind K. \textbf{Understanding User Context}. In: Journal Personal and Ubiquitous Computing Archive, v. 5, n. 1, p.4-7, fev. 2001.
\newline
\newline
\noindent
FERREIRA, Aurélio B. H. \textbf{Dicionário Aurélio da língua portuguesa}. 5. ed. Curitiba: Positivo, 2014. p. 180-180.
\newline
\newline
\noindent
FELIZARDO, KATIA R. et al. \textbf{REVISÃO SISTEMÁTICA DA LITERATURA EM ENGENHARIA DE SOFTWARE}: Teoria e Pratica. Rio de Janeiro: Elsevier Brasil, 2017. 144 p.
\newline
\newline
\noindent
FOHRHOLZ, Corinna; GRONAU, Norbert. \textbf{ERP-Systeme werden mobil}: ein Blick auf mögliche Anwendungsszenarien und die Anwender-Akzeptanz. Mobile Erp-systeme, Berlin, Alemanha, v. 1, n. 1, p.5-7, 1 jan. 2013. Disponível em: <https://goo.gl/5NnPD2>. Acesso em: 18 set. 2017.
\newline
\newline
\noindent
FOHRHOLZ, Corinna. \textbf{Indikatorbasierte Messung der ERP-Usability}. Erp Management, Gito Verlag, Berlim, v. 10, n. 4, p.25-28, abr. 2014.
\newline
\newline
\noindent
FREITAS, Henrique, OLIVEIRA, Mírian, SACCOL, Amarolinda Z., MOSCAROLA, Jean,  O método de pesquisa survey. \textbf{Revista de Administração}, São Paulo, v. 35, n. 3, p.105-112, jul. 2000.
\newline
\newline
\noindent
GIOVANELLA, Rafael.\textbf{Processos de aprendizagem em uma comunidade de prática virtual:} Um estudo de caso no grupo de usuários TOTVS RS. 2014. 198 f. Dissertação (Mestrado) - Curso de Pós Graduação em Administração, Universidade de Caxias do Sul, Caxias do Sul, 2014. Disponível em: <https://goo.gl/ZAm81b>. Acesso em: 27 jul. 2018.
\newline
\newline
\noindent
GONÇALVES, Mileni Kazedani. \textbf{Usabilidade de Software:} Estudo de Recomendações Básicas para Verificação do Nível de Conhecimento dos Alunos dos Cursos de Design Gráfico e Sistemas de Informação da UNESP/Bauru. 2009. 238 f. Dissertação (Mestrado) - Curso de Faculdade de Arquitetura, Artes e Comunicação, Desenho Industrial, Unesp, Bauru, 2009.
\newline
\newline
\noindent
GHOSH, Subir. \textbf{Multivariate Analysis, Design of Experiments, and Survey Sampling.}, CRC Press, 1999.
\newline
\newline
\noindent
HIX, D.; HARTSON, H. R. \textbf{Developing User Interfaces:} Ensuring Usability Through Product; Process. New York, NY, USA: John Wiley \& Sons, Inc., 1993. 
\newline
\newline
\noindent
INÁCIO, Livia R.. \textbf{Sistema de Gestão Integrado por Processo de Negócio}. 2017. 239 f. Dissertação (Mestrado) - Curso de Escola de Engenharia de São Carlos, Universidade de São Paulo, São Carlos, 2017.
\newline
\newline
\noindent
KITCHENHAM, Barbara; CHARTERS, Stuart. \textbf{Guidelines for performing systematic literature reviews in software engineering (version 2.3)}. Durham, Uk: Ebse Technical Report, Ebse-2007-01, 2007. 57 p.
\newline
\newline
\noindent
KOVALCZYK, Nelson; KOVALCZYK, Ramaiana Torres Godeia. Retorno sobre Investimento em TI com ênfase em ERP. In: SIMPÓSIO INTERDISCIPLINAR DE TECNOLOGIAS E EDUCAÇÃO, 3o, 2017, Boituva. \textbf{SInTE - Simpósio Interdisciplinar de Tecnologias e Educação}. [s.l.]: [s.n.], 2017?. p. 1 - 7. Disponível em: <http://200.133.214.240/
index.php/SInTE/rt/printerFriendly/332/0>. Acesso em: 29 ago. 2017.
\newline
\newline
\noindent
LAMBECK, Christian et al. (Re-)Evaluating User Interface Aspects in ERP Systems: An Empirical User Study. In: HAWAII INTERNATIONAL CONFERENCE ON SYSTEM SCIENCES, 47., 2014, Waikoloa, Hawaii. \textbf{Proceedings...}. Washington, D.c., Eua: Ieee Computer Society, 2014a. p. 396 - 405.
\newline
\newline
\noindent
LAMBECK, C.; FOHRHOLZ, C.; LEYH, C. Commonalities and Contrasts: An Investigation of ERP Usability in a Comparative User Study. In:  EUROPEAN CONFERENCE ON INFORMATION SYSTEMS, 22nd, 2014, Tel Aviv, Israel. \textbf{Proceedings...} Illinois:AIS Electronic Library, 2014b. p. 1 - 15. 
\newline
\newline
\noindent 
LEYH, Christian; GEBHARDT, Anne; BERTON, Philipp. Implementing ERP Systems in Higher Education Institutes: Critical Success Factors Revisited. In: FEDERATED CONFERENCE ON COMPUTER SCIENCE AND INFORMATION SYSTEMS, 15th, 2017, Prague \textbf{Proceedings...}, Washington, D.c., Eua: Ieee Computer Society, p.913-917, 24 set. 2017. IEEE. http://dx.doi.org/10.15439/2017f364.
\newline
\newline
\noindent
LEYH, Christian. \textbf{ERP systems, SME, Critical success factors, CSF, ERP implementation, Implementation approach, Teaching courses, Curriculum}. 2015. 130 f. Tese (Doutorado) - Curso de Négócios e Economia, Technical University Of Dresden, Dresden, Germany, 2014.
\newline
\newline
\noindent
MACÊDO, David Gradvohl; GAETE, Luciano; JOIA, Luiz Antonio. Antecedentes à Resistência a Sistemas Empresariais: A Perspectiva de Gestores Brasileiros. \textbf{Revista de Administração Contemporânea}, Curitiba, v. 18, n. 2, p.139-160, abr. 2014. Bimestral. FapUNIFESP. Disponível em: <https://doi.org/10.1590/s1415-65552014000200003>. Acesso em: 02 abr. 2019.
\newline
\newline
\noindent
MAGUIRE, Martin. Context of Use within usability activities. \textbf{International Journal Of Human-computer Studies. Amsterdam}, p. 453-483. 24 out. 2001. Disponível em: <https://doi.org/10.1006/ijhc.2001.0486>. Acesso em: 22 maio 2019.
\newline
\newline
\noindent
MAYHEW, Deborah J. \textbf{The Usability Engineering Lifecyle:} a practitioner's handbook for user interface design. Morgan Kaufmann, 1999.
\newline
\newline
\noindent
MELO, Espedito L. P. \textbf{Boa Usabilidade e comunicação eficiente de tarefas}: dois aliados na execução de processos em sistemas integrados de gestão. 2015. 152 f. Dissertação (Mestrado) - Curso de Ciência da Computação, Centro de Informática, Universidade Federal de Pernambuco, Pernambuco, 2015.
\newline
\newline
\noindent
MEIRELLES, Fernando S. \textbf{Pesquisa Anual do Uso de TI}. São Paulo: Fgv Gvcia, 2017. 24 slides, color. Disponível em: http://eaesp.fgv.br/sites/eaesp.fgv.br/files/pesti2017
gvciappt.pdf. Acesso em: 19 mar. 2017.
\newline
\newline
\noindent
NIELSEN, Jakob.  \textbf{Usability Engineering.} Mountain View. California, Ed. Academic Express, 1993. Acesso em 24 mar. 2018.
\newline
\newline
\noindent
NIELSEN, Jakob.  \textbf{10 Usability Heuristics for User Interface Design.} 1995a. Disponível em: http://www.nngroup.com/articles/ten-usability-heuristics/. Acesso em 24 mar. 2018.
\newline
\newline
\noindent
NIELSEN, Jakob.  \textbf{How to Conduct a Heuristic Evaluation.} 1995b. Disponível em: http:// www.nngroup.com/articles/how-to-conduct-a-heuristic-evaluation/. Acesso em 24 mar. 2018.
\newline
\newline
\noindent
NIELSEN, Jakob.  \textbf{Severity Ratings for Usability Problems. 1995c.} Disponível em: http://www.nngroup.com/articles/how-to-rate-the-severity-of-usability-problems/. \newline Acesso em 24 mar. 2018.
\newline
\newline
\noindent
NIELSEN, Jakob.  \textbf{The Use and Misuse of Focus Groups.} 1997. Disponível em: http:// www.nngroup.com/articles/focus-groups/. Acesso em 24 mar. 2018.
\newline
\newline
\noindent
NIELSEN, Jakob.  \textbf{Interviewing Users.} 2010. Disponível em: http://www.nngroup.com/ articles/interviewing-users/. Acesso em 24 mar. 2018.
\newline
\newline
\noindent
NIELSEN, Jakob. \textbf{Usability 101}: Introduction to Usability. 2012. Disponível em:
http://
www.nngroup.com/articles/usability-101-introduction-to-usability/. Acesso em
24 mar. 2018.
\newline
\newline
\noindent
OLIVEIRA, Maria B. A., \textbf{Usabilidade e Qualidade da Informação}: Avaliação do Portal do Aluno da Universidade Federal do Espirito Santo, 2014, 142 fl.
\newline
\newline
\noindent
PARKS, Nancy. E. Testing \& quantifying ERP usability. \textbf{Proceedings of the 1st Annual Conference on Research in Information Technology}, Calgary, Alberta, p.31–36, 2012, Canada: ACM Press.
\newline
\newline
\noindent
PREARO, L. C.; GOUVÊA, M. A.; MONARI, C. Avaliação do emprego da técnica de análise discriminante em teses e dissertações de algumas instituições de ensino superior. \textbf{Revista de Administração FACES Journal}, v. 9, n. 1, art. 19, p. 129-147, 2010.
\newline
\newline
\noindent
ROGERS, Y.; SHARP, H.; PREECE, J. \textbf{Interaction Design:} Beyond Human - Computer Interaction. Wiley, 2011. (Interaction Design: Beyond Human-computer Interaction). ISBN 9780470665763. Disponível em:
http://books.google.com.br/books?id=b-v\_6BeCwwQC.
\newline
\newline
\noindent
SADIQ, Mazhar; PIRHONEN, Antti. Usability in ERP: (Single and Multiple Document Interface) Application Environments. \textbf{International Journal Of Business: Humanities and Technology}. [s.l], p. 75-80. jul. 2014. Disponível em: <http://www.ijbhtnet.com/
journals/Vol\_4\_No\_4\_July\_2014/8.pdf>. Acesso em: 30 dez. 2017.
\newline
\newline
\noindent
SCHOLTZ, Brenda M; MAHMUD, Imran; T., Ramayah. Does Usability Matter? An Analysis of the Impact of Usability on Technology Acceptance in ERP Settin. \textbf{Interdisciplinary Journal Of Information, Knowledge, And Management}, [s.l.], v. 11, n. 1, p.309-330, 2016. Informing Science Institute. http://dx.doi.org/10.28945/3591.
\newline
\newline
\noindent
SINHORINI, Marcelo. \textbf{Pesquisa Panorama Mercado ERP 2016}, 2ª edição, São Paulo: Portal do ERP, 2017. 49 slides, color. Disponível em: https://goo.gl/BASpLW, Acesso em: 19 mar. 2017.
\newline
\newline
\noindent
SCHILIT, Bill; ADAMS, Norman; WANT, Roy. \textbf{Context-aware computing applications}. In: First International Workshop on Mobile Computing Systems and Applications, p.85-90, 1994.
\newline
\newline
\noindent
STABILE, Samuel. \textbf{Um estudo sobre a desconexão entre usuários e desenvolvedores de sistemas de informação e sua influência na obtenção de informação pelo decisor}. 2001. 163 f. Dissertação (Mestrado) - Curso de Engenharia de Produção, Universidade de São Paulo, São Carlos, 2001.
\newline
\newline
\noindent
SOUZA, Cesar A. \textbf{Sistemas Integrados de Gestão Empresarial}: Estudos de Casos de Implementação de Sistemas ERP. 2000. 306 f. Dissertação (Mestrado) - Curso de Faculdade de Economia, Administração e Contabilidade, Departamento de Administração, Universidade de São Paulo, São Paulo, 2000.
\newline
\newline
\noindent
ŠŪPULNIECE, Inese et al. Monitoring Perceived Usability of ERP Systems in Latvian Medium, Small and Micro Enterprises. \textbf{Information Technology And Management Science}, [s.l.], v. 16, n. 1, p.73-78, 1 jan. 2013. Walter de Gruyter GmbH. http://dx.doi.org/ 10.2478/itms-2013-0011.
\newline
\newline
\noindent
VENEZIANO, Vito et al. Usability analysis of ERP software: Education and experience of users' as moderators. In: INTERNATIONAL CONFERENCE ON SOFTWARE, KNOWLEDGE, INFORMATION MANAGEMENT AND APPLICATIONS, 8th, 2014, Dhaka, Bangladesh. \textbf{Proceedings...}. Washington, D.c., Eua: IEEE Computer Society, 2014. p. 1 - 7. Disponível em: <http://dx.doi.org/10.1109/skima.2014.7083560>. Acesso em: 30 jan. 2017.
\newline
\newline
\noindent
YASSIEN, Eman et al. The Impact of ERP System's Usability on Enterprise Resource Planning Project Implementation Success via the Mediating Role of User Satisfaction. \textbf{Journal Of Management Research}, [s.l.], v. 9, n. 3, p.49-71, 27 jun. 2017. Disponível em: <https://goo.gl/razpoq>. Acesso em: 23 nov. 2017.

\setboolean{isBib}{false} % Desobriga o titulo das referências à esquerda

%\setboolean{isBib}{false} % Desobriga o titulo das referências à esquerda

\apendices
\partapendices

% ----------------------------------------------------------
% Apêndices
% ----------------------------------------------------------
% Cícero-INICIO
%\chapter{Lorem ipsum dolor sit amet}
%\lipsum[21-26]

%\anexos
%\partanexos

% Cícero-FIM
% ----------------------------------------------------------
% Anexos
% ----------------------------------------------------------
% Cícero-INICIO


\chapter{Questionário aplicado}
%\section{Questionário}
\newcommand\Factor{1.9}

\noindent \begin{center}
\textbf{Avaliando a usabilidade dos sistemas ERP} \\	
Disponível em: http://teresios.com.br/pesquisa\_erp/index.php/768929
\end{center}
\\
\\
\noindent Caro participante,\\
\noindent em quase todos os ramos empresariais no Brasil e além, existem barreiras no uso de sistemas Enterprise Resource Planning (ERP). Este estudo visa identificar os problemas específicos que você enfrenta como usuário de destes sistemas. Com a sua participação, poderemos ter uma visão dos seus desafios diários ao interagir com tais sistemas. Talvez você já tenha experimentado inadequação ao usar sistemas ERP, o melhor do que ninguem sabe o que iria contribuir para melhorar a usabilidade. Com sua contribuição, os tais sistemas podem ser melhor personalizados criando assim sistemas corporativos intuitivos.\\
\noindent Obrigado pelo seu apoio!\\

\noindent \textbf{Declaração de Privacidade}\\
\noindent Os dados serão usados apenas anonimamente para garantir a proteção da privacidade (dos dados). As respostas dadas são completamente focadas na pesquisa em si. Garantimos que seus dados não serão encaminhados para nenhuma outra instância. Em caso de dúvidas, entre em contato com o pesquisador via e-mail, telefone ou endereço postal.\\

\noindent \textbf{Dados do Pesquisador}\\
\noindent Cícero Odílio Cruz, (11) 96196-2967, Avenida Professor Almeida Prado, 532 – Cidade Universitária, Butantã, São Paulo - SP - Coordenadoria de Ensino Tecnológico - Prédio 56\\

\noindent  \textbf{Duração da Pesquisa}\\
\noindent 15 Minutos\\

\newpage
\newlength{\tabquestwidth}
\setlength{\tabquestwidth}{7cm}

%\hspace{-2cm}\begin{tabularx}{\textwidth}{|X|}
\noindent \begin{tabularx}{\textwidth}{|X|}
  \hline
  \tabitem \textbf{Perguntas Relativas ao seu Ambiente Empresarial} \\
  \hline
  \begin{Spacing}{0.8} 
  \textbf{Quantos funcionários sua empresa tem?} \end{Spacing} 
  \begin{Spacing}{2.0} 
  	\tiny \textit{Escolha uma das seguintes respostas} \end{Spacing} 
  ( \ ) Até 9\\
  ( \ ) De 10 Até 49\\
  ( \ ) De 50 Até 250\\
  ( \ ) Mais de 250\\
  ( \ ) Eu não sei \\
  \begin{Spacing}{0.8} \end{Spacing}
  \begin{Spacing}{0.8} 
	\textbf{Qual o setor em que sua  empresa atua?} \end{Spacing} 
  \begin{Spacing}{2.0} 
	\tiny \textit{Escolha uma das seguintes respostas} \end{Spacing} 
	( \ ) Industrial \\
	( \ ) Comércio (Atacado ou Varejo) \\
	( \ ) Transporte, Logística ou Armazenamento \\
	( \ ) Informação e Comunicação \\
	( \ ) Atividades Financeiras e de Seguros \\
	( \ ) Atividade Científica ou Pesquisa/Desenvolvimento \\
	( \ ) Prestação de Serviço \\ 
	( \ ) Outro: \colorbox{white}{ ............................................................................................................ } \\
	\begin{Spacing}{0.8} \end{Spacing}
	\begin{Spacing}{0.8} 
		\textbf{Qual a área de atuação de sua empresa?} \end{Spacing} 
	\begin{Spacing}{2.0} 
		\tiny \textit{Selecione todas as que se apliquem} \end{Spacing} 
	( \ ) Municipal \\
	( \ ) Estadual \\
	( \ ) Federal \\
	( \ ) Continental \\
	( \ ) International \\
	( \ ) Eu não sei \\
	( \ ) Outro: \colorbox{white}{ ............................................................................................................ } \\
	\begin{Spacing}{0.8} \end{Spacing}
	\begin{Spacing}{0.8} 
		\textbf{Que posição você ocupa em sua empresa?} \end{Spacing} 
	\begin{Spacing}{2.0} 
		\tiny \textit{Escolha uma das seguintes respostas} \end{Spacing} 
	( \ ) Empregado / Terceirizado \\
	( \ ) Chefe de Departamento / Gerente \\
	( \ ) Diretor \\
	( \ ) CEO - Chief Executive Officer \\
	( \ ) CIO - Chief Information Officer \\
  \hline
\end{tabularx}

\bigskip

%\noindent \begin{tabularx}{\textwidth}{|X|}
\noindent  \begin{longtable}{|p{15.7cm}|}
	\hline
	\tabitem 	\begin{Spacing}{1.1} 
		\textbf{Informações sobre o seu sistema ERP} \end{Spacing} 
		\tiny \textit{Na seção a seguir, gostaríamos de conhecer mais sobre seu sistema ERP...} 
    \hline
	\begin{Spacing}{1.0} 
		\textbf{Sua empresa  usa algum sistema classificado como {\color{blue} \underline{Enterprise Resource Planning} } (ERP)?  } \end{Spacing} 
	\begin{Spacing}{2.0} 
		\tiny \textit{Escolha uma das seguintes respostas} \end{Spacing} 
	( \ ) Sim\\
	( \ ) Não\\
	\begin{Spacing}{0.8} \end{Spacing}
	\textbf{Por favor, informe o nome do seu ERP?} \\
	\colorbox{white}{ ................................................................................................................................ }\\
	\begin{Spacing}{0.8} \end{Spacing}
	\begin{Spacing}{0.8} 
		\textbf{Quando o sistema ERP foi implantado na sua Empresa?} \end{Spacing} 
	\begin{Spacing}{2.0} 
		\tiny \textit{Informe o Ano aproximado de Implantação} \end{Spacing} 
	\colorbox{white}{ .......................... }\\
	\begin{Spacing}{0.8} \end{Spacing}
	\begin{Spacing}{0.8} 
		\textbf{Seu sistema foi adaptado (customizado) nos seguintes pontos?} \end{Spacing} 
	\begin{Spacing}{2.0} 
		\tiny \textit{Selecione todas as que se apliquem} \end{Spacing} 
	( \ ) Sem adaptações/customizações\\
	( \ ) Eu não sei / desconheço\\
	( \ ) Menus\\
	( \ ) Interface gráfica do usuário e formulários (Telas e Relatórios específicos)\\
	( \ ) Integrações (Interfaces para outros sistemas, por exemplo, entrada / saída de dados)\\
	( \ ) Ambientes Inteiros (por exemplo, módulos específicos) \\
	\begin{Spacing}{0.8} \end{Spacing}
	\begin{Spacing}{0.8} 
		\textbf{Quais dos seguintes aplicativos você usa em conjunto com seu sistema ERP?} \end{Spacing} 
	\begin{Spacing}{2.0} 
		\tiny \textit{Selecione todas as que se apliquem} \end{Spacing} 
	( \ ) Pacote Microsoft Office (Ex: Excel, Word) \\
	( \ ) Sistemas criados por Terceiros (sistemas especializados) \\
	( \ ) Sistemas criados internamente (sistemas especializados) \\
	( \ ) Soluções WEB ou APP's \\
	( \ ) Eu não sei \\
	( \ ) Outro: \colorbox{white}{ ............................................................................................................ } \\
	\\
	\\
	\\
	\begin{Spacing}{0.8} \end{Spacing}
	\begin{Spacing}{0.8} 
		\textbf{Quais são os seus motivos para usar este aplicativo adicional (por exemplo, Excel é uma solução "Melhor", etc.)? Por favor, especifique suas respostas preenchendo a grade abaixo:} \end{Spacing} 
	\begin{Spacing}{2.0} 
		\tiny \textit{Escolha uma opção por cada afirmação...} \end{Spacing} 
	\begin{Spacing}{2.0} \end{Spacing}
	\tiny \begin{tabularx}{15.7 cm}{|X|X|X|X|X|X|X|}
		\hline
		& Concordo Completamente &	Concordo Parcialmente &	Não Concordo Nem Discordo &	Discordo  Parcialmente &	Discordo  Completamente &	Eu Não Sei\\
		\hline
		Os sistemas ERP disponíveis não atendem as nossas necessidades. &   &   &   &   &   &  \\
		\hline
		As funções dos sistemas ERP disponíveis é insuficiente para nossas necessidades. &   &   &   &   &   &  \\
		\hline
		Os custos dos módulos/ambientes adicionais e processos de adaptação de sistemas ERP, são muito altos. &   &   &   &   &   &  \\
		\hline
		O aplicativo adicional aumenta significativamente a flexibilidade. &   &   &   &   &   &  \\
		\hline
		Os sistemas ERP disponíveis não atendem as nossas necessidades. &   &   &   &   &   &  \\
		\hline
		Os sistemas ERP disponíveis não atendem as nossas necessidades. &   &   &   &   &   &  \\
		\hline
	\end{tabularx}\\
	\\
	\begin{Spacing}{0.8} \end{Spacing}
	\begin{Spacing}{0.8} 
		\textbf{Em quais departamentos você usa o sistema ERP?} \end{Spacing} 
	\begin{Spacing}{2.0} 
		\tiny \textit{Selecione todas as que se apliquem} \end{Spacing} 
	( \ ) Contabilidade\\
	( \ ) Recursos Humanos (RH)\\
	( \ ) Produção\\
	( \ ) Compras / Gerenciamento da Cadeia de Suprimentos (SCM)\\
	( \ ) Gerenciamento de Projetos\\
	( \ ) Gestão de Documentos\\
	( \ ) Vendas / Customer Relationship Management (CRM)\\
	( \ ) Gerenciamento de Armazenamento e Inventário\\
	\begin{Spacing}{0.8} \end{Spacing}
	\begin{Spacing}{0.8} 
		\textbf{Sua empresa  esta pensando em implantar um sistema ERP?} \end{Spacing} 
	\begin{Spacing}{2.0} 
		\tiny \textit{Escolha uma das seguintes respostas} \end{Spacing} 
	( \ ) Sim\\
	( \ ) Não\\
	( \ ) Eu Não Sei \ Desconheço\\
	\begin{Spacing}{0.8} \end{Spacing}
	\begin{Spacing}{0.8} 
		\textbf{Quais destes Aplicativos você esta usando em substituição ao sistema ERP?} \end{Spacing} 
	\begin{Spacing}{2.0} 
		\tiny \textit{Selecione todas as que se apliquem} \end{Spacing} 
	( \ ) Pacote Microsoft Office (Ex: Excel, Word) \\
	( \ ) Sistemas criados por Terceiros (sistemas especializados) \\
	( \ ) Sistemas criados internamente (sistemas especializados) \\
	( \ ) Soluções WEB ou APP's \\
	( \ ) Eu não sei \\
	( \ ) Outro: \colorbox{white}{ ............................................................................................................ } \\	
	\begin{Spacing}{0.8} \end{Spacing}
	\begin{Spacing}{0.8} 
		\textbf{Para quais departamentos corporativos você usa esse aplicativo alternativo?} \end{Spacing} 
	\begin{Spacing}{2.0} 
		\tiny \textit{Selecione todas as que se apliquem} \end{Spacing} 
	( \ ) Contabilidade \\
	( \ ) Recursos Humanos (RH) \\
	( \ ) Produção  \\
	( \ ) Compras / Gerenciamento da Cadeia de Suprimentos (SCM)  \\
	( \ ) Gerenciamento de Projetos  \\
	( \ ) Gestão de Documentos  \\
	( \ ) Vendas / Customer Relationship Management (CRM) \\
	( \ ) Gerenciamento de Armazenamento e Inventário \\
	( \ ) Outro: \colorbox{white}{ ............................................................................................................ } \\
	\begin{Spacing}{0.8} \end{Spacing}
	\begin{Spacing}{0.8} 
		\textbf{Quais são os seus motivos para não usar um sistema ERP? Por favor, avalie as declarações de acordo com na tabela a seguir:} \end{Spacing} 
	\begin{Spacing}{2.0} 
	\tiny \textit{Escolha uma opção por cada afirmação...} \end{Spacing} 
	\begin{Spacing}{2.0} \end{Spacing}
	\tiny \begin{tabularx}{15.7 cm}{|X|X|X|X|X|X|X|}
		\hline
		& {\rotatebox[origin=c]{90}{\parbox[c]{2.5cm}{\centering \textcolor{white}{.}\newline \medskip Concordo Completamente}}} 
		& {\rotatebox[origin=c]{90}{\parbox[c]{2.5cm}{\centering \textcolor{white}{.}\newline \medskip Concordo Parcialmente}}} 
		& {\rotatebox[origin=c]{90}{\parbox[c]{2.5cm}{\centering \textcolor{white}{.}\newline \medskip Não Concordo \newline Nem Discordo}}}	 
		& {\rotatebox[origin=c]{90}{\parbox[c]{2.5cm}{\centering \textcolor{white}{.}\newline \medskip Discordo  Parcialmente }}} 
		& {\rotatebox[origin=c]{90}{\parbox[c]{2.5cm}{\centering \textcolor{white}{.}\newline \medskip Discordo  Completamente  }}}
		& {\rotatebox[origin=c]{90}{\parbox[c]{2.5cm}{\centering \textcolor{white}{.}\newline  \textcolor{white}{.}\newline Eu Não Sei }}} \\
		\hline
		Os sistemas ERP existentes são inadequados para pequenas e médias empresas (PME). &   &   &   &   &   &  \\
		\hline
		Os sistemas ERP existentes não atendem a nossa linha de negócios. &   &   &   &   &   &  \\
		\hline
		As funções dos sistemas ERP padrão existentes são inadequadas. &   &   &   &   &   &  \\
		\hline
		Os custos de aquisição de um sistema ERP são muito altos. &   &   &   &   &   &  \\
		\hline
		Os custos de implantação de um sistema ERP são muito altos. &   &   &   &   &   &  \\
		\hline
		A implantação de um sistema ERP traria apenas pequenas melhorias. &   &   &   &   &   &  \\
		\hline
		O tamanho da empresa é muito pequeno para um sistema ERP padrão. &   &   &   &   &   &  \\
		\hline
	\end{tabularx}\\
	\tiny \begin{tabularx}{15.7 cm}{|X|X|X|X|X|X|X|}
	\hline
		& {\rotatebox[origin=c]{90}{\parbox[c]{2.5cm}{\centering \textcolor{white}{.}\newline \medskip Concordo Completamente}}} 
		& {\rotatebox[origin=c]{90}{\parbox[c]{2.5cm}{\centering \textcolor{white}{.}\newline \medskip Concordo Parcialmente}}} 
		& {\rotatebox[origin=c]{90}{\parbox[c]{2.5cm}{\centering \textcolor{white}{.}\newline \medskip Não Concordo \newline Nem Discordo}}}	 
		& {\rotatebox[origin=c]{90}{\parbox[c]{2.5cm}{\centering \textcolor{white}{.}\newline \medskip Discordo  Parcialmente }}} 
		& {\rotatebox[origin=c]{90}{\parbox[c]{2.5cm}{\centering \textcolor{white}{.}\newline \medskip Discordo  Completamente  }}}
		& {\rotatebox[origin=c]{90}{\parbox[c]{2.5cm}{\centering \textcolor{white}{.}\newline  \textcolor{white}{.}\newline Eu Não Sei }}} \\
		\hline
		Há uma incerteza na escolha de um sistema ERP apropriado. &   &   &   &   &   &  \\
		\hline
		Usar um sistema ERP é muito complicado. &   &   &   &   &   &  \\
		\hline
	\end{tabularx}\\
	\begin{Spacing}{1.0} \end{Spacing}
	\begin{Spacing}{0.8} 
		\textbf{Aqui você pode mencionar outras razões:} \end{Spacing} 
	\begin{Spacing}{2.5} \end{Spacing} 
	\colorbox{white}{ .................................................................................................................................. } \\
	\colorbox{white}{ .................................................................................................................................. } \\
	\colorbox{white}{ .................................................................................................................................. } \\
	\colorbox{white}{ .................................................................................................................................. } \\
	\hline
%\end{tabularx}
\end{longtable}

\bigskip

\noindent  \begin{longtable}{|p{15.7cm}|}
%\noindent \begin{tabularx}{\textwidth}{|X|}
	\hline
	\tabitem \textbf{Perguntas Relativas a Usabilidade} \\
	\hline
	%\begin{Spacing}{0.1} \end{Spacing} \\
	\begin{Spacing}{1.8} 
	\parbox[c]{1em}{\includegraphics[width=6in]{mens}} 
	\end{Spacing} 
	\begin{Spacing}{1.2} 
		\hspace{6} Figura 1. Tipos de Menu 
		\end{Spacing}
	\tiny \hspace{6} Imagem de C. Lambeck (imagem cedida pelo autor). \\
	%\vspace{-1} imagem cedida pelo autor \\ 
	%\tiny Tipos de Menu - Lambeck, (2014c), imagem cedida pelo autor \\
	\begin{Spacing}{0.8} 
	\textbf{Quais tipos de menu são oferecidos pelo seu sistema? Por favor, escolha os tipos que mais se aproximam do seu. O conteúdo dos menus não importa, pois são só um exemplo.} \end{Spacing} 	\begin{Spacing}{2.0} 
		\tiny \textit{Selecione todas as que se apliquem} \end{Spacing} 
	( \ ) 1 \\
	( \ ) 2 \\
	( \ ) 3 \\
	( \ ) 4 \\
	( \ ) 5 \\
	( \ ) 6 \\
	( \ ) Eu Não Sei \\
	\begin{Spacing}{0.8} \end{Spacing}
	\begin{Spacing}{0.8} 
		\textbf{Você conhece plenamente todas as etapas necessárias do processo para realizar suas tarefas (por exemplo, realizar uma transação, bancária, inserir um pedido, inserir uma Ordem de Produção)?} \end{Spacing} 
	\begin{Spacing}{2.0} 
		\tiny \textit{Escolha uma das seguintes respostas} \end{Spacing} 
	( \ ) Sim, sempre  \\
	( \ ) Sim, na maioria das vezes  \\
	( \ ) De vez em quando eu me sinto inseguro  \\
	( \ ) Não, a maioria das vezes eu me sinto inseguro  \\
	( \ ) Não, estou constantemente inseguro \\
	( \ ) Eu não sei \\
	( \ ) Outro: \colorbox{white}{ ............................................................................................................ } \\
	\begin{Spacing}{0.8} \end{Spacing}
	\begin{Spacing}{0.8} 
		\textbf{Por favor, avalie seu sistema ERP de acordo com a escala na tabela a seguir:} \end{Spacing} 
	\begin{Spacing}{2.0} 
		\tiny \textit{Escolha uma opção por cada afirmação...} \end{Spacing} 
	\begin{Spacing}{2.0} \end{Spacing}
	\tiny \begin{tabularx}{15.7 cm}{|X|X|X|X|X|X|X|}
		\hline
		& {\rotatebox[origin=c]{90}{\parbox[c]{2.5cm}{\centering \textcolor{white}{.}\newline \medskip Concordo Completamente}}} 
		& {\rotatebox[origin=c]{90}{\parbox[c]{2.5cm}{\centering \textcolor{white}{.}\newline \medskip Concordo Parcialmente}}} 
		& {\rotatebox[origin=c]{90}{\parbox[c]{2.5cm}{\centering \textcolor{white}{.}\newline \medskip Não Concordo \newline Nem Discordo}}}	 
		& {\rotatebox[origin=c]{90}{\parbox[c]{2.5cm}{\centering \textcolor{white}{.}\newline \medskip Discordo  Parcialmente }}} 
		& {\rotatebox[origin=c]{90}{\parbox[c]{2.5cm}{\centering \textcolor{white}{.}\newline \medskip Discordo  Completamente  }}}
		& {\rotatebox[origin=c]{90}{\parbox[c]{2.5cm}{\centering \textcolor{white}{.}\newline  \textcolor{white}{.}\newline Eu Não Sei }}} \\
		\hline
		O meu sistema ERP oferece uma ampla gama de funcionalidades de suporte para lidar com problemas. (por exemplo, explicar causas, oferecer soluções, assistência) &   &   &   &   &   &  \\
		\hline
	\end{tabularx}\\
	\tiny \begin{tabularx}{15.7 cm}{|X|X|X|X|X|X|X|}
		\hline
		& {\rotatebox[origin=c]{90}{\parbox[c]{2.5cm}{\centering \textcolor{white}{.}\newline \medskip Concordo Completamente}}} 
		& {\rotatebox[origin=c]{90}{\parbox[c]{2.5cm}{\centering \textcolor{white}{.}\newline \medskip Concordo Parcialmente}}} 
		& {\rotatebox[origin=c]{90}{\parbox[c]{2.5cm}{\centering \textcolor{white}{.}\newline \medskip Não Concordo \newline Nem Discordo}}}	 
		& {\rotatebox[origin=c]{90}{\parbox[c]{2.5cm}{\centering \textcolor{white}{.}\newline \medskip Discordo  Parcialmente }}} 
		& {\rotatebox[origin=c]{90}{\parbox[c]{2.5cm}{\centering \textcolor{white}{.}\newline \medskip Discordo  Completamente  }}}
		& {\rotatebox[origin=c]{90}{\parbox[c]{2.5cm}{\centering \textcolor{white}{.}\newline  \textcolor{white}{.}\newline Eu Não Sei }}} \\
		\hline
		O meu sistema ERP é muito complexo, o que muitas vezes me faz sentir perdido. &   &   &   &   &   &  \\
		\hline
		A quantidade de informações e detalhes fornecidos é alta para as minhas necessidades. &   &   &   &   &   &  \\
		\hline
		O meu sistema ERP oferece inúmeras e úteis visualizações, as quais eu posso escolher. (por exemplo, tabelas, diagramas, dashboards, organogramas ...) &   &   &   &   &   &  \\
		\hline
		O meu sistema ERP abre muitas janelas ou visualizações simultaneamente o qe prejudica minha compreensão do sistema. &   &   &   &   &   &  \\
		\hline
	\end{tabularx}\\
	\begin{Spacing}{0.8} \end{Spacing}
	\begin{Spacing}{0.8} 
		\textbf{Avalie seu software adicional (por exemplo, Excel ou solução de mercado, etc.) com a ajuda da grade a seguir:} \end{Spacing} 
	\begin{Spacing}{2.0} 
		\tiny \textit{Escolha uma opção por cada afirmação...} \end{Spacing} 
	\begin{Spacing}{2.0} \end{Spacing}
	\tiny \begin{tabularx}{15.7 cm}{|X|X|X|X|X|X|X|}
	\hline
	& {\rotatebox[origin=c]{90}{\parbox[c]{2.5cm}{\centering \textcolor{white}{.}\newline \medskip Concordo Completamente}}} 
	& {\rotatebox[origin=c]{90}{\parbox[c]{2.5cm}{\centering \textcolor{white}{.}\newline \medskip Concordo Parcialmente}}} 
	& {\rotatebox[origin=c]{90}{\parbox[c]{2.5cm}{\centering \textcolor{white}{.}\newline \medskip Não Concordo \newline Nem Discordo}}}	 
	& {\rotatebox[origin=c]{90}{\parbox[c]{2.5cm}{\centering \textcolor{white}{.}\newline \medskip Discordo  Parcialmente }}} 
	& {\rotatebox[origin=c]{90}{\parbox[c]{2.5cm}{\centering \textcolor{white}{.}\newline \medskip Discordo  Completamente  }}}
	& {\rotatebox[origin=c]{90}{\parbox[c]{2.5cm}{\centering \textcolor{white}{.}\newline  \textcolor{white}{.}\newline Eu Não Sei }}} \\
	\hline
	O meu sistema adicional oferece uma ampla gama de funcionalidades de suporte para lidar com problemas (por exemplo, explicar causas, oferecer soluções, assistência) &   &   &   &   &   &  \\
	\hline
	O meu sistema adicional é muito complexo, o que muitas vezes me faz sentir perdido. &   &   &   &   &   &  \\
	\hline
	A quantidade de informações e detalhes fornecidos é alta para as minhas necessidades. &   &   &   &   &   &  \\
	\hline
	O meu sistema adicional oferece inúmeras e úteis visualizações, as quais eu posso escolher (por exemplo, tabelas, diagramas, dashboards, organogramas...) &   &   &   &   &   &  \\
	\hline
	\end{tabularx}\\
	\tiny \begin{tabularx}{15.7 cm}{|X|X|X|X|X|X|X|}
	\hline
	& {\rotatebox[origin=c]{90}{\parbox[c]{2.5cm}{\centering \textcolor{white}{.}\newline \medskip Concordo Completamente}}} 
	& {\rotatebox[origin=c]{90}{\parbox[c]{2.5cm}{\centering \textcolor{white}{.}\newline \medskip Concordo Parcialmente}}} 
	& {\rotatebox[origin=c]{90}{\parbox[c]{2.5cm}{\centering \textcolor{white}{.}\newline \medskip Não Concordo \newline Nem Discordo}}}	 
	& {\rotatebox[origin=c]{90}{\parbox[c]{2.5cm}{\centering \textcolor{white}{.}\newline \medskip Discordo  Parcialmente }}} 
	& {\rotatebox[origin=c]{90}{\parbox[c]{2.5cm}{\centering \textcolor{white}{.}\newline \medskip Discordo  Completamente  }}}
	& {\rotatebox[origin=c]{90}{\parbox[c]{2.5cm}{\centering \textcolor{white}{.}\newline  \textcolor{white}{.}\newline Eu Não Sei }}} \\
	\hline
	O meu sistema adicional abre muitas janelas ou visualizações simultaneamente o qe prejudica minha compreenssão do sistema. &   &   &   &   &   &  \\
	\hline
	\end{tabularx}\\
	\\
	\begin{Spacing}{0.8} \end{Spacing}
	\begin{Spacing}{0.8} 
		\textbf{Como você executa funções (por exemplo, iniciando incluir um pedido)?} \end{Spacing} 
	\begin{Spacing}{2.0} 
		\tiny \textit{Selecione todas as que se apliquem} \end{Spacing} 
	( \ ) Palavra-Chave ou Atalho \\
	( \ ) Tecla de Função do Teclado (por exemplo, F2) \\
	( \ ) Cursor do mouse e clique do mouse \\
	( \ ) Selecione a função, seguida do comando de execução \\
	( \ ) Eu Não Sei \\
	( \ ) Outro: \colorbox{white}{ ............................................................................................................ } \\
	\begin{Spacing}{0.8} \end{Spacing}
	\begin{Spacing}{0.8} 
	\textbf{Você está sempre ciente das consequências de suas ações? (por exemplo, alterações no sistema resultantes, efeitos colaterais, operações de fluxo de trabalho afetadas)} \end{Spacing} 
	\begin{Spacing}{2.0} 
		\tiny \textit{Escolha uma das seguintes respostas} \end{Spacing} 
	( \ ) Sim, Sempre \\
	( \ ) Sim, a Maioria das Vezes\\
	( \ ) De Vez em Quando eu me Sinto Inseguro\\
	( \ ) Não, a Maioria das Vezes eu me Sinto Inseguro\\
	( \ ) Não, estou Constantemente Inseguro\\
	( \ ) Eu Não Sei \\
	\begin{Spacing}{0.8} \end{Spacing}
	\begin{Spacing}{0.8} 
		\textbf{Qual é o seu método preferido para pesquisar informações?} \end{Spacing} 
	\begin{Spacing}{2.0} 
		\tiny \textit{Selecione todas as que se apliquem} \end{Spacing} 
	( \ ) Pesquisa de texto completo (por exemplo, termos, IDs) \\
	( \ ) Auto-completar \\
	( \ ) Sinônimos e correção automática (opção do Google “Você quis dizer…?”) \\
	( \ ) Registrar e indexar (por exemplo, listagens de categorias em ordem alfabética) \\
	( \ ) Eu Não Sei \\
	( \ ) Outro: \colorbox{white}{ ............................................................................................................ } \\
	\begin{Spacing}{0.8} \end{Spacing}
	\begin{Spacing}{0.8} 
		\textbf{Como você avalia as seguintes estratégias para lidar com problemas no uso do sistema ERP?} \end{Spacing} 
	\begin{Spacing}{2.0} 
		\tiny \textit{Escolha uma opção por cada afirmação...} \end{Spacing} 
	\begin{Spacing}{2.0} \end{Spacing}
	\tiny \begin{tabularx}{15.7 cm}{|X|X|X|X|X|X|X|}
		\hline
		& {\rotatebox[origin=c]{90}{\parbox[c]{2.5cm}{\centering \textcolor{white}{.}\newline \medskip Concordo Completamente}}} 
		& {\rotatebox[origin=c]{90}{\parbox[c]{2.5cm}{\centering \textcolor{white}{.}\newline \medskip Concordo Parcialmente}}} 
		& {\rotatebox[origin=c]{90}{\parbox[c]{2.5cm}{\centering \textcolor{white}{.}\newline \medskip Não Concordo \newline Nem Discordo}}}	 
		& {\rotatebox[origin=c]{90}{\parbox[c]{2.5cm}{\centering \textcolor{white}{.}\newline \medskip Discordo  Parcialmente }}} 
		& {\rotatebox[origin=c]{90}{\parbox[c]{2.5cm}{\centering \textcolor{white}{.}\newline \medskip Discordo  Completamente  }}}
		& {\rotatebox[origin=c]{90}{\parbox[c]{2.5cm}{\centering \textcolor{white}{.}\newline  \textcolor{white}{.}\newline Eu Não Sei }}} \\
		\hline
		Nível Configurável de Detalhes da Informação &   &   &   &   &   &  \\
		\hline
		Quantidade Configurável de Informação &   &   &   &   &   &  \\
		\hline
		Amplas Formas de Visualizações &   &   &   &   &   &  \\
		\hline
		Feedback: Visual, Tátil ou Auditivo &   &   &   &   &   &  \\
		\hline
		Orientação e Suporte ao Usuário (por exemplo, informação de progresso, indicação de campos de obrigatórios nos formulários, alternativas de fluxo a seguir)  &   &   &   &   &   &  \\
		\hline
		Tipos de Menu e Estruturas Aprimorados &   &   &   &   &   &  \\
		\hline
		Suporte a Dispositivos Sensíveis ao Toque (por exemplo, multi-touch, sistema de mesa) &   &   &   &   &   &  \\
		\hline
	\end{tabularx}\\
	\tiny \begin{tabularx}{15.7 cm}{|X|X|X|X|X|X|X|}
		\hline
		& {\rotatebox[origin=c]{90}{\parbox[c]{2.5cm}{\centering \textcolor{white}{.}\newline \medskip Concordo Completamente}}} 
		& {\rotatebox[origin=c]{90}{\parbox[c]{2.5cm}{\centering \textcolor{white}{.}\newline \medskip Concordo Parcialmente}}} 
		& {\rotatebox[origin=c]{90}{\parbox[c]{2.5cm}{\centering \textcolor{white}{.}\newline \medskip Não Concordo \newline Nem Discordo}}}	 
		& {\rotatebox[origin=c]{90}{\parbox[c]{2.5cm}{\centering \textcolor{white}{.}\newline \medskip Discordo  Parcialmente }}} 
		& {\rotatebox[origin=c]{90}{\parbox[c]{2.5cm}{\centering \textcolor{white}{.}\newline \medskip Discordo  Completamente  }}}
		& {\rotatebox[origin=c]{90}{\parbox[c]{2.5cm}{\centering \textcolor{white}{.}\newline  \textcolor{white}{.}\newline Eu Não Sei }}} \\
		\hline
		Funcionalidade de Pesquisa Avançada &   &   &   &   &   &  \\
		\hline
	\end{tabularx}\\
	\begin{Spacing}{1.0} \end{Spacing}
	\begin{Spacing}{0.8} 
	\textbf{Aqui você pode mencionar outras razões:} \end{Spacing} 
	\begin{Spacing}{2.5} \end{Spacing} 
	\colorbox{white}{ .................................................................................................................................. } \\
	\colorbox{white}{ .................................................................................................................................. } \\
	\colorbox{white}{ .................................................................................................................................. } \\
	\colorbox{white}{ .................................................................................................................................. } \\
	\begin{Spacing}{0.4} \end{Spacing}
	\begin{Spacing}{1.4} 
		\textbf{ O conceito a seguir usa uma lista de materiais interativa e baseada em gráficos (BOM). Suporta:}  \end{Spacing} 	
	\begin{Spacing}{1.4} 
		\textbf{• ilustração de descartabilidades e situações problemáticas (código de cores);} \end{Spacing} 	
	\begin{Spacing}{1.4} 
		\textbf{• acesso ao formulário ERP associado clicando no nó;} \end{Spacing} 	
	\begin{Spacing}{1.4} 
		\textbf{• diferentes níveis de detalhes e aspectos (à esquerda: tempo, no meio: as etapas de produção discretos, à direita: custos);} \end{Spacing} 	
	\begin{Spacing}{1.4} 
		\textbf{Por favor, avalie sua primeira impressão do conceito apresentado em relação ao seu grau de inovação e a utilidade esperada.} \end{Spacing} 	
	\begin{Spacing}{1.4} 
	\tiny \textit{Selecione uma opção por afirmação} \end{Spacing} 
	\begin{Spacing}{2.5} 
	\parbox[c]{1em}{\includegraphics[width=6in]{CENTERIS_concept}} 
	\end{Spacing} 
	\begin{Spacing}{1.2} 
		\hspace{6} Figura 2. Modelo BOM 
	\end{Spacing}
	\tiny \hspace{6} Imagem de C. Lambeck (imagem cedida pelo autor). \\
	%\vspace{-1} imagem cedida pelo autor \\ 
	%\tiny Tipos de Menu - Lambeck, (2014c), imagem cedida pelo autor \\
	\tiny \begin{tabularx}{15.7 cm}{|X|X|X|X|X|X|X|X|}
	\hline
	          & Muito Boa & Boa  & Suficiente & Insuficiente & Ruim & Muito Ruim & Eu Não Sei\\
	\hline
	Inovação  &   &   &   &   &  &  &  \\
		\hline
	Utilidade &   &   &   &   &  &  &  \\
		\hline
	\end{tabularx} \\
	\begin{Spacing}{1.0} \end{Spacing}
	\begin{Spacing}{0.8} 
		\textbf{Outras observações sobre esta questão:} \end{Spacing} 
	\begin{Spacing}{2.5} \end{Spacing} 
	\colorbox{white}{ .................................................................................................................................. } \\
	\colorbox{white}{ .................................................................................................................................. } \\
	\colorbox{white}{ .................................................................................................................................. } \\
	\colorbox{white}{ .................................................................................................................................. } \\
	\begin{Spacing}{0.4} \end{Spacing}
	\begin{Spacing}{1.4} 
		\textbf{O conceito a seguir suporta o planejamento de produção utilizando:}  \end{Spacing} 	
	\begin{Spacing}{1.4} 
		\textbf{• um sistema de mesa;} \end{Spacing} 	
	\begin{Spacing}{1.4} 
		\textbf{• visão superior em máquinas ou bancadas de trabalho e seus fluxos de materiais;} \end{Spacing} 	
	\begin{Spacing}{1.4} 
		\textbf{• gráficos de Gantt relacionados e interativos;} \end{Spacing} 	
	\begin{Spacing}{1.4} 
		\textbf{Cada entrada e seleção de usuário será aplicada por toque direto ou objetos tangíveis (por exemplo, dados vermelhos na imagem à direita). Além disso, um esquema de cores facilita a legibilidade.} \end{Spacing} 	
	\begin{Spacing}{1.4} 
		\textbf{Por favor, avalie sua primeira impressão do conceito apresentado em relação ao seu grau de inovação e a utilidade esperada.} \end{Spacing} 
	\begin{Spacing}{1.4} 
		\tiny \textit{Selecione uma opção por afirmação} \end{Spacing} 
	\begin{Spacing}{2.5} 
		\parbox[c]{1em}{\includegraphics[width=6in]{AVI_concept}} 
	\end{Spacing} 
	\begin{Spacing}{1.2} 
		\hspace{6} Figura 3. Conceito AVI  
	\end{Spacing}
	\tiny \hspace{6} Imagem de C. Lambeck (imagem cedida pelo autor). \\
	%\vspace{-1} imagem cedida pelo autor \\ 
	%\tiny Tipos de Menu - Lambeck, (2014c), imagem cedida pelo autor \\
	\tiny \begin{tabularx}{15.7 cm}{|X|X|X|X|X|X|X|X|}
	\hline
	& Muito Boa & Boa  & Suficiente & Insuficiente & Ruim & Muito Ruim & Eu Não Sei\\
	\hline
	Inovação  &   &   &   &   &  &  &  \\
	\hline
	Utilidade &   &   &   &   &  &  &  \\
	\hline
\end{tabularx} \\
\begin{Spacing}{1.0} \end{Spacing}
\begin{Spacing}{0.8} 
	\textbf{Outras observações sobre esta questão:} \end{Spacing} 
\begin{Spacing}{2.5} \end{Spacing} 
\colorbox{white}{ .................................................................................................................................. } \\
\colorbox{white}{ .................................................................................................................................. } \\
\colorbox{white}{ .................................................................................................................................. } \\
\hline
%\end{tabularx}
\end{longtable}

\bigskip

%\noindent \begin{tabularx}{\textwidth}{|X|}
\noindent  \begin{longtable}{|p{15.7cm}|}
	\hline
	\tabitem \textbf{Acesso ao Sistema} \\
	\hline
	\begin{Spacing}{0.8} 
	\textbf{Como você acessa seu sistema ERP?} \end{Spacing} 
	\begin{Spacing}{2.0} 
		\tiny \textit{Escolha uma das seguintes respostas} \end{Spacing} 
	( \ ) Navegador (por exemplo, Firefox, Internet Explorer, Chrome) \\
	( \ ) Cliente Desktop (instalação do PC)\\
	( \ ) Ambos \\
	( \ ) Outro: \colorbox{white}{ ............................................................................................................ } \\
	\begin{Spacing}{0.8} \end{Spacing}
	\begin{Spacing}{0.8} 
	\textbf{Como você avalia essa maneira de acessar o sistema ERP em geral? (por exemplo, ocorrência de problemas, preferência pessoal, frequência de atualização)} \end{Spacing} 
	\begin{Spacing}{2.0} 
		\tiny \textit{Escolha uma das seguintes respostas} \end{Spacing} 
	( \ ) Muito Boa \\
	( \ ) Boa \\
	( \ ) Indiferente \\
	( \ ) Ruim \\
	( \ ) Muito Ruim \\
	( \ ) Outro: \colorbox{white}{ ............................................................................................................ } \\
	\\
	\begin{Spacing}{0.8} \end{Spacing}
	\begin{Spacing}{0.8} 
		\textbf{Por favor, nomeie as dificuldades ao acessar o sistema da maneira indicada.?} \end{Spacing} 
	\begin{Spacing}{2.0} 
		\tiny \textit{} \end{Spacing} 
	\colorbox{white}{ .................................................................................................................................. } \\
	\colorbox{white}{ .................................................................................................................................. } \\
	\colorbox{white}{ .................................................................................................................................. } \\
	\colorbox{white}{ .................................................................................................................................. } \\
	\begin{Spacing}{0.8} \end{Spacing}
	\begin{Spacing}{0.8} 
		\textbf{Você usa  quais dos seguintes dispositivos móveis para acessar as informações da sua empresa?} \end{Spacing} 
	\begin{Spacing}{2.0} 
		\tiny \textit{Selecione todas as que se apliquem} \end{Spacing} 
	( \ ) Nenhum  \\
	( \ ) Laptop  \\
	( \ ) NetBook  \\
	( \ ) Tablet  \\
	( \ ) Smartphone  \\
	( \ ) PDA/Handheld \\
	( \ ) Outro: \colorbox{white}{ ............................................................................................................ } \\
	\begin{Spacing}{0.8} \end{Spacing}
	\begin{Spacing}{0.8} 
		\textbf{Você usa seu dispositivo móvel para acessar os dados de quais departamentos?} \end{Spacing} 
	\begin{Spacing}{2.0} 
		\tiny \textit{Selecione todas as que se apliquem} \end{Spacing} 
	( \ ) Contabilidade \\
	( \ ) Recursos Humanos (RH) \\
	( \ ) Compras \\
	( \ ) Produção \\
	( \ ) Gerenciamento da Cadeia de Suprimentos (SCM) \\
	( \ ) Gerenciamento de Projetos  \\
	( \ ) Vendas / Customer Relationship Management (CRM) \\
	( \ ) Gerenciamento de Armazenamento e Inventário (WMS) \\
	( \ ) Outro: \colorbox{white}{ ............................................................................................................ } \\
	\hline
%\end{tabularx}
\end{longtable}

\bigskip

%\noindent \begin{tabularx}{\textwidth}{|X|}
\noindent  \begin{longtable}{|p{15.7cm}|}
	\hline
	\tabitem \textbf{Dados Funcionais} \\
	\hline
	\begin{Spacing}{0.8} 
		\textbf{Selecione a Faixa Etária a qual você pertence:} \end{Spacing} 
	\begin{Spacing}{2.0} 
		\tiny \textit{Escolha uma das seguintes respostas} \end{Spacing}
	( \ ) Até 30 Anos \\
	( \ ) 31  Anos até 40  Anos \\
	( \ ) 41  Anos até 50  Anos \\
	( \ ) 51  Anos até 60  Anos \\
	( \ ) 61 ou mais \\
	\begin{Spacing}{0.8} \end{Spacing}
	\begin{Spacing}{0.8} 
		\textbf{Qual seu Sexo:} \end{Spacing} 
	\begin{Spacing}{2.0} 
		\tiny \textit{Escolha uma das seguintes respostas} \end{Spacing} 
	( \ ) Feminino \\
	( \ ) Masculino \\
	\begin{Spacing}{0.8} \end{Spacing}
	\begin{Spacing}{0.8} 
		\textbf{Há quantos anos você trabalha na empresa?} \end{Spacing} 
	\begin{Spacing}{2.0} 
		\tiny \textit{Escolha uma das seguintes respostas} \end{Spacing} 
	( \ ) Menos de 1 ano \\
	( \ ) De 1 á 3 anos \\
	( \ ) De 3 á 6 anos \\
	( \ ) De 7 á 10 anos \\
	( \ ) Mais que 10 anos \\
	\begin{Spacing}{0.8} \end{Spacing}
	\begin{Spacing}{0.8} 
		\textbf{Há quanto tempo você usa Sistemas ERP no Geral?} \end{Spacing} 
	\begin{Spacing}{2.0} 
		\tiny \textit{Escolha uma das seguintes respostas} \end{Spacing} 
	( \ ) Menos de 1 Ano \\
	( \ ) De 1 á 3 Anos \\
	( \ ) De 4 á 6 Anos \\
	( \ ) De 7 á 10 Anos \\
	( \ ) Mais de 10 Anos \\
	\begin{Spacing}{0.8} \end{Spacing}
	\begin{Spacing}{0.8} 
		\textbf{Como você auto avalia sua experiência com sistemas ERP?} \end{Spacing} 
	\begin{Spacing}{2.0} 
		\tiny \textit{Escolha uma das seguintes respostas} \end{Spacing} 
	( \ ) Muito Boa \\
	( \ ) Muito Ruim \\
	( \ ) Indiferente \\
		\begin{Spacing}{0.8} \end{Spacing}
	\begin{Spacing}{0.8} 
		\textbf{Qual sua Região de Origem?} \end{Spacing} 
	\begin{Spacing}{2.0} 
		\tiny \textit{Escolha uma das seguintes respostas} \end{Spacing} 
	( \ ) Norte \\
	( \ ) Nordeste \\
	( \ ) Centro-Oeste \\
	( \ ) Sudeste \\
	( \ ) Sul \\
	\begin{Spacing}{0.8} \end{Spacing}
	\begin{Spacing}{0.8} 
		\textbf{Qual a região do País onde Mora?} \end{Spacing} 
	\begin{Spacing}{2.0} 
		\tiny \textit{Escolha uma das seguintes respostas} \end{Spacing} 
	( \ ) Norte \\
	( \ ) Nordeste \\
	( \ ) Centro-Oeste \\
	( \ ) Sudeste \\
	( \ ) Sul \\
	\begin{Spacing}{0.8} \end{Spacing}
	\begin{Spacing}{0.8} 
		\textbf{Qual dos seguintes dispositivos móveis você está usando no seu tempo livre?} \end{Spacing} 
	\begin{Spacing}{2.0} 
		\tiny \textit{Selecione todas as que se apliquem} \end{Spacing}
	( \ ) Nenhum \\
	( \ ) Notebook \\
	( \ ) Netbook \\
	( \ ) Tablet \\
	( \ ) Smartphone \\
	( \ ) PDA/Handheld \\
	\begin{Spacing}{0.8} \end{Spacing}
	\begin{Spacing}{0.8} 
		\textbf{No seu tempo livre você usa seu PC ou dispositivo móvel para quais fins?} \end{Spacing} 
	\begin{Spacing}{2.0} 
		\tiny \textit{Selecione todas as que se apliquem} \end{Spacing} 
	( \ ) Mídias Sociais e Chat (Facebook, Skype, LinkedIn etc.) \\
	( \ ) PC ou jogos online \\
	( \ ) Processamento de texto e planilha \\
	( \ ) Aplicações multimídia (música, vídeo, imagens) \\
	( \ ) Pesquisa de Informações \\
	( \ ) Outro: \colorbox{white}{ ............................................................................................................ } \\
	\begin{Spacing}{0.8} \end{Spacing}
	\begin{Spacing}{0.8} 
		\textbf{Com que frequência você usa esses dispositivos no seu tempo livre?}
	\end{Spacing} 
	\begin{Spacing}{2.0} 
		\tiny \textit{Escolha uma das seguintes respostas} \end{Spacing}
	( \ ) Menos de uma vez por semana \\
	( \ ) Uma vez por semana \\
	( \ ) Várias vezes por semana \\
	( \ ) Diariamente \\
	( \ ) Várias vezes por dia \\
	\hline
	%\end{tabularx}
\end{longtable}

\bigskip

%\noindent \begin{tabularx}{\textwidth}{|X|}
\noindent  \begin{longtable}{|p{15.7cm}|}
	\hline
	\tabitem \textbf{Dados de Contato} \\
	\hline
	\begin{Spacing}{0.8} 
		\textbf{Obrigado pela sua participação e, assim, apoiar a melhoria dos sistemas ERP! Caso você queira receber os resultados compilados da pesquisa, por favor digite seu endereço de e-mail:} \end{Spacing} \\
	\begin{Spacing}{0.0} 
		\tiny \end{Spacing} 
	\colorbox{white}{ .................................................................................................................................. } 
	\begin{Spacing}{2.0} \end{Spacing}
	\begin{Spacing}{0.8} 
		\textbf{Como agradecimento adicional por sua participação, gostaríamos de enviar uma cópia completa de nossa pesquisa impressa. Se você estiver interessado, digite seu endereço postal:} \end{Spacing} 
	\begin{Spacing}{0.5} \end{Spacing}\\
	\begin{Spacing}{0.5} \end{Spacing}
	Nome Completo: \colorbox{white}{ ..................................................................................................... } \\
	Empresa: \colorbox{white}{ ................................................................................................................ } \\
	Logradouro: \colorbox{white}{ ............................................................................................................ } \\
	Número: \colorbox{white}{ ................................................................................................................. } \\
	CEP: \colorbox{white}{ ....................................................................................................................... } \\
	Cidade: \colorbox{white}{ ................................................................................................................... } \\
	Estado: \colorbox{white}{ ................................................................................................................... } \\
	\\
	\hline
	%\end{tabularx}
\end{longtable}


%\chapter{Mapeamento Sistemático}
\addtocontents{toc}{\protect\setcounter{tocdepth}{0}}
\setcounter{chapter}{1}
%\renewcommand{\thesection}{\arabic{chapter}.\arabic{section}}
\renewcommand{\thesection}{\arabic{section}}
Este anexo tem o objetivo de documentar a execução da metodologia do Mapeamento Sistemático, com o intuito de agregar informações a pesquisa da Dissertação.

\section{Planejamento} %\label{RT - Planejamento}

\subsection{Objetivo}
\newline
\newline
\indent
O objetivo da pesquisa proposta é avaliar a Usabilidade em ferramentas ERP e traçar um perfil do usuário brasileiro comparando-o com o alemão, sob aspecto da usabilidade, definindo assim qual a real importância da usabilidade na escolha e na permanência do sistema ERP na empresa, sendo assim,  por se tratar de um tema amplo esta pesquisa não pretende adentrar nas características especificas de um determinado ERP, mas sim determinar “O quê é importante” do ponto de vista da usabilidade para que um ERP se mantenha na empresa. 
\newline

\subsection{Questão de Pesquisa para Revisão}
\newline
\newline
\noindent Quais e quantos estudos relacionam a Usabilidade com sistemas ERP na pós-implantação do sistema?\newline
Obs: aqui não tratamos de aspectos pois estamos fazendo um mapeamento sistemático das pesquisas relacionadas, buscamos a pós-implantação pois se relaciona com nossa referência de controle;
\newline

\underline{Intervenção}: Impacto da influência da usabilidade na aceitabilidade de sistemas ERP por parte dos usuários no pós-implantação.
\newline

\underline{Controle}: O estudo usado como apoio para iniciar o mapeamento sistemático foi obtido por meio de pesquisa exploratória de artigos publicados em periódicos e/ou anais de eventos, obtidos mediante busca em base de dados eletrônica, utilizando as palavras chave “ERP usability” em (IEEE Digital Library (http://ieeexplore.ieee.org/Xplore/)). Desta forma a referência utilizada como controle é  Lambeck et al (2014b). Este artigo serviu de base para definição das palavras chaves, fontes e período da busca, assim como a validação da abrangência do tema pesquisado.
\newline

\underline{População}: Estudos que relacionam usabilidade com ERP's, restritos a sistemas tradicionais não criados para uso em aparelhos móveis ou na world wide web (web);
\newline

\underline{Resultado}: possibilitar uma visão mais  abrangente do uso da Usabilidade relacionados a avaliação da eficiência do ERP pós implantação
\newline

\underline{Aplicação}: usuários de sistemas ERP na fase de pós-implantação.
\newline

\subsection{Seleção das Fontes}
\newline
\newline
\textbf{Critério de Definição de Fontes}: As fontes devem estar disponíveis via web, preferencialmente em bases científicas da área de computação (bibliotecas digitais online, bases eletrônicas indexadas, anais de eventos da área, periódicos, mapeamentos sistemáticos anteriores).
\newline
\newline
\textbf{Listagem das Fontes}: Estas fontes são reconhecidas mundialmente pela produção literária de alta qualidade considerando as principais revistas literárias e eventos científicos no campo da Engenharia de Software:

\indent a. IEEE Digital Library (http://ieeexplore.ieee.org/Xplore/);\newline
\indent b. Portal Brasileiro de Publicações Ciêntificas em Acesso Aberto (http://oasisbr.ibict. 
br/vufind/);\newline
\indent c. Portal de Publicações da Universidade de Dresden (https://tu-dresden.de/ing/infor
matik/smt/mg/forschung/publikationen);
\newline

\noindent \textbf{Idioma dos Trabalhos}: Inglês, por ser o idioma com maior aceitação internacional para trabalhos científicos, Alemão pela influência da maior empresa do setor (SAP) e também pela origem do trabalho de referência e Português, para que os trabalhos existentes de pesquisadores brasileiros pudessem ser contemplados.
\newline
\textbf{Palavras Chaves}: Palavras chaves em língua portuguesa: “ERP” relacionados com o termo “Usabilidade”. Palavras chaves em língua inglesa: “ERP” ou “Enterprise Resource Planning” relacionado com os termos “Usability”. Palavras chaves em língua alemã: “ERP” ou “ERP-Systemen” relacionado com os termos “Usability” usadas exclusivamente no site da Universidade de Dresden.
\newline
Recorreu-se ao operador lógico “OR” para combinação das palavras chaves. Além do uso dos operadores NEAR e ONEAR no site (a) e o equivalente no site (b) acima descriminados.
\newline
\newline
\subsection{Critérios de Inclusão e Exclusão dos Trabalhos}
\newline
\newline
Foram estabelecidos critérios de inclusão ou exclusão com o objetivo de limitar a seleção de trabalhos com base em avaliações qualitativas relevantes tendo como referência o objetivo do mapeamento sistemático neste trabalho. Partindo deste contexto, os seguintes critérios foram estabelecidos para esta revisão:
\newline
\newline
\noindent \textbf{Critérios de Inclusão}\newline
\newline
\indent a. Trabalhos publicados e disponíveis integralmente na web em formato eletrônico que atendam as strings de busca;\newline
\indent b. Trabalhos publicados no período de 2001 à Julho de 2018, descritos em inglês ou português;\newline
\indent c. Trabalhos publicados no período de 2001 à Julho de 2018, descritos em alemão na Universidade de Dresden;\newline
\indent d. Estudos que relacionam os aspectos de Usabilidade no uso do ERP's.\newline

\noindent \textbf{Critérios de Exclusão}\newline
\newline
\indent a. Trabalhos que apresentam avaliações sem apresentar o método utilizado;\newline 
\indent b. Estudos anteriores a 2013;\newline 
\indent c. Trabalhos que não relacionam Usabilidade com ERP caso venham a ocorrer;\newline 
\indent d. Estudo repetido, quando comparado com o resultado das demais buscas;\newline 
\indent e. Estudo incompleto (texto, conteúdo e resultados incompletos);\newline 
\indent f. Pôsteres, tutoriais, relatórios técnicos;\newline 
\indent g. Relacionar Usabilidade com sistemas erp nas plataformas web ou móveis (app's);\newline 
\indent h. Relacionar Usabilidade em ERP's na fase de implementação ou implantação do sistema;\newline 
\indent i. Relacionar Usabilidade em sistemas CRM, MRP ou outros sistemas que não ERP;\newline
\indent j. Relacionar Usabilidade com ERP em processos de Gamificação;\newline

\noindent \textbf{Definição da Estratégia de Seleção de Dados}\newline

\indent Por meio das palavras chave foram criadas as \textit{strings} de busca, executadas em cada uma das fontes selecionadas. As buscas são realizadas sempre nos campos ( “resumo” (abstract), titulo (title) ) do documento, com extração das seguintes informações: título do documento, autores, fonte, ano de publicação e resumo.\newline

\indent Devido às particularidades das máquinas de busca poderá haver adaptações nas strings, obedecendo as seguintes diretrizes:\newline 
\indent (1) a string derivada deverá ser logicamente equivalente à string original, ou;\newline 
\indent (2) na impossibilidade de se manter equivalência exata, deverá a string derivada ser mais abrangente para evitar perda de documentos potencialmente relevantes.\newline
\indent (3) a string inicial será modificada até que restem somente trabalhos duplicados.\newline

\noindent \underline{Primeira String na Língua Inglesa}: (( Usability AND ERP ) OR ( Usabilidade AND ERP ));\newline

\noindent \underline{Segunda String na Língua Inglesa}: ((usability NEAR/3 ERP) OR (usability ONEAR/3 ERP));\newline

\noindent \underline{Terceira String na Língua Inglesa}: (((usability NEAR/3 ERP) OR (ERP ONEAR/10 usability)));\newline

\noindent \underline{Quarta String na Língua Inglesa}: ((.QT.Enterprise Resource Planning.QT. NEAR/3 usability)  OR (.QT.Enterprise Resource Planning.QT. ONEAR/10 usability) );\newline

\noindent \underline{Quinta String na Língua Inglesa}: ((.QT.Enterprise Resource Planning.QT. NEAR/3 usability)  OR (.QT.Enterprise Resource Planning.QT. ONEAR/10 usability) ) OR ((usability NEAR/3 ERP) OR (ERP ONEAR/10 usability));\newline

\noindent \underline{Sexta String na Língua Inglesa}: (((usability NEAR/3 .QT.Enterprise Resource Planning.QT.) OR (usability ONEAR/10 .QT.Enterprise Resource Planning.QT.) ) OR ((usability NEAR/3 ERP) OR (usability ONEAR/10 ERP)));\newline

\noindent Obs: É importante ressaltar que o uso do NEAR e ONEAR pode ou não ser suportado pela ferramenta de busca, sendo que seu uso facilita a relação entre as strings.\newline

\noindent \underline{Primeira String na Língua Portuguesa}: "(Resumo Português:"usabilidade ERP"~10)";\newline 

\noindent \underline{Segunda String na Língua Portuguesa}: "(Resumo Português:" " "Enterprise Resource"~1 Planning"~1 usabilidade"~10)";\newline

\noindent \underline{String na Língua Alemã}: "( Usability AND ERP-Systemen )";\newline

\noindent Obs: A string na  lingua  alemã foi utilizada somente na  literatura cinzenta “ textos produzidos por todos os níveis do governo, institutos, academias, empresas e indústria, em formato impresso e eletrônico publicado locais específicos não indexados, e que não é controlado por editores, sejam, científicos ou comerciais”.\newline

Os trabalhos recuperados das bases serão documentados em uma  ferramenta especificamente projetada para a Revisão Sistemática e Mapeamento Sistemático, elaborada pelo LaPES da UFSCar denominada StaRt. \newline

A ferramenta StaRt armazena as referências bem como o abstract e  auxilia nos processos de Seleção, Extração e Sumarização. \newline

Após a leitura minuciosa e completa dos trabalhos incluídos, foi elaborado um resumo comparativo das obras, redigido pelo pesquisador, destacando os métodos ou técnicas utilizadas, além dos principais dados da pesquisa, quando for o caso. \newline

\subsection{Sumarização dos Resultados}
\newline
\indent Com os resultados obtidos, deve ser redigida uma síntese geral que descreve sinteticamente as análises críticas elaboradas pelo revisor. Análises qualitativas e quantitativas, com relação aos trabalhos pesquisados e algumas considerações sobre os resultados observados nos trabalhos selecionados.\newline

\section{Condução do Mapeamento Sistemático} \label{RT - Condução do Mapeamento Sistemático}
\newline
\newline
\indent A busca das referências utilizadas nesta revisão foi realizada em bases de dados eletrônicas. Por meio dos critérios de inclusão e exclusão definiram-se os trabalhos incluídos e excluídos da revisão. As referências que preencheram os critérios de inclusão foram avaliadas, independentemente do periódico.\newline
\newline
\indent A primeira coluna das Tabela 3, Tabela 4, Tabela 5 e Tabela 6 contém uma numeração sequencial para os trabalhos. As colunas intituladas “Critérios de Inclusão Atendidos” e “Critérios de Exclusão Atendidos” contêm os critérios atendidos para cada trabalho, descritos no tópico anterior. Por fim, a coluna “Status” contém a avaliação do trabalho, indicando se ele foi excluído da revisão ou se foi incluído para a próxima fase. A existência de um único critério de exclusão atendido torna o arquivo inválido para a seleção final.\newline
\newline
\indent Os artigos base utilizados para auxiliar a definição dos limitadores da busca estão descritos no item 1.2 deste anexo e não foram incluídos no resultado da revisão.\newline
\newline
\indent Os tópicos 2.1, 2.2, 2.3, 2.4, e 2.5 exibem todas as obras retornadas após a realização de cada ciclo de buscas (processo de seleção preliminar) e a sua avaliação após a leitura do resumo.\newline
\newline
\indent Os tópicos referentes a seleção final e a análise de resultados estão descritos no corpo da pesquisa, tópico 2.4 e 2.5 respectivamente.\newline
\newline
\indent Nos resumos comparativos dos trabalhos selecionados (seleção final) são apresentadas as principais diferenças e semelhanças detectadas nestas pesquisas.\newline
\newline
\indent Na análise dos resultados são apresentadas as informações quantitativas da revisão em conjunto com as considerações finais.

\subsection{Fonte 1 - IEEE Xplore}\newline
\newline
\noindent \textbf{Fonte}: IEEE Xplore Digital Library\newline
\noindent \textbf{Data de Busca}: 15/04/2018\newline
\noindent \textbf{Strings Utilizadas}\newline
\indent \textbf{Primeiro Ciclo}: (( Usability AND ERP ) OR ( Usabilidade AND ERP ));\newline
\indent \textbf{Segundo  Ciclo}: ((usability NEAR/3 ERP) OR (usability ONEAR/3 ERP));\newline
\indent \textbf{Terceiro Ciclo}: (((usability NEAR/3 ERP) OR (ERP ONEAR/10 usability)));\newline
\indent \textbf{Quarto   Ciclo}: ((.QT.Enterprise Resource Planning.QT. NEAR/3 usability)  OR (.QT.Enterprise Resource Planning.QT. ONEAR/10 usability) );\newline
\indent \textbf{Quinto   Ciclo}: ((.QT.Enterprise Resource Planning.QT. NEAR/3 usability)  OR (.QT.Enterprise Resource Planning.QT. ONEAR/10 usability) ) OR ((usability NEAR/3 ERP) OR (ERP ONEAR/10 usability));\newline
\indent \textbf{Sexto    Ciclo}: (((usability NEAR/3 .QT.Enterprise Resource Planning.QT.) OR (usability ONEAR/10 .QT.Enterprise Resource Planning.QT.) ) OR ((usability NEAR/3 ERP) OR (usability ONEAR/10 ERP)));\newline
\newline
\noindent \textbf{Campos Pesquisados}: Titulo (Title), Resumo (Abstract) \newline
\noindent \textbf{Período Considerado}: 2001 à Dezembro de 2017 \newline
\noindent \textbf{Filtros Utilizados}: “Full Text & Metadata” \newline

Os resultados desta pesquisa encontram-se nas Tabela 1 á 6, que se seguem:

\begin{landscape}
\noindent \textbf{Lista de artigos encontrados}:
\setcounter{table}{0}

% Ambiente longtable é similar à combinação de table com tabular
\begin{longtable}{||p{1.7cm}|p{11.0cm}|p{7.0cm}|p{1.7cm}|p{1cm}||} % use normalmente os parâmetros 
 % do tabular
 %\begin{longtable}{||c|r||} % usar a largura automática, com ítens 
 %1 centrada e segunda, alinhada a direit, na célula
 % c -> center, l-> left, r -> right
 \caption{Primeiro ciclo de pesquisa IEEE}
 \label{ltab:teste}
 
 %%%%%%%%%%%%%%%%%%%%%%%%%%%%%%%%%%%%%%%%%%%%%%%%%%%%%%%%%%%%%%%%%%%%%%%%%%%%%%%%%%%%%
 % Obs.: configuração padrão da quebra de tabelas quando                          %
 %  1. \hline é traçado entre linhas da tabela:                                    %
 %     traçará um \hline no final da tabela quando vai continuar na outra página, %
 %     para "fechar a tabela"                                                        %
 %  2. não tem \hline: não traçará \hline quando efetua quebra                   %
 % --------------------------------------------------------------------------------- %
 % para configurar ação da quebra de tabelas no modo                               %
 % não padrão, precisará definição mais complexa,                               %
 % definindo o que fazer em                                                          %
 % - final da tabela quando continua na próxima página                             %
 % - no começo da tabela quando se trata da continuação                           %
 %   da página anterior,                                                            %
 % - etc.														                     %
 % Para detalhes desta parte, veja o arquivo:                                        %
 % doc\latex\tools\longtable.dvi                                                     %
 %%%%%%%%%%%%%%%%%%%%%%%%%%%%%%%%%%%%%%%%%%%%%%%%%%%%%%%%%%%%%%%%%%%%%%%%%%%%%%%%%%%%%
 \\ % é necessário pular linha após definições
    % preliminares: caption, label, etc.
 %\hline
	\hline
	Nro Artigo	& Título & Local de Publicação & Páginas & Ano \\ % Note a separação de col. e a quebra de linhas
	\hline
	\endfirsthead
	
	\multicolumn{5}{c}%
	{{\bfseries \tablename\ \thetable{} -- continuação da página anterior}} \\
	\hline
	Nro Artigo	& Título & Local de Publicação & Páginas & Ano \\ % Note a separação de col. e a quebra de linhas
	\hline
	\endhead
	
	\hline \multicolumn{5}{|r|}{{Continua na página seguinte}} \\ \hline
	\endfoot
	
	\hline \hline
	\endlastfoot
	01 & Using quality models in software package selection & IEEE Software & 34-41 & 2003 \\ 
	\hline
	02 & An investigation of the proliferation of mobile ERP apps and their usability & 2017 8th International Conference on Information and Communication Systems (ICICS) & 352-357 & 2017 \\ 
	\hline
	03 & Artemis - an extensible natural language framework for data querying and manipulation & 2016 IEEE 12th International Conference on Intelligent Computer Communication and Processing (ICCP) & 85-91 & 2016 \\ 
	\hline
	04 & Learning through ERP in technical educational institutions & 2012 15th International Conference on Interactive Collaborative Learning (ICL) & 1-4 & 2012 \\ 
	\hline
	05 & Brain-Computer Interfaces: Beyond Medical Applications & Computer & 26-34 & 2012 \\ 
	\hline
	06 & Analyzing Speech Quality Perception Using Electroencephalography & IEEE Journal of Selected Topics in Signal Processing & 721-731 & 2012 \\ 
	\hline
	07 & Enterprise Resource Planning system continuance usage intention at an individual level: An approach from value perspective & 2016 International Conference on Data and Software Engineering (ICoDSE) & 1-6 & 2016 \\ 
	\hline
	08 & The relationship between ERP software selection criteria and ERP success & 2009 IEEE International Conference on Industrial Engineering and Engineering Management & 2222-2226 & 2009 \\
	\hline 
	09 & An adaptive system architecture for devising adaptive user interfaces for mobile ERP apps & 2017 2nd International Conference on the Applications of Information Technology in Developing Renewable Energy Processes Systems (IT-DREPS) & 1-6 & 2017 \\ 
	\hline
	10 & Metric based efficiency analysis of educational ERP system usability-using fuzzy model & 2015 Third International Conference on Image Information Processing (ICIIP) & 382-386 & 2015 \\ 
	\hline
	11 & SAP remote communications & 2012 7th IEEE International Symposium on Applied Computational Intelligence and Informatics (SACI) & 303-309 & 2012 \\ 
	\hline
	12 & A practical abstraction of ERP to cloud integration complexity: The easy way & 2016 15th RoEduNet Conference: Networking in Education and Research & 1-6 & 2016 \\ 
	\hline
	13 & Receiver access control and secured handoff in mobile multicast using IGMP-AC & 2008 33rd IEEE Conference on Local Computer Networks (LCN) & 411-418 & 2008 \\ 
	\hline
	14 & A selection model of ERP system in mobile ERP design science research: Case study: Mobile ERP usability & 2016 IEEE/ACS 13th International Conference of Computer Systems and Applications (AICCSA) & 1-8 & 2016 \\ 
	\hline
	15 & Usability analysis of ERP software: Education and experience of users' as moderators & The 8th International Conference on Software, Knowledge, Information Management and Applications (SKIMA 2014) & 1-7 & 2014 \\ 
	\hline
	16 & Exploratory study to identify critical success factors penetration in ERP implementations & Proceedings of 3rd International Conference on Reliability, Infocom Technologies and Optimization & 1-6 & 2014 \\ 
	\hline
	17 & Approach to analysis and assessment of ERP system. A software vendor's perspective & 2015 Federated Conference on Computer Science and Information Systems (FedCSIS) & 1415-1426 & 2015 \\ 
	\hline
	18 & Agile ERP: "You don't know what you've got 'till it's gone!" & Agile 2007 (AGILE 2007) & 143-149 & 2007 \\ 
	\hline
	19 & A High-Security EEG-Based Login System with RSVP Stimuli and Dry Electrodes & IEEE Transactions on Information Forensics and Security & 2635-2647 & 2016 \\ 
	\hline
	20 & Heuristic evaluation checklist for mobile ERP user interfaces & 2016 7th International Conference on Information and Communication Systems (ICICS) & 180-185 & 2016 \\ 
	\hline
	21 & Quality improvement of ERP system GUI using expert method: A case study & 2013 6th International Conference on Human System Interactions (HSI) & 145-152 & 2013 \\ 
	\hline
	22 & A demonstrator of the GINSENG-approach to performance and closed loop control in WSNs & 2012 Ninth International Conference on Networked Sensing (INSS) & 1-2 & 2012 \\ 
	\hline
	23 & A Collaboration Model for ERP User-System Interaction & 2010 43rd Hawaii International Conference on System Sciences & 1-9 & 2010 \\ 
	\hline
	24 & Assessment of knowledge ability towards decision-making in information systems from managers perspective & 2011 Malaysian Conference in Software Engineering & 465-468 & 2011 \\ 
	\hline
	25 & ICITSI & 2016 International Conference on Information Technology Systems and Innovation (ICITSI) & 1-1 & 2016 \\ 
	\hline
	26 & The implementation experience of an advanced service repository for supporting service-oriented architecture & 2012 XXXVIII Conferencia Latinoamericana En Informatica (CLEI) & 1-10 & 2012 \\ 
	\hline
	27 & Cognitive, affective, and experience correlates of speech quality perception in complex listening conditions & 2013 IEEE International Conference on Acoustics, Speech and Signal Processing & 3672-3676 & 2013 \\ 
	\hline
	28 & Applying human-computer collaboration for improving ERP usability & 2010 International Symposium on Collaborative Technologies and Systems & 396-397 & 2010 \\ 
	\hline
	29 & Transfer learning and active transfer learning for reducing calibration data in single-trial classification of visually-evoked potentials & 2014 IEEE International Conference on Systems, Man, and Cybernetics (SMC) & 2801-2807 & 2014 \\ 
	\hline
	30 & A Novel Usability Matrix for ERP Systems Using Heuristic Approach & 2012 International Conference on Management of e-Commerce and e-Government & 291-296 & 2012 \\ 
	\hline
	31 & (Re-)Evaluating User Interface Aspects in ERP Systems -- An Empirical User Study & 2014 47th Hawaii International Conference on System Sciences & 396-405 & 2014 \\ 
	\hline
	32 & Applying design principles for enhancing enterprise system usability & 2014 9th International Conference on Software Engineering and Applications (ICSOFT-EA) & 162-169 & 2014 \\ 
	\hline
	33 & Analyzing Usability Alternatives in Multi-criteria Decision Making During ERP Training & 2006 7th International Conference on Information Technology Based Higher Education and Training & 296-309 & 2006 \\ 
	\hline
	34 & Spatial auditory BCI paradigm based on real and virtual sound image generation & 2013 Asia-Pacific Signal and Information Processing Association Annual Summit and Conference & 1-5 & 2013 \\ 
	\hline
	35 & Using ERPs for assessing the (sub) conscious perception of noise & 2010 Annual International Conference of the IEEE Engineering in Medicine and Biology & 2690-2693 & 2010 \\ 
	\hline
	36 & A gamified system for learning enterprise resource planning systems: Investigating the user experience & 2017 1st International Conference on Next Generation Computing Applications (NextComp) & 209-214 & 2017 \\ 
	\hline
	37 & The study of technology acceptance based on gender's differences in ERP implementation & 2011 International Conference on E-Business and E-Government (ICEE) & 1-5 & 2011 \\ 
	\hline
	38 & Engineering Adaptive Model-Driven User Interfaces & IEEE Transactions on Software Engineering & 1118-1147 & 2016 \\ 
	\hline
	39 & Using human resource management suites to exploit team process improvement models & Proceedings. 28th Euromicro Conference & 382-387 & 2002 \\ 
	\hline
	40 & Initial assessment of artifact filtering for RSVP Keyboard #x2122; & 2013 IEEE Signal Processing in Medicine and Biology Symposium (SPMB) & 1-5 & 2013 \\ 
	\hline
	41 & Key Performance Indicators used in ERP performance measurement applications & 2012 IEEE 10th Jubilee International Symposium on Intelligent Systems and Informatics & 43-48 & 2012  
	\hline
 \end{longtable}
\end{landscape}

\begin{landscape}
%\noindent \textbf{Lista de artigos encontrados}:

 \begin{longtable}{||p{1.7cm}|p{11.0cm}|p{6.0cm}|p{1.7cm}|p{1cm}|p{1cm}||} % use normalmente os parâmetros 
 \caption{Segundo ciclo de pesquisa IEEE}
 \label{ltab:teste}
 \\ % é necessário pular linha após definições
 	\hline
	Nro Artigo	& Título & Local de Publicação & Páginas & Ano & Dupl. \\ % Note a separação de col. e a quebra de linhas
	\hline
	\endfirsthead
	
	\multicolumn{6}{c}%
	{{\bfseries \tablename\ \thetable{} -- continuação da página anterior}} \\
	\hline
	Nro Artigo	& Título & Local de Publicação & Páginas & Ano & Dupl. \\ % Note a separação de col. e a quebra de linhas
	\hline
	\endhead
	
	\hline \multicolumn{6}{|r|}{{Continua na página seguinte}} \\ \hline
	\endfoot
	
	\hline \hline
	\endlastfoot
	42 & The study of technology acceptance based on gender's differences in ERP implementation & 2011 International Conference on E-Business and E-Government (ICEE) & 1-5 & 2011 & SIM \\ 
	\hline
	43 & Evaluating mobile banking application: Usability dimensions and measurements & Proceedings of the 6th International Conference on Information Technology and Multimedia & 136-140 & 2014 & NÃO \\ 
	\hline
	44 & The process of ERP usage in manufacturing firms in China: An empirical investigation & 2009 International Conference on Management Science and Engineering & 3-9 & 2009 & NÃO \\ 
	\hline
	45 & Comprehensive evaluation of the auto parts supplier selection under the environment of things networking & Proceedings of 2013 IEEE International Conference on Service Operations and Logistics, and Informatics & 540-545 & 2013 & NÃO \\ 
	\hline
	46 & Analyzing Usability Alternatives in Multi-criteria Decision Making During ERP Training & 2006 7th International Conference on Information Technology Based Higher Education and Training & 296-309 & 2006 & SIM \\ 
	\hline
	47 & S.I. success models, 25 years of evolution & 2017 12th Iberian Conference on Information Systems and Technologies (CISTI) & 1-6 & 2017 & NÃO \\ 
	\hline
	48 & Metric based efficiency analysis of educational ERP system usability-using fuzzy model & 2015 Third International Conference on Image Information Processing (ICIIP) & 382-386 & 2015 & SIM \\ 
	\hline
	49 & Citarasa based vehicle planning system & 2007 IEEE International Conference on Industrial Engineering and Engineering Management & 1297-1301 & 2007 & NÃO \\ 
	\hline
	50 & (Re-)Evaluating User Interface Aspects in ERP Systems -- An Empirical User Study & 2014 47th Hawaii International Conference on System Sciences & 396-405 & 2014 & SIM \\ 
	\hline
	51 & PROMETHEUS: Procedural Methodology For Developing Heuristics Of Usability & IEEE Latin America Transactions & 541-549 & 2017 & NÃO \\ 
	\hline
	52 & Approach to analysis and assessment of ERP system. A software vendor's perspective & 2015 Federated Conference on Computer Science and Information Systems (FedCSIS) & 1415-1426 & 2015 & SIM \\ 
	\hline
	53 & Real Options and Subsequent Technology Adoption: An ERP System Perspective & 2015 48th Hawaii International Conference on System Sciences & 5020-5027 & 2015 & NÃO \\ 
	\hline
	54 & Key Performance Indicators used in ERP performance measurement applications & 2012 IEEE 10th Jubilee International Symposium on Intelligent Systems and Informatics & 43-48 & 2012 & SIM \\ 
	\hline
	55 & Exploratory study to identify critical success factors penetration in ERP implementations & Proceedings of 3rd International Conference on Reliability, Infocom Technologies and Optimization & 1-6 & 2014 & SIM \\ 
	\hline
	56 & Using human resource management suites to exploit team process improvement models & Proceedings. 28th Euromicro Conference & 382-387 & 2002 & SIM \\ 
	\hline
	57 & Heuristic evaluation checklist for mobile ERP user interfaces & 2016 7th International Conference on Information and Communication Systems (ICICS) & 180-185 & 2016 & SIM \\ 
	\hline
	58 & An adaptive system architecture for devising adaptive user interfaces for mobile ERP apps & 2017 2nd International Conference on the Applications of Information Technology in Developing Renewable Energy Processes Systems (IT-DREPS) & 1-6 & 2017 & SIM \\ 
	\hline
	59 & Customer Involvement in Requirements Management: Lessons from Mass Market Software Development & 2009 17th IEEE International Requirements Engineering Conference & 281-286 & 2009 & NÃO \\ 
	\hline
	60 & A Novel Usability Matrix for ERP Systems Using Heuristic Approach & 2012 International Conference on Management of e-Commerce and e-Government & 291-296 & 2012 & SIM \\ 
	\hline
	61 & Assessment of knowledge ability towards decision-making in information systems from managers perspective & 2011 Malaysian Conference in Software Engineering & 465-468 & 2011 & SIM \\ 
	\hline
	62 & A methodology for successful implementation of ERP in smaller companies & Proceedings of 2010 IEEE International Conference on Service Operations and Logistics, and Informatics & 380-385 & 2010 & NÃO \\ 
	\hline
	63 & ERP post-adoption: Value impact on firm performance: A survey study on Portuguese SMEs & 7th Iberian Conference on Information Systems and Technologies (CISTI 2012) & 1-6 & 2012 & NÃO \\ 
	\hline
	64 & Usability analysis of ERP software: Education and experience of users' as moderators & The 8th International Conference on Software, Knowledge, Information Management and Applications (SKIMA 2014) & 1-7 & 2014 & SIM \\ 
	\hline
	65 & Usability practice and awareness in UAE & The 2011 International Conference and Workshop on Current Trends in Information Technology (CTIT 11) & 1-6 & 2011 & NÃO \\ 
	\hline
	66 & Applying design principles for enhancing enterprise system usability & 2014 9th International Conference on Software Engineering and Applications (ICSOFT-EA) & 162-169 & 2014 & SIM \\ 
	\hline
	67 & Effort Estimation for ERP Projects #x2014; A Systematic Review & 2017 43rd Euromicro Conference on Software Engineering and Advanced Applications (SEAA) & 96-103 & 2017 & NÃO \\ 
	\hline
	68 & Mining Logs to Model the Use of a System & 2017 ACM/IEEE International Symposium on Empirical Software Engineering and Measurement (ESEM) & 334-343 & 2017 & NÃO \\ 
	\hline
	69 & Impact of Man-Machine Interaction Factors on Enterprise Resource Planning (ERP) Software Design & 2006 Technology Management for the Global Future - PICMET 2006 Conference & 2335-2341 & 2006 & NÃO \\ 
	\hline
	70 & Critical success factors for MES implementation in China & 2012 IEEE International Conference on Industrial Engineering and Engineering Management & 558-562 & 2012 & NÃO \\ 
	\hline
	71 & A demonstrator of the GINSENG-approach to performance and closed loop control in WSNs & 2012 Ninth International Conference on Networked Sensing (INSS) & 1-2 & 2012 & SIM \\ 
	\hline
	72 & Learning through ERP in technical educational institutions & 2012 15th International Conference on Interactive Collaborative Learning (ICL) & 1-4 & 2012 & SIM \\ 
	\hline
	73 & A Collaboration Model for ERP User-System Interaction & 2010 43rd Hawaii International Conference on System Sciences & 1-9 & 2010 & SIM \\ 
	\hline
	74 & Applying human-computer collaboration for improving ERP usability & 2010 International Symposium on Collaborative Technologies and Systems & 396-397 & 2010 & SIM \\ 
	\hline
	75 & Applying fuzzy sets for erp systems selection within the construction industry & 2010 IEEE International Conference on Industrial Engineering and Engineering Management & 320-324 & 2010 & NÃO \\ 
	\hline
	76 & A selection model of ERP system in mobile ERP design science research: Case study: Mobile ERP usability & 2016 IEEE/ACS 13th International Conference of Computer Systems and Applications (AICCSA) & 1-8 & 2016 & SIM \\ 
	\hline
	77 & An investigation of the proliferation of mobile ERP apps and their usability & 2017 8th International Conference on Information and Communication Systems (ICICS) & 352-357 & 2017 & SIM 
	\hline
 \end{longtable}
\end{landscape}

\begin{landscape}
%\noindent \textbf{Lista de artigos encontrados}:

 \begin{longtable}{||p{1.7cm}|p{11.0cm}|p{6.0cm}|p{1.7cm}|p{1cm}|p{1cm}||} % use normalmente os parâmetros 
 \caption{Terceiro ciclo de pesquisa IEEE}
 \label{ltab:teste}
 \\ % é necessário pular linha após definições
 	\hline
 	Nro Artigo	& Título & Local de Publicação & Páginas & Ano & Dupl. \\ % Note a separação de col. e a quebra de linhas
  	\hline
  	\endfirsthead
  	
  	\multicolumn{6}{c}%
  	{{\bfseries \tablename\ \thetable{} -- continuação da página anterior}} \\
  	\hline
  	Nro Artigo	& Título & Local de Publicação & Páginas & Ano & Dupl. \\ % Note a separação de col. e a quebra de linhas
  	\hline
  	\endhead
  	
  	\hline \multicolumn{6}{|r|}{{Continua na página seguinte}} \\ \hline
  	\endfoot
  	
  	\hline \hline
  	\endlastfoot
	78 & The study of technology acceptance based on gender's differences in ERP implementation & 2011 International Conference on E-Business and E-Government (ICEE) & 1-5 & 2011 & SIM \\ 
	 	\hline
	79 & Metric based efficiency analysis of educational ERP system usability-using fuzzy model & 2015 Third International Conference on Image Information Processing (ICIIP) & 382-386 & 2015 & SIM\\ 
 	\hline
	80 & (Re-)Evaluating User Interface Aspects in ERP Systems -- An Empirical User Study & 2014 47th Hawaii International Conference on System Sciences & 396-405 & 2014 & SIM \\ 
 	\hline
	81 & Exploratory study to identify critical success factors penetration in ERP implementations & Proceedings of 3rd International Conference on Reliability, Infocom Technologies and Optimization & 1-6 & 2014 & SIM \\ 
 	\hline
	82 & Heuristic evaluation checklist for mobile ERP user interfaces & 2016 7th International Conference on Information and Communication Systems (ICICS) & 180-185 & 2016 & SIM \\ 
 	\hline
	83 & A Novel Usability Matrix for ERP Systems Using Heuristic Approach & 2012 International Conference on Management of e-Commerce and e-Government & 291-296 & 2012 & SIM \\ 
 	\hline
	84 & Usability analysis of ERP software: Education and experience of users' as moderators & The 8th International Conference on Software, Knowledge, Information Management and Applications (SKIMA 2014) & 1-7 & 2014 & SIM \\ 
 	\hline
	85 & Applying design principles for enhancing enterprise system usability & 2014 9th International Conference on Software Engineering and Applications (ICSOFT-EA) & 162-169 & 2014 & SIM \\ 
 	\hline
	86 & The relationship between ERP software selection criteria and ERP success & 2009 IEEE International Conference on Industrial Engineering and Engineering Management & 2222-2226 & 2009 & SIM \\ 
 	\hline
	87 & A demonstrator of the GINSENG-approach to performance and closed loop control in WSNs & 2012 Ninth International Conference on Networked Sensing (INSS) & 1-2 & 2012 & SIM \\ 
 	\hline
	88 & A Collaboration Model for ERP User-System Interaction & 2010 43rd Hawaii International Conference on System Sciences & 1-9 & 2010 & SIM \\ 
 	\hline
	89 & Applying human-computer collaboration for improving ERP usability & 2010 International Symposium on Collaborative Technologies and Systems & 396-397 & 2010 & SIM \\ 
 	\hline
	90 & A selection model of ERP system in mobile ERP design science research: Case study: Mobile ERP usability & 2016 IEEE/ACS 13th International Conference of Computer Systems and Applications (AICCSA) & 1-8 & 2016 & SIM \\ 
 	\hline
	91 & An investigation of the proliferation of mobile ERP apps and their usability & 2017 8th International Conference on Information and Communication Systems (ICICS) & 352-357 & 2017 & SIM
	\hline
 \end{longtable}
\newline
 \begin{longtable}{||p{1.7cm}|p{11.0cm}|p{6.0cm}|p{1.7cm}|p{1cm}|p{1cm}||} % use normalmente os parâmetros 
	\caption{Quarto ciclo de pesquisa IEEE}
	\label{ltab:teste}
	\\ % é necessário pular linha após definições
	\hline
	Nro Artigo	& Título & Local de Publicação & Páginas & Ano & Dupl. \\ % Note a separação de col. e a quebra de linhas
	\hline
	\endfirsthead
	
	\multicolumn{6}{c}%
	{{\bfseries \tablename\ \thetable{} -- continuação da página anterior}} \\
	\hline
	Nro Artigo	& Título & Local de Publicação & Páginas & Ano & Dupl. \\ % Note a separação de col. e a quebra de linhas
	\hline
	\endhead
	
	\hline \multicolumn{6}{|r|}{{Continua na página seguinte}} \\ \hline
	\endfoot
	
	\hline \hline
	\endlastfoot
	92 & Studying the deficiencies and problems of different architecture in developing distributed systems and analyze the existing solution & 2015 2nd International Conference on Knowledge-Based Engineering and Innovation (KBEI) & 826-834 & 2015 & NÃO \\ 
	\hline 
	93 & (Re-)Evaluating User Interface Aspects in ERP Systems -- An Empirical User Study & 2014 47th Hawaii International Conference on System Sciences & 396-405 & 2014 & SIM \\ 
	\hline
\end{longtable}

\newline
\begin{longtable}{||p{1.7cm}|p{11.0cm}|p{6.0cm}|p{1.7cm}|p{1cm}|p{1cm}||} % use normalmente os parâmetros 
	\caption{Quinto ciclo de pesquisa IEEE}
	\label{ltab:teste}
	\\ % é necessário pular linha após definições
	\hline
	Nro Artigo	& Título & Local de Publicação & Páginas & Ano & Dupl. \\ % Note a separação de col. e a quebra de linhas
	\hline
	\endfirsthead
	
	\multicolumn{6}{c}%
	{{\bfseries \tablename\ \thetable{} -- continuação da página anterior}} \\
	\hline
	Nro Artigo	& Título & Local de Publicação & Páginas & Ano & Dupl. \\ % Note a separação de col. e a quebra de linhas
	\hline
	\endhead
	
	\hline \multicolumn{6}{|r|}{{Continua na página seguinte}} \\ \hline
	\endfoot
	
	\hline \hline
	\endlastfoot
	94 & Studying the deficiencies and problems of different architecture in developing distributed systems and analyze the existing solution & 2015 2nd International Conference on Knowledge-Based Engineering and Innovation (KBEI) & 826-834 & 2015 & SIM \\ 
	\hline	
	95 & Metric based efficiency analysis of educational ERP system usability-using fuzzy model & 2015 Third International Conference on Image Information Processing (ICIIP) & 382-386 & 2015 & SIM \\ 
	\hline	
	96 & (Re-)Evaluating User Interface Aspects in ERP Systems -- An Empirical User Study & 2014 47th Hawaii International Conference on System Sciences & 396-405 & 2014 & SIM \\ 
	\hline	
	97 & Exploratory study to identify critical success factors penetration in ERP implementations & Proceedings of 3rd International Conference on Reliability, Infocom Technologies and Optimization & 1-6 & 2014 & SIM \\ 
	\hline	
	98 & Heuristic evaluation checklist for mobile ERP user interfaces & 2016 7th International Conference on Information and Communication Systems (ICICS) & 180-185 & 2016 & SIM \\ 
	\hline	
	99 & A Novel Usability Matrix for ERP Systems Using Heuristic Approach & 2012 International Conference on Management of e-Commerce and e-Government & 291-296 & 2012 & SIM \\ 
	\hline	
	100 & Usability analysis of ERP software: Education and experience of users' as moderators & The 8th International Conference on Software, Knowledge, Information Management and Applications (SKIMA 2014) & 1-7 & 2014 & SIM \\ 
	\hline	
	101 & Applying design principles for enhancing enterprise system usability & 2014 9th International Conference on Software Engineering and Applications (ICSOFT-EA) & 162-169 & 2014 & SIM \\ 
	\hline	
	102 & The relationship between ERP software selection criteria and ERP success & 2009 IEEE International Conference on Industrial Engineering and Engineering Management & 2222-2226 & 2009 & SIM \\ 
	\hline	
	103 & A demonstrator of the GINSENG-approach to performance and closed loop control in WSNs & 2012 Ninth International Conference on Networked Sensing (INSS) & 1-2 & 2012 & SIM \\ 
	\hline	
	104 & A Collaboration Model for ERP User-System Interaction & 2010 43rd Hawaii International Conference on System Sciences & 1-9 & 2010 & SIM \\ 
	\hline	
	105 & Applying human-computer collaboration for improving ERP usability & 2010 International Symposium on Collaborative Technologies and Systems & 396-397 & 2010 & SIM \\ 
	\hline	
	106 & A selection model of ERP system in mobile ERP design science research: Case study: Mobile ERP usability & 2016 IEEE/ACS 13th International Conference of Computer Systems and Applications (AICCSA) & 1-8 & 2016 & SIM \\ 
	\hline	
	107 & An investigation of the proliferation of mobile ERP apps and their usability & 2017 8th International Conference on Information and Communication Systems (ICICS) & 352-357 & 2017 & SIM 
	\hline
\end{longtable}
\newline
\begin{longtable}{||p{1.7cm}|p{11.0cm}|p{6.0cm}|p{1.7cm}|p{1cm}|p{1cm}||} % use normalmente os parâmetros 
	\caption{Sexto ciclo de pesquisa IEEE}
	\label{ltab:teste}
	\\ % é necessário pular linha após definições
	\hline
	Nro Artigo	& Título & Local de Publicação & Páginas & Ano & Dupl. \\ % Note a separação de col. e a quebra de linhas
	\hline
	\endfirsthead
	
	\multicolumn{6}{c}%
	{{\bfseries \tablename\ \thetable{} -- continuação da página anterior}} \\
	\hline
	Nro Artigo	& Título & Local de Publicação & Páginas & Ano & Dupl. \\ % Note a separação de col. e a quebra de linhas
	\hline
	\endhead
	
	\hline \multicolumn{6}{|r|}{{Continua na página seguinte}} \\ \hline
	\endfoot
	
	\hline \hline
	\endlastfoot
	108 & An investigation of the proliferation of mobile ERP apps and their usability & 2017 8th International Conference on Information and Communication Systems (ICICS) & 352-357 & 2017 & SIM \\ 
	\hline
	109 & The relationship between ERP software selection criteria and ERP success & 2009 IEEE International Conference on Industrial Engineering and Engineering Management & 2222-2226 & 2009 & SIM \\ 
	\hline
	110 & An adaptive system architecture for devising adaptive user interfaces for mobile ERP apps & 2017 2nd International Conference on the Applications of Information Technology in Developing Renewable Energy Processes Systems (IT-DREPS) & 1-6 & 2017 & SIM \\ 
	\hline
	111 & Metric based efficiency analysis of educational ERP system usability-using fuzzy model & 2015 Third International Conference on Image Information Processing (ICIIP) & 382-386 & 2015 & SIM \\ 
	\hline
	112 & A selection model of ERP system in mobile ERP design science research: Case study: Mobile ERP usability & 2016 IEEE/ACS 13th International Conference of Computer Systems and Applications (AICCSA) & 1-8 & 2016 & SIM \\ 
	\hline
	113 & Usability analysis of ERP software: Education and experience of users' as moderators & The 8th International Conference on Software, Knowledge, Information Management and Applications (SKIMA 2014) & 1-7 & 2014 & SIM \\ 
	\hline
	114 & Exploratory study to identify critical success factors penetration in ERP implementations & Proceedings of 3rd International Conference on Reliability, Infocom Technologies and Optimization & 1-6 & 2014 & SIM \\ 
	\hline
	115 & Heuristic evaluation checklist for mobile ERP user interfaces & 2016 7th International Conference on Information and Communication Systems (ICICS) & 180-185 & 2016 & SIM \\ 
	\hline
	116 & A demonstrator of the GINSENG-approach to performance and closed loop control in WSNs & 2012 Ninth International Conference on Networked Sensing (INSS) & 1-2 & 2012 & SIM \\ 
	\hline
	117 & A Collaboration Model for ERP User-System Interaction & 2010 43rd Hawaii International Conference on System Sciences & 1-9 & 2010 & SIM \\ 
	\hline
	118 & Applying human-computer collaboration for improving ERP usability & 2010 International Symposium on Collaborative Technologies and Systems & 396-397 & 2010 & SIM \\ 
	\hline
	119 & A Novel Usability Matrix for ERP Systems Using Heuristic Approach & 2012 International Conference on Management of e-Commerce and e-Government & 291-296 & 2012 & SIM \\ 
	\hline
	120 & (Re-)Evaluating User Interface Aspects in ERP Systems -- An Empirical User Study & 2014 47th Hawaii International Conference on System Sciences & 396-405 & 2014 & SIM \\ 
	\hline
	121 & Applying design principles for enhancing enterprise system usability & 2014 9th International Conference on Software Engineering and Applications (ICSOFT-EA) & 162-169 & 2014 & SIM \\ 
	\hline
	122 & Analyzing Usability Alternatives in Multi-criteria Decision Making During ERP Training & 2006 7th International Conference on Information Technology Based Higher Education and Training & 296-309 & 2006 & SIM \\ 
	\hline
	123 & The study of technology acceptance based on gender's differences in ERP implementation & 2011 International Conference on E-Business and E-Government (ICEE) & 1-5 & 2011 & SIM \\  
	\hline
\end{longtable}
\end{landscape}

\subsection{Fonte 2 - Ibict OasisBr}\newline
\newline
\noindent \textbf{Fonte}: Portal brasileiro de publicações científicas em acesso aberto\newline
\noindent \textbf{Data de Busca}: 21/04/2018\newline
\noindent \textbf{Strings Utilizadas}\newline
\indent \textbf{Primeiro Ciclo}: "(Resumo Português:"usabilidade ERP"~10)";\newline
\indent \textbf{Segundo  Ciclo}: "(Resumo Português:" " "Enterprise Resource"~1 Planning"~1 usabilidade"~10)";
\newline
\noindent \textbf{Campos Pesquisados}: Titulo, Resumo \newline
\noindent \textbf{Período Considerado}: 2001 à Dezembro de 2017 \newline

Os resultados desta pesquisa encontram-se nas Tabelas 7 e 8, que se seguem:

\begin{landscape}
\noindent \textbf{Lista de artigos encontrados}:

 \begin{longtable}{||p{1.7cm}|p{11.0cm}|p{6.0cm}|p{1.7cm}|p{1cm}|p{1cm}||} % use normalmente os parâmetros 
 \caption{Primeiro ciclo de pesquisa IBICT}
 \label{ltab:teste}
 \\ % é necessário pular linha após definições
 	\hline
 	Nro Artigo	& Título & Tipo Documento & Páginas & Ano & Dupl. \\ % Note a separação de col. e a quebra de linhas
  	\hline
	124 & Uma abordagem orientada a modelos para desenvolvimento de sistemas ERP de varejo na Web utilizando características funcionais de usabilidade. & Tese & 149 & 2015 & NÃO \\ 
	\hline
	125 & Boa Usabilidade e comunicação eficiente de tarefas: dois aliados na execução de processos em sistemas integrados de gestão & Dissertação & 151 & 2015 & NÃO 
	\hline
 \end{longtable}
\newline
 \begin{longtable}{||p{1.7cm}|p{11.0cm}|p{6.0cm}|p{1.7cm}|p{1cm}|p{1cm}||} % use normalmente os parâmetros 
	\caption{Segundo ciclo de pesquisa IBICT}
	\label{ltab:teste}
	\\ % é necessário pular linha após definições
	\hline
	Nro Artigo	& Título & Tipo Documento & Páginas & Ano & Dupl. \\ % Note a separação de col. e a quebra de linhas
	\hline
	126 & Estudo de um caso de localização de um software ERP de código livre. & Tese & 133 & 2011 & NÃO 
	\hline
\end{longtable}

\end{landscape}

\subsection{Fonte 3 - Literatura Cinzenta}\newline
\newline
\noindent \textbf{Fonte}: Universidade de Dresden\newline
\noindent \textbf{Data de Busca}: 21/09/2017\newline
\noindent \textbf{Critério Utilizado}\newline
\indent Busca de Publicações da Universidade sobre o tema escolhido.
\newline
\noindent \textbf{Campos Pesquisados}: Titulo, Resumo \newline
\noindent \textbf{Período Considerado}: 2001 à Dezembro de 2017 \newline

Os resultados desta pesquisa encontram-se na Tabelas 9, a seguir:

\begin{landscape}
\noindent \textbf{Lista de artigos encontrados}:
 \begin{longtable}{||p{1.7cm}|p{11.0cm}|p{6.0cm}|p{1.7cm}|p{1cm}|p{1cm}||} % use normalmente os parâmetros 
 	\captionsetup{margin=14pt,labelfont=bf,justification=raggedright}
	\caption{Resultado Pesquisa Literatura Cinzenta}
	\label{ltab:teste}
	\\ % é necessário pular linha após definições
	\hline
	Nro Artigo	& Título & Local de Publicação & Páginas & Ano & Dupl. \\ % Note a separação de col. e a quebra de linhas
	\hline
  	\endfirsthead

	\multicolumn{6}{c}%
	{{\bfseries \tablename\ \thetable{} -- continuação da página anterior}} \\
	\hline
	Nro Artigo	& Título & Local de Publicação & Páginas & Ano & Dupl. \\ % Note a separação de col. e a quebra de linhas
	\hline
	\endhead
	
	\hline \multicolumn{6}{|r|}{{Continua na página seguinte}} \\ \hline
	\endfoot
	
	\hline \hline
	\endlastfoot
	127 & Discovering Potentials in Enterprise Interface Design - A review of our latest case studies in the enterprise domain & 15th International Conference on Enterprise Information Systems &  & 2013 & NÃO \\ 
	\hline
	128 & Framework to Enhance {ERP} Usability by Machine Learning Based Requirements Prioritization & Journal of Software & 664--670 & 2017 & NÃO \\ 
	\hline
	129 & The Impact of ERP System's Usability on Enterprise Resource Planning Project Implementation Success via the Mediating Role of User Satisfaction & Journal of Management Research & 49 & 2017 & NÃO \\ 
	\hline
	130 & Does Usability Matter? An Analysis of the Impact of Usability on Technology Acceptance in {ERP} Settings & Interdisciplinary Journal of Information,  Knowledge,  and Management & 309--330 & 2016 & NÃO \\ 
	\hline
	131 & Indikatorbasierte Messung der ERP-Usability &  &  & 2014 & NÃO \\ 
	\hline
	132 & (Re-)Evaluating User Interface Aspects in ERP Systems - An Empirical User Study & Proceedings of the 47th Hawaiian International Conference on System Sciences &  & 2014 & NÃO \\ \hline
	133 & Intuitive Interaktion in betrieblichen Anwendungen & ERP-Management - Zeitschrift far unternehmensweite Anwendungssysteme &  & 2011 & NÃO \\ 
	\hline
	134 & A Guide To Distance-Driven User Interfaces & COST 2101 Final Conference &  & 2011 & NÃO \\ \hline
	135 & Beyond Forms and Tables - A Visual and Task-oriented Approach to ERP Systems & Conference on Enterprise Information Systems (CENTERIS) 2012 &  & 2012 & NÃO \\ 
	\hline
	136 & Zufriedenheit von Anwendern - ERP-Systeme und weitere Unternehmensanwendungen im Vergleich & Zeitschrift für unternehmensweite Anwendungssysteme & 47--48 & 2013 & NÃO \\ 
	\hline
	137 & Usability von ERP-Systemen – Aktueller Stand und Perspektiven & Workshopband Mensch  Computer 2013 & & 2013 & NÃO \\ 
	\hline
	138 & Mastering ERP Interface Complexity - A Scalable User Interface Concept for ERP Systems & 15th International Conference on Enterprise Information Systems &  & 2013 & NÃO \\ 
	\hline
	139 & Herausforderungen für die Gestaltung von zukunftsfähigen ERP-Systemen & VDMA-Tagung: Zukunftsweisende Bedienkonzepte für die Unternehmenssoftware &  & 2013 & NÃO \\ 
	\hline
	140 & Discovering Potentials in Enterprise Interface Design - A review of our latest case studies in the enterprise domain & 15th International Conference on Enterprise Information Systems &  & 2013 & NÃO 
	\hline
\end{longtable}
\end{landscape}

\subsection{Lista de Arquivos com Status de Inclusão / Exclusão}

Foi criada uma Tabela com os resultados desta pesquisa, que encontram-se na Tabela 10, a seguir:

\newcolumntype{C}[1]{>{\centering\arraybackslash}m{#1}}
%\captionsetup[longtable]{labelfont=bf,textfont=it,labelsep=newline}
\begin{longtable}[h!]{||C{1.7cm}|C{5.0cm}|C{5.0cm}|C{3cm}||} % use normalmente os parâmetros 
    %\captionsetup{margin=-14pt,labelfont=bf,justification=justified}
    \captionsetup{margin=14pt,labelfont=bf,justification=raggedright}
    \caption{Avaliação dos Artigos  }   
    \label{tab:long} %\label{ltab:teste}
    \\% é necessário pular linha após definições
 	\hline
 	Nro Artigo	& Critérios de Inclusão & Critérios de Exclusão & Status \\ % Note a separação de col. e a quebra de linhas
  	\hline
  	\endfirsthead
  	
  	\multicolumn{4}{c}%
  	{{\bfseries \tablename\ \thetable{} -- continuação da página anterior}} \\
  	\hline
  	Nro Artigo	& Critérios de Inclusão & Critérios de Exclusão & Status \\ % Note a separação de col. e a quebra de linhas
  	\hline
  	\endhead
  	
  	\hline \multicolumn{4}{|r|}{{Continua na página seguinte}} \\ \hline
  	\endfoot
  	
  	\hline \hline
  	\endlastfoot
  	
	01 & (a)(b)    & (c)  		& Excluído 		\\ \hline
	02 & (a)(b)    & (g)  		& Excluído 		\\ \hline
	03 & (a)(b)    & (c)  		& Excluído 		\\ \hline
	04 & (a)(b)    & (b)(c)  	& Excluído 		\\ \hline
	05 & (a)(b)    & (b)(c)  	& Excluído 		\\ \hline
	06 & (a)(b)    & (b)(c)  	& Excluído 		\\ \hline
	07 & (a)(b)    & (c)     	& Excluído 		\\ \hline
	08 & (a)(b)(d) & (b)       	& Excluído 		\\ \hline
	09 & (a)(b)    & (c)(g)  	& Excluído 		\\ \hline
	10 & (a)(b)    & (c)     	& Excluído 		\\ \hline
	11 & (a)(b)    & (b)(c)(f) 	& Excluído 		\\ \hline
	12 & (a)(b)    & (c) 		& Excluído 		\\ \hline
	13 & (a)(b)    & (c) 		& Excluído 		\\ \hline
	14 & (a)(b)    & (c)(h) 	& Excluído 		\\ \hline
	15 & (a)(b)    & N/A		& Incluído 		\\ \hline
	16 & (a)(b)    & (h)		& Excluído 		\\ \hline
	17 & (a)(b)    & (h)		& Excluído 		\\ \hline
	18 & (a)(b)    & (b) 		& Excluído 		\\ \hline
	19 & (a)(b)    & (c) 		& Excluído 		\\ \hline
	20 & (a)(b)    & (c)(g)  	& Excluído 		\\ \hline
	21 & (a)(b)    & (c) 		& Excluído 		\\ \hline
	22 & (a)(b)    & (c) 		& Excluído 		\\ \hline
	23 & (a)(b)    & (c) 		& Excluído 		\\ \hline
	24 & (a)(b)    & (c) 		& Excluído 		\\ \hline
	25 & (a)(b)    & (f)(h) 	& Excluído 		\\ \hline
	26 & (a)(b)    & (f)	 	& Excluído 		\\ \hline
	27 & (a)(b)    & (c) 		& Excluído 		\\ \hline
	28 & (a)(b)    & (b) 		& Excluído 		\\ \hline
	29 & (a)(b)    & (c) 		& Excluído 		\\ \hline
	30 & (a)(b)    & (b) 		& Excluído 		\\ \hline
	31 & (a)(b)    & N/A		& Incluído 		\\ \hline
	32 & (a)(b)    & N/A		& Incluído 		\\ \hline
	33 & (a)(b)    & (b)(c)		& Excluído 		\\ \hline
	34 & (a)(b)    & (c)		& Excluído 		\\ \hline
	35 & (a)(b)    & (b)(c)		& Excluído 		\\ \hline
	36 & (a)(b)    & (c)   		& Excluído 		\\ \hline
	37 & (a)(b)    & (b) 		& Excluído 		\\ \hline
	38 & (a)(b)    & (c) 		& Excluído 		\\ \hline
	39 & (a)(b)    & (b)		& Excluído 		\\ \hline
	40 & (a)(b)    & (c) 		& Excluído 		\\ \hline
	41 & (a)(b)    & (b)		& Excluído 		\\ \hline
	42 & N/A 	   & N/A 		& Duplicado 	\\ \hline
	43 & (a)(b)    & (g) 		& Excluído 		\\ \hline
	44 & (a)(b)(d) & (b) 		& Excluído 		\\ \hline
	45 & (a)(b)    & (c) 		& Excluído 		\\ \hline
	46 & N/A 	   & N/A 		& Duplicado 	\\ \hline
	47 & (a)(b)    & (c) 		& Excluído 		\\ \hline
	48 & N/A 	   & N/A 		& Duplicado  	\\ \hline
	49 & (a)(b)    & (b)(c) 	& Excluído 		\\ \hline
	50 & N/A 	   & N/A 		& Duplicado  	\\ \hline
	51 & (a)(b)    & (c) 		& Excluído 		\\ \hline
	52 & N/A 	   & N/A 		& Duplicado  	\\ \hline
	53 & (a)(b)    & (c) 		& Excluído 		\\ \hline
	54 & N/A 	   & N/A 		& Duplicado  	\\ \hline
	55 & N/A 	   & N/A 		& Duplicado  	\\ \hline
	56 & N/A 	   & N/A 		& Duplicado  	\\ \hline
	57 & N/A 	   & N/A 		& Duplicado  	\\ \hline
	58 & N/A 	   & N/A 		& Duplicado  	\\ \hline
	59 & (a)(b)    & (b)(c) 	& Excluído 		\\ \hline
	60 & N/A 	   & N/A 		& Duplicado  	\\ \hline
	61 & N/A 	   & N/A 		& Duplicado  	\\ \hline
	62 & (a)(b)    & (b) 		& Excluído 		\\ \hline
	63 & (a)(b)    & (b) 		& Excluído 		\\ \hline
	64 & N/A 	   & N/A 		& Duplicado  	\\ \hline
	65 & (a)(b)(d) & (b) 		& Excluído 		\\ \hline
	66 & N/A 	   & N/A 		& Duplicado  	\\ \hline
	67 & (a)(b)    & (c) 		& Excluído 		\\ \hline
	68 & (a)(b)    & (c) 		& Excluído 		\\ \hline
	69 & (a)(b)(d) & (b) 		& Excluído 		\\ \hline
	70 & (a)(b)    & (b)(c) 	& Excluído 		\\ \hline
	71 & N/A 	   & N/A 		& Duplicado  	\\ \hline
	72 & N/A 	   & N/A 		& Duplicado  	\\ \hline
	73 & N/A 	   & N/A 		& Duplicado  	\\ \hline
	74 & N/A 	   & N/A 		& Duplicado  	\\ \hline
	75 & (a)(b)    & (b)(c)		& Excluído 		\\ \hline
	76 & N/A 	   & N/A 		& Duplicado  	\\ \hline
	77 & N/A 	   & N/A 		& Duplicado  	\\ \hline
	78 & N/A 	   & N/A 		& Duplicado  	\\ \hline
	79 & N/A 	   & N/A 		& Duplicado  	\\ \hline
	80 & N/A 	   & N/A 		& Duplicado  	\\ \hline
	81 & N/A 	   & N/A 		& Duplicado  	\\ \hline
	82 & N/A 	   & N/A 		& Duplicado  	\\ \hline
	83 & N/A 	   & N/A 		& Duplicado  	\\ \hline
	84 & N/A 	   & N/A 		& Duplicado  	\\ \hline
	85 & N/A 	   & N/A 		& Duplicado  	\\ \hline
	86 & N/A 	   & N/A 		& Duplicado  	\\ \hline
	87 & N/A 	   & N/A 		& Duplicado  	\\ \hline
	88 & N/A 	   & N/A 		& Duplicado  	\\ \hline
	89 & N/A 	   & N/A 		& Duplicado  	\\ \hline
	90 & N/A 	   & N/A 		& Duplicado  	\\ \hline
	91 & N/A 	   & N/A 		& Duplicado  	\\ \hline
	92 & (a)(b)    & (c)(g) 	& Excluído 		\\ \hline
	93 & N/A 	   & N/A 		& Duplicado  	\\ \hline
	94 & N/A 	   & N/A 		& Duplicado  	\\ \hline
	95 & N/A 	   & N/A 		& Duplicado  	\\ \hline
	96 & N/A 	   & N/A 		& Duplicado  	\\ \hline
	97 & N/A 	   & N/A 		& Duplicado  	\\ \hline
	98 & N/A 	   & N/A 		& Duplicado  	\\ \hline
	99 & N/A 	   & N/A 		& Duplicado  	\\ \hline
	100 & N/A 	   & N/A 		& Duplicado  	\\ \hline
	101 & N/A 	   & N/A 		& Duplicado  	\\ \hline
	102 & N/A 	   & N/A 		& Duplicado  	\\ \hline
	103 & N/A 	   & N/A 		& Duplicado  	\\ \hline
	104 & N/A 	   & N/A 		& Duplicado  	\\ \hline
	105 & N/A 	   & N/A 		& Duplicado  	\\ \hline
	106 & N/A 	   & N/A 		& Duplicado  	\\ \hline
	107 & N/A 	   & N/A 		& Duplicado  	\\ \hline
	108 & N/A 	   & N/A 		& Duplicado  	\\ \hline
	109 & N/A 	   & N/A 		& Duplicado  	\\ \hline
	110 & N/A 	   & N/A 		& Duplicado  	\\ \hline
	111 & N/A 	   & N/A 		& Duplicado  	\\ \hline
	112 & N/A 	   & N/A 		& Duplicado  	\\ \hline
	113 & N/A 	   & N/A 		& Duplicado  	\\ \hline
	114 & N/A 	   & N/A 		& Duplicado  	\\ \hline
	115 & N/A 	   & N/A 		& Duplicado  	\\ \hline
	116 & N/A 	   & N/A 		& Duplicado  	\\ \hline
	117 & N/A 	   & N/A 		& Duplicado  	\\ \hline
	118 & N/A 	   & N/A 		& Duplicado  	\\ \hline
	119 & N/A 	   & N/A 		& Duplicado  	\\ \hline
	120 & N/A 	   & N/A 		& Duplicado  	\\ \hline
	121 & N/A 	   & N/A 		& Duplicado  	\\ \hline
	122 & N/A 	   & N/A 		& Duplicado  	\\ \hline
	123 & N/A 	   & N/A 		& Duplicado  	\\ \hline
	124 & (a)(b)   & (g) 		& Excluído 		\\ \hline
	125 & (a)(b)   & N/A 		& Incluído 		\\ \hline
	126 & (a)(b)   & (a)(c) 	& Excluído 		\\ \hline
	127 & (a)(c)   & (c) 		& Excluído 		\\ \hline
	128 & (a)(c)   & (c) 		& Excluído 		\\ \hline
	129 & (a)(c)(d)& N/A 		& Incluído 		\\ \hline
	130 & (a)(c)(d)& N/A 		& Incluído 		\\ \hline
	131 & (a)(c)(d)& N/A 		& Incluído 		\\ \hline
	132 & N/A 	   & N/A 		& Duplicado  	\\ \hline
	133 & (a)(c)   & (c)(b)		& Excluído 		\\ \hline
	134 & (a)(c)   & (c)(b)	 	& Excluído 		\\ \hline
	135 & (a)(c)   & (c)(b)	 	& Excluído 		\\ \hline
	136 & (a)(c)   & (c) 		& Excluído 		\\ \hline
	137 & (a)(c)   & (c) 		& Excluído 		\\ \hline
	138 & (a)(c)   & (c) 		& Excluído 		\\ \hline
	139 & (a)(c)   & (c) 		& Excluído 		\\ \hline
	140 & (a)(c)   & (c) 		& Excluído 		
 \end{longtable}


\subsection{Lista de Trabalhos Selecionados}

Após a avaliação de todas as referências retornadas pelos diversos ciclos de pesquisa, foram selecionados os seguintes trabalhos:\newline
\newline
015. VENEZIANO, Vito et al. Usability analysis of ERP software: Education and experience of users' as moderators. In: INTERNATIONAL CONFERENCE ON SOFTWARE, KNOWLEDGE, INFORMATION MANAGEMENT AND APPLICATIONS, 8th, 2014, Dhaka, Bangladesh. \textbf{Proceedings...}. Washington, D.c., Eua: IEEE Computer Society, 2014. p. 1 - 7. Disponível em: <http://dx.doi.org/10.1109/skima.2014.7083560>. Acesso em: 30 jan. 2017.\newline
\newline
031. LAMBECK, Christian et al. (Re-)Evaluating User Interface Aspects in ERP Systems: An Empirical User Study. In: HAWAII INTERNATIONAL CONFERENCE ON SYSTEM SCIENCES, 47., 2014, Waikoloa, Hawaii. \textbf{Proceedings...}. Washington, D.c., Eua: Ieee Computer Society, 2014a. p. 396 - 405.\newline
\newline
032. BABAIAN, Tamara et al. Applying design principles for enhancing enterprise system usability. In: INTERNATIONAL CONFERENCE ON SOFTWARE ENGINEERING AND APPLICATIONS, 9., 2014, Vienna, Austria. \textbf{Proceedings...}. Washington, D.c., Eua: Ieee Computer Society, 2014. p. 162 - 169.\newline
\newline
125. MELO, Espedito L. P. \textbf{Boa Usabilidade e comunicação eficiente de tarefas}: dois aliados na execução de processos em sistemas integrados de gestão. 2015. 152 f. Dissertação (Mestrado) - Curso de Ciência da Computação, Centro de Informática, Universidade Federal de Pernambuco, Pernambuco, 2015.\newline
\newline
129. YASSIEN, Eman et al. The Impact of ERP System's Usability on Enterprise Resource Planning Project Implementation Success via the Mediating Role of User Satisfaction. Journal Of Management Research, [s.l.], v. 9, n. 3, p.49-71, 27 jun. 2017. Macrothink Institute, Inc.. http://dx.doi.org/10.5296/jmr.v9i3.11186.\newline
\newline
130. SCHOLTZ, Brenda M; MAHMUD, Imran; T., Ramayah. Does Usability Matter? An Analysis of the Impact of Usability on Technology Acceptance in ERP Settin. \textbf{Interdisciplinary Journal Of Information, Knowledge, And Management}, [s.l.], v. 11, n. 1, p.309-330, 2016. Informing Science Institute. http://dx.doi.org/10.28945/3591.\newline
\newline
131. FOHRHOLZ, Corinna. Indikatorbasierte Messung der ERP-Usability. Erp Management, Gito Verlag, Berlim, v. 10, n. 4, p.25-28, abr. 2014.

\section{Analise dos Resultados do Mapeamento Sistemático}

Neste tópico é apresentado um resumo comparativo dos trabalhos considerados válidos na revisão, enfatizando os métodos ou técnicas utilizados na pesquisa, além dos principais conceitos empregados.
Após a aplicação dos critérios de inclusão e exclusão das obras, sete trabalhos foram classificados como válidos conforme Tópico 2.5.\newline
\newline
\indent Veneziano et al. (2014) busca estabelecer a relação da influência das informações demográficas dos usuários (ex: antecedentes educacionais e experiências de trabalho) sob a avaliação da usabilidade.\newline
\newline
\indent Para efetuar sua pesquisa Veneziano et al. (2014), utilizou um questionário para a  aquisição de dados e o método survey para a analise das informações.\newline
\newline
\indent O estudo conduzido por  Veneziano et al. (2014),  concluiu que a existe uma relativa influência do grau de formação educacional na percepção da usabilidade, porém elenca como trabalhos futuros o aprofundamento deste estudo em outras populações.\newline
\newline
\indent Lambeck et al. (2014a) citam que os  estudos nos ultimos 20 anos são focados apenas  em ERP's individuais em filiais especificas com pequenos grupos de usuários, o estudo usa uma ampla  amostra de 184 usuários distribuido em pequenas e médias empresas.\newline
\newline
\indent A pesquisa de Lambeck et al. (2014a) é uma revisitação de uma pesquisa feita em 2005, por este motivo a pesquisa tem  dois objetivos, primeiramente  avaliar se os problemas de  usabilidade identificados em 2005 se repetem quase 10 anos depois e amplia o foco da pesquisa para considerações adicionais, como o papel do tipo de menu, a incerteza no uso do sistema ou o suporte em situações problemáticas.\newline
\newline
\indent Lambeck et al. (2014a) avaliam que os problemas encontrados em pesquisas anteriores ainda existem na  atualidade, porém, afirmam que estes problemas  são menos criticos, assim concluem que ainda há alguns esforços necessários para alcançar a visão de uma interface ERP fácil de usar.\newline
\newline
\indent Fohrholz (2014) utiliza figuras-chave para usabilidade e as representa por meio de indicadores levando ao desenvolvimento de uma solução que possibilita a comparação de indicadores.\newline
\newline
\indent Os indicadores utilizados por Fohrholz (2014) são descritos a seguir.\newline
\newline
\indent Busca de informações, quanto menor o tempo do usuário for gasto em busca de informações interpretação de códigos de erro ou ter em busca de informações de ajuda.\newline
\newline
\indent Suporte a erros, um sistema com um alto grau de usabilidade fornecerá mecanismos apropriados para evitar erros.\newline
\newline
\indent Adequação a tarefa descreve até que ponto um sistema suporta o usuário no desempenho de suas tarefas.\newline
\newline
\indent Facilidade na personalização frequência e facilidade com que a personalização precisa ser feita ao sistema pelo usuário.\newline
\newline
\indent Capacidade auto-descritiva de um sistema é satisfeita quando o usuário sabe, em todos os momentos, onde ele está, como as ações devem ser executadas e se a ajuda está sempre disponível.\newline
\newline
\indent Fohrholz (2014) utilizou estes indicadores em três estudos de caso, dois destes estudos foram realizados em sistemas ERP's e o terceiro em um sistema não caracterizado como ERP, concluiu após estes estudos de caso, que é possível estender esta metodologia para sistemas de  diferentes tamanhos e para os mais  diversos fins além dos sistemas ERP.\newline
\newline
\indent Babaian, Xu e Lucas (2014) criaram intervenções baseadas em usabilidade tendo em vista a  orientação do usuário através de reprodução de vídeos com situações semelhantes as tarefas que estão executando, objetivado uma melhora no desempenho da usabilidade do sistema ERP. Foi desenvolvido um protótipo do ERP integra dois tipos de recursos de suporte ao usuário: visualização de processos e reprodução automatizada de interações de tarefas anteriores.\newline
\newline
\indent Babaian, Xu e Lucas (2014) consideram após aplicação deste protótipo em um estudo de caso, que a aplicação desses recursos possibilitou uma relevante melhoria na usabilidade do sistema ERP.\newline
\newline
\indent O estudo conduzido por Scholtz, Mahmud e Ramayah (2016) teve o objetivo de identificar qual o papel da usabilidade na aceitação do ERP por parte do usuário, sendo que a  pesquisa foi efetuada em uma população de 112 usuários do sistema ERP SAP na Índia, nas suas conclusões os pesquisadores  indicam que a usabilidade tem influência sobre a percepção e aceitação do sistema ERP por parte do usuário.\newline
\newline
\newline
\indent A pesquisa Yassien et al. (2017) foi aplicada na Índia em uma população de 106 gerentes em diferentes organizações. O objetivo da pesquisa foi identificar a importância da usabilidade de software para alcançar o ERP-PIS (ERP Project Implementation Success), concluindo que a usabilidade tem uma importante influência para alcançar o ERP-PIS. A pesquisa trata o uso da  Usabilidade na  implementação do ERP e não na pós-implantação que é o foco deste trabalho.\newline

\section{Conclusão do Mapeamento Sistemático}

O mapeamento sistemático foi conduzido no mês de julho de 2018. Ao todo foram retornadas 140 abordagens (IEEE=124, IBICT=3, e Literatura Cinzenta=13). Os trabalhos repetidos (66) foram excluídos e os remanescentes tiveram os seus resumos avaliados. Dos trabalhos restantes (74) foram consideradas elegíveis para a segunda fase desta revisão, que consistiu da leitura dos resumos.\newline
\newline
\indent Os trabalhos que, pelos seus resumos, não satisfizeram o contexto da pesquisa e demais critérios de inclusão e exclusão também foram excluídos, sobrando 7 pesquisas, que por atenderem apenas os critérios de inclusão, tiveram o seu conteúdo analisado por completo, para que assim pudessem compor a síntese da pesquisa.\newline
\newline
\indent A leitura das obras possibilitou a identificação de trabalhos relevantes para o objetivo da revisão, ou seja, encontrar as obras que contribuíssem para que as perguntas estabelecidas no foco de pesquisa fossem respondidas.\newline
\newline
\indent De uma forma geral, buscaram-se obras que relacionassem usabilidade com aceitação de sistemas  ERP na pós-implantação ou estudos da usabilidade em ERP e sua influência na percepção do usuário  (6 pesquisas).\newline
\newline
\indent Em linhas gerais, esta revisão mostrou a escassez literária relacionada ao escopo em questão, e apontou as diferenças encontradas na visão de cada autor em relação a como deve ser tratada a usabilidade em relação aos sistemas ERP. Porém, um consenso entre os  autores é do papel de protagonista da usabilidade face a aceitação do sistema e do sucesso ou não da implantação. A leitura das obras possibilitou a identificação de trabalhos relevantes para o objetivo da revisão, contribuindo para que as perguntas estabelecidas no foco de pesquisa fossem respondidas. O debate referente ao conteúdo destes artigos serão utilizados especialmente no capitulo Usabilidade x ERP, juntamente com os dados da pesquisa exploratória.

\setcounter{chapter}{2}
\renewcommand{\thesection}{\Alph{chapter}.\arabic{section}}
\addtocontents{toc}{\protect\setcounter{tocdepth}{3}}

%\chapter{Teste}

%\section{Seção} 

%\subsection{Subseção}

%\lipsum[26-32] 
% Cícero-FIM

\end{document} 