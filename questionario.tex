\chapter{Questionário aplicado}
%\section{Questionário}
\newcommand\Factor{1.9}

\noindent \begin{center}
\textbf{Avaliando a usabilidade dos sistemas ERP} \\	
Disponível em: http://teresios.com.br/pesquisa\_erp/index.php/768929
\end{center}
\\
\\
\noindent Caro participante,\\
\noindent em quase todos os ramos empresariais no Brasil e além, existem barreiras no uso de sistemas Enterprise Resource Planning (ERP). Este estudo visa identificar os problemas específicos que você enfrenta como usuário de destes sistemas. Com a sua participação, poderemos ter uma visão dos seus desafios diários ao interagir com tais sistemas. Talvez você já tenha experimentado inadequação ao usar sistemas ERP, o melhor do que ninguem sabe o que iria contribuir para melhorar a usabilidade. Com sua contribuição, os tais sistemas podem ser melhor personalizados criando assim sistemas corporativos intuitivos.\\
\noindent Obrigado pelo seu apoio!\\

\noindent \textbf{Declaração de Privacidade}\\
\noindent Os dados serão usados apenas anonimamente para garantir a proteção da privacidade (dos dados). As respostas dadas são completamente focadas na pesquisa em si. Garantimos que seus dados não serão encaminhados para nenhuma outra instância. Em caso de dúvidas, entre em contato com o pesquisador via e-mail, telefone ou endereço postal.\\

\noindent \textbf{Dados do Pesquisador}\\
\noindent Cícero Odílio Cruz, (11) 96196-2967, Avenida Professor Almeida Prado, 532 – Cidade Universitária, Butantã, São Paulo - SP - Coordenadoria de Ensino Tecnológico - Prédio 56\\

\noindent  \textbf{Duração da Pesquisa}\\
\noindent 15 Minutos\\

\newpage
\newlength{\tabquestwidth}
\setlength{\tabquestwidth}{7cm}

%\hspace{-2cm}\begin{tabularx}{\textwidth}{|X|}
\noindent \begin{tabularx}{\textwidth}{|X|}
  \hline
  \tabitem \textbf{Perguntas Relativas ao seu Ambiente Empresarial} \\
  \hline
  \begin{Spacing}{0.8} 
  \textbf{Quantos funcionários sua empresa tem?} \end{Spacing} 
  \begin{Spacing}{2.0} 
  	\tiny \textit{Escolha uma das seguintes respostas} \end{Spacing} 
  ( \ ) Até 9\\
  ( \ ) De 10 Até 49\\
  ( \ ) De 50 Até 250\\
  ( \ ) Mais de 250\\
  ( \ ) Eu não sei \\
  \begin{Spacing}{0.8} \end{Spacing}
  \begin{Spacing}{0.8} 
	\textbf{Qual o setor em que sua  empresa atua?} \end{Spacing} 
  \begin{Spacing}{2.0} 
	\tiny \textit{Escolha uma das seguintes respostas} \end{Spacing} 
	( \ ) Industrial \\
	( \ ) Comércio (Atacado ou Varejo) \\
	( \ ) Transporte, Logística ou Armazenamento \\
	( \ ) Informação e Comunicação \\
	( \ ) Atividades Financeiras e de Seguros \\
	( \ ) Atividade Científica ou Pesquisa/Desenvolvimento \\
	( \ ) Prestação de Serviço \\ 
	( \ ) Outro: \colorbox{white}{ ............................................................................................................ } \\
	\begin{Spacing}{0.8} \end{Spacing}
	\begin{Spacing}{0.8} 
		\textbf{Qual a área de atuação de sua empresa?} \end{Spacing} 
	\begin{Spacing}{2.0} 
		\tiny \textit{Selecione todas as que se apliquem} \end{Spacing} 
	( \ ) Municipal \\
	( \ ) Estadual \\
	( \ ) Federal \\
	( \ ) Continental \\
	( \ ) International \\
	( \ ) Eu não sei \\
	( \ ) Outro: \colorbox{white}{ ............................................................................................................ } \\
	\begin{Spacing}{0.8} \end{Spacing}
	\begin{Spacing}{0.8} 
		\textbf{Que posição você ocupa em sua empresa?} \end{Spacing} 
	\begin{Spacing}{2.0} 
		\tiny \textit{Escolha uma das seguintes respostas} \end{Spacing} 
	( \ ) Empregado / Terceirizado \\
	( \ ) Chefe de Departamento / Gerente \\
	( \ ) Diretor \\
	( \ ) CEO - Chief Executive Officer \\
	( \ ) CIO - Chief Information Officer \\
  \hline
\end{tabularx}

\bigskip

%\noindent \begin{tabularx}{\textwidth}{|X|}
\noindent  \begin{longtable}{|p{15.7cm}|}
	\hline
	\tabitem 	\begin{Spacing}{1.1} 
		\textbf{Informações sobre o seu sistema ERP} \end{Spacing} 
		\tiny \textit{Na seção a seguir, gostaríamos de conhecer mais sobre seu sistema ERP...} 
    \hline
	\begin{Spacing}{1.0} 
		\textbf{Sua empresa  usa algum sistema classificado como {\color{blue} \underline{Enterprise Resource Planning} } (ERP)?  } \end{Spacing} 
	\begin{Spacing}{2.0} 
		\tiny \textit{Escolha uma das seguintes respostas} \end{Spacing} 
	( \ ) Sim\\
	( \ ) Não\\
	\begin{Spacing}{0.8} \end{Spacing}
	\textbf{Por favor, informe o nome do seu ERP?} \\
	\colorbox{white}{ ................................................................................................................................ }\\
	\begin{Spacing}{0.8} \end{Spacing}
	\begin{Spacing}{0.8} 
		\textbf{Quando o sistema ERP foi implantado na sua Empresa?} \end{Spacing} 
	\begin{Spacing}{2.0} 
		\tiny \textit{Informe o Ano aproximado de Implantação} \end{Spacing} 
	\colorbox{white}{ .......................... }\\
	\begin{Spacing}{0.8} \end{Spacing}
	\begin{Spacing}{0.8} 
		\textbf{Seu sistema foi adaptado (customizado) nos seguintes pontos?} \end{Spacing} 
	\begin{Spacing}{2.0} 
		\tiny \textit{Selecione todas as que se apliquem} \end{Spacing} 
	( \ ) Sem adaptações/customizações\\
	( \ ) Eu não sei / desconheço\\
	( \ ) Menus\\
	( \ ) Interface gráfica do usuário e formulários (Telas e Relatórios específicos)\\
	( \ ) Integrações (Interfaces para outros sistemas, por exemplo, entrada / saída de dados)\\
	( \ ) Ambientes Inteiros (por exemplo, módulos específicos) \\
	\begin{Spacing}{0.8} \end{Spacing}
	\begin{Spacing}{0.8} 
		\textbf{Quais dos seguintes aplicativos você usa em conjunto com seu sistema ERP?} \end{Spacing} 
	\begin{Spacing}{2.0} 
		\tiny \textit{Selecione todas as que se apliquem} \end{Spacing} 
	( \ ) Pacote Microsoft Office (Ex: Excel, Word) \\
	( \ ) Sistemas criados por Terceiros (sistemas especializados) \\
	( \ ) Sistemas criados internamente (sistemas especializados) \\
	( \ ) Soluções WEB ou APP's \\
	( \ ) Eu não sei \\
	( \ ) Outro: \colorbox{white}{ ............................................................................................................ } \\
	\\
	\\
	\\
	\begin{Spacing}{0.8} \end{Spacing}
	\begin{Spacing}{0.8} 
		\textbf{Quais são os seus motivos para usar este aplicativo adicional (por exemplo, Excel é uma solução "Melhor", etc.)? Por favor, especifique suas respostas preenchendo a grade abaixo:} \end{Spacing} 
	\begin{Spacing}{2.0} 
		\tiny \textit{Escolha uma opção por cada afirmação...} \end{Spacing} 
	\begin{Spacing}{2.0} \end{Spacing}
	\tiny \begin{tabularx}{15.7 cm}{|X|X|X|X|X|X|X|}
		\hline
		& Concordo Completamente &	Concordo Parcialmente &	Não Concordo Nem Discordo &	Discordo  Parcialmente &	Discordo  Completamente &	Eu Não Sei\\
		\hline
		Os sistemas ERP disponíveis não atendem as nossas necessidades. &   &   &   &   &   &  \\
		\hline
		As funções dos sistemas ERP disponíveis é insuficiente para nossas necessidades. &   &   &   &   &   &  \\
		\hline
		Os custos dos módulos/ambientes adicionais e processos de adaptação de sistemas ERP, são muito altos. &   &   &   &   &   &  \\
		\hline
		O aplicativo adicional aumenta significativamente a flexibilidade. &   &   &   &   &   &  \\
		\hline
		Os sistemas ERP disponíveis não atendem as nossas necessidades. &   &   &   &   &   &  \\
		\hline
		Os sistemas ERP disponíveis não atendem as nossas necessidades. &   &   &   &   &   &  \\
		\hline
	\end{tabularx}\\
	\\
	\begin{Spacing}{0.8} \end{Spacing}
	\begin{Spacing}{0.8} 
		\textbf{Em quais departamentos você usa o sistema ERP?} \end{Spacing} 
	\begin{Spacing}{2.0} 
		\tiny \textit{Selecione todas as que se apliquem} \end{Spacing} 
	( \ ) Contabilidade\\
	( \ ) Recursos Humanos (RH)\\
	( \ ) Produção\\
	( \ ) Compras / Gerenciamento da Cadeia de Suprimentos (SCM)\\
	( \ ) Gerenciamento de Projetos\\
	( \ ) Gestão de Documentos\\
	( \ ) Vendas / Customer Relationship Management (CRM)\\
	( \ ) Gerenciamento de Armazenamento e Inventário\\
	\begin{Spacing}{0.8} \end{Spacing}
	\begin{Spacing}{0.8} 
		\textbf{Sua empresa  esta pensando em implantar um sistema ERP?} \end{Spacing} 
	\begin{Spacing}{2.0} 
		\tiny \textit{Escolha uma das seguintes respostas} \end{Spacing} 
	( \ ) Sim\\
	( \ ) Não\\
	( \ ) Eu Não Sei \ Desconheço\\
	\begin{Spacing}{0.8} \end{Spacing}
	\begin{Spacing}{0.8} 
		\textbf{Quais destes Aplicativos você esta usando em substituição ao sistema ERP?} \end{Spacing} 
	\begin{Spacing}{2.0} 
		\tiny \textit{Selecione todas as que se apliquem} \end{Spacing} 
	( \ ) Pacote Microsoft Office (Ex: Excel, Word) \\
	( \ ) Sistemas criados por Terceiros (sistemas especializados) \\
	( \ ) Sistemas criados internamente (sistemas especializados) \\
	( \ ) Soluções WEB ou APP's \\
	( \ ) Eu não sei \\
	( \ ) Outro: \colorbox{white}{ ............................................................................................................ } \\	
	\begin{Spacing}{0.8} \end{Spacing}
	\begin{Spacing}{0.8} 
		\textbf{Para quais departamentos corporativos você usa esse aplicativo alternativo?} \end{Spacing} 
	\begin{Spacing}{2.0} 
		\tiny \textit{Selecione todas as que se apliquem} \end{Spacing} 
	( \ ) Contabilidade \\
	( \ ) Recursos Humanos (RH) \\
	( \ ) Produção  \\
	( \ ) Compras / Gerenciamento da Cadeia de Suprimentos (SCM)  \\
	( \ ) Gerenciamento de Projetos  \\
	( \ ) Gestão de Documentos  \\
	( \ ) Vendas / Customer Relationship Management (CRM) \\
	( \ ) Gerenciamento de Armazenamento e Inventário \\
	( \ ) Outro: \colorbox{white}{ ............................................................................................................ } \\
	\begin{Spacing}{0.8} \end{Spacing}
	\begin{Spacing}{0.8} 
		\textbf{Quais são os seus motivos para não usar um sistema ERP? Por favor, avalie as declarações de acordo com na tabela a seguir:} \end{Spacing} 
	\begin{Spacing}{2.0} 
	\tiny \textit{Escolha uma opção por cada afirmação...} \end{Spacing} 
	\begin{Spacing}{2.0} \end{Spacing}
	\tiny \begin{tabularx}{15.7 cm}{|X|X|X|X|X|X|X|}
		\hline
		& {\rotatebox[origin=c]{90}{\parbox[c]{2.5cm}{\centering \textcolor{white}{.}\newline \medskip Concordo Completamente}}} 
		& {\rotatebox[origin=c]{90}{\parbox[c]{2.5cm}{\centering \textcolor{white}{.}\newline \medskip Concordo Parcialmente}}} 
		& {\rotatebox[origin=c]{90}{\parbox[c]{2.5cm}{\centering \textcolor{white}{.}\newline \medskip Não Concordo \newline Nem Discordo}}}	 
		& {\rotatebox[origin=c]{90}{\parbox[c]{2.5cm}{\centering \textcolor{white}{.}\newline \medskip Discordo  Parcialmente }}} 
		& {\rotatebox[origin=c]{90}{\parbox[c]{2.5cm}{\centering \textcolor{white}{.}\newline \medskip Discordo  Completamente  }}}
		& {\rotatebox[origin=c]{90}{\parbox[c]{2.5cm}{\centering \textcolor{white}{.}\newline  \textcolor{white}{.}\newline Eu Não Sei }}} \\
		\hline
		Os sistemas ERP existentes são inadequados para pequenas e médias empresas (PME). &   &   &   &   &   &  \\
		\hline
		Os sistemas ERP existentes não atendem a nossa linha de negócios. &   &   &   &   &   &  \\
		\hline
		As funções dos sistemas ERP padrão existentes são inadequadas. &   &   &   &   &   &  \\
		\hline
		Os custos de aquisição de um sistema ERP são muito altos. &   &   &   &   &   &  \\
		\hline
		Os custos de implantação de um sistema ERP são muito altos. &   &   &   &   &   &  \\
		\hline
		A implantação de um sistema ERP traria apenas pequenas melhorias. &   &   &   &   &   &  \\
		\hline
		O tamanho da empresa é muito pequeno para um sistema ERP padrão. &   &   &   &   &   &  \\
		\hline
	\end{tabularx}\\
	\tiny \begin{tabularx}{15.7 cm}{|X|X|X|X|X|X|X|}
	\hline
		& {\rotatebox[origin=c]{90}{\parbox[c]{2.5cm}{\centering \textcolor{white}{.}\newline \medskip Concordo Completamente}}} 
		& {\rotatebox[origin=c]{90}{\parbox[c]{2.5cm}{\centering \textcolor{white}{.}\newline \medskip Concordo Parcialmente}}} 
		& {\rotatebox[origin=c]{90}{\parbox[c]{2.5cm}{\centering \textcolor{white}{.}\newline \medskip Não Concordo \newline Nem Discordo}}}	 
		& {\rotatebox[origin=c]{90}{\parbox[c]{2.5cm}{\centering \textcolor{white}{.}\newline \medskip Discordo  Parcialmente }}} 
		& {\rotatebox[origin=c]{90}{\parbox[c]{2.5cm}{\centering \textcolor{white}{.}\newline \medskip Discordo  Completamente  }}}
		& {\rotatebox[origin=c]{90}{\parbox[c]{2.5cm}{\centering \textcolor{white}{.}\newline  \textcolor{white}{.}\newline Eu Não Sei }}} \\
		\hline
		Há uma incerteza na escolha de um sistema ERP apropriado. &   &   &   &   &   &  \\
		\hline
		Usar um sistema ERP é muito complicado. &   &   &   &   &   &  \\
		\hline
	\end{tabularx}\\
	\begin{Spacing}{1.0} \end{Spacing}
	\begin{Spacing}{0.8} 
		\textbf{Aqui você pode mencionar outras razões:} \end{Spacing} 
	\begin{Spacing}{2.5} \end{Spacing} 
	\colorbox{white}{ .................................................................................................................................. } \\
	\colorbox{white}{ .................................................................................................................................. } \\
	\colorbox{white}{ .................................................................................................................................. } \\
	\colorbox{white}{ .................................................................................................................................. } \\
	\hline
%\end{tabularx}
\end{longtable}

\bigskip

\noindent  \begin{longtable}{|p{15.7cm}|}
%\noindent \begin{tabularx}{\textwidth}{|X|}
	\hline
	\tabitem \textbf{Perguntas Relativas a Usabilidade} \\
	\hline
	%\begin{Spacing}{0.1} \end{Spacing} \\
	\begin{Spacing}{1.8} 
	\parbox[c]{1em}{\includegraphics[width=6in]{mens}} 
	\end{Spacing} 
	\begin{Spacing}{1.2} 
		\hspace{6} Figura 1. Tipos de Menu 
		\end{Spacing}
	\tiny \hspace{6} Imagem de C. Lambeck (imagem cedida pelo autor). \\
	%\vspace{-1} imagem cedida pelo autor \\ 
	%\tiny Tipos de Menu - Lambeck, (2014c), imagem cedida pelo autor \\
	\begin{Spacing}{0.8} 
	\textbf{Quais tipos de menu são oferecidos pelo seu sistema? Por favor, escolha os tipos que mais se aproximam do seu. O conteúdo dos menus não importa, pois são só um exemplo.} \end{Spacing} 	\begin{Spacing}{2.0} 
		\tiny \textit{Selecione todas as que se apliquem} \end{Spacing} 
	( \ ) 1 \\
	( \ ) 2 \\
	( \ ) 3 \\
	( \ ) 4 \\
	( \ ) 5 \\
	( \ ) 6 \\
	( \ ) Eu Não Sei \\
	\begin{Spacing}{0.8} \end{Spacing}
	\begin{Spacing}{0.8} 
		\textbf{Você conhece plenamente todas as etapas necessárias do processo para realizar suas tarefas (por exemplo, realizar uma transação, bancária, inserir um pedido, inserir uma Ordem de Produção)?} \end{Spacing} 
	\begin{Spacing}{2.0} 
		\tiny \textit{Escolha uma das seguintes respostas} \end{Spacing} 
	( \ ) Sim, sempre  \\
	( \ ) Sim, na maioria das vezes  \\
	( \ ) De vez em quando eu me sinto inseguro  \\
	( \ ) Não, a maioria das vezes eu me sinto inseguro  \\
	( \ ) Não, estou constantemente inseguro \\
	( \ ) Eu não sei \\
	( \ ) Outro: \colorbox{white}{ ............................................................................................................ } \\
	\begin{Spacing}{0.8} \end{Spacing}
	\begin{Spacing}{0.8} 
		\textbf{Por favor, avalie seu sistema ERP de acordo com a escala na tabela a seguir:} \end{Spacing} 
	\begin{Spacing}{2.0} 
		\tiny \textit{Escolha uma opção por cada afirmação...} \end{Spacing} 
	\begin{Spacing}{2.0} \end{Spacing}
	\tiny \begin{tabularx}{15.7 cm}{|X|X|X|X|X|X|X|}
		\hline
		& {\rotatebox[origin=c]{90}{\parbox[c]{2.5cm}{\centering \textcolor{white}{.}\newline \medskip Concordo Completamente}}} 
		& {\rotatebox[origin=c]{90}{\parbox[c]{2.5cm}{\centering \textcolor{white}{.}\newline \medskip Concordo Parcialmente}}} 
		& {\rotatebox[origin=c]{90}{\parbox[c]{2.5cm}{\centering \textcolor{white}{.}\newline \medskip Não Concordo \newline Nem Discordo}}}	 
		& {\rotatebox[origin=c]{90}{\parbox[c]{2.5cm}{\centering \textcolor{white}{.}\newline \medskip Discordo  Parcialmente }}} 
		& {\rotatebox[origin=c]{90}{\parbox[c]{2.5cm}{\centering \textcolor{white}{.}\newline \medskip Discordo  Completamente  }}}
		& {\rotatebox[origin=c]{90}{\parbox[c]{2.5cm}{\centering \textcolor{white}{.}\newline  \textcolor{white}{.}\newline Eu Não Sei }}} \\
		\hline
		O meu sistema ERP oferece uma ampla gama de funcionalidades de suporte para lidar com problemas. (por exemplo, explicar causas, oferecer soluções, assistência) &   &   &   &   &   &  \\
		\hline
	\end{tabularx}\\
	\tiny \begin{tabularx}{15.7 cm}{|X|X|X|X|X|X|X|}
		\hline
		& {\rotatebox[origin=c]{90}{\parbox[c]{2.5cm}{\centering \textcolor{white}{.}\newline \medskip Concordo Completamente}}} 
		& {\rotatebox[origin=c]{90}{\parbox[c]{2.5cm}{\centering \textcolor{white}{.}\newline \medskip Concordo Parcialmente}}} 
		& {\rotatebox[origin=c]{90}{\parbox[c]{2.5cm}{\centering \textcolor{white}{.}\newline \medskip Não Concordo \newline Nem Discordo}}}	 
		& {\rotatebox[origin=c]{90}{\parbox[c]{2.5cm}{\centering \textcolor{white}{.}\newline \medskip Discordo  Parcialmente }}} 
		& {\rotatebox[origin=c]{90}{\parbox[c]{2.5cm}{\centering \textcolor{white}{.}\newline \medskip Discordo  Completamente  }}}
		& {\rotatebox[origin=c]{90}{\parbox[c]{2.5cm}{\centering \textcolor{white}{.}\newline  \textcolor{white}{.}\newline Eu Não Sei }}} \\
		\hline
		O meu sistema ERP é muito complexo, o que muitas vezes me faz sentir perdido. &   &   &   &   &   &  \\
		\hline
		A quantidade de informações e detalhes fornecidos é alta para as minhas necessidades. &   &   &   &   &   &  \\
		\hline
		O meu sistema ERP oferece inúmeras e úteis visualizações, as quais eu posso escolher. (por exemplo, tabelas, diagramas, dashboards, organogramas ...) &   &   &   &   &   &  \\
		\hline
		O meu sistema ERP abre muitas janelas ou visualizações simultaneamente o qe prejudica minha compreensão do sistema. &   &   &   &   &   &  \\
		\hline
	\end{tabularx}\\
	\begin{Spacing}{0.8} \end{Spacing}
	\begin{Spacing}{0.8} 
		\textbf{Avalie seu software adicional (por exemplo, Excel ou solução de mercado, etc.) com a ajuda da grade a seguir:} \end{Spacing} 
	\begin{Spacing}{2.0} 
		\tiny \textit{Escolha uma opção por cada afirmação...} \end{Spacing} 
	\begin{Spacing}{2.0} \end{Spacing}
	\tiny \begin{tabularx}{15.7 cm}{|X|X|X|X|X|X|X|}
	\hline
	& {\rotatebox[origin=c]{90}{\parbox[c]{2.5cm}{\centering \textcolor{white}{.}\newline \medskip Concordo Completamente}}} 
	& {\rotatebox[origin=c]{90}{\parbox[c]{2.5cm}{\centering \textcolor{white}{.}\newline \medskip Concordo Parcialmente}}} 
	& {\rotatebox[origin=c]{90}{\parbox[c]{2.5cm}{\centering \textcolor{white}{.}\newline \medskip Não Concordo \newline Nem Discordo}}}	 
	& {\rotatebox[origin=c]{90}{\parbox[c]{2.5cm}{\centering \textcolor{white}{.}\newline \medskip Discordo  Parcialmente }}} 
	& {\rotatebox[origin=c]{90}{\parbox[c]{2.5cm}{\centering \textcolor{white}{.}\newline \medskip Discordo  Completamente  }}}
	& {\rotatebox[origin=c]{90}{\parbox[c]{2.5cm}{\centering \textcolor{white}{.}\newline  \textcolor{white}{.}\newline Eu Não Sei }}} \\
	\hline
	O meu sistema adicional oferece uma ampla gama de funcionalidades de suporte para lidar com problemas (por exemplo, explicar causas, oferecer soluções, assistência) &   &   &   &   &   &  \\
	\hline
	O meu sistema adicional é muito complexo, o que muitas vezes me faz sentir perdido. &   &   &   &   &   &  \\
	\hline
	A quantidade de informações e detalhes fornecidos é alta para as minhas necessidades. &   &   &   &   &   &  \\
	\hline
	O meu sistema adicional oferece inúmeras e úteis visualizações, as quais eu posso escolher (por exemplo, tabelas, diagramas, dashboards, organogramas...) &   &   &   &   &   &  \\
	\hline
	\end{tabularx}\\
	\tiny \begin{tabularx}{15.7 cm}{|X|X|X|X|X|X|X|}
	\hline
	& {\rotatebox[origin=c]{90}{\parbox[c]{2.5cm}{\centering \textcolor{white}{.}\newline \medskip Concordo Completamente}}} 
	& {\rotatebox[origin=c]{90}{\parbox[c]{2.5cm}{\centering \textcolor{white}{.}\newline \medskip Concordo Parcialmente}}} 
	& {\rotatebox[origin=c]{90}{\parbox[c]{2.5cm}{\centering \textcolor{white}{.}\newline \medskip Não Concordo \newline Nem Discordo}}}	 
	& {\rotatebox[origin=c]{90}{\parbox[c]{2.5cm}{\centering \textcolor{white}{.}\newline \medskip Discordo  Parcialmente }}} 
	& {\rotatebox[origin=c]{90}{\parbox[c]{2.5cm}{\centering \textcolor{white}{.}\newline \medskip Discordo  Completamente  }}}
	& {\rotatebox[origin=c]{90}{\parbox[c]{2.5cm}{\centering \textcolor{white}{.}\newline  \textcolor{white}{.}\newline Eu Não Sei }}} \\
	\hline
	O meu sistema adicional abre muitas janelas ou visualizações simultaneamente o qe prejudica minha compreenssão do sistema. &   &   &   &   &   &  \\
	\hline
	\end{tabularx}\\
	\\
	\begin{Spacing}{0.8} \end{Spacing}
	\begin{Spacing}{0.8} 
		\textbf{Como você executa funções (por exemplo, iniciando incluir um pedido)?} \end{Spacing} 
	\begin{Spacing}{2.0} 
		\tiny \textit{Selecione todas as que se apliquem} \end{Spacing} 
	( \ ) Palavra-Chave ou Atalho \\
	( \ ) Tecla de Função do Teclado (por exemplo, F2) \\
	( \ ) Cursor do mouse e clique do mouse \\
	( \ ) Selecione a função, seguida do comando de execução \\
	( \ ) Eu Não Sei \\
	( \ ) Outro: \colorbox{white}{ ............................................................................................................ } \\
	\begin{Spacing}{0.8} \end{Spacing}
	\begin{Spacing}{0.8} 
	\textbf{Você está sempre ciente das consequências de suas ações? (por exemplo, alterações no sistema resultantes, efeitos colaterais, operações de fluxo de trabalho afetadas)} \end{Spacing} 
	\begin{Spacing}{2.0} 
		\tiny \textit{Escolha uma das seguintes respostas} \end{Spacing} 
	( \ ) Sim, Sempre \\
	( \ ) Sim, a Maioria das Vezes\\
	( \ ) De Vez em Quando eu me Sinto Inseguro\\
	( \ ) Não, a Maioria das Vezes eu me Sinto Inseguro\\
	( \ ) Não, estou Constantemente Inseguro\\
	( \ ) Eu Não Sei \\
	\begin{Spacing}{0.8} \end{Spacing}
	\begin{Spacing}{0.8} 
		\textbf{Qual é o seu método preferido para pesquisar informações?} \end{Spacing} 
	\begin{Spacing}{2.0} 
		\tiny \textit{Selecione todas as que se apliquem} \end{Spacing} 
	( \ ) Pesquisa de texto completo (por exemplo, termos, IDs) \\
	( \ ) Auto-completar \\
	( \ ) Sinônimos e correção automática (opção do Google “Você quis dizer…?”) \\
	( \ ) Registrar e indexar (por exemplo, listagens de categorias em ordem alfabética) \\
	( \ ) Eu Não Sei \\
	( \ ) Outro: \colorbox{white}{ ............................................................................................................ } \\
	\begin{Spacing}{0.8} \end{Spacing}
	\begin{Spacing}{0.8} 
		\textbf{Como você avalia as seguintes estratégias para lidar com problemas no uso do sistema ERP?} \end{Spacing} 
	\begin{Spacing}{2.0} 
		\tiny \textit{Escolha uma opção por cada afirmação...} \end{Spacing} 
	\begin{Spacing}{2.0} \end{Spacing}
	\tiny \begin{tabularx}{15.7 cm}{|X|X|X|X|X|X|X|}
		\hline
		& {\rotatebox[origin=c]{90}{\parbox[c]{2.5cm}{\centering \textcolor{white}{.}\newline \medskip Concordo Completamente}}} 
		& {\rotatebox[origin=c]{90}{\parbox[c]{2.5cm}{\centering \textcolor{white}{.}\newline \medskip Concordo Parcialmente}}} 
		& {\rotatebox[origin=c]{90}{\parbox[c]{2.5cm}{\centering \textcolor{white}{.}\newline \medskip Não Concordo \newline Nem Discordo}}}	 
		& {\rotatebox[origin=c]{90}{\parbox[c]{2.5cm}{\centering \textcolor{white}{.}\newline \medskip Discordo  Parcialmente }}} 
		& {\rotatebox[origin=c]{90}{\parbox[c]{2.5cm}{\centering \textcolor{white}{.}\newline \medskip Discordo  Completamente  }}}
		& {\rotatebox[origin=c]{90}{\parbox[c]{2.5cm}{\centering \textcolor{white}{.}\newline  \textcolor{white}{.}\newline Eu Não Sei }}} \\
		\hline
		Nível Configurável de Detalhes da Informação &   &   &   &   &   &  \\
		\hline
		Quantidade Configurável de Informação &   &   &   &   &   &  \\
		\hline
		Amplas Formas de Visualizações &   &   &   &   &   &  \\
		\hline
		Feedback: Visual, Tátil ou Auditivo &   &   &   &   &   &  \\
		\hline
		Orientação e Suporte ao Usuário (por exemplo, informação de progresso, indicação de campos de obrigatórios nos formulários, alternativas de fluxo a seguir)  &   &   &   &   &   &  \\
		\hline
		Tipos de Menu e Estruturas Aprimorados &   &   &   &   &   &  \\
		\hline
		Suporte a Dispositivos Sensíveis ao Toque (por exemplo, multi-touch, sistema de mesa) &   &   &   &   &   &  \\
		\hline
	\end{tabularx}\\
	\tiny \begin{tabularx}{15.7 cm}{|X|X|X|X|X|X|X|}
		\hline
		& {\rotatebox[origin=c]{90}{\parbox[c]{2.5cm}{\centering \textcolor{white}{.}\newline \medskip Concordo Completamente}}} 
		& {\rotatebox[origin=c]{90}{\parbox[c]{2.5cm}{\centering \textcolor{white}{.}\newline \medskip Concordo Parcialmente}}} 
		& {\rotatebox[origin=c]{90}{\parbox[c]{2.5cm}{\centering \textcolor{white}{.}\newline \medskip Não Concordo \newline Nem Discordo}}}	 
		& {\rotatebox[origin=c]{90}{\parbox[c]{2.5cm}{\centering \textcolor{white}{.}\newline \medskip Discordo  Parcialmente }}} 
		& {\rotatebox[origin=c]{90}{\parbox[c]{2.5cm}{\centering \textcolor{white}{.}\newline \medskip Discordo  Completamente  }}}
		& {\rotatebox[origin=c]{90}{\parbox[c]{2.5cm}{\centering \textcolor{white}{.}\newline  \textcolor{white}{.}\newline Eu Não Sei }}} \\
		\hline
		Funcionalidade de Pesquisa Avançada &   &   &   &   &   &  \\
		\hline
	\end{tabularx}\\
	\begin{Spacing}{1.0} \end{Spacing}
	\begin{Spacing}{0.8} 
	\textbf{Aqui você pode mencionar outras razões:} \end{Spacing} 
	\begin{Spacing}{2.5} \end{Spacing} 
	\colorbox{white}{ .................................................................................................................................. } \\
	\colorbox{white}{ .................................................................................................................................. } \\
	\colorbox{white}{ .................................................................................................................................. } \\
	\colorbox{white}{ .................................................................................................................................. } \\
	\begin{Spacing}{0.4} \end{Spacing}
	\begin{Spacing}{1.4} 
		\textbf{ O conceito a seguir usa uma lista de materiais interativa e baseada em gráficos (BOM). Suporta:}  \end{Spacing} 	
	\begin{Spacing}{1.4} 
		\textbf{• ilustração de descartabilidades e situações problemáticas (código de cores);} \end{Spacing} 	
	\begin{Spacing}{1.4} 
		\textbf{• acesso ao formulário ERP associado clicando no nó;} \end{Spacing} 	
	\begin{Spacing}{1.4} 
		\textbf{• diferentes níveis de detalhes e aspectos (à esquerda: tempo, no meio: as etapas de produção discretos, à direita: custos);} \end{Spacing} 	
	\begin{Spacing}{1.4} 
		\textbf{Por favor, avalie sua primeira impressão do conceito apresentado em relação ao seu grau de inovação e a utilidade esperada.} \end{Spacing} 	
	\begin{Spacing}{1.4} 
	\tiny \textit{Selecione uma opção por afirmação} \end{Spacing} 
	\begin{Spacing}{2.5} 
	\parbox[c]{1em}{\includegraphics[width=6in]{CENTERIS_concept}} 
	\end{Spacing} 
	\begin{Spacing}{1.2} 
		\hspace{6} Figura 2. Modelo BOM 
	\end{Spacing}
	\tiny \hspace{6} Imagem de C. Lambeck (imagem cedida pelo autor). \\
	%\vspace{-1} imagem cedida pelo autor \\ 
	%\tiny Tipos de Menu - Lambeck, (2014c), imagem cedida pelo autor \\
	\tiny \begin{tabularx}{15.7 cm}{|X|X|X|X|X|X|X|X|}
	\hline
	          & Muito Boa & Boa  & Suficiente & Insuficiente & Ruim & Muito Ruim & Eu Não Sei\\
	\hline
	Inovação  &   &   &   &   &  &  &  \\
		\hline
	Utilidade &   &   &   &   &  &  &  \\
		\hline
	\end{tabularx} \\
	\begin{Spacing}{1.0} \end{Spacing}
	\begin{Spacing}{0.8} 
		\textbf{Outras observações sobre esta questão:} \end{Spacing} 
	\begin{Spacing}{2.5} \end{Spacing} 
	\colorbox{white}{ .................................................................................................................................. } \\
	\colorbox{white}{ .................................................................................................................................. } \\
	\colorbox{white}{ .................................................................................................................................. } \\
	\colorbox{white}{ .................................................................................................................................. } \\
	\begin{Spacing}{0.4} \end{Spacing}
	\begin{Spacing}{1.4} 
		\textbf{O conceito a seguir suporta o planejamento de produção utilizando:}  \end{Spacing} 	
	\begin{Spacing}{1.4} 
		\textbf{• um sistema de mesa;} \end{Spacing} 	
	\begin{Spacing}{1.4} 
		\textbf{• visão superior em máquinas ou bancadas de trabalho e seus fluxos de materiais;} \end{Spacing} 	
	\begin{Spacing}{1.4} 
		\textbf{• gráficos de Gantt relacionados e interativos;} \end{Spacing} 	
	\begin{Spacing}{1.4} 
		\textbf{Cada entrada e seleção de usuário será aplicada por toque direto ou objetos tangíveis (por exemplo, dados vermelhos na imagem à direita). Além disso, um esquema de cores facilita a legibilidade.} \end{Spacing} 	
	\begin{Spacing}{1.4} 
		\textbf{Por favor, avalie sua primeira impressão do conceito apresentado em relação ao seu grau de inovação e a utilidade esperada.} \end{Spacing} 
	\begin{Spacing}{1.4} 
		\tiny \textit{Selecione uma opção por afirmação} \end{Spacing} 
	\begin{Spacing}{2.5} 
		\parbox[c]{1em}{\includegraphics[width=6in]{AVI_concept}} 
	\end{Spacing} 
	\begin{Spacing}{1.2} 
		\hspace{6} Figura 3. Conceito AVI  
	\end{Spacing}
	\tiny \hspace{6} Imagem de C. Lambeck (imagem cedida pelo autor). \\
	%\vspace{-1} imagem cedida pelo autor \\ 
	%\tiny Tipos de Menu - Lambeck, (2014c), imagem cedida pelo autor \\
	\tiny \begin{tabularx}{15.7 cm}{|X|X|X|X|X|X|X|X|}
	\hline
	& Muito Boa & Boa  & Suficiente & Insuficiente & Ruim & Muito Ruim & Eu Não Sei\\
	\hline
	Inovação  &   &   &   &   &  &  &  \\
	\hline
	Utilidade &   &   &   &   &  &  &  \\
	\hline
\end{tabularx} \\
\begin{Spacing}{1.0} \end{Spacing}
\begin{Spacing}{0.8} 
	\textbf{Outras observações sobre esta questão:} \end{Spacing} 
\begin{Spacing}{2.5} \end{Spacing} 
\colorbox{white}{ .................................................................................................................................. } \\
\colorbox{white}{ .................................................................................................................................. } \\
\colorbox{white}{ .................................................................................................................................. } \\
\hline
%\end{tabularx}
\end{longtable}

\bigskip

%\noindent \begin{tabularx}{\textwidth}{|X|}
\noindent  \begin{longtable}{|p{15.7cm}|}
	\hline
	\tabitem \textbf{Acesso ao Sistema} \\
	\hline
	\begin{Spacing}{0.8} 
	\textbf{Como você acessa seu sistema ERP?} \end{Spacing} 
	\begin{Spacing}{2.0} 
		\tiny \textit{Escolha uma das seguintes respostas} \end{Spacing} 
	( \ ) Navegador (por exemplo, Firefox, Internet Explorer, Chrome) \\
	( \ ) Cliente Desktop (instalação do PC)\\
	( \ ) Ambos \\
	( \ ) Outro: \colorbox{white}{ ............................................................................................................ } \\
	\begin{Spacing}{0.8} \end{Spacing}
	\begin{Spacing}{0.8} 
	\textbf{Como você avalia essa maneira de acessar o sistema ERP em geral? (por exemplo, ocorrência de problemas, preferência pessoal, frequência de atualização)} \end{Spacing} 
	\begin{Spacing}{2.0} 
		\tiny \textit{Escolha uma das seguintes respostas} \end{Spacing} 
	( \ ) Muito Boa \\
	( \ ) Boa \\
	( \ ) Indiferente \\
	( \ ) Ruim \\
	( \ ) Muito Ruim \\
	( \ ) Outro: \colorbox{white}{ ............................................................................................................ } \\
	\\
	\begin{Spacing}{0.8} \end{Spacing}
	\begin{Spacing}{0.8} 
		\textbf{Por favor, nomeie as dificuldades ao acessar o sistema da maneira indicada.?} \end{Spacing} 
	\begin{Spacing}{2.0} 
		\tiny \textit{} \end{Spacing} 
	\colorbox{white}{ .................................................................................................................................. } \\
	\colorbox{white}{ .................................................................................................................................. } \\
	\colorbox{white}{ .................................................................................................................................. } \\
	\colorbox{white}{ .................................................................................................................................. } \\
	\begin{Spacing}{0.8} \end{Spacing}
	\begin{Spacing}{0.8} 
		\textbf{Você usa  quais dos seguintes dispositivos móveis para acessar as informações da sua empresa?} \end{Spacing} 
	\begin{Spacing}{2.0} 
		\tiny \textit{Selecione todas as que se apliquem} \end{Spacing} 
	( \ ) Nenhum  \\
	( \ ) Laptop  \\
	( \ ) NetBook  \\
	( \ ) Tablet  \\
	( \ ) Smartphone  \\
	( \ ) PDA/Handheld \\
	( \ ) Outro: \colorbox{white}{ ............................................................................................................ } \\
	\begin{Spacing}{0.8} \end{Spacing}
	\begin{Spacing}{0.8} 
		\textbf{Você usa seu dispositivo móvel para acessar os dados de quais departamentos?} \end{Spacing} 
	\begin{Spacing}{2.0} 
		\tiny \textit{Selecione todas as que se apliquem} \end{Spacing} 
	( \ ) Contabilidade \\
	( \ ) Recursos Humanos (RH) \\
	( \ ) Compras \\
	( \ ) Produção \\
	( \ ) Gerenciamento da Cadeia de Suprimentos (SCM) \\
	( \ ) Gerenciamento de Projetos  \\
	( \ ) Vendas / Customer Relationship Management (CRM) \\
	( \ ) Gerenciamento de Armazenamento e Inventário (WMS) \\
	( \ ) Outro: \colorbox{white}{ ............................................................................................................ } \\
	\hline
%\end{tabularx}
\end{longtable}

\bigskip

%\noindent \begin{tabularx}{\textwidth}{|X|}
\noindent  \begin{longtable}{|p{15.7cm}|}
	\hline
	\tabitem \textbf{Dados Funcionais} \\
	\hline
	\begin{Spacing}{0.8} 
		\textbf{Selecione a Faixa Etária a qual você pertence:} \end{Spacing} 
	\begin{Spacing}{2.0} 
		\tiny \textit{Escolha uma das seguintes respostas} \end{Spacing}
	( \ ) Até 30 Anos \\
	( \ ) 31  Anos até 40  Anos \\
	( \ ) 41  Anos até 50  Anos \\
	( \ ) 51  Anos até 60  Anos \\
	( \ ) 61 ou mais \\
	\begin{Spacing}{0.8} \end{Spacing}
	\begin{Spacing}{0.8} 
		\textbf{Qual seu Sexo:} \end{Spacing} 
	\begin{Spacing}{2.0} 
		\tiny \textit{Escolha uma das seguintes respostas} \end{Spacing} 
	( \ ) Feminino \\
	( \ ) Masculino \\
	\begin{Spacing}{0.8} \end{Spacing}
	\begin{Spacing}{0.8} 
		\textbf{Há quantos anos você trabalha na empresa?} \end{Spacing} 
	\begin{Spacing}{2.0} 
		\tiny \textit{Escolha uma das seguintes respostas} \end{Spacing} 
	( \ ) Menos de 1 ano \\
	( \ ) De 1 á 3 anos \\
	( \ ) De 3 á 6 anos \\
	( \ ) De 7 á 10 anos \\
	( \ ) Mais que 10 anos \\
	\begin{Spacing}{0.8} \end{Spacing}
	\begin{Spacing}{0.8} 
		\textbf{Há quanto tempo você usa Sistemas ERP no Geral?} \end{Spacing} 
	\begin{Spacing}{2.0} 
		\tiny \textit{Escolha uma das seguintes respostas} \end{Spacing} 
	( \ ) Menos de 1 Ano \\
	( \ ) De 1 á 3 Anos \\
	( \ ) De 4 á 6 Anos \\
	( \ ) De 7 á 10 Anos \\
	( \ ) Mais de 10 Anos \\
	\begin{Spacing}{0.8} \end{Spacing}
	\begin{Spacing}{0.8} 
		\textbf{Como você auto avalia sua experiência com sistemas ERP?} \end{Spacing} 
	\begin{Spacing}{2.0} 
		\tiny \textit{Escolha uma das seguintes respostas} \end{Spacing} 
	( \ ) Muito Boa \\
	( \ ) Muito Ruim \\
	( \ ) Indiferente \\
		\begin{Spacing}{0.8} \end{Spacing}
	\begin{Spacing}{0.8} 
		\textbf{Qual sua Região de Origem?} \end{Spacing} 
	\begin{Spacing}{2.0} 
		\tiny \textit{Escolha uma das seguintes respostas} \end{Spacing} 
	( \ ) Norte \\
	( \ ) Nordeste \\
	( \ ) Centro-Oeste \\
	( \ ) Sudeste \\
	( \ ) Sul \\
	\begin{Spacing}{0.8} \end{Spacing}
	\begin{Spacing}{0.8} 
		\textbf{Qual a região do País onde Mora?} \end{Spacing} 
	\begin{Spacing}{2.0} 
		\tiny \textit{Escolha uma das seguintes respostas} \end{Spacing} 
	( \ ) Norte \\
	( \ ) Nordeste \\
	( \ ) Centro-Oeste \\
	( \ ) Sudeste \\
	( \ ) Sul \\
	\begin{Spacing}{0.8} \end{Spacing}
	\begin{Spacing}{0.8} 
		\textbf{Qual dos seguintes dispositivos móveis você está usando no seu tempo livre?} \end{Spacing} 
	\begin{Spacing}{2.0} 
		\tiny \textit{Selecione todas as que se apliquem} \end{Spacing}
	( \ ) Nenhum \\
	( \ ) Notebook \\
	( \ ) Netbook \\
	( \ ) Tablet \\
	( \ ) Smartphone \\
	( \ ) PDA/Handheld \\
	\begin{Spacing}{0.8} \end{Spacing}
	\begin{Spacing}{0.8} 
		\textbf{No seu tempo livre você usa seu PC ou dispositivo móvel para quais fins?} \end{Spacing} 
	\begin{Spacing}{2.0} 
		\tiny \textit{Selecione todas as que se apliquem} \end{Spacing} 
	( \ ) Mídias Sociais e Chat (Facebook, Skype, LinkedIn etc.) \\
	( \ ) PC ou jogos online \\
	( \ ) Processamento de texto e planilha \\
	( \ ) Aplicações multimídia (música, vídeo, imagens) \\
	( \ ) Pesquisa de Informações \\
	( \ ) Outro: \colorbox{white}{ ............................................................................................................ } \\
	\begin{Spacing}{0.8} \end{Spacing}
	\begin{Spacing}{0.8} 
		\textbf{Com que frequência você usa esses dispositivos no seu tempo livre?}
	\end{Spacing} 
	\begin{Spacing}{2.0} 
		\tiny \textit{Escolha uma das seguintes respostas} \end{Spacing}
	( \ ) Menos de uma vez por semana \\
	( \ ) Uma vez por semana \\
	( \ ) Várias vezes por semana \\
	( \ ) Diariamente \\
	( \ ) Várias vezes por dia \\
	\hline
	%\end{tabularx}
\end{longtable}

\bigskip

%\noindent \begin{tabularx}{\textwidth}{|X|}
\noindent  \begin{longtable}{|p{15.7cm}|}
	\hline
	\tabitem \textbf{Dados de Contato} \\
	\hline
	\begin{Spacing}{0.8} 
		\textbf{Obrigado pela sua participação e, assim, apoiar a melhoria dos sistemas ERP! Caso você queira receber os resultados compilados da pesquisa, por favor digite seu endereço de e-mail:} \end{Spacing} \\
	\begin{Spacing}{0.0} 
		\tiny \end{Spacing} 
	\colorbox{white}{ .................................................................................................................................. } 
	\begin{Spacing}{2.0} \end{Spacing}
	\begin{Spacing}{0.8} 
		\textbf{Como agradecimento adicional por sua participação, gostaríamos de enviar uma cópia completa de nossa pesquisa impressa. Se você estiver interessado, digite seu endereço postal:} \end{Spacing} 
	\begin{Spacing}{0.5} \end{Spacing}\\
	\begin{Spacing}{0.5} \end{Spacing}
	Nome Completo: \colorbox{white}{ ..................................................................................................... } \\
	Empresa: \colorbox{white}{ ................................................................................................................ } \\
	Logradouro: \colorbox{white}{ ............................................................................................................ } \\
	Número: \colorbox{white}{ ................................................................................................................. } \\
	CEP: \colorbox{white}{ ....................................................................................................................... } \\
	Cidade: \colorbox{white}{ ................................................................................................................... } \\
	Estado: \colorbox{white}{ ................................................................................................................... } \\
	\\
	\hline
	%\end{tabularx}
\end{longtable}