\chapter{Mapeamento Sistemático}
\addtocontents{toc}{\protect\setcounter{tocdepth}{0}}
\setcounter{chapter}{1}
%\renewcommand{\thesection}{\arabic{chapter}.\arabic{section}}
\renewcommand{\thesection}{\arabic{section}}
Este anexo tem o objetivo de documentar a execução da metodologia do Mapeamento Sistemático, com o intuito de agregar informações a pesquisa da Dissertação.

\section{Planejamento} %\label{RT - Planejamento}

\subsection{Objetivo}
\newline
\newline
\indent
O objetivo da pesquisa proposta é avaliar a Usabilidade em ferramentas ERP e traçar um perfil do usuário brasileiro comparando-o com o alemão, sob aspecto da usabilidade, definindo assim qual a real importância da usabilidade na escolha e na permanência do sistema ERP na empresa, sendo assim,  por se tratar de um tema amplo esta pesquisa não pretende adentrar nas características especificas de um determinado ERP, mas sim determinar “O quê é importante” do ponto de vista da usabilidade para que um ERP se mantenha na empresa. 
\newline

\subsection{Questão de Pesquisa para Revisão}
\newline
\newline
\noindent Quais e quantos estudos relacionam a Usabilidade com sistemas ERP na pós-implantação do sistema?\newline
Obs: aqui não tratamos de aspectos pois estamos fazendo um mapeamento sistemático das pesquisas relacionadas, buscamos a pós-implantação pois se relaciona com nossa referência de controle;
\newline

\underline{Intervenção}: Impacto da influência da usabilidade na aceitabilidade de sistemas ERP por parte dos usuários no pós-implantação.
\newline

\underline{Controle}: O estudo usado como apoio para iniciar o mapeamento sistemático foi obtido por meio de pesquisa exploratória de artigos publicados em periódicos e/ou anais de eventos, obtidos mediante busca em base de dados eletrônica, utilizando as palavras chave “ERP usability” em (IEEE Digital Library (http://ieeexplore.ieee.org/Xplore/)). Desta forma a referência utilizada como controle é  Lambeck et al (2014b). Este artigo serviu de base para definição das palavras chaves, fontes e período da busca, assim como a validação da abrangência do tema pesquisado.
\newline

\underline{População}: Estudos que relacionam usabilidade com ERP's, restritos a sistemas tradicionais não criados para uso em aparelhos móveis ou na world wide web (web);
\newline

\underline{Resultado}: possibilitar uma visão mais  abrangente do uso da Usabilidade relacionados a avaliação da eficiência do ERP pós implantação
\newline

\underline{Aplicação}: usuários de sistemas ERP na fase de pós-implantação.
\newline

\subsection{Seleção das Fontes}
\newline
\newline
\textbf{Critério de Definição de Fontes}: As fontes devem estar disponíveis via web, preferencialmente em bases científicas da área de computação (bibliotecas digitais online, bases eletrônicas indexadas, anais de eventos da área, periódicos, mapeamentos sistemáticos anteriores).
\newline
\newline
\textbf{Listagem das Fontes}: Estas fontes são reconhecidas mundialmente pela produção literária de alta qualidade considerando as principais revistas literárias e eventos científicos no campo da Engenharia de Software:

\indent a. IEEE Digital Library (http://ieeexplore.ieee.org/Xplore/);\newline
\indent b. Portal Brasileiro de Publicações Ciêntificas em Acesso Aberto (http://oasisbr.ibict. 
br/vufind/);\newline
\indent c. Portal de Publicações da Universidade de Dresden (https://tu-dresden.de/ing/infor
matik/smt/mg/forschung/publikationen);
\newline

\noindent \textbf{Idioma dos Trabalhos}: Inglês, por ser o idioma com maior aceitação internacional para trabalhos científicos, Alemão pela influência da maior empresa do setor (SAP) e também pela origem do trabalho de referência e Português, para que os trabalhos existentes de pesquisadores brasileiros pudessem ser contemplados.
\newline
\textbf{Palavras Chaves}: Palavras chaves em língua portuguesa: “ERP” relacionados com o termo “Usabilidade”. Palavras chaves em língua inglesa: “ERP” ou “Enterprise Resource Planning” relacionado com os termos “Usability”. Palavras chaves em língua alemã: “ERP” ou “ERP-Systemen” relacionado com os termos “Usability” usadas exclusivamente no site da Universidade de Dresden.
\newline
Recorreu-se ao operador lógico “OR” para combinação das palavras chaves. Além do uso dos operadores NEAR e ONEAR no site (a) e o equivalente no site (b) acima descriminados.
\newline
\newline
\subsection{Critérios de Inclusão e Exclusão dos Trabalhos}
\newline
\newline
Foram estabelecidos critérios de inclusão ou exclusão com o objetivo de limitar a seleção de trabalhos com base em avaliações qualitativas relevantes tendo como referência o objetivo do mapeamento sistemático neste trabalho. Partindo deste contexto, os seguintes critérios foram estabelecidos para esta revisão:
\newline
\newline
\noindent \textbf{Critérios de Inclusão}\newline
\newline
\indent a. Trabalhos publicados e disponíveis integralmente na web em formato eletrônico que atendam as strings de busca;\newline
\indent b. Trabalhos publicados no período de 2001 à Julho de 2018, descritos em inglês ou português;\newline
\indent c. Trabalhos publicados no período de 2001 à Julho de 2018, descritos em alemão na Universidade de Dresden;\newline
\indent d. Estudos que relacionam os aspectos de Usabilidade no uso do ERP's.\newline

\noindent \textbf{Critérios de Exclusão}\newline
\newline
\indent a. Trabalhos que apresentam avaliações sem apresentar o método utilizado;\newline 
\indent b. Estudos anteriores a 2013;\newline 
\indent c. Trabalhos que não relacionam Usabilidade com ERP caso venham a ocorrer;\newline 
\indent d. Estudo repetido, quando comparado com o resultado das demais buscas;\newline 
\indent e. Estudo incompleto (texto, conteúdo e resultados incompletos);\newline 
\indent f. Pôsteres, tutoriais, relatórios técnicos;\newline 
\indent g. Relacionar Usabilidade com sistemas erp nas plataformas web ou móveis (app's);\newline 
\indent h. Relacionar Usabilidade em ERP's na fase de implementação ou implantação do sistema;\newline 
\indent i. Relacionar Usabilidade em sistemas CRM, MRP ou outros sistemas que não ERP;\newline
\indent j. Relacionar Usabilidade com ERP em processos de Gamificação;\newline

\noindent \textbf{Definição da Estratégia de Seleção de Dados}\newline

\indent Por meio das palavras chave foram criadas as \textit{strings} de busca, executadas em cada uma das fontes selecionadas. As buscas são realizadas sempre nos campos ( “resumo” (abstract), titulo (title) ) do documento, com extração das seguintes informações: título do documento, autores, fonte, ano de publicação e resumo.\newline

\indent Devido às particularidades das máquinas de busca poderá haver adaptações nas strings, obedecendo as seguintes diretrizes:\newline 
\indent (1) a string derivada deverá ser logicamente equivalente à string original, ou;\newline 
\indent (2) na impossibilidade de se manter equivalência exata, deverá a string derivada ser mais abrangente para evitar perda de documentos potencialmente relevantes.\newline
\indent (3) a string inicial será modificada até que restem somente trabalhos duplicados.\newline

\noindent \underline{Primeira String na Língua Inglesa}: (( Usability AND ERP ) OR ( Usabilidade AND ERP ));\newline

\noindent \underline{Segunda String na Língua Inglesa}: ((usability NEAR/3 ERP) OR (usability ONEAR/3 ERP));\newline

\noindent \underline{Terceira String na Língua Inglesa}: (((usability NEAR/3 ERP) OR (ERP ONEAR/10 usability)));\newline

\noindent \underline{Quarta String na Língua Inglesa}: ((.QT.Enterprise Resource Planning.QT. NEAR/3 usability)  OR (.QT.Enterprise Resource Planning.QT. ONEAR/10 usability) );\newline

\noindent \underline{Quinta String na Língua Inglesa}: ((.QT.Enterprise Resource Planning.QT. NEAR/3 usability)  OR (.QT.Enterprise Resource Planning.QT. ONEAR/10 usability) ) OR ((usability NEAR/3 ERP) OR (ERP ONEAR/10 usability));\newline

\noindent \underline{Sexta String na Língua Inglesa}: (((usability NEAR/3 .QT.Enterprise Resource Planning.QT.) OR (usability ONEAR/10 .QT.Enterprise Resource Planning.QT.) ) OR ((usability NEAR/3 ERP) OR (usability ONEAR/10 ERP)));\newline

\noindent Obs: É importante ressaltar que o uso do NEAR e ONEAR pode ou não ser suportado pela ferramenta de busca, sendo que seu uso facilita a relação entre as strings.\newline

\noindent \underline{Primeira String na Língua Portuguesa}: "(Resumo Português:"usabilidade ERP"~10)";\newline 

\noindent \underline{Segunda String na Língua Portuguesa}: "(Resumo Português:" " "Enterprise Resource"~1 Planning"~1 usabilidade"~10)";\newline

\noindent \underline{String na Língua Alemã}: "( Usability AND ERP-Systemen )";\newline

\noindent Obs: A string na  lingua  alemã foi utilizada somente na  literatura cinzenta “ textos produzidos por todos os níveis do governo, institutos, academias, empresas e indústria, em formato impresso e eletrônico publicado locais específicos não indexados, e que não é controlado por editores, sejam, científicos ou comerciais”.\newline

Os trabalhos recuperados das bases serão documentados em uma  ferramenta especificamente projetada para a Revisão Sistemática e Mapeamento Sistemático, elaborada pelo LaPES da UFSCar denominada StaRt. \newline

A ferramenta StaRt armazena as referências bem como o abstract e  auxilia nos processos de Seleção, Extração e Sumarização. \newline

Após a leitura minuciosa e completa dos trabalhos incluídos, foi elaborado um resumo comparativo das obras, redigido pelo pesquisador, destacando os métodos ou técnicas utilizadas, além dos principais dados da pesquisa, quando for o caso. \newline

\subsection{Sumarização dos Resultados}
\newline
\indent Com os resultados obtidos, deve ser redigida uma síntese geral que descreve sinteticamente as análises críticas elaboradas pelo revisor. Análises qualitativas e quantitativas, com relação aos trabalhos pesquisados e algumas considerações sobre os resultados observados nos trabalhos selecionados.\newline

\section{Condução do Mapeamento Sistemático} \label{RT - Condução do Mapeamento Sistemático}
\newline
\newline
\indent A busca das referências utilizadas nesta revisão foi realizada em bases de dados eletrônicas. Por meio dos critérios de inclusão e exclusão definiram-se os trabalhos incluídos e excluídos da revisão. As referências que preencheram os critérios de inclusão foram avaliadas, independentemente do periódico.\newline
\newline
\indent A primeira coluna das Tabela 3, Tabela 4, Tabela 5 e Tabela 6 contém uma numeração sequencial para os trabalhos. As colunas intituladas “Critérios de Inclusão Atendidos” e “Critérios de Exclusão Atendidos” contêm os critérios atendidos para cada trabalho, descritos no tópico anterior. Por fim, a coluna “Status” contém a avaliação do trabalho, indicando se ele foi excluído da revisão ou se foi incluído para a próxima fase. A existência de um único critério de exclusão atendido torna o arquivo inválido para a seleção final.\newline
\newline
\indent Os artigos base utilizados para auxiliar a definição dos limitadores da busca estão descritos no item 1.2 deste anexo e não foram incluídos no resultado da revisão.\newline
\newline
\indent Os tópicos 2.1, 2.2, 2.3, 2.4, e 2.5 exibem todas as obras retornadas após a realização de cada ciclo de buscas (processo de seleção preliminar) e a sua avaliação após a leitura do resumo.\newline
\newline
\indent Os tópicos referentes a seleção final e a análise de resultados estão descritos no corpo da pesquisa, tópico 2.4 e 2.5 respectivamente.\newline
\newline
\indent Nos resumos comparativos dos trabalhos selecionados (seleção final) são apresentadas as principais diferenças e semelhanças detectadas nestas pesquisas.\newline
\newline
\indent Na análise dos resultados são apresentadas as informações quantitativas da revisão em conjunto com as considerações finais.

\subsection{Fonte 1 - IEEE Xplore}\newline
\newline
\noindent \textbf{Fonte}: IEEE Xplore Digital Library\newline
\noindent \textbf{Data de Busca}: 15/04/2018\newline
\noindent \textbf{Strings Utilizadas}\newline
\indent \textbf{Primeiro Ciclo}: (( Usability AND ERP ) OR ( Usabilidade AND ERP ));\newline
\indent \textbf{Segundo  Ciclo}: ((usability NEAR/3 ERP) OR (usability ONEAR/3 ERP));\newline
\indent \textbf{Terceiro Ciclo}: (((usability NEAR/3 ERP) OR (ERP ONEAR/10 usability)));\newline
\indent \textbf{Quarto   Ciclo}: ((.QT.Enterprise Resource Planning.QT. NEAR/3 usability)  OR (.QT.Enterprise Resource Planning.QT. ONEAR/10 usability) );\newline
\indent \textbf{Quinto   Ciclo}: ((.QT.Enterprise Resource Planning.QT. NEAR/3 usability)  OR (.QT.Enterprise Resource Planning.QT. ONEAR/10 usability) ) OR ((usability NEAR/3 ERP) OR (ERP ONEAR/10 usability));\newline
\indent \textbf{Sexto    Ciclo}: (((usability NEAR/3 .QT.Enterprise Resource Planning.QT.) OR (usability ONEAR/10 .QT.Enterprise Resource Planning.QT.) ) OR ((usability NEAR/3 ERP) OR (usability ONEAR/10 ERP)));\newline
\newline
\noindent \textbf{Campos Pesquisados}: Titulo (Title), Resumo (Abstract) \newline
\noindent \textbf{Período Considerado}: 2001 à Dezembro de 2017 \newline
\noindent \textbf{Filtros Utilizados}: “Full Text & Metadata” \newline

Os resultados desta pesquisa encontram-se nas Tabela 1 á 6, que se seguem:

\begin{landscape}
\noindent \textbf{Lista de artigos encontrados}:
\setcounter{table}{0}

% Ambiente longtable é similar à combinação de table com tabular
\begin{longtable}{||p{1.7cm}|p{11.0cm}|p{7.0cm}|p{1.7cm}|p{1cm}||} % use normalmente os parâmetros 
 % do tabular
 %\begin{longtable}{||c|r||} % usar a largura automática, com ítens 
 %1 centrada e segunda, alinhada a direit, na célula
 % c -> center, l-> left, r -> right
 \caption{Primeiro ciclo de pesquisa IEEE}
 \label{ltab:teste}
 
 %%%%%%%%%%%%%%%%%%%%%%%%%%%%%%%%%%%%%%%%%%%%%%%%%%%%%%%%%%%%%%%%%%%%%%%%%%%%%%%%%%%%%
 % Obs.: configuração padrão da quebra de tabelas quando                          %
 %  1. \hline é traçado entre linhas da tabela:                                    %
 %     traçará um \hline no final da tabela quando vai continuar na outra página, %
 %     para "fechar a tabela"                                                        %
 %  2. não tem \hline: não traçará \hline quando efetua quebra                   %
 % --------------------------------------------------------------------------------- %
 % para configurar ação da quebra de tabelas no modo                               %
 % não padrão, precisará definição mais complexa,                               %
 % definindo o que fazer em                                                          %
 % - final da tabela quando continua na próxima página                             %
 % - no começo da tabela quando se trata da continuação                           %
 %   da página anterior,                                                            %
 % - etc.														                     %
 % Para detalhes desta parte, veja o arquivo:                                        %
 % doc\latex\tools\longtable.dvi                                                     %
 %%%%%%%%%%%%%%%%%%%%%%%%%%%%%%%%%%%%%%%%%%%%%%%%%%%%%%%%%%%%%%%%%%%%%%%%%%%%%%%%%%%%%
 \\ % é necessário pular linha após definições
    % preliminares: caption, label, etc.
 %\hline
	\hline
	Nro Artigo	& Título & Local de Publicação & Páginas & Ano \\ % Note a separação de col. e a quebra de linhas
	\hline
	\endfirsthead
	
	\multicolumn{5}{c}%
	{{\bfseries \tablename\ \thetable{} -- continuação da página anterior}} \\
	\hline
	Nro Artigo	& Título & Local de Publicação & Páginas & Ano \\ % Note a separação de col. e a quebra de linhas
	\hline
	\endhead
	
	\hline \multicolumn{5}{|r|}{{Continua na página seguinte}} \\ \hline
	\endfoot
	
	\hline \hline
	\endlastfoot
	01 & Using quality models in software package selection & IEEE Software & 34-41 & 2003 \\ 
	\hline
	02 & An investigation of the proliferation of mobile ERP apps and their usability & 2017 8th International Conference on Information and Communication Systems (ICICS) & 352-357 & 2017 \\ 
	\hline
	03 & Artemis - an extensible natural language framework for data querying and manipulation & 2016 IEEE 12th International Conference on Intelligent Computer Communication and Processing (ICCP) & 85-91 & 2016 \\ 
	\hline
	04 & Learning through ERP in technical educational institutions & 2012 15th International Conference on Interactive Collaborative Learning (ICL) & 1-4 & 2012 \\ 
	\hline
	05 & Brain-Computer Interfaces: Beyond Medical Applications & Computer & 26-34 & 2012 \\ 
	\hline
	06 & Analyzing Speech Quality Perception Using Electroencephalography & IEEE Journal of Selected Topics in Signal Processing & 721-731 & 2012 \\ 
	\hline
	07 & Enterprise Resource Planning system continuance usage intention at an individual level: An approach from value perspective & 2016 International Conference on Data and Software Engineering (ICoDSE) & 1-6 & 2016 \\ 
	\hline
	08 & The relationship between ERP software selection criteria and ERP success & 2009 IEEE International Conference on Industrial Engineering and Engineering Management & 2222-2226 & 2009 \\
	\hline 
	09 & An adaptive system architecture for devising adaptive user interfaces for mobile ERP apps & 2017 2nd International Conference on the Applications of Information Technology in Developing Renewable Energy Processes Systems (IT-DREPS) & 1-6 & 2017 \\ 
	\hline
	10 & Metric based efficiency analysis of educational ERP system usability-using fuzzy model & 2015 Third International Conference on Image Information Processing (ICIIP) & 382-386 & 2015 \\ 
	\hline
	11 & SAP remote communications & 2012 7th IEEE International Symposium on Applied Computational Intelligence and Informatics (SACI) & 303-309 & 2012 \\ 
	\hline
	12 & A practical abstraction of ERP to cloud integration complexity: The easy way & 2016 15th RoEduNet Conference: Networking in Education and Research & 1-6 & 2016 \\ 
	\hline
	13 & Receiver access control and secured handoff in mobile multicast using IGMP-AC & 2008 33rd IEEE Conference on Local Computer Networks (LCN) & 411-418 & 2008 \\ 
	\hline
	14 & A selection model of ERP system in mobile ERP design science research: Case study: Mobile ERP usability & 2016 IEEE/ACS 13th International Conference of Computer Systems and Applications (AICCSA) & 1-8 & 2016 \\ 
	\hline
	15 & Usability analysis of ERP software: Education and experience of users' as moderators & The 8th International Conference on Software, Knowledge, Information Management and Applications (SKIMA 2014) & 1-7 & 2014 \\ 
	\hline
	16 & Exploratory study to identify critical success factors penetration in ERP implementations & Proceedings of 3rd International Conference on Reliability, Infocom Technologies and Optimization & 1-6 & 2014 \\ 
	\hline
	17 & Approach to analysis and assessment of ERP system. A software vendor's perspective & 2015 Federated Conference on Computer Science and Information Systems (FedCSIS) & 1415-1426 & 2015 \\ 
	\hline
	18 & Agile ERP: "You don't know what you've got 'till it's gone!" & Agile 2007 (AGILE 2007) & 143-149 & 2007 \\ 
	\hline
	19 & A High-Security EEG-Based Login System with RSVP Stimuli and Dry Electrodes & IEEE Transactions on Information Forensics and Security & 2635-2647 & 2016 \\ 
	\hline
	20 & Heuristic evaluation checklist for mobile ERP user interfaces & 2016 7th International Conference on Information and Communication Systems (ICICS) & 180-185 & 2016 \\ 
	\hline
	21 & Quality improvement of ERP system GUI using expert method: A case study & 2013 6th International Conference on Human System Interactions (HSI) & 145-152 & 2013 \\ 
	\hline
	22 & A demonstrator of the GINSENG-approach to performance and closed loop control in WSNs & 2012 Ninth International Conference on Networked Sensing (INSS) & 1-2 & 2012 \\ 
	\hline
	23 & A Collaboration Model for ERP User-System Interaction & 2010 43rd Hawaii International Conference on System Sciences & 1-9 & 2010 \\ 
	\hline
	24 & Assessment of knowledge ability towards decision-making in information systems from managers perspective & 2011 Malaysian Conference in Software Engineering & 465-468 & 2011 \\ 
	\hline
	25 & ICITSI & 2016 International Conference on Information Technology Systems and Innovation (ICITSI) & 1-1 & 2016 \\ 
	\hline
	26 & The implementation experience of an advanced service repository for supporting service-oriented architecture & 2012 XXXVIII Conferencia Latinoamericana En Informatica (CLEI) & 1-10 & 2012 \\ 
	\hline
	27 & Cognitive, affective, and experience correlates of speech quality perception in complex listening conditions & 2013 IEEE International Conference on Acoustics, Speech and Signal Processing & 3672-3676 & 2013 \\ 
	\hline
	28 & Applying human-computer collaboration for improving ERP usability & 2010 International Symposium on Collaborative Technologies and Systems & 396-397 & 2010 \\ 
	\hline
	29 & Transfer learning and active transfer learning for reducing calibration data in single-trial classification of visually-evoked potentials & 2014 IEEE International Conference on Systems, Man, and Cybernetics (SMC) & 2801-2807 & 2014 \\ 
	\hline
	30 & A Novel Usability Matrix for ERP Systems Using Heuristic Approach & 2012 International Conference on Management of e-Commerce and e-Government & 291-296 & 2012 \\ 
	\hline
	31 & (Re-)Evaluating User Interface Aspects in ERP Systems -- An Empirical User Study & 2014 47th Hawaii International Conference on System Sciences & 396-405 & 2014 \\ 
	\hline
	32 & Applying design principles for enhancing enterprise system usability & 2014 9th International Conference on Software Engineering and Applications (ICSOFT-EA) & 162-169 & 2014 \\ 
	\hline
	33 & Analyzing Usability Alternatives in Multi-criteria Decision Making During ERP Training & 2006 7th International Conference on Information Technology Based Higher Education and Training & 296-309 & 2006 \\ 
	\hline
	34 & Spatial auditory BCI paradigm based on real and virtual sound image generation & 2013 Asia-Pacific Signal and Information Processing Association Annual Summit and Conference & 1-5 & 2013 \\ 
	\hline
	35 & Using ERPs for assessing the (sub) conscious perception of noise & 2010 Annual International Conference of the IEEE Engineering in Medicine and Biology & 2690-2693 & 2010 \\ 
	\hline
	36 & A gamified system for learning enterprise resource planning systems: Investigating the user experience & 2017 1st International Conference on Next Generation Computing Applications (NextComp) & 209-214 & 2017 \\ 
	\hline
	37 & The study of technology acceptance based on gender's differences in ERP implementation & 2011 International Conference on E-Business and E-Government (ICEE) & 1-5 & 2011 \\ 
	\hline
	38 & Engineering Adaptive Model-Driven User Interfaces & IEEE Transactions on Software Engineering & 1118-1147 & 2016 \\ 
	\hline
	39 & Using human resource management suites to exploit team process improvement models & Proceedings. 28th Euromicro Conference & 382-387 & 2002 \\ 
	\hline
	40 & Initial assessment of artifact filtering for RSVP Keyboard #x2122; & 2013 IEEE Signal Processing in Medicine and Biology Symposium (SPMB) & 1-5 & 2013 \\ 
	\hline
	41 & Key Performance Indicators used in ERP performance measurement applications & 2012 IEEE 10th Jubilee International Symposium on Intelligent Systems and Informatics & 43-48 & 2012  
	\hline
 \end{longtable}
\end{landscape}

\begin{landscape}
%\noindent \textbf{Lista de artigos encontrados}:

 \begin{longtable}{||p{1.7cm}|p{11.0cm}|p{6.0cm}|p{1.7cm}|p{1cm}|p{1cm}||} % use normalmente os parâmetros 
 \caption{Segundo ciclo de pesquisa IEEE}
 \label{ltab:teste}
 \\ % é necessário pular linha após definições
 	\hline
	Nro Artigo	& Título & Local de Publicação & Páginas & Ano & Dupl. \\ % Note a separação de col. e a quebra de linhas
	\hline
	\endfirsthead
	
	\multicolumn{6}{c}%
	{{\bfseries \tablename\ \thetable{} -- continuação da página anterior}} \\
	\hline
	Nro Artigo	& Título & Local de Publicação & Páginas & Ano & Dupl. \\ % Note a separação de col. e a quebra de linhas
	\hline
	\endhead
	
	\hline \multicolumn{6}{|r|}{{Continua na página seguinte}} \\ \hline
	\endfoot
	
	\hline \hline
	\endlastfoot
	42 & The study of technology acceptance based on gender's differences in ERP implementation & 2011 International Conference on E-Business and E-Government (ICEE) & 1-5 & 2011 & SIM \\ 
	\hline
	43 & Evaluating mobile banking application: Usability dimensions and measurements & Proceedings of the 6th International Conference on Information Technology and Multimedia & 136-140 & 2014 & NÃO \\ 
	\hline
	44 & The process of ERP usage in manufacturing firms in China: An empirical investigation & 2009 International Conference on Management Science and Engineering & 3-9 & 2009 & NÃO \\ 
	\hline
	45 & Comprehensive evaluation of the auto parts supplier selection under the environment of things networking & Proceedings of 2013 IEEE International Conference on Service Operations and Logistics, and Informatics & 540-545 & 2013 & NÃO \\ 
	\hline
	46 & Analyzing Usability Alternatives in Multi-criteria Decision Making During ERP Training & 2006 7th International Conference on Information Technology Based Higher Education and Training & 296-309 & 2006 & SIM \\ 
	\hline
	47 & S.I. success models, 25 years of evolution & 2017 12th Iberian Conference on Information Systems and Technologies (CISTI) & 1-6 & 2017 & NÃO \\ 
	\hline
	48 & Metric based efficiency analysis of educational ERP system usability-using fuzzy model & 2015 Third International Conference on Image Information Processing (ICIIP) & 382-386 & 2015 & SIM \\ 
	\hline
	49 & Citarasa based vehicle planning system & 2007 IEEE International Conference on Industrial Engineering and Engineering Management & 1297-1301 & 2007 & NÃO \\ 
	\hline
	50 & (Re-)Evaluating User Interface Aspects in ERP Systems -- An Empirical User Study & 2014 47th Hawaii International Conference on System Sciences & 396-405 & 2014 & SIM \\ 
	\hline
	51 & PROMETHEUS: Procedural Methodology For Developing Heuristics Of Usability & IEEE Latin America Transactions & 541-549 & 2017 & NÃO \\ 
	\hline
	52 & Approach to analysis and assessment of ERP system. A software vendor's perspective & 2015 Federated Conference on Computer Science and Information Systems (FedCSIS) & 1415-1426 & 2015 & SIM \\ 
	\hline
	53 & Real Options and Subsequent Technology Adoption: An ERP System Perspective & 2015 48th Hawaii International Conference on System Sciences & 5020-5027 & 2015 & NÃO \\ 
	\hline
	54 & Key Performance Indicators used in ERP performance measurement applications & 2012 IEEE 10th Jubilee International Symposium on Intelligent Systems and Informatics & 43-48 & 2012 & SIM \\ 
	\hline
	55 & Exploratory study to identify critical success factors penetration in ERP implementations & Proceedings of 3rd International Conference on Reliability, Infocom Technologies and Optimization & 1-6 & 2014 & SIM \\ 
	\hline
	56 & Using human resource management suites to exploit team process improvement models & Proceedings. 28th Euromicro Conference & 382-387 & 2002 & SIM \\ 
	\hline
	57 & Heuristic evaluation checklist for mobile ERP user interfaces & 2016 7th International Conference on Information and Communication Systems (ICICS) & 180-185 & 2016 & SIM \\ 
	\hline
	58 & An adaptive system architecture for devising adaptive user interfaces for mobile ERP apps & 2017 2nd International Conference on the Applications of Information Technology in Developing Renewable Energy Processes Systems (IT-DREPS) & 1-6 & 2017 & SIM \\ 
	\hline
	59 & Customer Involvement in Requirements Management: Lessons from Mass Market Software Development & 2009 17th IEEE International Requirements Engineering Conference & 281-286 & 2009 & NÃO \\ 
	\hline
	60 & A Novel Usability Matrix for ERP Systems Using Heuristic Approach & 2012 International Conference on Management of e-Commerce and e-Government & 291-296 & 2012 & SIM \\ 
	\hline
	61 & Assessment of knowledge ability towards decision-making in information systems from managers perspective & 2011 Malaysian Conference in Software Engineering & 465-468 & 2011 & SIM \\ 
	\hline
	62 & A methodology for successful implementation of ERP in smaller companies & Proceedings of 2010 IEEE International Conference on Service Operations and Logistics, and Informatics & 380-385 & 2010 & NÃO \\ 
	\hline
	63 & ERP post-adoption: Value impact on firm performance: A survey study on Portuguese SMEs & 7th Iberian Conference on Information Systems and Technologies (CISTI 2012) & 1-6 & 2012 & NÃO \\ 
	\hline
	64 & Usability analysis of ERP software: Education and experience of users' as moderators & The 8th International Conference on Software, Knowledge, Information Management and Applications (SKIMA 2014) & 1-7 & 2014 & SIM \\ 
	\hline
	65 & Usability practice and awareness in UAE & The 2011 International Conference and Workshop on Current Trends in Information Technology (CTIT 11) & 1-6 & 2011 & NÃO \\ 
	\hline
	66 & Applying design principles for enhancing enterprise system usability & 2014 9th International Conference on Software Engineering and Applications (ICSOFT-EA) & 162-169 & 2014 & SIM \\ 
	\hline
	67 & Effort Estimation for ERP Projects #x2014; A Systematic Review & 2017 43rd Euromicro Conference on Software Engineering and Advanced Applications (SEAA) & 96-103 & 2017 & NÃO \\ 
	\hline
	68 & Mining Logs to Model the Use of a System & 2017 ACM/IEEE International Symposium on Empirical Software Engineering and Measurement (ESEM) & 334-343 & 2017 & NÃO \\ 
	\hline
	69 & Impact of Man-Machine Interaction Factors on Enterprise Resource Planning (ERP) Software Design & 2006 Technology Management for the Global Future - PICMET 2006 Conference & 2335-2341 & 2006 & NÃO \\ 
	\hline
	70 & Critical success factors for MES implementation in China & 2012 IEEE International Conference on Industrial Engineering and Engineering Management & 558-562 & 2012 & NÃO \\ 
	\hline
	71 & A demonstrator of the GINSENG-approach to performance and closed loop control in WSNs & 2012 Ninth International Conference on Networked Sensing (INSS) & 1-2 & 2012 & SIM \\ 
	\hline
	72 & Learning through ERP in technical educational institutions & 2012 15th International Conference on Interactive Collaborative Learning (ICL) & 1-4 & 2012 & SIM \\ 
	\hline
	73 & A Collaboration Model for ERP User-System Interaction & 2010 43rd Hawaii International Conference on System Sciences & 1-9 & 2010 & SIM \\ 
	\hline
	74 & Applying human-computer collaboration for improving ERP usability & 2010 International Symposium on Collaborative Technologies and Systems & 396-397 & 2010 & SIM \\ 
	\hline
	75 & Applying fuzzy sets for erp systems selection within the construction industry & 2010 IEEE International Conference on Industrial Engineering and Engineering Management & 320-324 & 2010 & NÃO \\ 
	\hline
	76 & A selection model of ERP system in mobile ERP design science research: Case study: Mobile ERP usability & 2016 IEEE/ACS 13th International Conference of Computer Systems and Applications (AICCSA) & 1-8 & 2016 & SIM \\ 
	\hline
	77 & An investigation of the proliferation of mobile ERP apps and their usability & 2017 8th International Conference on Information and Communication Systems (ICICS) & 352-357 & 2017 & SIM 
	\hline
 \end{longtable}
\end{landscape}

\begin{landscape}
%\noindent \textbf{Lista de artigos encontrados}:

 \begin{longtable}{||p{1.7cm}|p{11.0cm}|p{6.0cm}|p{1.7cm}|p{1cm}|p{1cm}||} % use normalmente os parâmetros 
 \caption{Terceiro ciclo de pesquisa IEEE}
 \label{ltab:teste}
 \\ % é necessário pular linha após definições
 	\hline
 	Nro Artigo	& Título & Local de Publicação & Páginas & Ano & Dupl. \\ % Note a separação de col. e a quebra de linhas
  	\hline
  	\endfirsthead
  	
  	\multicolumn{6}{c}%
  	{{\bfseries \tablename\ \thetable{} -- continuação da página anterior}} \\
  	\hline
  	Nro Artigo	& Título & Local de Publicação & Páginas & Ano & Dupl. \\ % Note a separação de col. e a quebra de linhas
  	\hline
  	\endhead
  	
  	\hline \multicolumn{6}{|r|}{{Continua na página seguinte}} \\ \hline
  	\endfoot
  	
  	\hline \hline
  	\endlastfoot
	78 & The study of technology acceptance based on gender's differences in ERP implementation & 2011 International Conference on E-Business and E-Government (ICEE) & 1-5 & 2011 & SIM \\ 
	 	\hline
	79 & Metric based efficiency analysis of educational ERP system usability-using fuzzy model & 2015 Third International Conference on Image Information Processing (ICIIP) & 382-386 & 2015 & SIM\\ 
 	\hline
	80 & (Re-)Evaluating User Interface Aspects in ERP Systems -- An Empirical User Study & 2014 47th Hawaii International Conference on System Sciences & 396-405 & 2014 & SIM \\ 
 	\hline
	81 & Exploratory study to identify critical success factors penetration in ERP implementations & Proceedings of 3rd International Conference on Reliability, Infocom Technologies and Optimization & 1-6 & 2014 & SIM \\ 
 	\hline
	82 & Heuristic evaluation checklist for mobile ERP user interfaces & 2016 7th International Conference on Information and Communication Systems (ICICS) & 180-185 & 2016 & SIM \\ 
 	\hline
	83 & A Novel Usability Matrix for ERP Systems Using Heuristic Approach & 2012 International Conference on Management of e-Commerce and e-Government & 291-296 & 2012 & SIM \\ 
 	\hline
	84 & Usability analysis of ERP software: Education and experience of users' as moderators & The 8th International Conference on Software, Knowledge, Information Management and Applications (SKIMA 2014) & 1-7 & 2014 & SIM \\ 
 	\hline
	85 & Applying design principles for enhancing enterprise system usability & 2014 9th International Conference on Software Engineering and Applications (ICSOFT-EA) & 162-169 & 2014 & SIM \\ 
 	\hline
	86 & The relationship between ERP software selection criteria and ERP success & 2009 IEEE International Conference on Industrial Engineering and Engineering Management & 2222-2226 & 2009 & SIM \\ 
 	\hline
	87 & A demonstrator of the GINSENG-approach to performance and closed loop control in WSNs & 2012 Ninth International Conference on Networked Sensing (INSS) & 1-2 & 2012 & SIM \\ 
 	\hline
	88 & A Collaboration Model for ERP User-System Interaction & 2010 43rd Hawaii International Conference on System Sciences & 1-9 & 2010 & SIM \\ 
 	\hline
	89 & Applying human-computer collaboration for improving ERP usability & 2010 International Symposium on Collaborative Technologies and Systems & 396-397 & 2010 & SIM \\ 
 	\hline
	90 & A selection model of ERP system in mobile ERP design science research: Case study: Mobile ERP usability & 2016 IEEE/ACS 13th International Conference of Computer Systems and Applications (AICCSA) & 1-8 & 2016 & SIM \\ 
 	\hline
	91 & An investigation of the proliferation of mobile ERP apps and their usability & 2017 8th International Conference on Information and Communication Systems (ICICS) & 352-357 & 2017 & SIM
	\hline
 \end{longtable}
\newline
 \begin{longtable}{||p{1.7cm}|p{11.0cm}|p{6.0cm}|p{1.7cm}|p{1cm}|p{1cm}||} % use normalmente os parâmetros 
	\caption{Quarto ciclo de pesquisa IEEE}
	\label{ltab:teste}
	\\ % é necessário pular linha após definições
	\hline
	Nro Artigo	& Título & Local de Publicação & Páginas & Ano & Dupl. \\ % Note a separação de col. e a quebra de linhas
	\hline
	\endfirsthead
	
	\multicolumn{6}{c}%
	{{\bfseries \tablename\ \thetable{} -- continuação da página anterior}} \\
	\hline
	Nro Artigo	& Título & Local de Publicação & Páginas & Ano & Dupl. \\ % Note a separação de col. e a quebra de linhas
	\hline
	\endhead
	
	\hline \multicolumn{6}{|r|}{{Continua na página seguinte}} \\ \hline
	\endfoot
	
	\hline \hline
	\endlastfoot
	92 & Studying the deficiencies and problems of different architecture in developing distributed systems and analyze the existing solution & 2015 2nd International Conference on Knowledge-Based Engineering and Innovation (KBEI) & 826-834 & 2015 & NÃO \\ 
	\hline 
	93 & (Re-)Evaluating User Interface Aspects in ERP Systems -- An Empirical User Study & 2014 47th Hawaii International Conference on System Sciences & 396-405 & 2014 & SIM \\ 
	\hline
\end{longtable}

\newline
\begin{longtable}{||p{1.7cm}|p{11.0cm}|p{6.0cm}|p{1.7cm}|p{1cm}|p{1cm}||} % use normalmente os parâmetros 
	\caption{Quinto ciclo de pesquisa IEEE}
	\label{ltab:teste}
	\\ % é necessário pular linha após definições
	\hline
	Nro Artigo	& Título & Local de Publicação & Páginas & Ano & Dupl. \\ % Note a separação de col. e a quebra de linhas
	\hline
	\endfirsthead
	
	\multicolumn{6}{c}%
	{{\bfseries \tablename\ \thetable{} -- continuação da página anterior}} \\
	\hline
	Nro Artigo	& Título & Local de Publicação & Páginas & Ano & Dupl. \\ % Note a separação de col. e a quebra de linhas
	\hline
	\endhead
	
	\hline \multicolumn{6}{|r|}{{Continua na página seguinte}} \\ \hline
	\endfoot
	
	\hline \hline
	\endlastfoot
	94 & Studying the deficiencies and problems of different architecture in developing distributed systems and analyze the existing solution & 2015 2nd International Conference on Knowledge-Based Engineering and Innovation (KBEI) & 826-834 & 2015 & SIM \\ 
	\hline	
	95 & Metric based efficiency analysis of educational ERP system usability-using fuzzy model & 2015 Third International Conference on Image Information Processing (ICIIP) & 382-386 & 2015 & SIM \\ 
	\hline	
	96 & (Re-)Evaluating User Interface Aspects in ERP Systems -- An Empirical User Study & 2014 47th Hawaii International Conference on System Sciences & 396-405 & 2014 & SIM \\ 
	\hline	
	97 & Exploratory study to identify critical success factors penetration in ERP implementations & Proceedings of 3rd International Conference on Reliability, Infocom Technologies and Optimization & 1-6 & 2014 & SIM \\ 
	\hline	
	98 & Heuristic evaluation checklist for mobile ERP user interfaces & 2016 7th International Conference on Information and Communication Systems (ICICS) & 180-185 & 2016 & SIM \\ 
	\hline	
	99 & A Novel Usability Matrix for ERP Systems Using Heuristic Approach & 2012 International Conference on Management of e-Commerce and e-Government & 291-296 & 2012 & SIM \\ 
	\hline	
	100 & Usability analysis of ERP software: Education and experience of users' as moderators & The 8th International Conference on Software, Knowledge, Information Management and Applications (SKIMA 2014) & 1-7 & 2014 & SIM \\ 
	\hline	
	101 & Applying design principles for enhancing enterprise system usability & 2014 9th International Conference on Software Engineering and Applications (ICSOFT-EA) & 162-169 & 2014 & SIM \\ 
	\hline	
	102 & The relationship between ERP software selection criteria and ERP success & 2009 IEEE International Conference on Industrial Engineering and Engineering Management & 2222-2226 & 2009 & SIM \\ 
	\hline	
	103 & A demonstrator of the GINSENG-approach to performance and closed loop control in WSNs & 2012 Ninth International Conference on Networked Sensing (INSS) & 1-2 & 2012 & SIM \\ 
	\hline	
	104 & A Collaboration Model for ERP User-System Interaction & 2010 43rd Hawaii International Conference on System Sciences & 1-9 & 2010 & SIM \\ 
	\hline	
	105 & Applying human-computer collaboration for improving ERP usability & 2010 International Symposium on Collaborative Technologies and Systems & 396-397 & 2010 & SIM \\ 
	\hline	
	106 & A selection model of ERP system in mobile ERP design science research: Case study: Mobile ERP usability & 2016 IEEE/ACS 13th International Conference of Computer Systems and Applications (AICCSA) & 1-8 & 2016 & SIM \\ 
	\hline	
	107 & An investigation of the proliferation of mobile ERP apps and their usability & 2017 8th International Conference on Information and Communication Systems (ICICS) & 352-357 & 2017 & SIM 
	\hline
\end{longtable}
\newline
\begin{longtable}{||p{1.7cm}|p{11.0cm}|p{6.0cm}|p{1.7cm}|p{1cm}|p{1cm}||} % use normalmente os parâmetros 
	\caption{Sexto ciclo de pesquisa IEEE}
	\label{ltab:teste}
	\\ % é necessário pular linha após definições
	\hline
	Nro Artigo	& Título & Local de Publicação & Páginas & Ano & Dupl. \\ % Note a separação de col. e a quebra de linhas
	\hline
	\endfirsthead
	
	\multicolumn{6}{c}%
	{{\bfseries \tablename\ \thetable{} -- continuação da página anterior}} \\
	\hline
	Nro Artigo	& Título & Local de Publicação & Páginas & Ano & Dupl. \\ % Note a separação de col. e a quebra de linhas
	\hline
	\endhead
	
	\hline \multicolumn{6}{|r|}{{Continua na página seguinte}} \\ \hline
	\endfoot
	
	\hline \hline
	\endlastfoot
	108 & An investigation of the proliferation of mobile ERP apps and their usability & 2017 8th International Conference on Information and Communication Systems (ICICS) & 352-357 & 2017 & SIM \\ 
	\hline
	109 & The relationship between ERP software selection criteria and ERP success & 2009 IEEE International Conference on Industrial Engineering and Engineering Management & 2222-2226 & 2009 & SIM \\ 
	\hline
	110 & An adaptive system architecture for devising adaptive user interfaces for mobile ERP apps & 2017 2nd International Conference on the Applications of Information Technology in Developing Renewable Energy Processes Systems (IT-DREPS) & 1-6 & 2017 & SIM \\ 
	\hline
	111 & Metric based efficiency analysis of educational ERP system usability-using fuzzy model & 2015 Third International Conference on Image Information Processing (ICIIP) & 382-386 & 2015 & SIM \\ 
	\hline
	112 & A selection model of ERP system in mobile ERP design science research: Case study: Mobile ERP usability & 2016 IEEE/ACS 13th International Conference of Computer Systems and Applications (AICCSA) & 1-8 & 2016 & SIM \\ 
	\hline
	113 & Usability analysis of ERP software: Education and experience of users' as moderators & The 8th International Conference on Software, Knowledge, Information Management and Applications (SKIMA 2014) & 1-7 & 2014 & SIM \\ 
	\hline
	114 & Exploratory study to identify critical success factors penetration in ERP implementations & Proceedings of 3rd International Conference on Reliability, Infocom Technologies and Optimization & 1-6 & 2014 & SIM \\ 
	\hline
	115 & Heuristic evaluation checklist for mobile ERP user interfaces & 2016 7th International Conference on Information and Communication Systems (ICICS) & 180-185 & 2016 & SIM \\ 
	\hline
	116 & A demonstrator of the GINSENG-approach to performance and closed loop control in WSNs & 2012 Ninth International Conference on Networked Sensing (INSS) & 1-2 & 2012 & SIM \\ 
	\hline
	117 & A Collaboration Model for ERP User-System Interaction & 2010 43rd Hawaii International Conference on System Sciences & 1-9 & 2010 & SIM \\ 
	\hline
	118 & Applying human-computer collaboration for improving ERP usability & 2010 International Symposium on Collaborative Technologies and Systems & 396-397 & 2010 & SIM \\ 
	\hline
	119 & A Novel Usability Matrix for ERP Systems Using Heuristic Approach & 2012 International Conference on Management of e-Commerce and e-Government & 291-296 & 2012 & SIM \\ 
	\hline
	120 & (Re-)Evaluating User Interface Aspects in ERP Systems -- An Empirical User Study & 2014 47th Hawaii International Conference on System Sciences & 396-405 & 2014 & SIM \\ 
	\hline
	121 & Applying design principles for enhancing enterprise system usability & 2014 9th International Conference on Software Engineering and Applications (ICSOFT-EA) & 162-169 & 2014 & SIM \\ 
	\hline
	122 & Analyzing Usability Alternatives in Multi-criteria Decision Making During ERP Training & 2006 7th International Conference on Information Technology Based Higher Education and Training & 296-309 & 2006 & SIM \\ 
	\hline
	123 & The study of technology acceptance based on gender's differences in ERP implementation & 2011 International Conference on E-Business and E-Government (ICEE) & 1-5 & 2011 & SIM \\  
	\hline
\end{longtable}
\end{landscape}

\subsection{Fonte 2 - Ibict OasisBr}\newline
\newline
\noindent \textbf{Fonte}: Portal brasileiro de publicações científicas em acesso aberto\newline
\noindent \textbf{Data de Busca}: 21/04/2018\newline
\noindent \textbf{Strings Utilizadas}\newline
\indent \textbf{Primeiro Ciclo}: "(Resumo Português:"usabilidade ERP"~10)";\newline
\indent \textbf{Segundo  Ciclo}: "(Resumo Português:" " "Enterprise Resource"~1 Planning"~1 usabilidade"~10)";
\newline
\noindent \textbf{Campos Pesquisados}: Titulo, Resumo \newline
\noindent \textbf{Período Considerado}: 2001 à Dezembro de 2017 \newline

Os resultados desta pesquisa encontram-se nas Tabelas 7 e 8, que se seguem:

\begin{landscape}
\noindent \textbf{Lista de artigos encontrados}:

 \begin{longtable}{||p{1.7cm}|p{11.0cm}|p{6.0cm}|p{1.7cm}|p{1cm}|p{1cm}||} % use normalmente os parâmetros 
 \caption{Primeiro ciclo de pesquisa IBICT}
 \label{ltab:teste}
 \\ % é necessário pular linha após definições
 	\hline
 	Nro Artigo	& Título & Tipo Documento & Páginas & Ano & Dupl. \\ % Note a separação de col. e a quebra de linhas
  	\hline
	124 & Uma abordagem orientada a modelos para desenvolvimento de sistemas ERP de varejo na Web utilizando características funcionais de usabilidade. & Tese & 149 & 2015 & NÃO \\ 
	\hline
	125 & Boa Usabilidade e comunicação eficiente de tarefas: dois aliados na execução de processos em sistemas integrados de gestão & Dissertação & 151 & 2015 & NÃO 
	\hline
 \end{longtable}
\newline
 \begin{longtable}{||p{1.7cm}|p{11.0cm}|p{6.0cm}|p{1.7cm}|p{1cm}|p{1cm}||} % use normalmente os parâmetros 
	\caption{Segundo ciclo de pesquisa IBICT}
	\label{ltab:teste}
	\\ % é necessário pular linha após definições
	\hline
	Nro Artigo	& Título & Tipo Documento & Páginas & Ano & Dupl. \\ % Note a separação de col. e a quebra de linhas
	\hline
	126 & Estudo de um caso de localização de um software ERP de código livre. & Tese & 133 & 2011 & NÃO 
	\hline
\end{longtable}

\end{landscape}

\subsection{Fonte 3 - Literatura Cinzenta}\newline
\newline
\noindent \textbf{Fonte}: Universidade de Dresden\newline
\noindent \textbf{Data de Busca}: 21/09/2017\newline
\noindent \textbf{Critério Utilizado}\newline
\indent Busca de Publicações da Universidade sobre o tema escolhido.
\newline
\noindent \textbf{Campos Pesquisados}: Titulo, Resumo \newline
\noindent \textbf{Período Considerado}: 2001 à Dezembro de 2017 \newline

Os resultados desta pesquisa encontram-se na Tabelas 9, a seguir:

\begin{landscape}
\noindent \textbf{Lista de artigos encontrados}:
 \begin{longtable}{||p{1.7cm}|p{11.0cm}|p{6.0cm}|p{1.7cm}|p{1cm}|p{1cm}||} % use normalmente os parâmetros 
 	\captionsetup{margin=14pt,labelfont=bf,justification=raggedright}
	\caption{Resultado Pesquisa Literatura Cinzenta}
	\label{ltab:teste}
	\\ % é necessário pular linha após definições
	\hline
	Nro Artigo	& Título & Local de Publicação & Páginas & Ano & Dupl. \\ % Note a separação de col. e a quebra de linhas
	\hline
  	\endfirsthead

	\multicolumn{6}{c}%
	{{\bfseries \tablename\ \thetable{} -- continuação da página anterior}} \\
	\hline
	Nro Artigo	& Título & Local de Publicação & Páginas & Ano & Dupl. \\ % Note a separação de col. e a quebra de linhas
	\hline
	\endhead
	
	\hline \multicolumn{6}{|r|}{{Continua na página seguinte}} \\ \hline
	\endfoot
	
	\hline \hline
	\endlastfoot
	127 & Discovering Potentials in Enterprise Interface Design - A review of our latest case studies in the enterprise domain & 15th International Conference on Enterprise Information Systems &  & 2013 & NÃO \\ 
	\hline
	128 & Framework to Enhance {ERP} Usability by Machine Learning Based Requirements Prioritization & Journal of Software & 664--670 & 2017 & NÃO \\ 
	\hline
	129 & The Impact of ERP System's Usability on Enterprise Resource Planning Project Implementation Success via the Mediating Role of User Satisfaction & Journal of Management Research & 49 & 2017 & NÃO \\ 
	\hline
	130 & Does Usability Matter? An Analysis of the Impact of Usability on Technology Acceptance in {ERP} Settings & Interdisciplinary Journal of Information,  Knowledge,  and Management & 309--330 & 2016 & NÃO \\ 
	\hline
	131 & Indikatorbasierte Messung der ERP-Usability &  &  & 2014 & NÃO \\ 
	\hline
	132 & (Re-)Evaluating User Interface Aspects in ERP Systems - An Empirical User Study & Proceedings of the 47th Hawaiian International Conference on System Sciences &  & 2014 & NÃO \\ \hline
	133 & Intuitive Interaktion in betrieblichen Anwendungen & ERP-Management - Zeitschrift far unternehmensweite Anwendungssysteme &  & 2011 & NÃO \\ 
	\hline
	134 & A Guide To Distance-Driven User Interfaces & COST 2101 Final Conference &  & 2011 & NÃO \\ \hline
	135 & Beyond Forms and Tables - A Visual and Task-oriented Approach to ERP Systems & Conference on Enterprise Information Systems (CENTERIS) 2012 &  & 2012 & NÃO \\ 
	\hline
	136 & Zufriedenheit von Anwendern - ERP-Systeme und weitere Unternehmensanwendungen im Vergleich & Zeitschrift für unternehmensweite Anwendungssysteme & 47--48 & 2013 & NÃO \\ 
	\hline
	137 & Usability von ERP-Systemen – Aktueller Stand und Perspektiven & Workshopband Mensch  Computer 2013 & & 2013 & NÃO \\ 
	\hline
	138 & Mastering ERP Interface Complexity - A Scalable User Interface Concept for ERP Systems & 15th International Conference on Enterprise Information Systems &  & 2013 & NÃO \\ 
	\hline
	139 & Herausforderungen für die Gestaltung von zukunftsfähigen ERP-Systemen & VDMA-Tagung: Zukunftsweisende Bedienkonzepte für die Unternehmenssoftware &  & 2013 & NÃO \\ 
	\hline
	140 & Discovering Potentials in Enterprise Interface Design - A review of our latest case studies in the enterprise domain & 15th International Conference on Enterprise Information Systems &  & 2013 & NÃO 
	\hline
\end{longtable}
\end{landscape}

\subsection{Lista de Arquivos com Status de Inclusão / Exclusão}

Foi criada uma Tabela com os resultados desta pesquisa, que encontram-se na Tabela 10, a seguir:

\newcolumntype{C}[1]{>{\centering\arraybackslash}m{#1}}
%\captionsetup[longtable]{labelfont=bf,textfont=it,labelsep=newline}
\begin{longtable}[h!]{||C{1.7cm}|C{5.0cm}|C{5.0cm}|C{3cm}||} % use normalmente os parâmetros 
    %\captionsetup{margin=-14pt,labelfont=bf,justification=justified}
    \captionsetup{margin=14pt,labelfont=bf,justification=raggedright}
    \caption{Avaliação dos Artigos  }   
    \label{tab:long} %\label{ltab:teste}
    \\% é necessário pular linha após definições
 	\hline
 	Nro Artigo	& Critérios de Inclusão & Critérios de Exclusão & Status \\ % Note a separação de col. e a quebra de linhas
  	\hline
  	\endfirsthead
  	
  	\multicolumn{4}{c}%
  	{{\bfseries \tablename\ \thetable{} -- continuação da página anterior}} \\
  	\hline
  	Nro Artigo	& Critérios de Inclusão & Critérios de Exclusão & Status \\ % Note a separação de col. e a quebra de linhas
  	\hline
  	\endhead
  	
  	\hline \multicolumn{4}{|r|}{{Continua na página seguinte}} \\ \hline
  	\endfoot
  	
  	\hline \hline
  	\endlastfoot
  	
	01 & (a)(b)    & (c)  		& Excluído 		\\ \hline
	02 & (a)(b)    & (g)  		& Excluído 		\\ \hline
	03 & (a)(b)    & (c)  		& Excluído 		\\ \hline
	04 & (a)(b)    & (b)(c)  	& Excluído 		\\ \hline
	05 & (a)(b)    & (b)(c)  	& Excluído 		\\ \hline
	06 & (a)(b)    & (b)(c)  	& Excluído 		\\ \hline
	07 & (a)(b)    & (c)     	& Excluído 		\\ \hline
	08 & (a)(b)(d) & (b)       	& Excluído 		\\ \hline
	09 & (a)(b)    & (c)(g)  	& Excluído 		\\ \hline
	10 & (a)(b)    & (c)     	& Excluído 		\\ \hline
	11 & (a)(b)    & (b)(c)(f) 	& Excluído 		\\ \hline
	12 & (a)(b)    & (c) 		& Excluído 		\\ \hline
	13 & (a)(b)    & (c) 		& Excluído 		\\ \hline
	14 & (a)(b)    & (c)(h) 	& Excluído 		\\ \hline
	15 & (a)(b)    & N/A		& Incluído 		\\ \hline
	16 & (a)(b)    & (h)		& Excluído 		\\ \hline
	17 & (a)(b)    & (h)		& Excluído 		\\ \hline
	18 & (a)(b)    & (b) 		& Excluído 		\\ \hline
	19 & (a)(b)    & (c) 		& Excluído 		\\ \hline
	20 & (a)(b)    & (c)(g)  	& Excluído 		\\ \hline
	21 & (a)(b)    & (c) 		& Excluído 		\\ \hline
	22 & (a)(b)    & (c) 		& Excluído 		\\ \hline
	23 & (a)(b)    & (c) 		& Excluído 		\\ \hline
	24 & (a)(b)    & (c) 		& Excluído 		\\ \hline
	25 & (a)(b)    & (f)(h) 	& Excluído 		\\ \hline
	26 & (a)(b)    & (f)	 	& Excluído 		\\ \hline
	27 & (a)(b)    & (c) 		& Excluído 		\\ \hline
	28 & (a)(b)    & (b) 		& Excluído 		\\ \hline
	29 & (a)(b)    & (c) 		& Excluído 		\\ \hline
	30 & (a)(b)    & (b) 		& Excluído 		\\ \hline
	31 & (a)(b)    & N/A		& Incluído 		\\ \hline
	32 & (a)(b)    & N/A		& Incluído 		\\ \hline
	33 & (a)(b)    & (b)(c)		& Excluído 		\\ \hline
	34 & (a)(b)    & (c)		& Excluído 		\\ \hline
	35 & (a)(b)    & (b)(c)		& Excluído 		\\ \hline
	36 & (a)(b)    & (c)   		& Excluído 		\\ \hline
	37 & (a)(b)    & (b) 		& Excluído 		\\ \hline
	38 & (a)(b)    & (c) 		& Excluído 		\\ \hline
	39 & (a)(b)    & (b)		& Excluído 		\\ \hline
	40 & (a)(b)    & (c) 		& Excluído 		\\ \hline
	41 & (a)(b)    & (b)		& Excluído 		\\ \hline
	42 & N/A 	   & N/A 		& Duplicado 	\\ \hline
	43 & (a)(b)    & (g) 		& Excluído 		\\ \hline
	44 & (a)(b)(d) & (b) 		& Excluído 		\\ \hline
	45 & (a)(b)    & (c) 		& Excluído 		\\ \hline
	46 & N/A 	   & N/A 		& Duplicado 	\\ \hline
	47 & (a)(b)    & (c) 		& Excluído 		\\ \hline
	48 & N/A 	   & N/A 		& Duplicado  	\\ \hline
	49 & (a)(b)    & (b)(c) 	& Excluído 		\\ \hline
	50 & N/A 	   & N/A 		& Duplicado  	\\ \hline
	51 & (a)(b)    & (c) 		& Excluído 		\\ \hline
	52 & N/A 	   & N/A 		& Duplicado  	\\ \hline
	53 & (a)(b)    & (c) 		& Excluído 		\\ \hline
	54 & N/A 	   & N/A 		& Duplicado  	\\ \hline
	55 & N/A 	   & N/A 		& Duplicado  	\\ \hline
	56 & N/A 	   & N/A 		& Duplicado  	\\ \hline
	57 & N/A 	   & N/A 		& Duplicado  	\\ \hline
	58 & N/A 	   & N/A 		& Duplicado  	\\ \hline
	59 & (a)(b)    & (b)(c) 	& Excluído 		\\ \hline
	60 & N/A 	   & N/A 		& Duplicado  	\\ \hline
	61 & N/A 	   & N/A 		& Duplicado  	\\ \hline
	62 & (a)(b)    & (b) 		& Excluído 		\\ \hline
	63 & (a)(b)    & (b) 		& Excluído 		\\ \hline
	64 & N/A 	   & N/A 		& Duplicado  	\\ \hline
	65 & (a)(b)(d) & (b) 		& Excluído 		\\ \hline
	66 & N/A 	   & N/A 		& Duplicado  	\\ \hline
	67 & (a)(b)    & (c) 		& Excluído 		\\ \hline
	68 & (a)(b)    & (c) 		& Excluído 		\\ \hline
	69 & (a)(b)(d) & (b) 		& Excluído 		\\ \hline
	70 & (a)(b)    & (b)(c) 	& Excluído 		\\ \hline
	71 & N/A 	   & N/A 		& Duplicado  	\\ \hline
	72 & N/A 	   & N/A 		& Duplicado  	\\ \hline
	73 & N/A 	   & N/A 		& Duplicado  	\\ \hline
	74 & N/A 	   & N/A 		& Duplicado  	\\ \hline
	75 & (a)(b)    & (b)(c)		& Excluído 		\\ \hline
	76 & N/A 	   & N/A 		& Duplicado  	\\ \hline
	77 & N/A 	   & N/A 		& Duplicado  	\\ \hline
	78 & N/A 	   & N/A 		& Duplicado  	\\ \hline
	79 & N/A 	   & N/A 		& Duplicado  	\\ \hline
	80 & N/A 	   & N/A 		& Duplicado  	\\ \hline
	81 & N/A 	   & N/A 		& Duplicado  	\\ \hline
	82 & N/A 	   & N/A 		& Duplicado  	\\ \hline
	83 & N/A 	   & N/A 		& Duplicado  	\\ \hline
	84 & N/A 	   & N/A 		& Duplicado  	\\ \hline
	85 & N/A 	   & N/A 		& Duplicado  	\\ \hline
	86 & N/A 	   & N/A 		& Duplicado  	\\ \hline
	87 & N/A 	   & N/A 		& Duplicado  	\\ \hline
	88 & N/A 	   & N/A 		& Duplicado  	\\ \hline
	89 & N/A 	   & N/A 		& Duplicado  	\\ \hline
	90 & N/A 	   & N/A 		& Duplicado  	\\ \hline
	91 & N/A 	   & N/A 		& Duplicado  	\\ \hline
	92 & (a)(b)    & (c)(g) 	& Excluído 		\\ \hline
	93 & N/A 	   & N/A 		& Duplicado  	\\ \hline
	94 & N/A 	   & N/A 		& Duplicado  	\\ \hline
	95 & N/A 	   & N/A 		& Duplicado  	\\ \hline
	96 & N/A 	   & N/A 		& Duplicado  	\\ \hline
	97 & N/A 	   & N/A 		& Duplicado  	\\ \hline
	98 & N/A 	   & N/A 		& Duplicado  	\\ \hline
	99 & N/A 	   & N/A 		& Duplicado  	\\ \hline
	100 & N/A 	   & N/A 		& Duplicado  	\\ \hline
	101 & N/A 	   & N/A 		& Duplicado  	\\ \hline
	102 & N/A 	   & N/A 		& Duplicado  	\\ \hline
	103 & N/A 	   & N/A 		& Duplicado  	\\ \hline
	104 & N/A 	   & N/A 		& Duplicado  	\\ \hline
	105 & N/A 	   & N/A 		& Duplicado  	\\ \hline
	106 & N/A 	   & N/A 		& Duplicado  	\\ \hline
	107 & N/A 	   & N/A 		& Duplicado  	\\ \hline
	108 & N/A 	   & N/A 		& Duplicado  	\\ \hline
	109 & N/A 	   & N/A 		& Duplicado  	\\ \hline
	110 & N/A 	   & N/A 		& Duplicado  	\\ \hline
	111 & N/A 	   & N/A 		& Duplicado  	\\ \hline
	112 & N/A 	   & N/A 		& Duplicado  	\\ \hline
	113 & N/A 	   & N/A 		& Duplicado  	\\ \hline
	114 & N/A 	   & N/A 		& Duplicado  	\\ \hline
	115 & N/A 	   & N/A 		& Duplicado  	\\ \hline
	116 & N/A 	   & N/A 		& Duplicado  	\\ \hline
	117 & N/A 	   & N/A 		& Duplicado  	\\ \hline
	118 & N/A 	   & N/A 		& Duplicado  	\\ \hline
	119 & N/A 	   & N/A 		& Duplicado  	\\ \hline
	120 & N/A 	   & N/A 		& Duplicado  	\\ \hline
	121 & N/A 	   & N/A 		& Duplicado  	\\ \hline
	122 & N/A 	   & N/A 		& Duplicado  	\\ \hline
	123 & N/A 	   & N/A 		& Duplicado  	\\ \hline
	124 & (a)(b)   & (g) 		& Excluído 		\\ \hline
	125 & (a)(b)   & N/A 		& Incluído 		\\ \hline
	126 & (a)(b)   & (a)(c) 	& Excluído 		\\ \hline
	127 & (a)(c)   & (c) 		& Excluído 		\\ \hline
	128 & (a)(c)   & (c) 		& Excluído 		\\ \hline
	129 & (a)(c)(d)& N/A 		& Incluído 		\\ \hline
	130 & (a)(c)(d)& N/A 		& Incluído 		\\ \hline
	131 & (a)(c)(d)& N/A 		& Incluído 		\\ \hline
	132 & N/A 	   & N/A 		& Duplicado  	\\ \hline
	133 & (a)(c)   & (c)(b)		& Excluído 		\\ \hline
	134 & (a)(c)   & (c)(b)	 	& Excluído 		\\ \hline
	135 & (a)(c)   & (c)(b)	 	& Excluído 		\\ \hline
	136 & (a)(c)   & (c) 		& Excluído 		\\ \hline
	137 & (a)(c)   & (c) 		& Excluído 		\\ \hline
	138 & (a)(c)   & (c) 		& Excluído 		\\ \hline
	139 & (a)(c)   & (c) 		& Excluído 		\\ \hline
	140 & (a)(c)   & (c) 		& Excluído 		
 \end{longtable}


\subsection{Lista de Trabalhos Selecionados}

Após a avaliação de todas as referências retornadas pelos diversos ciclos de pesquisa, foram selecionados os seguintes trabalhos:\newline
\newline
015. VENEZIANO, Vito et al. Usability analysis of ERP software: Education and experience of users' as moderators. In: INTERNATIONAL CONFERENCE ON SOFTWARE, KNOWLEDGE, INFORMATION MANAGEMENT AND APPLICATIONS, 8th, 2014, Dhaka, Bangladesh. \textbf{Proceedings...}. Washington, D.c., Eua: IEEE Computer Society, 2014. p. 1 - 7. Disponível em: <http://dx.doi.org/10.1109/skima.2014.7083560>. Acesso em: 30 jan. 2017.\newline
\newline
031. LAMBECK, Christian et al. (Re-)Evaluating User Interface Aspects in ERP Systems: An Empirical User Study. In: HAWAII INTERNATIONAL CONFERENCE ON SYSTEM SCIENCES, 47., 2014, Waikoloa, Hawaii. \textbf{Proceedings...}. Washington, D.c., Eua: Ieee Computer Society, 2014a. p. 396 - 405.\newline
\newline
032. BABAIAN, Tamara et al. Applying design principles for enhancing enterprise system usability. In: INTERNATIONAL CONFERENCE ON SOFTWARE ENGINEERING AND APPLICATIONS, 9., 2014, Vienna, Austria. \textbf{Proceedings...}. Washington, D.c., Eua: Ieee Computer Society, 2014. p. 162 - 169.\newline
\newline
125. MELO, Espedito L. P. \textbf{Boa Usabilidade e comunicação eficiente de tarefas}: dois aliados na execução de processos em sistemas integrados de gestão. 2015. 152 f. Dissertação (Mestrado) - Curso de Ciência da Computação, Centro de Informática, Universidade Federal de Pernambuco, Pernambuco, 2015.\newline
\newline
129. YASSIEN, Eman et al. The Impact of ERP System's Usability on Enterprise Resource Planning Project Implementation Success via the Mediating Role of User Satisfaction. Journal Of Management Research, [s.l.], v. 9, n. 3, p.49-71, 27 jun. 2017. Macrothink Institute, Inc.. http://dx.doi.org/10.5296/jmr.v9i3.11186.\newline
\newline
130. SCHOLTZ, Brenda M; MAHMUD, Imran; T., Ramayah. Does Usability Matter? An Analysis of the Impact of Usability on Technology Acceptance in ERP Settin. \textbf{Interdisciplinary Journal Of Information, Knowledge, And Management}, [s.l.], v. 11, n. 1, p.309-330, 2016. Informing Science Institute. http://dx.doi.org/10.28945/3591.\newline
\newline
131. FOHRHOLZ, Corinna. Indikatorbasierte Messung der ERP-Usability. Erp Management, Gito Verlag, Berlim, v. 10, n. 4, p.25-28, abr. 2014.

\section{Analise dos Resultados do Mapeamento Sistemático}

Neste tópico é apresentado um resumo comparativo dos trabalhos considerados válidos na revisão, enfatizando os métodos ou técnicas utilizados na pesquisa, além dos principais conceitos empregados.
Após a aplicação dos critérios de inclusão e exclusão das obras, sete trabalhos foram classificados como válidos conforme Tópico 2.5.\newline
\newline
\indent Veneziano et al. (2014) busca estabelecer a relação da influência das informações demográficas dos usuários (ex: antecedentes educacionais e experiências de trabalho) sob a avaliação da usabilidade.\newline
\newline
\indent Para efetuar sua pesquisa Veneziano et al. (2014), utilizou um questionário para a  aquisição de dados e o método survey para a analise das informações.\newline
\newline
\indent O estudo conduzido por  Veneziano et al. (2014),  concluiu que a existe uma relativa influência do grau de formação educacional na percepção da usabilidade, porém elenca como trabalhos futuros o aprofundamento deste estudo em outras populações.\newline
\newline
\indent Lambeck et al. (2014a) citam que os  estudos nos ultimos 20 anos são focados apenas  em ERP's individuais em filiais especificas com pequenos grupos de usuários, o estudo usa uma ampla  amostra de 184 usuários distribuido em pequenas e médias empresas.\newline
\newline
\indent A pesquisa de Lambeck et al. (2014a) é uma revisitação de uma pesquisa feita em 2005, por este motivo a pesquisa tem  dois objetivos, primeiramente  avaliar se os problemas de  usabilidade identificados em 2005 se repetem quase 10 anos depois e amplia o foco da pesquisa para considerações adicionais, como o papel do tipo de menu, a incerteza no uso do sistema ou o suporte em situações problemáticas.\newline
\newline
\indent Lambeck et al. (2014a) avaliam que os problemas encontrados em pesquisas anteriores ainda existem na  atualidade, porém, afirmam que estes problemas  são menos criticos, assim concluem que ainda há alguns esforços necessários para alcançar a visão de uma interface ERP fácil de usar.\newline
\newline
\indent Fohrholz (2014) utiliza figuras-chave para usabilidade e as representa por meio de indicadores levando ao desenvolvimento de uma solução que possibilita a comparação de indicadores.\newline
\newline
\indent Os indicadores utilizados por Fohrholz (2014) são descritos a seguir.\newline
\newline
\indent Busca de informações, quanto menor o tempo do usuário for gasto em busca de informações interpretação de códigos de erro ou ter em busca de informações de ajuda.\newline
\newline
\indent Suporte a erros, um sistema com um alto grau de usabilidade fornecerá mecanismos apropriados para evitar erros.\newline
\newline
\indent Adequação a tarefa descreve até que ponto um sistema suporta o usuário no desempenho de suas tarefas.\newline
\newline
\indent Facilidade na personalização frequência e facilidade com que a personalização precisa ser feita ao sistema pelo usuário.\newline
\newline
\indent Capacidade auto-descritiva de um sistema é satisfeita quando o usuário sabe, em todos os momentos, onde ele está, como as ações devem ser executadas e se a ajuda está sempre disponível.\newline
\newline
\indent Fohrholz (2014) utilizou estes indicadores em três estudos de caso, dois destes estudos foram realizados em sistemas ERP's e o terceiro em um sistema não caracterizado como ERP, concluiu após estes estudos de caso, que é possível estender esta metodologia para sistemas de  diferentes tamanhos e para os mais  diversos fins além dos sistemas ERP.\newline
\newline
\indent Babaian, Xu e Lucas (2014) criaram intervenções baseadas em usabilidade tendo em vista a  orientação do usuário através de reprodução de vídeos com situações semelhantes as tarefas que estão executando, objetivado uma melhora no desempenho da usabilidade do sistema ERP. Foi desenvolvido um protótipo do ERP integra dois tipos de recursos de suporte ao usuário: visualização de processos e reprodução automatizada de interações de tarefas anteriores.\newline
\newline
\indent Babaian, Xu e Lucas (2014) consideram após aplicação deste protótipo em um estudo de caso, que a aplicação desses recursos possibilitou uma relevante melhoria na usabilidade do sistema ERP.\newline
\newline
\indent O estudo conduzido por Scholtz, Mahmud e Ramayah (2016) teve o objetivo de identificar qual o papel da usabilidade na aceitação do ERP por parte do usuário, sendo que a  pesquisa foi efetuada em uma população de 112 usuários do sistema ERP SAP na Índia, nas suas conclusões os pesquisadores  indicam que a usabilidade tem influência sobre a percepção e aceitação do sistema ERP por parte do usuário.\newline
\newline
\newline
\indent A pesquisa Yassien et al. (2017) foi aplicada na Índia em uma população de 106 gerentes em diferentes organizações. O objetivo da pesquisa foi identificar a importância da usabilidade de software para alcançar o ERP-PIS (ERP Project Implementation Success), concluindo que a usabilidade tem uma importante influência para alcançar o ERP-PIS. A pesquisa trata o uso da  Usabilidade na  implementação do ERP e não na pós-implantação que é o foco deste trabalho.\newline

\section{Conclusão do Mapeamento Sistemático}

O mapeamento sistemático foi conduzido no mês de julho de 2018. Ao todo foram retornadas 140 abordagens (IEEE=124, IBICT=3, e Literatura Cinzenta=13). Os trabalhos repetidos (66) foram excluídos e os remanescentes tiveram os seus resumos avaliados. Dos trabalhos restantes (74) foram consideradas elegíveis para a segunda fase desta revisão, que consistiu da leitura dos resumos.\newline
\newline
\indent Os trabalhos que, pelos seus resumos, não satisfizeram o contexto da pesquisa e demais critérios de inclusão e exclusão também foram excluídos, sobrando 7 pesquisas, que por atenderem apenas os critérios de inclusão, tiveram o seu conteúdo analisado por completo, para que assim pudessem compor a síntese da pesquisa.\newline
\newline
\indent A leitura das obras possibilitou a identificação de trabalhos relevantes para o objetivo da revisão, ou seja, encontrar as obras que contribuíssem para que as perguntas estabelecidas no foco de pesquisa fossem respondidas.\newline
\newline
\indent De uma forma geral, buscaram-se obras que relacionassem usabilidade com aceitação de sistemas  ERP na pós-implantação ou estudos da usabilidade em ERP e sua influência na percepção do usuário  (6 pesquisas).\newline
\newline
\indent Em linhas gerais, esta revisão mostrou a escassez literária relacionada ao escopo em questão, e apontou as diferenças encontradas na visão de cada autor em relação a como deve ser tratada a usabilidade em relação aos sistemas ERP. Porém, um consenso entre os  autores é do papel de protagonista da usabilidade face a aceitação do sistema e do sucesso ou não da implantação. A leitura das obras possibilitou a identificação de trabalhos relevantes para o objetivo da revisão, contribuindo para que as perguntas estabelecidas no foco de pesquisa fossem respondidas. O debate referente ao conteúdo destes artigos serão utilizados especialmente no capitulo Usabilidade x ERP, juntamente com os dados da pesquisa exploratória.

\setcounter{chapter}{2}
\renewcommand{\thesection}{\Alph{chapter}.\arabic{section}}
\addtocontents{toc}{\protect\setcounter{tocdepth}{3}}