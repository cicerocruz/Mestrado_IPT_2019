\chapter{Execução da Pesquisa} \label{Execução da Pesquisa}

Este capítulo é dividido em elaboração e estruturação do questionário, que trata das características do questionário que será aplicado, meio de distribuição da pesquisas e amostra e as características das amostras de dados que, se espera conseguir com a aplicação da pesquisa.\newline 
\indent A pesquisa que terá o prazo de 8 semanas, serão distribuídos os links para 2500 pessoas e espera-se um retorno de 10 \%, ou seja, 250 questionários respondidos por completo. O perfil dos participantes se resume a duas variáveis: eles são usuários de sistemas ERP que usam o ERP há no mínimo um ano.\newline 
\indent Usando a Sinhorini (2017), como referência estima-se a seguinte meta de respostas por região: sudeste (60\%), sul (18\%), centro-oeste e nordeste (20\%), norte(2\%).

\section{Elaboração e Estruturação do Questionário} \label{Elaboração e Estruturação do Questionário}

O questionário da \textit{survey} desta pesquisa foi utilizado em pesquisa que deu origem aos artigos Lambeck et al. (2014a) e Lambeck et al. (2014b), cedidos pelos autores, inclusive com os dados que foram colhidos, o que possibilitará a comparação dos resultados. Conforme informaram os autores da pesquisa original, a preocupação durante a criação do questionário foi minimizar o número de perguntas e estruturá-las de forma a poupar esforço e tempo dos respondentes durante a coleta de dados, sendo que o tempo de resposta máximo estipulado foi de 15 minutos.\newline 
\indent A escala definida para o registro de algumas variáveis foi: “Concordo Completamente”, “Concordo Parcialmente”, “Não Concordo, Nem Discordo”, “Discordo Parcialmente”, ”Discordo Completamente” e “Eu Não Sei”. A decisão de formatar a pergunta dessa forma deveu-se ao fato de se poder inferir mais que uma questão em uma única pergunta. O questionário foi dividido em seis partes e suas seções são descritas abaixo, o questionário completo, pode ser consultado no Apêndice A deste documento.\newline 
\newline
\noindent \textbf{Parte 1: Ambiente Empresarial} \newline

\indent Esta Seção tem o objetivo de posicionar a empresa no mercado em que atua, sendo solicitadas informações a respeito do número de funcionários, setor de atuação, área de influência e qual a posição hierárquica do pesquisado dentro da empresa.\newline 
\newline
\noindent \textbf{Parte 2: Informações sobre o sistema ERP} \newline

\indent O objetivo desta Seção é saber se o pesquisado usa o ERP, plenamente, sem o auxílio de ferramentas ou sistemas auxiliares, se usa o ERP em conjunto de ferramentas ou sistemas auxiliares ou se não usa o ERP e faz uso de ferramentas ou outros sistemas para executar seu trabalho. Esta seção possibilitará avaliar quais ERPs são utilizados pelo público pesquisado.\newline 
\newline
\noindent \textbf{Parte 3: Usabilidade} \newline

\indent Nesta Seção, são feitas questões sobre a usabilidade do sistema ERP, com o objetivo de avaliar a capacidade crítica do usuário em relação ao seu sistema ERP e das ferramentas ou softwares adicionais, sendo avaliadas a percepção do usuário acerca dos recursos disponibilizados nas ferramentas, formato do menu, acesso ao conhecimento dos processos e operações na ferramenta, acesso a funcionalidades e facilitadores. Esta seção contribuirá com a avaliação da percepção do usuário relativa à usabilidade do sistema ERP.\newline 
\newline
\noindent \textbf{Parte 4: Acesso ao Sistema} \newline

\indent Nesta Seção são avaliadas as formas de acesso ao ERP e se o ERP é compatível para utilização em dispositivos móveis. A utilidade dessas informações será avaliar a percepção do usuário relativa à forma de acesso ao sistema ERP.\newline 
\newline
\noindent \textbf{Parte 5: Dados Funcionais} \newline

\indent Composta de oito perguntas que ajudam a caracterizar o perfil do respondente e garantir que sua participação na pesquisa agregue valor aos resultados coletados. A anonimidade dos dados foi garantida na apresentação da pesquisa. Foram definidas questões para identificar o nível de experiência do respondente em relação ao uso de sistemas ERP. \newline 
\indent Essas informações são úteis na fase de análise para identificar correlações entre determinados grupos e suas posturas em relação à percepção do uso dos sistemas ERP. Em relação à pesquisa original foi incluída uma questão sobre a região do país onde o pesquisado trabalha. Essa pergunta adicionada teve o objetivo de possibilitar e considerar a diferenciação da influência cultural por região em nosso país.\newline 
\newline
\noindent \textbf{Parte 6: Dados de Contato} \newline

\indent Para motivar os participantes foi oferecido o envio dos dados resumidos da pesquisa, por e-mail ou cópia impressa, da dissertação para aqueles que fornecerem seus dados de contato.

\section{Meio de Distribuição} \label{Meio de Distribuição}

A construção do questionário é feita no \textit{Lime Survey}, hospedado em um \textit{site}, incluindo uma apresentação inicial, contendo o objetivo da pesquisa, o foco do trabalho, a garantia de anonimato dos dados coletados e informações para contato em caso de necessidade de maiores esclarecimentos. As perguntas foram configuradas para que não fosse possível o envio de respostas em branco, a não ser em campos adicionais de texto livre, disponíveis para os respondentes que sintam a necessidade de conceder maiores informações sobre suas respostas.\newline
\indent Também foram incluídos campos de nome e \textit{e-mail}, em uma tentativa de validar as respostas dos pesquisados, garantindo uma resposta por participante. O questionário proposto encontra-se no apêndice A da pesquisa.\newline

\section{Amostra} \label{Amostra}

Para Freitas et al. (2000), nenhuma amostra é perfeita, podendo haver variação no erro ou no viés. Para mitigar o risco em uma amostra alguns aspectos devem ser, fortemente, considerados como ter, claramente, definido o objetivo que se tem com a realização da pesquisa, o que dará melhores condições de assegurar que a amostra será adequada ou não; definir, objetivamente, os critérios de elegibilidade dos respondentes, ou seja, quais as condições definem se uma pessoa pode ou não participar da pesquisa.\newline
\indent Uma amostra pode ser caracterizada como probabilística e não probabilística. A principal característica de uma amostra probabilística é o fato de todos os elementos da população terem a mesma chance de serem escolhidos, resultando em uma amostra representativa da população. Uma amostra probabilística pode ser classificada em estratificada ou não estratificada. A amostra probabilística estratificada assegura que todos os tipos de intervenientes estejam presentes; cada subgrupo da população considerada dará origem a uma amostra, segundo o fator discriminante para a segmentação da população (FREITAS et al., 2000).\newline
\indent A amostra não probabilística é obtida a partir de algum tipo de critério e nem todos os elementos da população tem a mesma chance de ser selecionados, o que torna os resultados não generalizáveis. Guardando suas limitações esse tipo de amostra pode ser conveniente quando os responsáveis são difíceis de se identificar ou pertencem a grupos específicos ou, ainda, quando existe restrição no orçamento da pesquisa (FREITAS et al., 2000).\newline
\indent Conforme Cooper e Schindler (2011) há dois tipos principais de amostragem não probabilística, por julgamento e amostragem por quota. A amostragem por julgamento é uma forma de amostragem em que os elementos da população são selecionados, deliberadamente, com base no julgamento do pesquisador. Quando usada nos estágios iniciais de um estudo exploratório, uma amostra por julgamento é apropriada.\newline
\indent Quando desejamos selecionar um grupo diferenciado para fins de filtragem, esse método de amostragem também é uma boa escolha. A amostragem por quota é uma forma em que os participantes são escolhidos, proporcionalmente, atendendo a determinados critérios e a amostra é composta por subgrupos.\newline
\indent A decisão de adotar uma amostra não probabilística é influenciada por: tempo reduzido disponível para a distribuição, execução e análise da pesquisa; ausência de recursos financeiros e materiais; necessidade de um processo de amostragem acessível e descomplicado. Neste trabalho será usada uma amostragem não probabilística por julgamento.\newline
\indent Nesse formato são escolhidos membros da população, que são boas fontes de informação sobre o assunto pesquisado e que possam representar, da melhor maneira possível, dentro do contexto de uma amostra não probabilística, a população estudada.\newline
\indent Como uma maneira de garantir a amostragem por julgamento, algumas perguntas abordam o tempo de experiência do pesquisado com sistemas ERP e se é usuário do sistema ou um técnico. Essas informações servirão como uma ferramenta de filtragem na seleção dos elementos que serão incluídos na amostra.\newline
\indent O \textit{link} do questionário será distribuído, principalmente, por \textit{Whatsapp}, mas também será enviado por \textit{e-mail} e compartilhamento na rede social \textit{Facebook}.\newline
\indent Após o término da pesquisa, será efetuada uma filtragem para assegurar o critério usado na amostragem por julgamento, quando serão excluídos os registros que não se enquadravam no perfil da população pesquisada.

\section{Planejamento} \label{Planejamento}

Segundo Babbie (2005), uma \textit{survey} se divide em sete etapas: (1) identificação da questão de pesquisa; (2) elaboração do questionário; (3) pré-teste do questionário; (4) aplicação; (5) coleta dos dados; (6) tabulação dos dados e (7) análise dos dados.\newline
\indent O primeiro passo para realizar um \textit{survey} foi identificar, claramente, uma questão de pesquisa e objetivos a serem atingidos, o que foi feito na introdução deste trabalho. Depois disso, o próximo passo foi traduzir e adaptar o questionário, subsequentemente, foram testadas as funcionalidades da ferramenta de pesquisa e submetido o questionário para a avaliação da banca de qualificação. Após os ajustes solicitados pela banca, o questionário será divulgado e aplicado.\newline
\indent Após a finalização da pesquisa \textit{on-line} serão coletados os dados e realizada a tabulação das informações. Por fim serão analisados os dados do Brasil comparando-os com os da Alemanha.
