\imprimircapa

\imprimirfolhadeaprovacao

\imprimirfolhaderosto

% ---
% Inserir a ficha bibliografica
% ---

% Isto é um exemplo de Ficha Catalográfica, ou ``Dados internacionais de
% catalogação-na-publicação''. Você pode utilizar este modelo como referência.
% Porém, provavelmente a biblioteca da sua universidade lhe fornecerá um PDF
% com a ficha catalográfica definitiva após a defesa do trabalho. Quando estiver
% com o documento, salve-o como PDF no diretório do seu projeto e substitua todo
% o conteúdo de implementação deste arquivo pelo comando abaixo:
%
% \begin{fichacatalografica}
%     \includepdf{fig_ficha_catalografica.pdf}
% \end{fichacatalografica}
%\begin{fichacatalografica}
%	\vspace*{\fill}					% Posição vertical
%	\hrule							% Linha horizontal
%	\begin{center}					% Minipage Centralizado
%	\begin{minipage}[c]{12.5cm}		% Largura
%
%	\imprimirautor
%
%	\hspace{0.5cm} \imprimirtitulo  / \imprimirautor. --
%	\imprimirlocal, \imprimirdata-
%
%	\hspace{0.5cm} \pageref{LastPage} p. : il. (algumas color.) ; 30 cm.\\
%
%	\hspace{0.5cm} \imprimirorientadorRotulo~\imprimirorientador\\
%
%	\hspace{0.5cm}
%	\parbox[t]{\textwidth}{\imprimirtipotrabalho~--~\imprimirinstituicao,
%	\imprimirdata.}\\
%
%	\hspace{0.5cm}
%		1. Palavra-chave1.
%		2. Palavra-chave2.
%		I. Orientador.
%		II. Universidade xxx.
%		III. Faculdade de xxx.
%		IV. Título\\
%
%	\hspace{8.75cm} CDU 02:141:005.7\\
%
%	\end{minipage}
%	\end{center}
%	\hrule
%\end{fichacatalografica}
% ---



% ---
% Dedicatória
% ---
% Cícero-INICIO
%\begin{dedicatoria}
%\vspace*{\fill}
%\OnehalfSpacing
%Dedico este trabalho...

%\lipsum[1]

%\vspace*{\fill}
%\end{dedicatoria}
% ---
% Cícero-fim
% ---
% Agradecimentos
% ---
% Cícero-INICIO
%\begin{agradecimentos}
%\vspace*{\fill}
%\OnehalfSpacing
%Gostaria de agradecer...

%\lipsum[2]

%\vspace*{\fill}
%\end{agradecimentos}
% Cícero-FIM
% ---

% ---
% Epígrafe
% ---
%\begin{epigrafe}
%    \vspace*{\fill}
%	\begin{flushright}
%		\textit{``Não vos amoldeis às estruturas deste mundo, \\
%		mas transformai-vos pela renovação da mente, \\
%		a fim de distinguir qual é a vontade de Deus: \\
%		o que é bom, o que Lhe é agradável, o que é perfeito.\\
%		(Bíblia Sagrada, Romanos 12, 2)}
%	\end{flushright}
%\end{epigrafe}
% ---


\begin{resumo}
\normalsize

A realidade atual do mercado de sistemas integrados de gestão que popularmente são conhecidos como \textit{Enterprise Resource Planning} ou pela sigla (ERP), demandam soluções dinâmicas, fáceis e eficazes. Ocorre que os fornecedores de tais sistemas, não conseguem atender as expectativas cada vez mais prementes de celeridade e interação exigidas pelo mercado. Apesar da demanda crescente por sistemas simples e ambientes intuitivos, atendendo assim as expectativas cotidianas cada vez mais simples dos usuários, há uma barreira na usabilidade destes sistemas, que ao longo dos anos se faz cada vez mais insuperável. Existe, portanto, uma demanda por sistemas ERP fáceis de usar com o melhor resultado possível, sendo que a característica da usabilidade ganha importância, especialmente, quando se trata da seleção de ERP, o que torna o tema da usabilidade dos sistemas ERP de grande relevância com importância indiscutível nos campos da pesquisa acadêmica e na prática dos contextos da seleção, implantação e utilização de tais sistemas. Essa tendência mundial é antiga com referências que datam de 2005, podendo, ainda, ser verificada nos dias atuais em vasto material sobre o tema, sendo que nos trabalhos atuais pode-se citar as contribuições de Parks (2012), Veneziano , et al (2014), Lambeck, et al (2014b), Corinna (2014), Babaian, Xu e Lucas (2014), Sadiq, Pirhonen (2017). Em 2014 um estudo amplo realizado com 208 usuários em pequenas e  médias empresas em dois países (Alemanha e Letônia), expôs o tema sobre a percepção da Usabilidade em populações diferentes. A hipótese principal é que diversas características regionais levam a uma avaliação diferente dos problemas de usabilidade em sistemas ERP, deste modo, comparamos a pesquisa efetuada na Alemanha com a pesquisa efetuada no Brasil, utilizando-se a metodologia survey como na pesquisa alemã. Verificamos muitos indicadores semelhantes e outros que divergiram entre a população dos dois países.


\vspace{\onelineskip}

\noindent
\textbf{Palavras-chave:} Usabilidade, \textit{Enterprise Resource Planning}, Sistema Integrado de Gestão, Interface do Usuário, Iteração Humano Computador
 %\imprimirpalavraschave
\end{resumo}

% resumo em inglês
% Cícero-INICIO
%\begin{resumo}[Abstract]
%Resumo da dissertação em inglês.

%\lipsum[4]

%\vspace{\onelineskip}

%\noindent
%\textbf{Keywords:} \imprimirkeyword   
%\end{resumo}
% Cícero-FIM

% ---
% inserir lista de ilustrações
% ---
% Cícero-INICIO
{\SingleSpacing
\pdfbookmark[0]{\listfigurename}{lof}
\listoffigures*
\cleardoublepage
}
% Cícero-FIM
% ---

% ---
% inserir lista de quadros
% ---
%{\SingleSpacing
%\pdfbookmark[0]{\listofquadrosname}{loq}
%\listofquadros*
%\cleardoublepage
%}
% ---

% ---
% inserir lista de tabelas
% ---
% Cícero-INICIO
{\SingleSpacing
\pdfbookmark[0]{\listtablename}{lot}
\listoftables*
\cleardoublepage
}
% ---
% Cícero-fim

% ---
% inserir lista de abreviaturas e siglas
% ---

\begin{siglas}
  \item[ERP] Enterprise Resource Planning
  \item[IPT] Instituto de Pesquisas Tecnológicas do Estado de São Paulo
  \item[MRP] Material Requirement Planning
  \item[MRP II] Manufacturing Resources Planning
  \item[SGI] Sistema de Gestão Integrado
  \item[SIG] Sistema Integrado de Gestão
\end{siglas}

% ---

% ---
% inserir lista de símbolos
% ---
%\begin{simbolos}
%  \item[$ \Gamma $] Letra grega Gama
%  \item[$ \Lambda $] Lambda
%  \item[$ \zeta $] Letra grega minúscula zeta
%  \item[$ \in $] Pertence
%\end{simbolos}
% ---

% ---
% inserir o sumario
% ---
% Cícero-INICIO
{\SingleSpacing
\pdfbookmark[0]{\contentsname}{toc}
\tableofcontents*
\cleardoublepage
}
% ---